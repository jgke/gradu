\documentclass{article}

\usepackage{maarittely}

\title{Miksi C on vieläkin käytössä? \\ {\large Aihemäärittely}}
\author{
    Jaakko Hannikainen
}
\date{\today}

\begin{document}

\maketitle

C on ollut vallitseva ohjelmointikieli järjestelmäohjelmoinnissa C:n
alkuajoista lähtien. Useita ohjelmointikieliä on luotu historian saatossa,
joiden oli tarkoitus syrjäyttää C. C on vieläkin johtavana kielenä varsinkin
sulautetuissa järjestelmissä ja UNIX-pohjaisten käyttöjärjestelmien
vallitsevana järjestelmäohjelmointikielenä. Tarkoitus on tutkia, miksi
vaihtoehdoista huolimatta C on vieläkin laajalti käytössä myös uusissa
projekteissa, ja minkälainen ohjelmointikieli voisi syrjäyttää C:n.

Määritetään C:tä paremmaksi kieleksi jokin kieli, mikä C:hen verrattuna:

\begin{itemize}
    \item on yhtä nopea tai nopeampi
    \item käyttää saman verran muistia tai vähemmän
    \item on helpompi käyttää
    \item toimii järjestelmissä, joissa C toimii
\end{itemize}

Tällainen kieli on selkeästi olemassa, sillä C:stä voi tehdä helpomman käyttää
poistamalla C:n ominaisuuksista trigraphit, joiden poistaminen ei vaikuta
kielen ominaisuuksiin.

C:n vaihtoehdoiksi tutkitaan seuraavia kieliä: C++, Go, Ada, D, Rust. Näistä
kielistä tutkitaan, mikä tai mitkä ominaisuudet ovat estäneet C:n korvaamisen,
ja mitkä ominaisuudet ovat olleet parannuksia C:hen verrattuna. Lisäksi
tutkitaan muista suosituista ohjelmointikielistä ominaisuuksia, jotka ovat
hyödyllisiä matalan tason ohjelmoinnissa ja jotka voi toteuttaa korvaavan
kielen rajoitteissa. Suosituista ohjelmointikielistä tutkitaan seuraavia
kieliä: Python, Java, Haskell. Erityisesti tutkitaan kielen mahdollisuuksia

\begin{itemize}
    \item tyyppiturvallisuuteen
        (Haskell)
    \item kääntäjän suorittamaan optimointiin
        (esim. vektorisointi)
    \item kääntöaikaiseen koodin varmistamiseen
        (alustamattomat muuttujat)
    \item funktionaaliseen ohjelmointiin
    \item sivukanavahyökkäysten torjumiseen
        (ajoitushyökkäykset, välimuistihyökkäykset)
\end{itemize}

Tutkittavana on myös, mitä optimointeja C:ssä ei voi tehdä helposti johtuen
kielen rajoitteista, ja miten tämän voisi korjata. Näitä ominaisuuksia ovat
esimerkiksi sivuvaikutuksettoman ohjelmakoodin merkitseminen,
optimointivinkkien alustariippumaton ilmaiseminen (assertiot funktion
parametreihin) ja useat eri funktiot riippuen parametrien arvoista, mikäli ne
voidaan kääntöaikaisesti päätellä.

\nocite{ADA12}
\nocite{C11}
\nocite{CPP14}
\nocite{D}
\nocite{golang}

\bibliographystyle{abbrv}
\bibliography{references}

\end{document}

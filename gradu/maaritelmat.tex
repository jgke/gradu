\section{Määritelmät} 
\hilight{Tyylittely: Alkaako kappaleet omalta sivultaan?}

\hilight{Tähän enemmän humanismia ennen seuraavaa palautusta}

Määritetään C:tä paremmaksi kieleksi jokin kieli, mikä C:hen verrattuna:

\begin{itemize}
    \item on yhtä nopea tai nopeampi
    \item käyttää saman verran muistia tai vähemmän
    \item on helpompi käyttää
    \item toimii järjestelmissä, joissa C toimii
\end{itemize}

On huomioitava, että lukuisten olemassaolevien C:n kirjastojen, rajapintojen ja
projektien vuoksi yhteensopivuus C:n kanssa tulee olla saumatonta mahdollisten
vaihtoehtoisten ohjelmointikielten osalta, jotta kielen vaihtaminen olisi
mahdollista. Tämä sisältää myös kirkastorutiinien kutsumisen.

Tutkittavana on myös, mitä optimointeja C:ssä ei voi tehdä helposti johtuen
kielen rajoitteista ja miten tämän voisi korjata. Näitä ominaisuuksia ovat
esimerkiksi sivuvaikutuksettoman ohjelmakoodin merkitseminen,
optimointivinkkien alustariippumaton ilmaiseminen (annotaatiot funktion
parametreihin) ja useat eri funktiot riippuen parametrien arvoista, mikäli ne
voidaan kääntöaikaisesti päätellä.

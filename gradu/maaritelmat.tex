\section{Ohjelmointikielten vertailun määritelmät}

\hl{Tässä luvussa kerrotaan ohjelmointikielten taustoista ja yleisistä
ominaisuuksista, luodaan määrittelyt vertailuun liittyville kriteereille,
perustelut verrattavien kielten valinnoille sekä pohditaan erilaisia
mahdollisia syitä kielten suosioon. Tämän luvun sisältö on hyvin samankaltainen
tutkimussuunnitelman kirjallisuusosan toisen luvun kanssa.}

\hl{ 2 Taustaa: luvun alussa pitää selostaa, mitä asioita luvussa käsitellään.
Varsinkin, kun luku on vähän sillisalaatti sisällöltään muutenkin. }

\subsection{Ohjelmointikielten vertailun kriteerit}
\label{sec:abs}

\hl{Jotta kielten vertailun voi tehdä objektiivisesti, valitaan vertailua
varten jotkin kriteerit. Kriteereiksi valitaan suorituskyky, muistinkäyttö ja
yhteensopivuus C:n ja muiden ohjelmointikielten kanssa. Vertailtavat kielet
eivät saa olla C:tä huonompia millään osa-alueella.}

\hl{ Luvussa 2.4 voisi noita vertailukriteereitä tuoda selvemmin esiin
esimerkiksi taulukoimalla ne. }

Verrattavissa ohjelmointikielissä on pyritty parantamaan C:n huonoja puolia
hyvien puolien kustannuksella, usein lisäämällä kieleen turvallisuutta
parantavia ominaisuuksia tai tehden kielestä helppokäyttöisemmän esimerkiksi
automaattisella muistinhallinnalla. Tämä kuitenkin heikentää kielen tehokkuutta
tai alustariippumattomuutta tehden kielten suorasta vertailusta hankalaa.
Määrittelemällä absoluuttiset reunaehdot voidaan vertailla kieliä ehtojen
puitteissa objektiivisesti. Jos yksikin näistä kriteereistä ei pidä, verrattava
kieli ei ole aidosti C:tä parempi, vaan se häviää C:lle joissain osa-alueissa
ja vastaavasti voi olla parempi joissakin toisissa.

\hl{tehden -> toisto}

Tutkielmassa vertaillaan kolmea osa-aluetta ohjelmointikielistä: suorituskykyä,
muistinkäyttöä sekä yhteensopivuutta C:n ja muiden ohjelmointikielten kanssa
\hl{Miksi juuri nämä?}. Tutkielmassa käsitellään myös subjektiivisia kielten
ominaisuuksia, kuten kielen tiiviyttä\defword{terseness, expressiveness}, mutta
näitä ominaisuuksia ei huomioida kielten paremmuusvertailussa.

Ohjelmointikielellä kirjoitetun ohjelman tulee olla suoritusaikaisesti
vähintään yhtä nopea kuin vastaava C:llä kirjoitettu ohjelma. Kieli siis ei saa
vaatia ohjelmoijaa käyttämään mitään kielen ominaisuuksia, jotka voisivat
hidastaa ohjelmien suoritusta C:hen verrattuna. Monet suoritusaikaiset
turvallisuutta lisäävät ominaisuudet, kuten muistialueiden tarkistukset,
hidastavat kielen suoritusaikaista nopeutta.

Ohjelmointikielellä toteutettu ohjelma ei myöskään saa käyttää enempää muistia
niin suoritusaikaisesti kuin talletusvälineelläkään verrattuna vastaavaan
C-ohjelmaan. Tämä koskee myös vakiokirjastoa\defword{standard library} --
yksikin konkreettinen vakiokirjaston funktio linkitettynä ohjelmaan kasvattaa
ohjelman kokoa. Mikäli jokin vakiokirjasto toteutetaan, sen käyttäminen tulee
olla ohjelmoijalle täysin vapaaehtoista. Yksi tapa toteuttaa tämä on liittää
vain käytetyt funktiot osaksi ohjelmaa, jolloin käyttämättömät funktiot eivät
kasvata ohjelman kokoa.

Jos samaa vakiokirjastoa käytetään useassa ohjelmassa, tilankäytön kannalta on
edullisempaa säilöä vakiokirjasto jaettuna kirjastona, mutta tämä heikentää
kääntäjän mahdollisuuksia käännösaikaiseen optimointiin. Jos vakiokirjastoa ei
ole ladattu muistiin ohjelman käynnistyessä, ohjelman käynnistys voi kestää
hieman kauemmin. C:n vakiokirjasto liitetään usein moderneissa
käyttöjärjestelmissä ohjelmiin jaettuna kirjastona, sillä C:tä käytetään lähes
jokaisessa käyttöjärjestelmän ohjelmassa, ja näin jaetun kirjaston käyttäminen
välttää osan jaetun kirjaston huonoista puolista: kirjasto löytyy vain yhtenä
kopiona kiintolevyltä, ja se on valmiina ladattuna välimuistiin.

Ohjelmointikielen tulee olla täysin yhteensopiva C:n suoritusympäristön kanssa.
Tämä koskee C-koodin kutsumista C:n vierasfunktiorajapinnan
läpi\defword{Foreign function interface, FFI, \emph{käännös
\citealt[s.~8]{vierasfunktiorajapinta}}} sekä kielen funktioiden kutsumista
muiden ohjelmointikielten C-rajapinnan läpi. Kielen tulee näiden lisäksi toimia
kaikissa ympäristöissä, joissa C toimii. Kielen olisi myös hyvä tukea C:n
esikäsittelijää, jotta C:n käyttäminen verrattavan ohjelmointikielen kanssa
olisi mahdollisimman saumatonta.

\subsection{C:hen verrattavissa olevat ohjelmointikielet}

\hl{Vertailtaviksi kieliksi valitaan Ada, C++, D, Go ja Rust. Kaikki
vertailtavat kielet ovat kohtalaisen tehokkaita, jonka lisäksi kaikilla on
yritetty korvata C:n tai C++:n käyttöä.}

Historian saatossa on tehty useita C:n kilpailijoita, jotka ovat yrittäneet
parantaa C:tä joidenkin C:n hyvien puolien kustannuksella. Muutamat näistä ovat
päätyneet hyvin suosituiksi ohjelmointikieliksi, kuten esimerkiksi C++ ja Go.
Kielten suosion mittaamiseen on tehty useita projekteja, jotka vertailevat
kieliä esimerkiksi hakutulosten tai projektien mukaan. Näitä ovat esimerkiksi
TIOBEn ohjelmointikielten suosion indeksi~\citep{tiobe} ja GitHub-palvelun
julkaisema \mbox{Octoverse}~\citep{octoverse}.\hl{Voisi verrata muitakin
suosiolistoja?}
%Valitsemalla vertailuun suosittuja kieliä voidaan tutkia, miksi muuten
%selkeästi täysin käyttökelpoinen kieli ei ole syrjäyttänyt C:tä.

Koska tutkimuskysymyksessä vertaillaan ohjelmointikieliä suorituskyvyn ja
muistinkäytön suhteen, vertailuun kannattaa ottaa mukaan vain tehokkaita kieliä
-- korkeamman tason ohjelmointikielet on tarkoitettu ohjelmointinopeuden
parantamiseen ja turvallisempien ohjelmistojen toteuttamiseen nopeiden
ohjelmien sijaan. Tällöin kielen suorituskyky on heikompi. Yksi kattava
suorituskykyä mittaava vertailu on Benchmarks Game~\citep{benchmarks}, jossa
pyritään kirjoittamaan mahdollisimman nopea ohjelma pysyen silti kielelle
idiomaattisessa lähdekoodissa\footnote{Hyvin monessa kielessä voi kirjoittaa
C:hen verrattavissa olevaa matalan tason ohjelmointia, mutta Benchmarks Gamessa
on tarkoituksena välttää tätä.}.

TIOBEn listasta Ada, C++, D, Go ja Rust nousevat esiin verrattavina kielinä.
Kaikki viisi kieltä ovat tehokkaita. Tämän lisäksi jokaisen kielten historiassa
on ollut tavoitteena korvata C:n tai C++:n käyttö, kuten luvussa~\ref{sec:muut}
kerrotaan.

Viime vuosina on tehty myös useita C:hen käännettäviä ohjelmointikieliä, jotka
ovat jääneet pitkälti ilman mitään näkyvyyttä, kuten LISP/c~\citep{clisp1},
C-Mera~\citep{clisp2}, Carp~\citep{clisp3} ja Nymph~\citep{nymph}. LISP/c,
C-Mera ja Carp ovat LISP-perheeseen kuuluvia C:ksi kääntyviä ohjelmointikieliä,
jotka pyrkivät parantamaan C:n syntaksia korvaamalla sen LISP-perheen
syntaksilla. Nymph taas on olio-ohjelmointikieli. Erityisesti Carp on tämän
tutkielman kannalta kiintoisa ohjelmointikieli, sillä se on C:ksi kääntyvä
ohjelmointikieli, joka on suunniteltu mahdollisimman suorituskykyiseksi.

%\hl{Tässä voisi olla jotain pohdintaa, miksi nuo ovat jääneet huomiotta ja
%miten tämän voisi välttää. Erityisesti Carp tuntuu tätä tutkielmaa vastaavalta
%kieleltä, olisi ikävää jättää se täysin huomiotta. Tosin tällaiset pohdinnat
%jäänee puhtaasti spekulaatioksi, sillä noista ei ole erityisen paljoa dataa
%saatavilla.}

\subsection{Kielten suosioon vaikuttavat tekijät}

\hl{Analyysiä muista syistä kielen suosioon, erityisesti Meyerovichin ja
Rabkinin tekemän tutkimuksen perusteella. Tämä antaa tietoa ominaisuuksista,
jotka ohjelmointikielessä olisi hyvä olla.}

Eräässä tutkimuksessa~\citep{empiricalpopularity} tutkittiin syitä
ohjelmointikielten valintaan. \hl{Olisi hyvä olla ainakin 1 muu tutkielma
viittauksessa.} Yhden tutkimuksen järjestämän kyselyn (s. 8,
\mbox{Slashdotissa} julkaistu kysely, n=1679) perusteella kielen valintaan
vaikuttaa avoimen lähdekoodin kirjastojen saatavuus, olemassa olevien ohjelmien
jatkokehitys sekä kielen tunnettavuus ohjelmoijien keskuudessa -- tutkimuksen
mukaan ohjelmoijat siis suosivat jo käytettyjä ohjelmointikieliä uusien kielten
sijaan. Saman kyselyn vastaajat arvioivat suorituskyvyn turvallisuutta
tärkeämmäksi.

Saman tutkimuksen järjestämässä Slashdot-sivustolla julkaistussa kyselyssä noin
40\% vastaajista arvioi tärkeäksi kriteeriksi työkalut. Kyselyn perusteella
kielen olisi siis hyvä tarjota toimivat työkalut, kuten
ohjelmistopakettien~\defword{software package, \emph{käännös
\citealt[s.~18]{ohjelmistopaketti}}} hakemiseen
paketinhallintajärjestelmän\defword{package manager}, nopean ja
käyttäjäystävällisen käännöstyökalun sekä valmiudet olemassa oleviin
kehitysympäristöihin integroitumiselle. Olemassa olevien C-ohjelmistojen
tukeminen on välttämätöntä mutta haastavaa johtuen C:n ekosysteemin
monimuotoisuudesta, erityisesti lukuisista kääntämistyökaluista.

Tutkimuksessa selvitettiin myös suosittuja ominaisuuksia ohjelmointikieliltä
(s. 13, SaaS MOOC -kurssin yhteydessä oleva kysely, n=415). Useita
tutkimuksessa selvitetyistä suosituimmista ominaisuuksista ei ole mahdollista
toteuttaa johtuen luvussa~\ref{sec:abs} määritetyistä rajoitteista, kuten
poikkeuksia ja rajapintoja. Useat muut tutkimuksessa esiin nousseet
ominaisuudet, kuten suorituskyky, ovat taas suoraan rajotteiden mukaisesti osa
verrattavan ohjelmointikielen tavoitteita.

Tutkimuksessa myös verrattiin tiettyjen toteamuksien, kuten
\emph{''\foreignlanguage{english}{This language has a strong static type
system}''} keskinäistä korrelaatiota. Kielen tiiviys
(''\emph{\foreignlanguage{english}{This language is expressive}}'') korreloi
eniten (korrelaatiokertoimella 0.76) kielestä pitämisen kanssa (s. 13, The
Hammer Principle -sivustolla julkaistu kysely).

%Eräässä blogikirjoituksessa \citep{microsoftdictperf} vertaillaan C\#:n ja
%C++:n suorituskykyä. Kirjoituksessa C\#:llä kirjoitettu yksinkertainen toteutus
%suoriutuu tehtävästä nopeammin ja yksinkertaisemmin kuin optimoitu C++-ohjelma.
%Vasta usean C++-ohjelma päihitti C\#-toteuden nopeudella. C\#-toteutus
%kuitenkin vei noin neljä kertaa enemmän muistia kuin C++-toteutus.

Tutkimuksen perusteella valmiiksi suosittuja kieliä käytetään enemmän myös
uusissa projekteissa. Olemassa olevien kirjastojen tärkeyttä korostetaan
useassa kohdassa tutkimusta. Täysin C:n kanssa yhteensopiva kieli voi käyttää
C:lle tehtyjä kirjastoja, jolloin kielellä on käytettävissään laaja C:n
ekosysteemi\footnote{Esimerkiksi GitHubista hakusanalla 'library' löytyy yli
26\,000 C:llä kirjoitettua projektia.}.

Uusia ohjelmointikieliä opetellessa ohjelmoijat turvautuvat aikaisemmista
kielistä opittuihin käytäntöihin~\citep{languagelearning}. Uusien
ohjelmointikielten käyttöönottoa helpottaa siis muiden vastaavien kielten
osaaminen, sillä aikaisempi kokemus tukee uuden kielen opiskelua.
Suunnittelemalla C:n korvaajan C:n kanssa samankaltaiseksi kieleksi voidaan
pienentää uuden kielen opettelun kynnystä. Kielen eriävät ominaisuudet olisi
siis hyvä toteuttaa siten, että ne ovat mahdollisimman helppoja oppia C:stä
uuteen ohjelmointikieleen siirtyvälle ohjelmoijalle.

%Helppokäyttöisyys \\
%- turvallisuus (esim. tyypit) \\
%- esim. annotaatiot \\
%- selkeä mitä tekee

%\subsection{Mahdollisia kielen ominaisuuksia}
%
%Kielen ominaisuudet vaikuttavat sillä rakennettujen ohjelmistojen
%arkkitehtuuriin. Tyyppiluokat tai rajapinnat kannustavat vahvan
%tyypityksen kautta tyyppiturvalliseen ohjelmointiin, kun taas dynaamiset kielet
%(esimerkiksi LISP-perheen kielet) kannustavat nopeaan kehitystahtiin staattisen
%turvallisuuden hinnalla. Yksinkertaiset kielet, kuten C ja Go, tarjoavat usein
%vain yhden selkeän tavan toteuttaa yksittäiset funktiot, kun taas
%monimutkaisemmat kielet tarjoavat lukuisia vaihtoehtoja.\citationneeded
%
%BitC-ohjelmointikielen sähköpostilistalla käydyssä keskustelussa pohdittiin
%mahdollisia ongelmia vahvan tyypityksen (etenkin tyyppiluokkien) käytöstä
%matalan tason ohjelmoinnissa~\citep{bitc}. Shapiron sähköpostissa todetaan,
%että tyyppiluokkia ei voi toteuttaa ilman suoritusaikaista tukea.
%
%On huomioitava, että lukuisten olemassa olevien C:n kirjastojen, rajapintojen
%ja projektien vuoksi yhteensopivuus C:n kanssa tulee olla saumatonta
%mahdollisten vaihtoehtoisten ohjelmointikielten osalta, jotta kielen
%vaihtaminen olisi mahdollista. Tämä sisältää myös kirkastorutiinien kutsumisen
%muista ohjelmointikielistä, sillä C on lukuisissa järjestelmissä \emph{lingua
%franca}, jonka avulla ohjelmointikielet pystyvät kommunikoimaan keskenään.
%Esimerkiksi Python-ohjelmointikieli~\citep{python} sisältää tuen C-funktioiden
%kutsumiseen~\citep{pythonffi}, jota voi käyttää muiden ohjelmointikielten
%funktioiden kutsumiseen, mikäli käyttäjä kirjoittaa ''sillan'' C-ohjelmana.
%Käytännössä jokaisesta aktiivisesti käytetystä ohjelmointikielestä on
%mahdollista kutsua C-koodia.
%
%Ohjelmointikielten ekosysteemit koostuvat eri tahojen luomista kirjastoista. On
%tärkeää, että näihin kirjastoihin pääsee käsiksi mahdollisimman helposti.

% Kirjastojen lisenssien tulee myös olla yhteensopivia, jotta kirjastoja voi
% käyttää yhdessä toistensa kanssa. Eri ohjelmointikieliekosysteemeissä on
% käytössä erilaisia lisenssejä -- JavaScript-kirjastot ovat usein
% MIT-yhteensopivia, kun taas Java-kirjastot ovat usein Apache 2 -yhteensopivia.
% Liitteessä~\ref{app:github} on taulukko GitHub-verkkopalvelun sisältämien
% julkisten projektien määrä ryhmiteltynä lisenssin ja kielen mukaan.
%
% Kirjastot voi myös julkaista useammalla kuin yhdellä lisenssillä, jolloin
% käyttäjät voivat päättää, mitä lisenssiä haluaa seurata. On kuitenkin
% filosofinen kysymys, onko esimerkiksi Apache2+GPL parempi lisenssi kuin pelkkä
% GPL, sillä Apache2 ei vaadi tiettyjä oikeuksia loppukäyttäjille, kuten pääsyä
% ohjelmiston lähdekoodiin~\citep{apachetldr, gpl3tldr}. Apache2 siis antaa
% ohjelmoijille enemmän vapauksia GPL3:een verrattuna, mutta käyttäjien vapaus
% esimerkiksi muokata ohjelmistoa kärsii tästä.

\subsection{Makrojärjestelmät}

\hl{Tässä aliluvussa selitetään makrojärjestelmien käsite, kerrotaan
makrojärjestelmien tarpeellisuudesta ja käyttökohteista sekä verrataan
vaihtoehtoisia toteutuksia makrojärjestelmiin.}

\hl{ Luku 2.7: Tämä on jotenkin orpo aliluku. Ohjelmaesimerkit on selostettu
huonosti tai ei ollenkaan. }

Ohjelmointikielten makrojärjestelmillä tarkoitetaan ohjelmointikielen
ominaisuuksia, joita voidaan käyttää käännösaikaiseen lähdekoodin luomiseen ja
muuntamiseen. Lähdekoodin luonti käännösaikaisesti voi pienentää huomattavasti
tarvittavaa lähdekoodin määrää tietyissä tilanteissa aiheuttamatta
suoritusaikaisia haittoja. Ensimmäiset makrojärjestelmät rakennettiin
symbolisen konekielen käsittelyyn, jolloin makrojärjestelmiä käytettiin
poistamaan turhaa toistoa ohjelmien lähdekoodista.

Rust käyttää vakiokirjastonsa lähdekoodissa paljon makroja erityisesti
alkeistyyppien liittyvien määrittelyjen kohdalla. Rustin vakiokirjaston
kokonaislukutyyppejä määrittelevässä moduulissa \texttt{std::num} on käytetty
makroja huomattavan paljon lähdekoodin määrän vähentämiseen -- tiedoston
\texttt{mod.rs} lähdekoodin koko kasvaa noin 5\,000 rivistä yli 20\,000 riviin,
kun makrot laajennetaan. Kaikki makroilla määritellyt funktiot ovat käytännössä
identtisiä tyyppejä tai tiettyjä vakioita lukuun ottamatta.
Ohjelmassa~\ref{fig:ruststdnum} on kaksi tällaista funktiota, jotka toteuttavat
\texttt{From}-piirteen eri tyyppiparametreille). Käyttämällä makrojärjestelmää
taataan paras mahdollinen suorituskyky, pitäen kuitenkin lähdekoodin selkeänä
ja ylläpidettävänä. Makrojärjestelmän käyttäminen määrittelyihin pakottaa
yhtenäiset määrittelyt kaikille alkeistyypeille, jolloin ohjelmointivirheiden
määrä pysyy mahdollisimman pienenä. \hl{Ehkä parempi: saattaa vähentää virheitä}

\begin{listing}[ht!]
    \inputminted{Rust}{koodi/ruststdnum.rs}
    \caption{Esimerkissä olevat Rust-funktiot ovat esimerkin
    \texttt{impl\_from} -makrolla generoituja. Esimerkki on hieman
    yksinkertaistettu Rustin standardikirjastosta löytyvästä esimerkistä.}
    \label{fig:ruststdnum}
\end{listing}

Lähdekoodin luominen käännösaikaisesti voi kasvattaa lopullisen ohjelman kokoa,
sillä makron sisältö kopioidaan makrokutsun paikalle. Toisaalta ohjelmakoodin
luonti voi myös mahdollistaa optimointeja, joita kääntäjä ei olisi pystynyt
tekemään tavalliselle funktiolle. Kääntäjästä riippuen lähdekoodin
sijoittaminen funktiokutsun paikalle voi esimerkiksi mahdollistaa paremman
silmukoiden vektorisoinnin tai kuolleen koodin poistamisen~\citep{cinlining}.
Ohjelmoijalle tulisi siis antaa mahdollisuus päättää, haluaako hän
ohjelmoidessaan käyttää ohjelmointikielen makrojärjestelmää vai perinteisiä
funktioita.

Makroja käyttäessä voi tapahtua ristiriitoja esimerkiksi luodessa väliaikaisia
muuttujia, joilla on sama tunniste kuin aikaisemmin lähdekoodissa
määritellyillä muuttujilla~\citep{macrohygiene}. Esimerkiksi C:n makrojen
tulokset voivat symbolien törmäyksien lisäksi aiheuttaa syntaksivirheitä, kuten
ohjelmassa~\ref{fig:cmacrofail} näytetään. Jos ohjelmointikielen
makrojärjestelmä estää tällaiset ristiriidat, sitä kutsutaan hygieeniseksi
makrojärjestelmäksi.

\begin{listing}[ht!]
    \inputminted{C}{koodi/cmacrofail.c}
    \caption{C-makrot voivat aiheuttaa laajennetussa muodossa yllättäviä
    tuloksia muuttujanimien törmäyksien ja syntaksivirheiden vuoksi. Ohjelmoija
    odottaa tulokseksi ohjelmaa, joka tulostaisi arvon 2, mutta lopullisen
    ohjelman tulostama viesti on määrittelemätöntä toimintaa alustamattoman
    muuttujan käytön vuoksi.}
    \label{fig:cmacrofail}
\end{listing}

Makrojärjestelmiin on runsaasti vaihtoehtoisia toteutustapoja. Yksinkertaisin
vaihtoehto on jättää makrot kokonaan pois ohjelmointikielestä, mikä tosin
rajoittaa ohjelmointikielen mahdollisuuksia käännösaikaiseen
laajennettavuuteen.

Tekstialkioiden korvaamiseen perustuvat makrojärjestelmät ovat erittäin
yksinkertainen ja joustava tapa toteuttaa käännösaikainen makrojärjestelmä.
Tekstialkioiden korvaamisessa makroprosessori etsii syötteestä tiettyjä
tekstialkioita ja korvaa niitä joukoilla tekstialkioita. Tekstialkioiden
korvaaminen toimii yleensä erillisenä osana kääntämistä, eikä makroilla ole
tällöin tietoa varsinaisen ohjelmointikielen tyypeistä tai tunnisteista.

Mallipohjaisissa makrojärjestelmissä asetetaan tyyppejä tai arvoja ennalta
määritettyyn malliin, yleensä luokkiin tai funktioihin. Mallipohjaiset
makrojärjestelmät eivät voi tehdä kaikkea, mitä tekstialkioiden korvaamiseen
perustuvat makroprosessorit voivat tehdä. Mallipohjaiset järjestelmät voivat
käyttää hyväksi ohjelmointikielen tyyppijärjestelmää ja asettaa käännösaikaisia
rajoitteita mallin parametreille, mikä mahdollistaa paremmat käännösaikaiset
tarkistukset ohjelmien oikeellisuudesta. Monet mallipohjaisen ohjelmoinnin
käyttökohteista ovat suunnattu geneeristen funktioiden tai luokkien luomiseen
eikä syntaksin muuttamiseen.

Syntaksipuupohjaiset makrojärjestelmät käsittelevät ohjelmointikielten
syntaksipuita. Syntaksipuiden käsittely pelkkien tekstialkioiden sijaan
mahdollistaa uusien kontrollirakenteiden määrittämisen yksinkertaisesti ja
turvallisesti. Puhtaassa syntaksipuiden käsittelyssä sekä syötteen että makron
laajennetun version on oltava ohjelmointikielen syntaksin mukaisia, eli
syntaksipuita käsittelevä makro ei voi käsitellä mitä tahansa syötettä.

Monikäyttöisimmät makrojärjestelmät ovat proseduraalisia makrojärjestelmiä.
Proseduraalisissa makrojärjestelmissä ohjelmoija voi käyttää samaa
ohjelmointikieltä sekä tavalliseen ohjelmointiin että makrojen toteutukseen.
Proseduraaliset makrojärjestelmät vaativat yleensä enemmän koodia makrojen
toteutukseen verrattuna muihin makrojärjestelmiin, mutta toisaalta
mahdollistavat kaiken, mitä ohjelmointikielellä voisi tehdä. Proseduraalinen
makrojärjestelmä voisi esimerkiksi lukea tiedostoja käännösaikaisesti ja
vaikuttaa lopulliseen lähdekoodiin tiedoston sisällöstä riippuen.

Mikään makrojärjestelmä ei ole täydellinen, vaan jokaisessa makrojärjestelmässä
on sekä hyviä että huonoja puolia. Kun makrojärjestelmän ilmaisuvoimaa
kasvatetaan, sen käyttäminen muuttuu monimutkaisemmaksi. \hl{Missä mielessä?}

%\hl{Osa tästä seuraavaan lukuun?}
%
%C:n ja C++:n tekstin korvaamiseen perustuvassa makrojärjestelmissä korvataan
%tekstialkioita ja funktion kaltaisia makroja uusilla sekvensseillä alkioita. Ei
%rekursiota yleisessä tapauksessa, ei turing-täydellinen. Yksinkertainen
%yksinkertaisissa tapauksissa.
%
%C++:n ja D:n mallipohjaisissa järjestelmissä ... C++ erittäin hidas kääntää.
%Turing-täydellinen. D:n mallipohjainen ei turing-täydellinen.
%
%Rust syntaksipohjainen, ottaa tyypitettyjä ja pattern match. Oma DSL, joka
%eroaa jonkin verran varsinaisesta kielestä.
%
%D ja Rust myös proseduraalinen (D: string mixin + ctfe), you can do everything!
%
%C:n ja C++:n esikäsittelijä on tekstialkioiden korvaamiseen perustuva
%makroprosessori.
%
%C++:n ja D:n mallit mahdollistavat metaohjelmoinnin, tosin D vaatii kaikkien
%parametrien olevan tyyppejä.
%
%Rustin makrojärjestelmä perustuu syntaksipuiden käsittelyyn.
%Rustin makrojärjestelmä on käännösaikaisesti tyypitetty, ja sisältää tyypit
%esimerkiksi tunnisteille ja lausekkeille.
%
%Rust sisältää syntaksipuupohjaisen makrojärjestelmän lisäksi kokeellisen
%proseduraalisen makrojärjestelmän.
%

%\subsection{Vertailtavien kielten makrojärjestelmät}
%
%\hl{Tarkista kieli}

%Ada, D ja Go eivät sisällä mahdollisuutta käännösaikaisille makroille. C ja C++
%käyttävät samaa makrojärjestelmää, joka on kuitenkin hyvin
%rajoittunut~(\citeauthor{CPP17}, \citeyear{CPP17}, luku~19; \citeauthor{C18},
%\citeyear{C18}, luku~6.10).
%
%C:n ja C++:n makrot ajetaan ennen muuta koodin käsittelyä, eli makroja voi
%sijoittaa mihin tahansa koodia., kun taas C:n ja C++:n ei\footnote{C:n ja
%C++:n makrojärjestelmä ei ole edes primitiivirekursiivinen, sillä makrot eivät
%voi sisältää itseään.}.

\section{Määritelmät} 

Verrattavissa ohjelmointikielissä on yritetty parantaa C:n huonoja puolia
hyvien puolien kustannuksella, usein lisäämällä kieleen turvallisuutta
parantavia ominaisuuksia. Tämä kuitenkin heikentää kielen tehokkuutta tai
alustayhteensopivuutta, tehden kielten suorasta vertailusta mahdotonta. Jos
määritellään absoluuttiset reunaehdot, voidaan näiden puitteissa vertailla
kieliä.

Määritellään C:tä paremmaksi kieleksi jokin kieli, mikä C:hen verrattuna aina:

\begin{itemize}
    \item on yhtä nopea tai nopeampi
    \item käyttää saman verran muistia tai vähemmän
    \item toimii kaikissa järjestelmissä, joissa C toimii
    \item on helpompi käyttää \hilight{paitsi että tämä on subjektiivista}
    \item \hilight{Onko jotain, mistä voisi sanoa suoraan, että se on parannus?}
\end{itemize}

Jos yksikin näistä kriteereistä ei pidä, verrattava kieli ei ole absoluuttisesti
C:tä parempi.

On huomioitava, että lukuisten olemassaolevien C:n kirjastojen, rajapintojen ja
projektien vuoksi yhteensopivuus C:n kanssa tulee olla saumatonta mahdollisten
vaihtoehtoisten ohjelmointikielten osalta, jotta kielen vaihtaminen olisi
mahdollista. Tämä sisältää myös kirkastorutiinien kutsumisen muista
ohjelmointikielistä, sillä C on lukuisissa järjestelmissä \emph{lingua franca},
jolla voidaan sillata ohjelmointikieliä toisiinsa.

Ohjelmointikielten ekosysteemit koostuvat eri tahojen luomista kirjastoista. On
tärkeää, että näihin kirjastoihin pääsee käsiksi mahdollisimman helposti. Tämä
vaatii toimiakseen hyvät työkalut, kuten paketinhallinnat, käännöstyökalu sekä
kehitysympäristöt.

Kirjastojen lisenssen tulee myös olla yhteensopivia, jotta kirjastoja voi
käyttää yhdessä toistensa kanssa. Eri ekosysteemeissä on erilaisia lisenssejä
-- JavaScript-kirjastot ovat usein MIT-yhteensopivia~\citationneeded, kun taas
Java-kirjastot ovat usein Apache 2 -yhteensopivia~\citationneeded. (Mitä C:ssä
nyt onkaan eniten, GPL, LGPL, BSD, Apache2... taitaa olla aika sekasikiö.
Tekstiä lisää tästä kunhan saan tutkittua)
Kirjastot voi myös julkaista useammalla kuin yhdellä lisenssillä, jolloin
käyttäjät voivat päättää, mitä lisenssiä haluaa seurata. On kuitenkin
filosofinen kysymys, onko esimerkiksi Apache2+GPL parempi kuin pelkkä GPL,
sillä Apache2 ei vaadi tiettyjä oikeuksia loppukäyttäjille.\hilight{Teksti
vähän tökkii}

\subsection{Verrattavien ohjelmointikielten valinta}

C:hen verrattavien ohjelmointikielten valintaan käytettiin

TIOBEn indeksi~\citep{tiobe} ja Octoverse~\citep{octoverse} suosion
mukaan järjestämiseksi, ei valita aivan tuntematonta kieltä

Benchmarks gamen~\citep{benchmarks} tuloksia, nähdään tehokkaat kielet

Valittiin Ada, C++, D, Go, Rust

Muita liittyviä kieliä: Fortran, Objective-C \\
\hilight{Pitäisikö nämä vain ottaa joukon jatkoksi? Menisi helposti samaan syssyyn} \\
\hilight{Fortran toisaalta vähän hassu tähän koska se on tehty ennen C:tä}

%Muita tehokkaita kieliä:

%Halide; innovatiivinen DSL, jossa on erikseen
%algoritmi ja sen rinnakkaistaminen. Ei yleiskäyttöinen, mutta hieno ideatasolla

Muita huomattavia kieliä:

LISP-perhe: makrot jotka käsittelee S-exprejä eikä
vain läimi tekstiä ympäriinsä,

ML-perhe: erinomainen syntaksi tyyppi- ja funktiomäärittelyihin sekä pattern
matchaukseen, sääli että kaikki kovemman tason tyypitys käytännössä vaatii
interfacejen toteutuksen mikä taas on hankalaa absolutismissa (tradeoff
vtable/function pointereita kaikkialle/C++:n ja Rustin inlinetys jolloin
binääri kasvaa).  Esim. enumeihin voisi silti saada hyvän matchauksen

C-subsetit jotka voi formaalisti verifioida esim. 
\url{https://github.com/PrincetonUniversity/VST}
\url{http://vst.cs.princeton.edu/}

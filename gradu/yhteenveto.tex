\section{Yhteenveto}

Tutkielmassa selvitettiin mahdollisuuksia parantaa C-ohjelmointikieltä. Tätä
tarkoitusta varten luotiin uusi ohjelmointikieli, Purkka. Tutkimuksessa
verrattiin useita erilaisia ohjelmointikieliä C:hen suorituskyvyn sekä
muistinkäytön perusteella, mutta yksikään verrattavista ohjelmointikielistä ei
ollut C:tä nopeampi kuin yksittäisissä mittauksissa. Purkka tarjoaa pienen
syntaktisen parannuksen C:hen ilman ajoaikaisia haittoja. C:n parantaminen on
siis mahdollista ainakin syntaksin osalta.

Tutkielmassa selvitettiin tärkeimpiä C:n ominaisuuksia, joiden takia C on yhä
käytössä. Ominaisuuksista nousi esiin kielen yksinkertaisuus, tehokkuus
ja alustariippumattomuus. Mahdollisten korvaavien kielten tulisi ymmärtää
C:n esikäsittelijää varten kirjoitettuja otsikkotiedostoja, jotta kielestä
toiseen siirtyminen olisi mahdollisimman yksinkertaista. Kielen tulisi myös
mahdollistaa yksittäisten funktioiden muuntaminen kerrallaan uuden
ohjelmointikielen mukaisiksi, jotta kielestä toiseen siirtyminen ei vaadi
huomattavia investointeja.

Jotta kielestä toiseen siirtyminen olisi kannattavaa, uuden kielen tulisi
tarjota huomattavia parannuksia C:hen verrattuna. Tutkielmassa löydettiin
mahdollisiksi ominaisuuksiksi makrojärjestelmän ja syntaksin parantaminen sekä
summatyypit. Nämä eivät kuitenkaan välttämättä ole yksinään tarpeeksi suuri
tekijä, jotta C:stä luopuminen olisi järkevää.

Verrattavat ohjelmointikielet häviävät C:lle eniten yhteensopivuuden osalta.
Vain C++ osaa käsitellä C:n otsikkotiedostoja, kun muissa verrattavissa
kielissä on epäkäytännöllistä kutsua jopa yksittäisiä C-funktioita.

Tutkielmassa määritellystä ohjelmointikielestä voisi ottaa yksittäisiä
ominaisuuksia uusiin C-standardeihin vaikuttamatta yhteensopivuuteen vanhempien
standardien mukaisiin ohjelmiin. Tämä onnistuu ainakin summatyypeille, jotka
voisi lisätä C-standardiin tavallisen \texttt{enum}- tai
\texttt{union}-tyyppien yhteyteen.

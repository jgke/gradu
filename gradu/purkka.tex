\section{Purkka-ohjelmointikieli}
\label{sec:purkka}

Jotta uudesta ohjelmointikielestä saisi tutkielman vertailukriteerien
mukaisesti C:tä paremman kielen, uuden kielen suunnittelussa tulee seurata
tarkasti vertailukriteerejä. Erityisesti C-yhteensopivuus on tärkeä ominaisuus,
jota muissa ohjelmointikielissä ei ole pidetty tärkeänä ohjelmointikielen
suunnittelussa. Tässä luvussa ensin käsitellään periaatteet, joilla Purkasta
voisi tulla C:tä parempi ohjelmointikieli tutkielman kontekstissa, jonka
jälkeen käsitellään tarkemmin Purkan tyypitystä, syntaksia sekä
C-yhteensopivuutta.

\subsection{Purkan suunnitteluperiaatteet}

Luvut~\ref{sec:clyhyesti},~\ref{sec:ctaustaa}~ja~\ref{sec:cominaisuudet}
nostavat tärkeinä C:n ominaisuuksina kielen yksinkertaisuuden ja tehokkuuden,
jotka ovat mahdollistaneet kielen leviämisen järjestelmästä toiseen.
Luvussa~\ref{sec:cominaisuudet} nostetaan tämän lisäksi esiin vaatimus
C-ohjelmien kirjoittamisesta funktio kerrallaan uudella kielellä, jotta
kielestä toiseen siirtyminen olisi ylipäätään mahdollista ilman kohtuutonta
investointia. Luvussa~\ref{sec:ckehitettavat} esitellään lukuisia C:n
syntaktisia ongelmia, jotka voi muuttaa tehden kielestä helppolukuisemman,
kuten tyyppipäättelyn lisääminen ohjelmointikieleen.

Käännösaikaista varmennettavuutta voi parantaa lisäämällä esimerkiksi
kieleen summatyypit, jotka löytyvät esimerkiksi Rustista. Yhteensopivuus nousee
myös esiin luvussa~\ref{sec:suosio}, jossa kirjastojen saatavuus näytetään
tärkeäksi ohjelmointikielen valintakriteeriksi.

Uuden kielen määrittelyssä tulee pitää luvun~\ref{sec:abs} kriteerit, jotta
ominaisuuksia päättäessä ei muodostu esimerkiksi yhteensopivuusongelmia C:n
kanssa. Tämä rajoittaakin suurta osaa luvussa~\ref{sec:muut} esiteltyjä
ominaisuuksia, kuten automaattista muistinhallintaa ja poikkeuksia. Monet
ominaisuudet myös monimutkaistaisivat kieltä tarjoamatta kuitenkaan
tehokkuusparannuksia. Yksittäisiä ominaisuuksia kuitenkin pystyy lisäämään,
kuten luvuissa~\ref{sec:go}~ja~\ref{sec:rust} esiintyvät useat paluuarvot sekä
luvussa~\ref{sec:rust} esitellyt summatyypit, sillä näiden sisällyttäminen
kieleen mahdollistaa paremman käännösaikaisen todentamisen ilman
suoritusaikaisia haittoja. Summatyypit voivat myös parantaa
kääntäjäoptimointia, jos kääntäjä pystyy poistamaan ohjelmakoodia päättelemällä
tietyt summatyypin arvot mahdottomiksi.

Uusi kieli tulisi kääntää C:ksi, jotta sen käyttäminen on mahdollista kaikissa
järjestelmissä, joissa C:tä käytetään. C:ksi kääntäminen myös mahdollistaa
C-yhteensopivien kirjastojen käyttämisen ilman erillistä
yhteensopivuuskerrosta. Kielen tulee myös ymmärtää C:llä kirjoitettuja ohjelmia
sisältäen myös C:llä kirjoitetut makrot ja otsikkotiedostot. Esimerkiksi
POSIX-C:n \texttt{errno}-muuttuja voi olla määritelty makrona ja
\texttt{errno}-muuttujan käyttäminen ilman esikäsittelijää esimerkiksi
viittaamalla suoraan \texttt{errno}-nimiseen muuttujaan on määrittelemätöntä
toimintaa~\citep[s. 234]{POSIX}, joten \texttt{errno}-arvon lukeminen vaatii
tuen C:n esikäsittelijälle.

\subsection{Tyypit}

Jotta yhteensopivuus C:n kanssa olisi joustavaa, Purkan tyyppijärjestelmä
muodostetaan mahdollisimman paljon C:n kaltaiseksi. C:n tyyppisyntaksi sisältää
useita erilaisia tapoja ilmaista samaa pohjatyyppiä -- esimerkiksi
\texttt{long}, \texttt{signed~long}, \texttt{long~int} ja
\texttt{signed~long~int} ilmaisevat kokonaislukutyyppiä, joka pystyy
sisältämään ainakin luvut $[-(2^{31} - 1), 2^{31}-1]$\footnote{Käytännössä
modernit toteutukset käyttävät kahden komplementtia kokonaislukujen
ilmaisemiseen, jolloin 32-bittinen kokonaislukumuuttuja voi sisältää luvut
$[-2^{31}, 2^{31} - 1]$.}~\citep{C18}. Tyypit kuten \texttt{long} ja
\texttt{signed long} ovat kuitenkin standardien mukaisessa C:ssä keskenään
täysin vaihdettavissa, eli Purkan ei tarvitse pystyä kääntymään jokaiseen
mahdolliseen vaihtoehtoon, vaan tyyppijärjestelmässä voi olla yksi tyyppi joka
vastaa kaikkia \texttt{long}-tyypin vaihtoehtoisia kirjoitusasuja.

C:n kokonaislukutyypeistä (\texttt{char}, \texttt{short}, \texttt{int},
\texttt{long}) \texttt{char}-tyyppi on ainoa, jolla \texttt{char},
\texttt{signed char} ja \texttt{unsigned char} ovat kolme erillistä
tyyppiä~\citep{C18}, kun muilla alkeistyypillä esimerkiksi \texttt{signed int}
on sama tyyppi kuin \texttt{int}. Merkkijonot koostuvat
\texttt{char}-tyyppisistä alkioista, kun taas \texttt{signed char} ja
\texttt{unsigned char} vastaavat tavun kokoista kokonaislukua ja
epänegatiivista kokonaislukua. Purkka-kielessä C:n \texttt{char}-tyyppiä
vastaa \texttt{char}, kun taas \texttt{signed char} ja \texttt{unsigned char}
ovat \texttt{i8} ja \texttt{u8}.

Tyyppien kääntäminen C:stä Purkaksi ja takaisin on yksinkertaista: jokaista
C-tyyppiä vastaa täsmälleen yksi Purkka-tyyppi. Jokaiselle Purkan
alkeistyypille taas on määritelty yksi C-tyyppi, johon kyseinen Purkka-tyyppi
käännetään. Liitteessä~\ref{app:purkka} on taulukko C:n ja Purkan tyypeistä. 

C:n alkeistyyppien, osoittimien, taulukoiden ja yhdistetyyppien lisäksi
määritellään ohjelmoinnin tehostamiseksi sekä käännösaikaisen optimoinnin ja
oikeellisuuden parantamiseksi yksittäisiä lisätyyppejä. Nämä tyypit ovat
epätyhjät osoittimet sekä summatyypit.

Ensimmäiseksi määritellään epätyhjä osoitin, joka on osoitin, joka ei salli
arvokseen C:n \texttt{NULL}-arvoa. Tyyppi kääntyy Purkasta standardien
mukaiseksi C:ksi yksinkertaisesti vastaavaan C:n osoitintyyppiin. Epätyhjä
osoitin mahdollistaa kääntäjäoptimointeja, jotka eivät muuten olisi
mahdollisia. GCC-kääntäjä mahdollistaa epätyhjät osoittimet
\texttt{\_\_attribute\_\_((nonnull))}-määreellä, jota Purkka-kääntäjä voi
käyttää ohjelmoijan niin pyytäessä.

\begin{listing}[ht!]
    \inputminted{Rust}{koodi/sumtype.prk}
    \caption{Summatyyppi Purkka-kielessä ja sama summatyyppi käännettynä C-kielelle.}
    \label{fig:purkkatree}
\end{listing}

Toiseksi määritellään summatyyppi. Summatyyppi on yhdistelmä C:n
\texttt{struct}, \texttt{union} ja \texttt{enum} -tyypeistä, ja se mahdollistaa
käännösaikaisesti varmennetun \texttt{union}-tyyppien käytön. Summatyyppi kääntyy
\texttt{struct}-tyypiksi, jossa on \texttt{enum}-tyyppinen muuttuja summatyypin
diskriminanttina ja \texttt{union}-tyyppi, joka sisältää kaikki summatyypin
variantit.

Ohjelmassa~\ref{fig:purkkatree} on Purkka-kielellä kirjoitettu summatyyppi
\texttt{Puu}, jossa on \texttt{Lehti}- ja \texttt{Haara}-variantit.
\texttt{Lehti}-variantti sisältää 32-bittisen kokonaisluvun, kun taas
\texttt{Haara}-variantti sisältää osoittimet kahteen alipuuhun. Kääntäessä
tietorakenne muutetaan liitteen~\ref{app:purkka}
ohjelman~\ref{fig:purkkatreecompile} kaltaiseksi lähdekoodiksi. Tässä
erikoistapauksessa kääntäjä voisi optimoida tietorakennetta
liitteen~\ref{app:purkka} ohjelman~\ref{fig:purkkatreecompile2} mukaisesti
säilömällä \texttt{\_Puu\_d} -tyypin tiedon \texttt{\_Puu\_Lehti}- ja
\texttt{\_Puu\_Haara}-tyyppien ensimmäiseen alkioon: jos osoitin on tyhjä, on
kyseessä \texttt{Lehti}-variantti, muulloin \texttt{Haara}-variantti.

\subsection{Syntaksi}

Purkan syntaksi on hyvin samankaltainen C:n syntaksiin verrattuna. Tämä
helpottaa kielen oppimista aikaisemman tutkimuksen
mukaisesti~\citep{languagelearning}. Samankaltainen syntaksi myös helpottaa
nykyisten ohjelmien uudelleenkirjoitusta, sillä ohjelmien rakennetta ei
tarvitse muuntaa. Suurimmat erot C:hen liittyvät tyyppien kirjoittamiseen,
jossa syntaksia on muutettu modernien ohjelmointikielten syntaksin mukaiseksi.

Funktiomäärittelyt alkavat \texttt{fun}-avainsanalla. Tämä noudattaa usean muun
modernin ohjelmointikielen tapaa aloittaa funktiomäärittely avainsanalla.
Vertailun vuoksi Rust käyttää avainsanaa \texttt{fn}, Go käyttää avainsanaa
\texttt{func} ja Kotlin käyttää avainsanaa \texttt{fun}. Avainsanan
käyttäminen yksinkertaistaa kielen kielioppia ja jäsentäjän toteutusta.

Purkka-kieli sisältää tyyppipäättelyn, joka helpottaa koodin kirjoittamista,
kun kääntäjä pystyy päättelemään muuttujien tyypit. C-kielen mahdollisuuksia
automaattiseen tyyppipäättelyä on tutkittu
aikaisemminkin~\citep[mm.][]{ctypeinference}. Koska Purkan syntaksi on
yksiselitteinen, monet artikkelissa esitetyt C:n ongelmat eivät ole Purkassa
haittana. Toisaalta esimerkiksi yhteenlasku toimii samoin kuin C:ssä, joten sen
tyyppipäättely ei ole triviaalia: jos C:ssä tai Purkassa toinen
yhteenlaskettavista on osoitin, summan tyyppi on osoitin, muutoin tyyppi
muodostuu C:n ''tavallisten'' lukujen muunnossääntöjen mukaisesti.

Koska tyyppipäättely toteutetaan puhtaasti käännösaikaisesti, se ei voi
aiheuttaa suoritusaikaisia haittoja. Tyyppisyntaksi itsessään muistuttaa paljon
Rustin tyyppisyntaksia.

C:n \texttt{static}-avainsana on jaettu käytön mukaisesti kahteen avainsanaan.
Funktiomäärittelyissä ja globaaleissa muuttujissa C:n
\texttt{static}-avainsanaa vastaa Purkan \texttt{pub}-avainsana (''public'' eli
julkinen), mutta käänteisellä merkityksellä: jos \texttt{pub}-avainsanaa ei ole
käytetty, koodi käyttäytyy kuin siihen olisi lisätty C:n
\texttt{static}-avainsana. Funktioiden sisällä Purkka tukee myös
\texttt{static}-avainsanaa, joka toimii samalla tavalla kuin C:n
\texttt{static} käytettynä funktioiden sisällä.

C käsittelee tietueiden ja taulukoiden alustamista identtisellä syntaksilla.
Purkka erottaa nämä kaksi syntaksia erikseen ohjelman~\ref{fig:structinit}
tavalla. Ohjelmassa alustetaan \texttt{Esimerkki}-tietue sekä kolmen
kokonaisluvun taulukko. Tietueen tyypin nimen pitäminen mukana
tietueliteraaleissa pitää tyypin selkeästi näkyvissä myös tyyppipäättelyä
käytettäessä. Yksinkertaisemmissa tapauksissa tämä ei vaikuta lähdekoodin
pituuteen, mutta monimutkaisten tietueiden alustuksessa Purkka-koodi on hieman
pidempää kuin vastaava C-koodi.

\begin{listing}[ht!]
    \inputminted{Rust}{koodi/structinit.prk}
    \inputminted{C}{koodi/structinit.c}
    \caption{Tietueen ja taulukon alustaminen Purkassa ja C:ssä.}
    \label{fig:structinit}
\end{listing}

Suurin osa C:n lauseista on Purkassa lausekkeita. Esimerkiksi \texttt{if-else}
lausepari on Purkassa lauseke, joka mahdollistaa selkeämmän koodin
kirjoittamisen. C:ssä vastaavia lausekkeita voi kirjoittaa käyttämällä
välimuuttujaa tai yksinkertaisissa tapauksissa \texttt{?:}-operaattorilla.

\subsection{C-yhteensopivuus}

Koska Purkan tulee olla myös makrojen osalta C-yhteensopiva, Purkka osaa
laajentaa C-makroja. Purkka-kääntäjä pystyy muuntamaan makrokutsun
C-lähdekoodiksi, laajentamaan sen C-esikäsittelijällä ja muuntamaan makron
laajennetun muodon takaisin Purkka-koodiksi.

%Purkassa on myös oma makrojärjestelmä, joka toimii hahmotunnistuksella ja
%rekursiolla.\footnote{Referenssikääntäjään (ks. luku \ref{sec:results}) ei ole
%toteutettu makroja.} Jos makrot eivät sisällä rekursiota tai hahmontunnistusta,
%ne voidaan muuntaa C-makroiksi.

Purkka ei sisällä yhtään suoritusaikaista ominaisuutta, joita ei ole C:ssä.
Tämä pitää kielen yksinkertaisena ja C-yhteensopivana. Tämä toisaalta tekee
kielellä ohjelmoinnista työläämpää verrattuna muihin moderneihin
ohjelmointikieliin.

Useat C-toteutukset sisältävät erilaisia laajennoksia, joiden tarkoituksena on
mahdollistaa erilaisten alustariippuvaisten ominaisuuksien käyttö. Benchmarks
Gamessa yksi käytetyimmistä laajennoksista on SIMD-tyypit, joita varten
esimerkiksi GCC-kääntäjällä on oma syntaksinsa. GCC:llä SIMD-tyypit voidaan
määritellä käyttämällä \texttt{\_\_attribute\_\_((vector\_size))} -määrettä.
Purkka tukee useita GCC:n laajennoksia, muun muassa vektorityyppejä.

Syntaksi sisältää myös \texttt{pragma}-avainsanan, jota voidaan käyttää
vastaavasti kuin C:n esikäsittelijän \texttt{pragma}-direktiiviä erilaisten
laajennosten käyttämiseen. Esimerkiksi OpenMP-laajennosta käytetään pragmojen
läpi.

Jotta Purkkaa voisi käyttää mahdollisimman helposti nykyisten järjestelmien
kanssa, Purkka käännetään C-koodiksi. Tämä mahdollistaa olemassa olevien
C-kääntäjien käyttöä Purkan kääntämiseen kaikille mahdollisille alustoille,
joille on olemassa standardien mukainen C-kääntäjä. Esimerkiksi
\texttt{make}-työkalua käyttäville projekteille riittää yksi Makefile-sääntö
Purkka-ohjelmien kääntäminen C-ohjelmiksi. Tämä sääntö on esitetty
ohjelmassa~\ref{fig:makefile}. Muissa käännösautomaatiosovelluksissa Purkan
integrointi osa kääntämistä on luultavasti yhtä helppoa.

\begin{listing}[ht!]
    \inputminted{Makefile}{koodi/Makefile.kieli}
    \caption{Kaksi riviä Makefile-syntaksia riittää Purkan integroimiseen
    Make-ohjelmaa käyttäviin projekteihin.}
    \label{fig:makefile}
\end{listing}

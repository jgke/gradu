\section{C-ohjelmointikielen taustaa}

\hl{Miten C on siirtynyt tähän päivään? Miten C:tä voisi parantaa?}

\subsection{C-ohjelmointikieli lyhyesti}
\label{sec:clyhyesti}

\hl{Lähteitä lisää.}

\begin{listing}[ht!]
    \inputminted{C}{koodi/hello.c}
    \caption{Yksinkertainen hello world -ohjelma toteutettuna C:llä.}
    \label{fig:helloc}
\end{listing}

C on Dennis Ritchien 1970-luvun taitteessa kehittämä yksinkertainen matalan
tason ohjelmointikieli~\citep{chistory}, jota käytetään nykypäivänä erityisesti
järjestelmäohjelmoinnissa sekä sulautetuissa järjestelmissä. C ei aseta
ohjelmoijalle rajoituksia muistinkäsittelyyn, mikä mahdollistaa tehokkaan mutta
turvattoman ohjelmoinnin. C on \citeauthor{tiobe}:n (\citeyear{tiobe}) mukaan
yksi tällä hetkellä käytetyimmistä ohjelmointikielistä. C ei ole vuoden 1989
ANSI-standardin jälkeen muuttunut mainittavasti~\citep{chistory, C18}, vaan
ohjelma~\ref{fig:helloc} näyttää samalta kuin vuonna 1988 julkaistussa
\emph{The C Programming Language} -kirjassa~\citep{krsecond}. Ohjelma tulostaa
näytölle merkkijonon \emph{Hello,~World}.

C kehitettiin B- ja BCPL-ohjelmointikielten pohjalta näiden osoittautuessa
kömpelöiksi vaihtaessa laitteistoarkkitehtuuria, kuten
luvussa~\ref{sec:ctaustaa} kerrotaan. Uudella ohjelmointikielellä tavoiteltiin
B:tä nopeampaa, mutta BCPL:ää yksinkertaisempaa kieltä Unix-käyttöjärjestelmän
kehitykseen~\citep{chistory}.

Toisin kuin monissa nykypäivänä suosituissa kielissä, C ei tarjoa automaattista
muistinhallintaa, vaan ohjelmoijan täytyy itse hallita sekä muistin varaaminen
että sen vapauttaminen. Tämä on pitänyt kielen toteutuksen hyvin tehokkaana ja
yksinkertaisena, mikä on mahdollistanut C:n leviämisen järjestelmästä toiseen.
Kääntöpuolena muistinhallinta jätetään ohjelmoijan vastuulle, monimutkaistaen
ohjelmien toteutusta.

C:n mahdollistama suorituskyky on johtanut kielen suosioon tehokkuutta
vaativissa sovelluksissa, kuten verkkopalvelimissa ja tietokannoissa.
Toisaalta C:n yksinkertainen lähestymistapa muistinkäsittelyyn on
mahdollistanut tarkkaa muistinhallintaa vaativien ohjelmistojen toteutuksen,
kuten käyttöjärjestelmien ytimien tai sulautettujen järjestelmien ohjelmien.
Tämä yhdistelmä on johtanut C:n laajaan käyttöön käyttötarkoituksesta
riippumatta.

Koska C:llä on helppoa saada aikaan erilaisia muistinkäsittelyyn liittyviä
tietoturvaongelmia, useita työkaluja on kehitetty havaitsemaan ja estämään
näitä~(mm.~\citet{valgrind,asan}). Myös lukuisia ohjelmointikieliä on luotu
toteuttamaan ''turvallinen C'', usein lisäämällä jokaisen osoittimen käytön
yhteyteen tarkistuksen, osoittaako osoitin ohjelman omistamaan muistiin.

\hl{Kieliperheistä liittyvään juttuun ei löydy yhtään järkevää lähdettä. Tästä
bulkkipaperi-idea?}

Koska C on suunniteltu mahdollisimman yksinkertaiseksi kieleksi, siitä ei löydy
omaa moduulijärjestelmää. Mikäli ohjelmoija haluaa käyttää jonkun toisen
tiedostojen sisältämää funktiota, hänen pitää joko kirjoittaa
funktioprototyypit tai sisällyttää ohjelmaan erityinen otsikkotiedosto käyttäen
C:n makrojärjestelmää.

C on vaikuttanut lukuisien kielten kehitykseen, ja useiden kielien sanotaankin
olevan osa C-kieliperhettä. C-kieliperheeseen kuuluviin kieliin liittyy usein
muun muassa imperatiivinen ohjelmointityyli, perinteinen merkintätapa (eli
infix) lausekkeiden muodostamiseen, muuttujien näkyvyysalueiden rajoittaminen
kaarisuluilla sekä staattinen tyypitys. Näitä piirteitä näkyy
ohjelmassa~\ref{fig:helloc}: kaarisuluilla rajoitettu \texttt{main}-funktio
kutsuu \texttt{printf}-funktiota, joka tulostaa käyttäjälle merkkijonon
\emph{Hello,~World}. Funktio ottaa parametriksi kokonaisluvun \texttt{argc} ja
osoittimen merkkijonolistaan \texttt{argv}.

Suurin osa nykypäivänä käytetyimmistä ohjelmointikielistä on osa
C-kieliperhettä, kuten TIOBEn indeksissä~(\citeyear{tiobe}) kärkiviisikosta
löytyvät C:n lisäksi Java, C++ ja C\#. Python on ainoa kärkiviisikossa oleva
ohjelmointikieli, joka ei suoraan ole osa C-kieliperhettä, mutta Pythonin
referenssi-implementaatio CPython on nimensä mukaisesti toteutettu
C:llä ja Pythonilla~\citep{cpython}.

\subsection{C-ohjelmointikielen historiaa ja nykypäivää}
\label{sec:ctaustaa}

\hl{Tässä aliluvussa kerrotaan C-kielen taustasta, alkuperäisistä
käyttökohteista, historiasta, suosion kehityksestä sekä nykytilanteesta.
Erityisesti keskitytään C:n tämänhetkisiin käyttökohteisiin, joista selviää,
mitä ohjelmointikielten ominaisuuksia tulisi käsitellä ohjelmointikielten
vertailussa.}

\hl{ Luvussa 2.2 ihmetyttää, että ovatko nämä asiat nyt keskeisimmän piirteet
C-kielessä? Miksei muita kielen ominaisuuksia selosteta? }

Dennis~\citet{chistory} käsittelee C-kielen historiaa ja tekemiään
suunnittelupäätöksiä artikkelissaan \emph{The Development of the C Language}.
Artikkelin mukaan C kehitettiin pitkälti 70-luvulla B- ja
BCPL-ohjelmointikielien pohjalta. Ritchie tavoitteli uudella
ohjelmointikielellä B-ohjelmointikielen tehokkuuden parantamista tarkemmalla
tyypityksellä. B-kielen ainoa primitiivityyppi oli sana\defword{word\emph{,
B-kielen tyyppinä \texttt{cell}}}, sillä B oli suunniteltu ajettavaksi
tietokoneilla, joissa muistiosoittimet osoittivat aina yksittäisiin sanoihin.
Tämä kuitenkin käytännössä osoittautui haastavaksi esimerkiksi merkkijonojen
käsittelyyn, sillä jokaiseen sanaan mahtuu useita tavun kokoisia merkkejä. Tämä
tarkoittaa ohjelmoidessa sitä, että ohjelmoija joutuu purkamaan käsin
yksittäisen sanan merkeiksi, käsittelemään näitä yksittäin ja lopuksi
yhdistämään merkit uudelleen sanoiksi. Epäkäytännöllisyyden lisäksi tämä oli
tehotonta, kun B-kieli muutettiin tavupohjaiselle PDP-11 -tietokoneelle
säilyttäen kuitenkin sanapohjaiset muistiosoittimet -- ohjelmoijan oli pakko
käyttää sanarajoilla toimivia muistiosoittimia, vaikka tavupohjainen
muistiosoitin olisi tehokkaampi ratkaisu.

B-kielen taulukot poikkesivat huomattavasti nykyisistä C:n taulukoista. Jos
B-koodissa luodaan kymmenen alkion taulukko \texttt{A}, niin ohjelmaa
ajettaessa varattaisiin kymmenen sanan kokoinen taulukko ja \texttt{A}:han
asetettaisiin tämän taulukon osoitin. C:ssä \texttt{A} muutettaisiin
osoittimeksi määrittelyhetken sijaan vasta, kun muuttujaa käytettäisiin
lausekkeessa.

Tämän lisäksi Ritchien mukaan B-ohjelmointikielestä puuttui kokonaan
liukulukujen käsittely. Vaikka PDP-11 ei tukenut liukuluvuilla laskentaa,
valmistaja oli luvannut tuen tälle. BCPL-ohjelmointikieleen lisättiin
liukulukulaskenta olettaen, että liukuluku mahtuisi yhteen sanaan, mikä ei
pitänyt paikkaansa 16-bittisellä PDP-11 -tietokoneella.

\grayrule

Artikkelin mukaan C-ohjelmointikielen kehitys alkoi vahvemmalla tyypityksellä:
C-kielen ensimmäiseen varsinaiseen esiasteeseen oli lisätty \texttt{char}- ja
\texttt{int}-tyypit sekä muistiosoittimet näihin tyyppeihin. Tässä vaiheessa
C-tyylisten taulukoiden sijaan taulukot toimivat kuin B-kielen taulukot. Kun
C:hen lisättiin tietuetyypit, tämä muistimalli ei toiminut enää, vaan
taulukoiden käsittelyä muutettiin vastaamaan nykystä C:n mallia. Nyt taulukot
muunnettiin muistiosoittimiksi vasta, kun taulukkoa käytettiin lausekkeissa, ja
taulukon määrittelyssä alkioiden ja osoittimen sijaan varattiin muistia vain
taulukon alkioille.

Tästä tavasta käsitellä taulukoita seurasi kuitenkin ominaisuus, joka on myös
nykypäivän C:ssä: jos funktio ottaa parametrikseen taulukon, kääntäjä muuttaa
parametrin tyypin muistiosoittimeksi, sillä funktiokutsut ovat
lausekkeita\footnote{Jos funktio ottaa esimerkiksi \texttt{char[2]} -tyyppisen
parametrin, funktio saakin parametrikseen muistiosoittimen taulukon sijaan.
Tässä tapauksessa funktio saa siis moderneilla tietokoneilla kahden tavun
sijaan kahdeksan tavun kokoisen parametrin. Tämän ominaisuuden voi kiertää
säilömällä taulukon tietueen sisään.}.

Tämän lisäksi C-kieleen lisättiin tyyppi funktio-osoittimille.
Määrittelysyntaksin logiikan perusteena toimi lausekkeiden syntaksi. Jos
jostain muuttujasta saa \texttt{int}-arvon, kun lausekkeessa lukee
\texttt{(*muuttuja)()}, niin \texttt{muuttuja} määritellään kirjoittamalla
\texttt{int~(*muuttuja)();}. Monimutkaisissa tapauksissa on kuitenkin hankala
erottaa tyyppejä toisistaan, kuten erot tyyppien ''osoitin taulukkoon
kokonaislukuja'' eli \texttt{int~(*muuttuja)[]} ja ''taulukko osoittimia
kokonaislukuihin'' eli \texttt{int~*muuttuja[]}.

B-yhteensopivuuden tavoittelu ohjasi kielen suunnittelua syntaksin osalta
mahdollisimman B:n kaltaiseksi. B-lause \texttt{if(a~\&~b)} vastaa C-lausetta
\texttt{if(a~\&\&~b)}, mutta B-kielessä \texttt{\&}:tä käytettiin myös
bittioperaatioihin loogisten operaatioiden lisäksi. Koska B-ohjelmien haluttiin
toimivan C-ohjelmien tavoin mahdollisimman pienillä muutoksilla, C-idiomissa
\texttt{(a\&mask)~==~b} lausekkeessa käytetyn \texttt{\&}-bittioperaattorin
ympärille joutuu lisäämään sulut. Tämä johtuu B-kielen \texttt{if(a == b \& c)}
-tyylisistä lausekkeista, joiden haluttiin toimivan ilman muutoksia C:ssä
kielen vaihtamisen helpottamiseksi.

Seuraavaksi alkoi C-kielen esikäsittelijän kehitys. Aluksi esikäsittelijässä
oli vain toiminnot tiedostojen sisällyttämiseen (\texttt{\#include}) ja
yksinkertaiseen korvaamiseen (\texttt{\#define}), mutta hyvin nopeasti
kieleen lisättiin funktiomakrot sekä \texttt{\#if}-lauseet. Aluksi
esikäsittelijää pidettiin vain vapaaehtoisena laajennoksena C:hen, mikä
selittää myös nykypäivänä esikäsittelijän huomattavat erot muuhun C-kieleen
verrattuna. Ensimmäisen C-standardin jälkeen esikäsittelijä on pysynyt lähes
koskemattomana. Ainoa lisätty ominaisuus oli C99-standardin mukana tulleet
funktiomakrot, jotka voivat ottaa mielivaltaisen määrän argumentteja.

\grayrule

Myöhemmin, kun C oli levinnyt usealle eri alustalle, alkoi olla selkeää, että C
tarvitsi standardin. Brian Kernighan, jonka kanssa Ritchie oli kirjoittanut
\emph{The C Programming Language} -kirjan~\citep{krfirst}, kirjoitti Ritchien
kanssa C:n ensimmäisen standardin ANSIn X3J11 -työryhmässä. Kuuden vuoden
jälkeen työryhmä sai valmiiksi nk. C89 -standardin, joka tunnetaan myös ANSI
C:nä\footnote{ISO-järjestö hyväksyi standardin pienillä muutoksilla vuonna
1990, jonka vuoksi standardi tunnetaan myös C90-standardina.}~\citep{C89}.
Samoihin aikoihin valmistui myös toinen painos \emph{The C Programming
Language} -kirjasta, jossa korjattiin lukuisia eroja ensimmäisen version ja
C-standardin välillä~\citep{krsecond}.

\grayrule

Nykypäivänä C:tä käytetään käytännössä jokaisessa tietokoneessa
käyttöjärjestelmästä riippuen joko pelkästään ydinkomponenttien toteutukseen
tai koko käyttöjärjestelmän toteutukseen. Sulautetuissa järjestelmissä C on
yksi suosituimmista kielistä johtuen kielen yksinkertaisuudesta ja
suorituskyvystä. C:n suosion myötä myös kielen huonot puolet nousevat esiin
erilaisissa tietoturvaongelmissa, jotka johtuvat kielen sallimasta
rajoittamattomasta muistinkäsittelystä yhdistettynä ohjelmoijan tekemiin
virheisiin. Esimerkiksi puskuriylivuodoissa C:llä kirjoitettu ohjelma tallentaa
tietoa muualle tai lukee muistia muualta kuin mitä ohjelmoija on tarkoittanut,
kun ohjelmoinja jättää tekemättä kriittisen syötteen oikeellisuustarkistuksen.
C-kääntäjä voi optimoidessa poistaa tällaisia tarkistuksia ja aiheuttaa
tietoturvaongelmia, jos kääntäjä päättelee tarkistusten olevan
turhia~\citep{redhatsecurity}.

C on hyvin yksinkertainen kieli, ja se on selviytynyt nykypäivään asti lähes
identtisenä ANSI C:hen. Uudemmat standardit~\citep{C99, C11, C18} ovat lähinnä
tehneet pieniä parannuksia kielen tehokkuuteen esimerkiksi lisäämällä
\texttt{restrict}-avainsanan. Erilaiset kääntäjät ovat kuitenkin tuottaneet
omia laajennoksiaan kieleen mahdollistaen tehokkaampien mutta
kääntäjäriippuvaisten C-ohjelmien kirjoituksen. Moderni esimerkki
kääntäjäriippuvaisesta syntaksia muokkaavasta laajennoksista on vektorityypit,
joita esimerkiksi GCC-kääntäjä tukee omalla \texttt{\_\_attribute\_\_(())}
-syntaksillaan. GCC-kääntäjä tukee myös lukuisia muita laajennoksia, kuten
tietueliteraaleja. C:stä löytyy myös standardien mukainen tapa käyttää
kääntäjäriippuvaisia ominaisuuksia, \texttt{\#pragma}. Pragmoja käytetään
erityisesti OpenMP-kirjaston~\citep{openmp} yhteydessä.

Nykypäivänä käytännössä jokainen alusta tukee C:tä. C:tä käytetään alustoilla
muun muassa ohjelmointikielten väliseen kommunikaatioon -- jos C\#-ohjelma
haluaa käyttää Java-ohjelman kirjastorutiineja, C\#-ohjelman on helpointa
käyttää Java-ohjelman C-rajapintaa.

C on nykyään käytössä erityisesti matalan tason ohjelmoinnissa, kuten
käyttöjärjestelmien ytimissä, sulautetuissa järjestelmissä, UNIX-työkaluissa,
vapaan lähdekoodin ohjelmistoissa, tietokannoissa ja muissa tehokkuutta
vaativissa ohjelmistoissa.

\subsection{Tärkeimmät C-ohjelmointikielen ominaisuudet}
\label{sec:cominaisuudet}

Artikkelissa \emph{The Next 7000 Programming Languages}~\citep{next7000}
käsitellään ohjelmointikielten kehitystä ja pohditaan mahdollisia ominaisuuksia
tulevissa ohjelmointikielissä, joita nykyiset ohjelmointikielet eivät sisällä.
Artikkelissa selitetään C:n nykyistä suosiota kielen yksinkertaisuudella ja
tehokkuudella. Nämä ominaisuudet ovat mahdollistaneet C:n laajan käytön
käyttöjärjestelmistä työkaluihin huolimatta C:n turvattomuudesta.

Artikkelin mukaan C:n (ja C++:n) korvaaminen lyhyellä tähtäimellä on mahdotonta
johtuen kielen suosiosta. Koska lukuisat työkalut kääntäjistä
virheenjäljittäjiin\defword{debugger} on kirjoitettu yksinomaan C:tä varten,
vastaavien työkalujen luominen muita ohjelmointikieliä varten on huomattava
investointi. C:n yleisyydestä myös seuraa suuri määrä ohjelmoijia, kirjastoja
ja työkaluja, mikä tekee C:stä luonnollisen valinnan myös uusiin projekteihin.
Artikkelissa kuitenkin todetaan, että useat ohjelmointikielet ovat vhentäneet
C:n suosiota tarjoten yksittäisillä osa-alueilla parannuksia. Yksi mainituista
ohjelmointikielistä on Rust, joka mahdollistaa paremman ohjelmien
käännösaikaisen oikeellisuustarkistamisen heikentämättä tehokkuutta C:hen
verrattuna. Toiseksi vaihtoehdoksi tarjotaan ajoaikaisia tarkistuksia
turvallisuuden parantamiseksi.

Artikkelissa puhutaan myös mahdollisuudesta oikeelisuuden tasaiseen
parantamiseen\defword{gradual verification}, joka mahdollistaisi ohjelmien
ensimmäisten versioiden toteuttamisen ilman kattavaa käännösaikaista
oikeellisuustarkistusta, mutta antaen kehityksen jatkuessa työkalut ohjelmiston
oikeellisuuden varmistamiseen. Esimerkiksi
TypeScript~\citep{typescript}-ohjelmointikieli mahdollistaa tyypittämättömien
JavaScript-ohjelmien vaillinaisen tyypityken\defword{gradual typing}, jolloin
ohjelmien oikeellisuutta voi käännösaikaisesti tarkistaa funktio kerrallaan. 

Artikkelissa \emph{Some Were Meant For C} \citet{somemeantforc} myös nostaa
esiin tarpeen yksittäisten funktioiden kerrallaan muuntamisesta. Useissa
kielissä ei ole saumatonta C-yhteensopivuutta, jolloin kielestä toiseen
siirtyminen vaatii joko koko ohjelman uudelleenkirjoituksen tai erillisen
yhteensopivuuskerroksen alkuperäisen ja uuden kielen väliin. Koska lukuisat
työkalut ja kirjastot on toteutettu C:tä varten, C tulee pysymään jatkossakin
tärkeänä osana tietokoneita.

Yhteensopivuuden sijaan \citeauthor{somemeantforc} kuitenkin painottaa C:n
alusta-agnostisuutta kielen tärkeimpänä ominaisuutena. Useilla alustoilla
voi normaalin välimuistin lisäksi käyttää laitteistoa tai tiedostojärjestelmää
suoraan muistiosoitteina, minkä C:n yksinkertainen lähestymistapa
muistinhallintaan mahdollistaa. Artikkelissa esitetään tästä esimerkkinä
iteraation ohjelman omien konekielisen käskyjen yli, mikä muissa kielissä olisi
kohtuuttoman hankalaa, mutta C:ssä triviaalia.

Molemmissa artikkeleissa käsitellään C:n määrittelemätöntä toimintaa kielen
sekä hyvänä että huonona puolena. Lukuisat tietoturvaongelmat ovat johtuneet
C:n turvattomuudesta, mutta toisaalta kielen turvattomuutta voi käyttää hyväksi
mahdollisimman tehokkaiden ohjelmien toteutuksessa.

\subsection{Kehitettävissä olevat ominaisuudet C-ohjelmointikielessä}
\label{sec:ckehitettavat}

\hl{Tässä aliluvussa pohditaan, miksi C pitäisi korvata uudella
ohjelmointikielellä, mitä ominaisuuksia C:stä voisi kehittää, mitkä C:n
ominaisuudet voisi jättää pois ja mitä uusia ominaisuuksia voisi tulla.}

\hl{Näitä ominaisuuksia ovat ainakin tarkempi käännösaikainen tyypitys etenkin
tyhjien osoittimien osalta, makrojärjestelmän uusiminen sekä syntaksin
selkeyttäminen.}

\citeauthor{chistory} (\citeyear{chistory}) nostaa esiin kaksi usein
keskustelua herättänyttä C:n ominaisuutta. Toinen näistä on C:n tyyppisyntaksi,
ja toinen on C:n tapa käsitellä taulukoita ja osoittimia keskenään.
Ensimmäiselle näistä voi tehdä verrattaen helposti jotain, sillä syntaksin
muuntaminen käännösaikaisesti on triviaalia. Taulukoiden ja osoittimien välistä
käytöstä on paljon hankalampi muuntaa, sillä olemassa olevat C-ohjelmat
käyttävät osoittimia ja taulukoita hyvin sekalaisesti. Yksi C:stä puuttuva
ominaisuus on taulukon antaminen funktion parametrina, jonka voi kuitenkin
tehdä käärimällä taulukko tietueen sisään.

Muita syntaktisia parannuksia lausekkeisiin voi tehdä tietueiden kohdalla. Jos
lausekkeessa käyttää tietuetta, niin tietueen \texttt{foo} jäsenen \texttt{bar}
saa lausekkeella \texttt{foo.bar}, mutta jos \texttt{foo} onkin osoitin,
lausekkeen tulee olla \texttt{foo->bar}. Jos \texttt{foo} olisi osoitin
osoittimeen, lauseke olisi \texttt{(*foo)->bar}. Yksinkertaisempi syntaksi
olisi käyttää jokaisessa tapauksessa lauseketta \texttt{foo.bar}, ja jättää
tarvittava muoto kääntäjän pääteltäväksi. Esimerkiksi Rust-ohjelmointikielen
syntaksi tietueiden käsittelyyn on tällainen.

C:n \texttt{static}-avainsana on jaettu käytön mukaisesti kahteen avainsanaan.
Funktiomäärittelyissä ja globaaleissa muuttujissa C:n
\texttt{static}-avainsana vastaa muiden kielten avainsanaa yksityiselle
funktiolle. Globaaleja \texttt{static}-määreellä määritettyjä funktioita ja
muuttujia ei voi käyttää muusta kuin samasta tiedostosta, jossa kyseinen
symboli on määritelty.

Toinen \texttt{static}-määreen käyttö on funktioiden sisällä muuttujien
määrittelyyn. Staattiset muuttujat alustetaan vain kerran, vaikka funktiota
kutsuttaisiin useita kertoja. Ohjelmassa~\ref{fig:cstatic} käytetään tällaista
muuttujaa. Ohjelma tulostaa ensin numeron 0, jonka jälkeen ohjelma tulostaa
numeron 1.

\begin{listing}[ht!]
    \inputminted{C}{koodi/static.c}
    \caption{Staattinen muuttuja C:ssä}
    \label{fig:cstatic}
\end{listing}

\grayrule

C:n tyypitys sallii monia ilmaisuja, jotka voivat johtaa salakavaliin ongelmiin
suoritusaikaisesti. Koska C sallii suurempien kokonaislukutyyppien asettamisen
pienempiin kokonaislukutyyppeihin, nämä arvot voivat suoritusaikaisesti muuttua
ilman käännösaikaisia varoituksia. Mainittavasti lauseet \texttt{int a = 256;
unsigned char b = a;} eivät aiheuta käännösaikaisesti edes varoituksia
GCC-kääntäjällä, vaikka ''kaikki'' kääntäjän varoitukset olisivat päällä
\texttt{-Wall} -komentolipun avulla. Kääntäjälle pitää antaa erillinen
\texttt{-Wconversion} -komentolippu, jotta kääntäjä edes varoittaisi
mahdollisesti yllättävästä käytöksestä. Tämän ominaisuuden muuttaminen
virheeksi tai edes varoitukseksi ei ole ongelmatonta, sillä esimerkiksi
C-makrojen tuottama koodi voi olettaa tällaisten lausekkeiden toimivan.

C:hen voisi myös lisätä useita uusia tyyppejä, erityisesti tyypin ei-tyhjälle
osoittimelle. Tätä tyyppiä voisi käyttää turvallisempien ja nopeampien
ohjelmien kirjoittamiseen. GCC- ja Clang-kääntäjät tukevat ei-tyhjiä osoittimia
\texttt{\_\_attribute\_\_((nonnull))} -määreellä. Kääntäjä voi käyttää määrettä
optimoidessa funktioita -- erityisesti turhat tarkistukset tyhjien muuttujien
varalta optimoidaan pois.

Muita hyödyllisiä tyyppejä ohjelmoinnin helpottamiseen olisivat
monikot\defword{tuple} ja summatyypit\defword{tagged union, sum type}. Nämä
molemmat tyypit löytyvät esimerkiksi Rustista. Ensimmäisen saa käännettyä
triviaalisti tietueeksi, ja toisen saa käännettyä tietueeksi, jossa on sisällä
vaihtoehdoista luetelman\defword{enumeration} ja yhdisteen\defword{union}
yhdistelmä. Ohjelmassa~\ref{fig:csumtype} on Rustin ja C:n syntaksien mukaiset
summatyypit tyypille, jossa on joko operaation onnistuessa jokin kokonaisluku
tai epäonnistumisen yhteydessä epäonnistumisen syy merkkijonona. Erillisen
summatyypin määritteleminen mahdollistaa yhtä aikaa sekä paremman
oikeellisuustarkistuksen että tehokkaampien ohjelmien tuottamisen.
C-esimerkissä \texttt{failure}-kentässä voisi olla arvo, vaikka
\texttt{type}-kentän arvo olisi \texttt{\_SUCCESS}. Tämä voi johtaa
määrittelemättömään toimintaan, jos \texttt{success}-kentän arvoa yritetään
käyttää ja sillä hetkellä alustettu kenttä on \texttt{failure}.

\begin{listing}[ht!]
    \inputminted{Rust}{koodi/rustenum.rs}
    \inputminted[firstline=3]{C}{koodi/csumtype.c}
    \caption{Summatyyppi Rustissa ja C:ssä.}
    \label{fig:csumtype}
\end{listing}

%C:n kaltaisiin kieliin saa toteutettua helposti tyyppipäättelyn, sillä C:n
%tyypit ovat tyyppitasolla yksinkertaisia. Koska C:ssä tyypit eivät ole
%geneerisiä, lausekkeiden tyyppipäättely on lähes triviaalia paria operaatiota
%lukuun ottamatta. Tyyppipäättely helpottaa ohjelmointia, sillä ohjelmoija
%voi keskittyä ohjelmien logiikkaan muuttujien tyyppien kirjoittamisen sijaan.
%\hl{Viittaus C:n dokseihin?}

C:ssä ei ole erillistä moduulijärjestelmää, jonka lisääminen voisi nopeuttaa
kääntämistä ja tehdä ohjelmien ymmärtämisestä yksinkertaisempaa. C-ohjelmat
voivat käyttää kirjastojen funktioita kirjoittamalla kirjastofunktioista
prototyypit, jotka yleisesti ottaen ovat kirjastojen otsikkotiedostoissa. Koska
otsikkotiedostojen sisältö käytännössä ottaen kopioidaan
\texttt{\#include}-kutsun tilalle, kääntäjä joutuu käsittelemään samoja
otsikkotiedostoja lukuisia kertoja käännösprosessin aikana\footnote{Käytännössä
kääntäjät pystyvät optimoimaan tietyllä yleisellä tavalla kirjoitettuja
otsikkotiedostoja ja jättämään jo kertaalleen luetut tiedostot kokonaan pois.}.
Erillisen moduulijärjestelmän lisääminen voisi kääntämisen nopeuttamisen
lisäksi yksinkertaistaa kirjastojen toteuttamista, sillä ohjelmoijien ei
tarvitsisi kirjoittaa kirjastoilleen otsikkotiedostoja. 

C:n makrojärjestelmässä on paljon parantamisen varaa. C:n esikäsittelijä toimii
hyvin yksinkertaisissa tapauksissa, mutta sen rajoitteet tulevat nopeasti
esille monimutkaisempia makroja kirjoittaessa. Voimakkaampi makrojärjestelmä
mahdollistaa lyhyempien ja selkeämpien makrojen kirjoittamisen
monimutkaisemmissa tapauksissa.
%Uutta makrojärjestelmää kirjoittaessa tulee
%kuitenkin pitää mielessä C-yhteensopivuus. \citeauthor{cabuse}
%(\citeyear{cabuse}) esittelee artikkelissaan tapoja hyväksikäyttää C:n
%esikäsittelijää esimerkiksi iteraation toteuttamiseen. Monimutkaisempi
%makrojärjestelmä voisi muuntaa voimakkaampaa kieltä C:n esikäsittelijän
%makroiksi käyttäen artikkelin mukaisia C-makroja, jolloin C-ohjelmat voisivat
%käyttää näitä yksinkertaisemmalla kielellä kirjoitettuja makroja. C:n standardi
%on kuitenkin hyvin avoin makrojen käytöksestä rajatapauksissa, joten
%monimutkaiset makrot eivät välttämättä toimi kääntäjäriippumattomasti, vaikka
%ne olisivatkin standardin mukaisia.


C:n makrojärjestelmä myös hankaloittaa työkalujen kirjoittamista. Koska
jokainen pieni muutos lähdekoodiin voi vaikuttaa radikaalisti saatavilla
oleviin symboleihin, esimerkiksi automaattitäydennyksen tarjoavien työkalujen
kirjoittaminen on hankalaa. Standardoitu moduulijärjelmä voisi myös
yksinkertaistaa kääntämistyökalujen kirjoittamista.\hl{relevanssi?}

\section{Uuden ohjelmointikielen vertaaminen muihin ohjelmointikieliin}

\subsection{Vertailun tulokset}

Purkan referenssitoteutus \citep{purkka}

Purkkaa verrataan samalla tavalla kuin muita ohjelmointikieliä luvussa xx.
Kaavioissa on esitetty kaikki performanssitestit erikseen. Raakadata liitteeseen.

\begin{figure}[ht!]
    \begin{adjustbox}{center}
        \begin{minipage}{1.2\textwidth}
        \input{data/benchmarkscpu2.tex}
        \end{minipage}
    \end{adjustbox}

    \begin{adjustbox}{center}
        \begin{minipage}{1.2\textwidth}
        \input{data/benchmarksmem2.tex}
        \end{minipage}
    \end{adjustbox}

    \begin{adjustbox}{center}
        \begin{minipage}{1.2\textwidth}
        \input{data/benchmarkslines2.tex}
        \end{minipage}
    \end{adjustbox}
    \caption{
        Benchmarks Gamen ohjelmiin perustuvat kuvaajat Purkalla kirjoitettujen ohjelmien
        suorituskyvystä, muistinkäytöstä ja ohjelmien koosta verrattuna C:llä
        kirjoitettujen ohjelmien tuloksiin.}
    \label{fig:benchmarksgame}
\end{figure}

\hl{Tässä luvussa uutta ohjelmointikieltä vertaillaan muihin tutkielmassa
käsiteltyihin ohjelmointikieliin, ja kerrotaan, miksi tai miksei se päihitä
C:tä vertailuehtojen mukaisesti.}

Kuvass zx nähdään, että Purkka on yhtä tehokas kieli kuin C, eikä se vie
enempää muistia. Purkka vie keskimäärin xx prosenttia vähemmän lähdekoodia.

\subsection{Johtopäätökset ja vertailun arviointi}

Tutkimuksessa verrattin useita ohjelmointikiiä C:hen, mutta yksikään ei
pärjää kriteereistä johtuen. Purkka pärjää.

Performanssitestaus ei vastaa todellista maailmaa, mutta koska Purkka käännetään
vastaavaksi C:ksi, sen tulisi aina päästä identtiseen suorituskykyyn.

Muiden kielten osalta: koska mittuksrt on mittauksia, ovat epätarkkoja.

Purkkaa tuskin otetaan käyttöön, vaikka se tämän paperin mukaan olisi parempi, sillä
kukaan ei tunne kieltä, ainoan kääntäjän toteuttaja ei ole suuri firma.

Jatkotutkimus... tutkimus osoittaa, että C:tä on mahdollista parantaa syntaktisesti,
jos joku firma hsluaisi niin parannettavaa löytyy. Tarkempi tyypitys C:hen voi
paljastaa mahdollisia ohjelmointi irheitä ja mahdollistaa turvallisemman ja tehokkaamman
optimoinnin.

\hl{Jos tutkielmassa on mahdollisia puutteita, tai sen tulokset eivät
välttämättä tutkielman rajoituksista johtuen päde tutkielman ulkopuolella,
tässä aliluvussa perustellaan syitä tälle.}

\hl{Erityisesti tutkimusaiheen rajaus ei välttämättä yleisty tutkimuksen
ulkopuolelle, sillä ohjelmointikielten valintaan liittyy usein monimutkaisia
syitä, kuten aikaisempi kokemus tai ohjelmoijan henkilökohtainen mieltymys
johonkin kieleen, eikä projekteissa käytettyjä ohjelmointikieliä valita
pelkästään kielen ominaisuuksien perusteella.}

\hl{Jos tutkielmassa ilmenee mahdollisia jatkotutkimuskohteita, niistä
kerrotaan tässä aliluvussa.}

\section{Uuden ohjelmointikielen vertaaminen C:hen}

Tässä luvussa verrataan uuden ohjelmointikielen suorituskykymittauksia C:hen
sekä muihin tutkielmassa käsiteltyihin kieliin. Suorituskykymittaukset
näyttävät, että Purkan toteutus ei hidasta C:tä. Tämä taas johtuu siitä, että
Purkassa ei ole yhtään ominaisuutta, joita C:ssä ei ole. Tässä luvussa myös
tutkielman oikeellisuutta ja pohditaan jatkotutkimuskohteita.

\subsection{Vertailun tulokset}
\label{sec:results}

Tätä tutkielmaa varten on toteutettu kääntäjä, joka kääntää yksinkertaiset
ohjelmat Purkka-kielestä C-kieleen \citep{purkka}. Kääntäjään ei ole toteutettu
kaikkia luvussa~\ref{sec:purkka} esitellyistä ominaisuuksia, mutta kääntäjä
tukee esimerkiksi C:n esikäsittelijää.

Nopeimmat Benchmarks Gamen C-toteutukset on käännetty Purkka-kielelle. Nämä
ohjelmat on sitten käännetty Purkka-kääntäjällä takaisin C:ksi ja käännetty
sitten GCC-kääntäjällä vastaavilla asetuksilla kuin verrattavat C-ohjelmat.
Kuvassa~\ref{fig:purkkabenchmarksgame} verrataan Purkka-toteutuksia muihin
verrattaviin kieliin ja kuvassa~\ref{fig:purkkabenchmarksgame2} verrataan
Purkka-toteutuksia nopeimpaan C-ohjelmaan. Koska Purkka-toteutukset on
muunnettu nopeimmasta C-toteutuksesta ja käännetty takaisin lähes identtiseen
muotoon, suoritusajan ja muistinkäytön tulisi olla samat verrattuna
C-toteutukseen.

Suorituskykymittaukset mittaavat ainoastaan prosessoriin ja muistinkäyttöön
liittyviä mittauksia, eivätkä esimerkiksi näytönohjaimella suoritettua
laskentaa. Mittaukset tehdään vain yksittäisillä kääntäjäversioilla, joten
mittaukset eivät välttämättä päde muilla saman ohjelmointikielen kääntäjillä.

Suorituskykymittaukset on suoritettu Ubuntu 19.10 -käyttöjärjestelmällä Linux
5.3.0 -kernelillä. Mittaustietokoneessa on neliytiminen Intel i7-8550U
-prosessori Hy\-per\-Thread\-ing-tuella (4.0 GHz) ja 32 gigatavua DDR3-muistia.
C- ja C++-ohjelmat on käännetty GCC-kääntäjän versiolla 9.2.1, Ada-ohjelmat
GNAT-kääntäjän versiolla 8.3.0, Go-ohjelmat Go-kääntäjän versiolla 1.12.10 ja
Rust-ohjelmat Rust-kääntäjän versiolla 1.41.0.

Mittaukset on toteutettu suorittamalla ohjelma suurimmalla syötteellä viisi
kertaa ja mittaamalla kunkin suorituksen viemä aika ja muistinkäyttö.
Muistinkäyttö on mitattu 200 millisekunnin välein \texttt{libgtop2}-kirjastolla
ja kunkin suorituksen muistinkäytöksi on merkitty suurin mitattu muistikäyttö.

Kuvaajista nähdään, että Benchmarks Gamen tarjoamissa esimerkeissä Purkka on
täsmälleen yhtä tehokas kieli kuin C, eikä se vie enempää muistia. Pienet
eroavaisuudet suuntaan ja toiseen on selitettävissä testauksen aiheuttamalla
luonnollisella vaihtelulla. Purkalla toteutettu \texttt{fannkuchredux}-ohjelma
on kuitenkin noin 50\% nopeampi kuin C:llä toteutettu lähes identtinen versio
ja \texttt{nbody}-ohjelma vie vain 80\% muistia verrattuna C-toteutukseen.

Purkka-tiedostot ovat keskimäärin 6\% pienempiä kuin vastaavat C-tiedostot.
Suurimmat kokoerot tiedostoissa tulevat muuttujien määrittelyistä sekä
tyyppipäättelystä. Kuvassa~\ref{fig:declarations} on yksittäisiä
tyyppimäärittelyjä, jotka ovat Purkassa hieman yksinkertaisempia kuin vastaavat
C-määrittelyt.

\begin{figure}[ht!]
    \begin{adjustbox}{center}
        \begin{tabular}{@{} m{0.55\textwidth} m{0.55\textwidth} @{}} \toprule
            Purkka-määrittely ja C-määrittely & Selite \\ \midrule

            \texttt{let a: u32; \newline unsigned int a;} & Epänegatiivinen 32-bittinen kokonaisluku \\
            \noalign{\vspace{0.2cm}}

            \texttt{let a, b: [\&i8;5]; \newline signed char *a[5], *b[5];} & Taulukko viidestä osoitinmuuttujasta \newline 8-bittiseen kokonaislukuun \\
            \noalign{\vspace{0.2cm}}

            \texttt{const a = ["1", "2", "3"]; \newline const char * const a[] = \{"1", "2", "3"\};} & Kolmen merkkijonon taulukon vakiomuuttuja \\
            \noalign{\vspace{0.2cm}}

            \texttt{let a = fun (a, b) => a + b; \newline int a(int a, int b) \{ return a + b; \}} & Funktio, joka laskee argumenttiensa summan \\ \bottomrule
        \end{tabular}
    \end{adjustbox}
    \label{fig:declarations}
    \caption{Purkan muuttujien määrittelyt verrattuna C:n vastaaviin
    määrittelyihin.}
\end{figure}

\begin{figure}[ht!]
    \begin{adjustbox}{center}
    \begin{minipage}{1.15\textwidth}
    \begin{minipage}{0.5\textwidth}
        \input{data/benchmarkscpu3.tex}
        \vspace*{-0.8cm}
    \end{minipage}
    \begin{minipage}{0.5\textwidth}
        \input{data/benchmarksmem3.tex}
        \vspace*{-0.9cm}
    \end{minipage}
    \end{minipage}
    \end{adjustbox}

    \caption{
        Benchmarks Gamen ohjelmiin perustuvat kuvaajat Purkalla kirjoitettujen
        ohjelmien suorituskyvystä, muistinkäytöstä ja ohjelmien koosta
        verrattuna muilla kielillä kirjoitettujen ohjelmien tuloksiin.}
    \label{fig:purkkabenchmarksgame}
\end{figure}

\begin{figure}[ht!]
    \begin{adjustbox}{center}
        \begin{minipage}{1.25\textwidth}
        \input{data/benchmarkscpu2.tex}
        \end{minipage}
    \end{adjustbox}

    \begin{adjustbox}{center}
        \begin{minipage}{1.25\textwidth}
        \input{data/benchmarksmem2.tex}
        \end{minipage}
    \end{adjustbox}

    \begin{adjustbox}{center}
        \begin{minipage}{1.25\textwidth}
        \input{data/benchmarkslines2.tex}
        \end{minipage}
    \end{adjustbox}
    \caption{
        Benchmarks Gamen ohjelmiin perustuvat kuvaajat Purkalla kirjoitettujen ohjelmien
        suorituskyvystä, muistinkäytöstä ja ohjelmien koosta verrattuna C:llä
        kirjoitettujen ohjelmien tuloksiin.}
    \label{fig:purkkabenchmarksgame2}
\end{figure}

\FloatBarrier

\subsection{Johtopäätökset ja vertailun arviointi}

Purkka on vertailun ainoa kieli, joka on yhtä tehokas kuin C. Tämä johtunee
pitkälti siitä, että Purkka käännetään lähes identtisenä C:ksi, eikä tarjoa
suoritusaikaisia ominaisuuksia, jotka hidastaisivat kieltä. Purkka ei kuitenkaan
päihitä C:tä tutkimuksen määrittelyn mukaisesti, vaan on suorituskyvyltään
täsmälleen yhtä tehokas kuin C.

Kaikkien vertailtavien kielten suunnittelutavoitteissa on mainittu C:n lisäksi
myös C++:n käytön korvaaminen. Tämä on todennäköisesti vaikuttanut kielen
suunnitteluun monimutkaistaen kieliä, jotta kieli pystyisi korvaamaan C++:n
monimutkaiset malliohjelmointirakenteet. Muista kielistä poiketen Purkan
tavoitteena on korvata ainoastaan C, mikä on ohjannut suunnittelua tutkielman
määrittelyjen mukaisesti.

Suorituskykymittaukset eivät kuitenkaan vastaa todellista maailmaa, sillä ne
usein mittaavat vain yksittäistä pientä osa-aluetta, kuten yksittäisten
operaatioiden nopeutta monimutkaisten ohjelmistojen sijaan. Lisäksi
suorituskykymittaukset ovat usein epätarkkoja, sillä monimutkaiset
moniajoympäristöt eivät mahdollista deterministisiä mittauksia.

Mittaukset koskevat vain yksittäisiä ohjelmia ajettuna tietyssä ympäristössä
tietyllä kääntäjällä. Ne eivät siis anna kattavaa kuvaa ohjelmointikielistä,
vaan mittaustuloksia yksittäisen kääntäjätoteutuksen kääntämistä ohjelmista.
Kääntäjät voivat käyttää jopa yksittäisten versioiden välillä erilaisia
optimointeja, jotka voivat nopeuttaa tai hidastaa ohjelmaa juuri mitatulla
syötteellä, mikä pienentää tulosten vertailukelpoisuutta. Kääntäjät voivat
myös toimia paremmin tai huonommin erilaisilla tietokonearkkitehtuureilla,
jolloin esimerkiksi prosessorin valinta vaikuttaa mittaustuloksiin.

Purkkaa ei todennäköisesti oteta laajaan käyttöön, sillä se ei tarjoa
merkittäviä parannuksia C:hen, vaan mahdollistaa lähinnä hieman lyhyemmän
lähdekoodin. Ohjelmointikielten valintaan liittyy usein monimutkaisia syitä,
kuten aikaisempi kokemus tai ohjelmoijan henkilökohtainen mieltymys johonkin
kieleen, eikä projekteissa käytettyjä ohjelmointikieliä valita pelkästään
kielen ominaisuuksien perusteella. Tämän lisäksi Purkan toteuttaja on
yksittäinen opiskelija, kun taas esimerkiksi Go-kielen ja Rustin toteuttajat
ovat kokeneita ohjelmointikielten suunnittelijoita, jonka lisäksi kieliä
ylläpidetään aktiivisesti tunnettujen organisaatioiden toimesta. Tämä on
todennäköisesti myös syy sille, miksi esimerkiksi LISP/c~\citep{clisp1},
C-Mera~\citep{clisp2}, Carp~\citep{clisp3} ja Nymph~\citep{nymph} ovat jääneet
ilman laajempaa huomiota.

Tutkimuksesta voidaan kuitenkin päätellä, että C:tä parempi kieli on
todennäköisesti toteutettavissa. Käännösaikaista turvallisuutta voi parantaa
tiukemmalla tyyppijärjestelmällä ilman suoritusaikaisia haittoja. Tarkemmilla
tyyppimäärittelyillä voi myös tehdä optimointeja, esimerkiksi vaatimalla
osoitinargumentit aina ei-tyhjiksi osoittimiksi.

Osan Purkan ominaisuuksista voisi ottaa käyttöön jopa uusissa C:n versioissa.
Tyyppipäättelyn lisääminen esimerkiksi korvaamalla käyttämättömän
\texttt{auto}-avainsanan\footnote{Avainsanaa käytetään tarkkaan ottaen
\texttt{static}-avainsanan vastinparina, eli ei-staattisten muuttujien
luomiseen. Koska kaikki muuttujat ilman \texttt{static}-määrettä ovat
ei-staattisia, \texttt{auto}-avainsana on turha.} tarkoittamaan pääteltyä
tyyppiä voisi auttaa moderneja C-kääntäjiä käyttävien projektien
kirjoittamista. GCC-kääntäjä tukee tyyppipäättelyä
\texttt{\_\_auto\_type}-määreellä. Summatyyppien lisääminen laajennoksena
\texttt{enum}- tai \texttt{union}-tyyppisyntaksiin ei vaikuta tämänhetkisiin
standardien mukaisiin ohjelmiin, mutta mahdollistaisi summatyyppien käytön.
Suurempia syntaktisia muutoksia, kuten tyyppisyntaksin uudelleenkirjoitusta
tuskin pystyy toteuttamaan säilyttäen yhä tuen nykyiselle C:lle.

Purkan jatkokehityksen voisi aloittaa omalla makrojärjestelmällä, sillä
Purkassa ei ole määriteltynä omaa makrojärjestelmää. Purkka tukee C:n
makrojärjestelmää, mutta C:n makrojärjestelmää kattavamman makrojärjestelmä
voisi helpottaa monimutkaisten käännösaikaisten laskujen laskelmista.

Purkkaa varten ei ole kääntäjän lisäksi kehitetty yhtään kehitystyökalua,
vaikka luvussa~\ref{sec:suosio} valmiit työkalut nostetaan tärkeäksi
ominaisuudeksi. Esimerkiksi hyvälaatuinen automaattinen käännösjärjestelmä
helpottaisi kielen käyttämissä uusissa ohjelmissa. Tämän voisi nimetä
ohjelmointikielen nimen mukaisella teemalla Jesariksi, sillä se sitoo
käännettävät ohjelman palaset toisiinsa yhdistettynä mahdollisiin
riippuvuuksiin.

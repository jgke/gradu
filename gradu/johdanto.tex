\section{Johdanto}

\hl{Johdantoluku, jossa avataan tutkielman aihe eli C-ohjelmointikielen
vertaaminen muihin ohjelmointikieliin sekä uuden ohjelmointikielen luominen
tiettyjen kriteerien ohjaamana.}

\hl{Kokonaisuutena esitys on luettelomainen ja jäsentymätön: selkeää yleistä
teemaa tai tavoitetta ei ole tai sitä on vaikea hahmottaa kaikesta
sillisalaatista. Tutkielma vaikuttaa myös melko suppealta (vain 30 tekstisivua)
- ottaen huomioon käsitellyt lukuisat eri kielet ja niihin liittyvät teemat.}

\hl{ (1) Aikoinaan Simula ja olio-ohjelmointi rakennettiin näkemykselle, että
'ohjelmointi on simulointia'. Simula lisäsi Algol-60 -kieleen luokat ja
oliomekanismit sekä välineet diskreettiaikasimulointiin.}

\hl{(2) Pascal tyyppijärjestelmä oli staattinen ja kömpelö ja alun perin myös
heikosti tyypitetty.  Ada tarjosi ilmaisuvoimaisen vahvasti tyypitetyn
tyyppijärjestelmän sekä välineet modulaariseen ohjelmointiin eli pakkaukset
(josta tulee alkuperäinen 'package'-termi). Lähtökohtana kielen kehittämiselle
olivat ohjelmistotuotannon tarpeet ja vaatimukset.}

\hl{(3) B. Stroustrup lisäsi C:hen Simulan luokat ja oliomekanismit tukemaan
versiotaan 'käsitteellisestä' ohjelmoinnista.  Kielen rakenteet tukivat samalla
sekä abstrakteja tietotyyppejä että olio-ohjelmointia. Yhteensopivuus C:hen
pyrittiin säilyttämään eikä lisäpiirteistä saanut aiheutua osakielelle C
yleisrasitetta.  Toisin sanoen jos uusia piirteitä ei käytetty, C++-kielen tuli
vastata C-kielen käyttöä.  Vahva tyypitys ja luotettavuus tms.
ohjelmistotuotantoaspektit eivät olleet erityisesti painotettuina tavoitteina.}

\hl{ (4 ) Rust näyttää (minusta) ainakin yrittävän korjata joitakin C++:n
heikkouksia oikeellisuuden ja luotettavuuden suhteen.}

\hl{ Tämänkaltaista näkemyksellistä tavoitteita jää tutkielman sisällöstä
kaipaamaan.   Seuraavassa vielä joitakin yleisluontoisia kommentteja.

Erityisesti jäin kaipaamaan tekstissä viittauksia ohjelmistotuotannon
asettamiin vaatimuksiin (oikeellisuus, luotettavuus, ylläpidettävyys)
ohjelmointikielille. Tai sitten näiden tekijöiden ohittamista tulee aukaista
sopivissa paikoissa (Johdanto ja Yhteenveto).

C:n merkitys tälle tutkielmalle on niin olennainen, että sisällöstä sen
käsittelylle olisi syytä omistaa yksi oma pääluku. Lukijalle jää nyt
epäselväksi, ovatko kaikki olennaiset C:n piirteet ja heikkoudet  tulleet
käsiteltyä. Tästä myös Antti-Pekka erikseen mainitsi.

Tutkielmassa ei toisaalta saa olla mitään tarpeetonta täytetekstiä.
Tällaisilta tuntuvat omaa kieltä ja sen toteutusta teknisesti dokumentoivat
lyhyet irralliset kohdat.  Vaikuttaa siltä ettei niitä käytetä
jatkokäsittelyssä mitenkään hyväksi, joten nämä kohdat voi hyvin siirtää
tutkielman liitteiksi.  Kielen määrittelyyn ja toteutukseen (annettuna
liitteissä) tulee kuitenkin viitata tekstissä, ja tietysti jos niistä on saanut
irti olennaista tärkeää mainittavaa itse analyysiin, tämä nostetaan osaksi
tekstiä. }

\hrule

\hl{ 1 Johdanto: tämä on edelleenkin sekavasti ja epämääräisesti kirjoitettu
tutkielman tavoitteiden ja tutkimusmenetelmien osalta. Tavoitteita ja
tutkimuskysymyksiä ei ole ylipäätään kunnolla eritelty (tästä olen jo riittävän
montaa kertaa huomauttanut). Esimerkiksi väität, että 'selvitetetään mitkä
ominaisuudet ovat voineet estää kielen käytön C:n sijaan...'  , mutta ei
tämmöistä analyysiä tekstistä löydy. Toisaalta taas näiden tulosten pohjalta
'suunniteltaan ohjelmointikieli' - siis minkä tulosten?? Minun mielestäni
vertailun tuloksia ei käytetä mihinkään. } 

C~\citep{C18} on ollut vallitseva ohjelmointikieli järjestelmäohjelmoinnissa
kielen alkuajoista lähtien. Useita ohjelmointikieliä on luotu historian
saatossa, joiden oli tarkoitus syrjäyttää C, mutta C on vieläkin johtavana
kielenä varsinkin sulautetuissa järjestelmissä ja UNIX-pohjaisten
käyttöjärjestelmien vallitsevana ohjelmointikielenä. C on myös käytössä
Windows-käyttöjärjestelmäperheen ydinkomponenttien toteutuksessa.

Tutkielmassa selvitetään C:n ominaisuuksia, joiden takia se on ollut
suosituimpien ohjelmointikielten joukossa vuosikymmeniä, kuten myös
ominaisuuksia, joita C:stä voisi kehittää. Vaihtoehtoisista kielistä
selvitetään, mitkä ominaisuudet ovat voineet estää kielen käytön C:n sijaan
uusissa ja olemassa olevissa projekteissa ja mitkä ominaisuudet ovat taas
olleet parannuksia C:n ominaisuuksiin verrattuna. Tutkielmassa suunnitellaan
näiden tulosten pohjalta uusi ohjelmointikieli, Purkka.
%Vertailussa tutkitaan erityisesti mahdollisuuksia olemassa olevien C-kielellä
%kirjoitettujen ohjelmien jatkokehittämistä uudella kielellä ja uusien
%projektien toteuttamista C:n sijaan uudella kielellä.

C:n vaihtoehdoiksi tutkitaan seuraavia tehokkaaseen ohjelmointiin tarkoitettuja
kieliä: Ada~\citep{ADA12}, C++~\citep{CPP17}, D~\citep{D}, Go~\citep{golang}
sekä Rust~\citep{rust}. \hl{Miksi juuri nämä kielet? Satunnainen otanta?}
Näistä kielistä tutkitaan, mikä tai mitkä ominaisuudet ovat estäneet C:n
korvaamisen ja mitkä ominaisuudet ovat olleet parannuksia C:hen verrattuna.
Lisäksi tutkitaan muista suosituista ohjelmointikielistä korkean tason
ominaisuuksia, jotka ovat hyödyllisiä matalan tason ohjelmoinnissa ja jotka voi
toteuttaa korvaavan kielen määrittelyssä luotujen rajoitusten puitteissa.
Vertailun tuloksia käytetään uuden ohjelmointikielen suunnitteluun, jossa
otetaan tavoitteeksi luoda C:tä parempi ohjelmointikieli tutkielman
määrittelyjen puitteissa.

%Tutkittavana on myös, mitä optimointeja C:ssä ei voi tehdä helposti johtuen
%kielen rajoitteista ja miten tämän voisi korjata. Näitä ominaisuuksia ovat
%esimerkiksi sivuvaikutuksettoman ohjelmakoodin merkitseminen,
%optimointivinkkien kääntäjäriippumaton ilmaiseminen (esimerkiksi funktioiden
%koristelulla\defword{function annotation}) ja funktioiden
%yliajaminen\defword{function overloading} riippuen parametrien tyypistä tai
%arvoista, mikäli ne voidaan kääntöaikaisesti päätellä. Koska C:n spesifikaatio
%ei salli useaa samannimistä funktiota, funktioiden yliajaminen on haastavaa
%toteuttaa alustariippumattomasti.

%Tämän tutkielmasuunnitelman toisessa luvussa määritellään seuraavat
%ominaisuudet kielelle, joka voisi korvata C:n kokonaan:
%
%Vaihtoehtoinen kieli aina
%\begin{itemize}[topsep=0pt,itemsep=0pt]
%    \item on yhtä nopea tai nopeampi kuin C,
%    \item käyttää saman verran muistia tai vähemmän kuin C,
%    \item toimii kaikissa järjestelmissä, joissa C toimii,
%    \item toimii saumattomasti C:n kanssa ja
%    \item toimii saumattomasti muiden kielien C-rajapintojen kanssa.
%\end{itemize}
%Jotta jokin kieli olisi absoluuttisesti parempi, vähintään yhden näistä
%kriteereistä tulee olla aidosti parempi.

%Lisäksi kielistä tutkitaan muita suosioon vaikuttavia syitä, kuten
%helppokäyttöisyyttä ja sosiaalisia aspekteja, kuten Googlen tarjoaman tuen
%vaikutusta Go-kielen suosioon.%\hl{Vai tutkitaanko?}

Tutkielman toisessa luvussa esitetään tutkielman kannalta oleellinen
teoria, eli määritetään tutkielman vertailussa käytetyt kielten ominaisuuksien
vertailukohteet, pohditaan kielen suosioon vaikuttavia tekijöitä sekä kerrotaan
lyhyesti muista tutkittavista kielistä. Kolmannessa luvussa kerrotaan, miksi
muut tutkitut kielet eivät täytä näitä ominaisuuksia. Neljännessä luvussa
määritellään näiden tulosten pohjalta uusi ohjelmointikieli, Purkka, joka voisi
tutkielman määritelmien mukaisesti korvata C:n käytön. Viidennessä luvussa
verrataan tätä ohjelmointikieltä C:hen ja muihin tutkielmassa käsitellyihin
kieliin. Kuudennessa luvussa kerrataan tutkimuksen tulokset.

\hl{Ohjelmistotuotannon näkökulma: oikeellisuus, luotettavuus, ylläpito?}

\hl{C:stä hyvää tuntemista ei voi edellyttää lukijalta, mutta tutkielman
sisältö edellyttää sitä}

\section{Johdanto} 

C~\citep{C11} on ollut vallitseva ohjelmointikieli järjestelmäohjelmoinnissa C:n
alkuajoista lähtien. Useita ohjelmointikieliä on luotu historian saatossa,
joiden oli tarkoitus syrjäyttää C, mutta C on vieläkin johtavana kielenä
varsinkin sulautetuissa järjestelmissä ja UNIX-pohjaisten käyttöjärjestelmien
vallitsevana ohjelmointikielenä. C on myös käytössä
Windows-käyttöjärjestelmäperheen ydinkomponenttien toteutuksessa. Tässä
opinnäytetyössä tutkitaan, miksi vaihtoehdoista huolimatta C on vieläkin
laajalti käytössä myös uusissa projekteissa ja minkälainen ohjelmointikieli
voisi syrjäyttää C:n.

C:n vaihtoehdoiksi tutkitaan seuraavia tehokkaaseen ohjelmointiin tarkoitettuja kieliä: Ada~\citep{ADA12},
C++~\citep{CPP14}, D~\citep{D}, Go~\citep{golang} sekä Rust~\citep{rust}.
Näistä kielistä tutkitaan, mikä tai mitkä ominaisuudet ovat estäneet C:n
korvaamisen ja mitkä ominaisuudet ovat olleet parannuksia C:hen verrattuna.
Lisäksi tutkitaan muista suosituista ohjelmointikielistä ominaisuuksia, jotka
ovat hyödyllisiä matalan tason ohjelmoinnissa ja jotka voi toteuttaa korvaavan
kielen rajoitteissa.

Tutkittavana on myös, mitä optimointeja C:ssä ei voi tehdä helposti johtuen
kielen rajoitteista ja miten tämän voisi korjata. Näitä ominaisuuksia ovat
esimerkiksi sivuvaikutuksettoman ohjelmakoodin merkitseminen,
optimointivinkkien alustariippumaton ilmaiseminen (esimerkiksi Rustin tavalla
funktioiden annotointi) ja useat eri funktiot riippuen parametrien arvoista,
mikäli ne voidaan kääntöaikaisesti päätellä.


Toisessa luvussa esitetään tutkielman kannalta oleellinen teoria, eli
C-kielen historiaa ja nykypäivää, esitellään C:n hyviä ja huonoja puolia sekä
kerrotaan lyhyesti muista tutkittavista kielistä.
Toisessa luvussa myös määritellään, minkälaiset ominaisuudet tarvitaan
kieleltä, joka voisi korvata C:n kokonaan.
Kolmannessa luvussa kerrotaan, miksi muut tutkitut kielet eivät täytä näitä
ominaisuuksia.
Neljännessä luvussa kuvaillaan toisen luvun määrittämien  ominaisuuksien
täyttävä ohjelmointikieli.
Lopettavassa luvussa kerrataan tutkielman tavoitteet, vaiheet ja tulokset
sekä pohditaan tutkielman puutteita ja jatkotutkimuskohteita.

\section{Johdanto}

\hl{Lukujen alussa voisi olla metatekstiä lukujen sisällöstä.}

\grayrule

C~\citep{C18} on nykypäivänä yksi eniten käytetyimmistä ohjelmointikielistä. C
on ollut vallitseva ohjelmointikieli järjestelmäohjelmoinnissa kielen
alkuajoista lähtien. Useita ohjelmointikieliä on luotu historian saatossa,
joiden oli tarkoitus syrjäyttää C, mutta C on vieläkin johtavana kielenä
varsinkin sulautetuissa järjestelmissä ja UNIX-pohjaisten käyttöjärjestelmien
vallitsevana ohjelmointikielenä. C on myös käytössä
Windows-käyttöjärjestelmäperheen ydinkomponenttien toteutuksessa.

Tutkielmassa selvitetään C:n ominaisuuksia, joiden takia se on ollut
suosituimpien ohjelmointikielten joukossa vuosikymmeniä, kuten myös
ominaisuuksia, joita C:stä voisi kehittää. Vaihtoehtoisista kielistä
selvitetään, mitkä ominaisuudet ovat voineet estää kielen käytön C:n sijaan
uusissa ja olemassa olevissa projekteissa ja mitkä ominaisuudet ovat taas
olleet parannuksia C:n ominaisuuksiin verrattuna. Tutkielmassa suunnitellaan
näiden tulosten pohjalta uusi ohjelmointikieli, Purkka.

C:n vaihtoehdoiksi tutkitaan seuraavia tehokkaaseen ohjelmointiin tarkoitettuja
kieliä: Ada~\citep{ADA12}, C++~\citep{CPP17}, D~\citep{D}, Go~\citep{golang}
sekä Rust~\citep{rust}. Näistä kielistä tutkitaan, mikä tai mitkä ominaisuudet
ovat estäneet C:n korvaamisen ja mitkä ominaisuudet ovat olleet parannuksia
C:hen verrattuna. Vertailun tuloksia käytetään uuden ohjelmointikielen
suunnitteluun, jossa otetaan tavoitteeksi luoda C:tä parempi ohjelmointikieli
tutkielman määrittelyjen puitteissa.

Kaikkia verrattavia kieliä tutkitaan sekä analyyttisesti että
suorituskykymittausten muodossa. Suorituskykymittauksiin käytetään Benchmarks
Gamea~\citep{benchmarks}, johon on toteutettu lukuisia pieniä ohjelmia
suorituskyvyn mittaamiseen. D on ainoa tutkielman käsittelemä kieli, jota
Benchmarks Game ei sisällä. Benchmarks Gamen Purkka-ohjelmat on toteutettu
nopeimpien C-ohjelmien pohjalta, jotta ohjelmien arkkitehtuuriset valinnat
eivät vaikuta vertailuun. 

Tutkielman toisessa luvussa käsitellään C-ohjelmointikieltä. Kielestä
käsitellään perusteiden lisäksi kielen historiaa ja nykypäivää, tärkeimpiä
ominaisuuksia ja kehityskohteita. Historian käsitteleminen mahdollistaa
ymmärryksen siitä, mitkä C:n ominaisuudet ovat tärkeitä kielen nykykäytön
kannalta ja mitkä ovat jäänteitä historiallisista syistä. Tärkeimpien
ominaisuuksien ja kehitettävien ominaisuuksien tunnistamiseen käytetään
ohjelmointikielten tulevaisuutta käsittelevää artikkelia \emph{The Next 7000
Programming Languages}~\citep{next7000}, C-kielen suosiota käsittelevää
artikkelia \emph{Some Were Meant For C}~\citep{somemeantforc} sekä Dennis
Ritchien artikkelissa \emph{The Development of the C Language}~\citep{chistory}
esiin nostettuja C:n ominaisuuksia.

Kolmannessa luvussa määritetään toisen luvun esiin nostamien ominaisuuksien
pohjalta vertailukriteerit kielten vertaamiseen, käsitellään lyhyesti
tutkielmaan valittuja vertailtavia kieliä sekä käsitellään erilaisia yleisiä
ohjelmointikielten valintaan liittyviä tekijöitä. Edellämainitut kielet
valitaan, sillä ne ovat suosittuja~\citep{tiobe} sekä kunkin kielen historiassa
on ollut tavoitteena korvata C:n käyttö. Ohjelmistojen toteutuskielen
valintaprosessia käsitellään artikkelin \emph{Empirical Analysis of Programming
Language Adoption}~\citep{empiricalpopularity} avulla, jossa tutkitaan
kyselyillä erilaisia syitä ohjelmointikielen valintaan.

%\citetitle{empiricalpopularity}

esitetään tutkielman kannalta oleellinen
teoria, eli määritetään tutkielman vertailussa käytetyt kielten ominaisuuksien
vertailukohteet, pohditaan kielen suosioon vaikuttavia tekijöitä sekä kerrotaan
lyhyesti muista tutkittavista kielistä. Kolmannessa luvussa kerrotaan, miksi
muut tutkitut kielet eivät täytä näitä ominaisuuksia. Neljännessä luvussa
määritellään näiden tulosten pohjalta uusi ohjelmointikieli, Purkka, joka voisi
tutkielman määritelmien mukaisesti korvata C:n käytön. Viidennessä luvussa
verrataan tätä ohjelmointikieltä C:hen ja muihin tutkielmassa käsitellyihin
kieliin. Kuudennessa luvussa kerrataan tutkimuksen tulokset.

%\hl{Ohjelmistotuotannon näkökulma: oikeellisuus, luotettavuus, ylläpito?}

\section{Johdanto} 

Tämä on gradumääritelmän kopio.

C~\citep{C11} on ollut vallitseva ohjelmointikieli järjestelmäohjelmoinnissa C:n
alkuajoista lähtien. Useita ohjelmointikieliä on luotu historian saatossa,
joiden oli tarkoitus syrjäyttää C, mutta C on vieläkin johtavana kielenä
varsinkin sulautetuissa järjestelmissä ja UNIX-pohjaisten käyttöjärjestelmien
vallitsevana ohjelmointikielenä. C on myös käytössä
Windows-käyttöjärjestelmäperheen ydinkomponenttien toteutuksessa. Tarkoitus on
tutkia, miksi vaihtoehdoista huolimatta C on vieläkin laajalti käytössä myös
uusissa projekteissa ja minkälainen ohjelmointikieli voisi syrjäyttää C:n.

Määritetään C:tä paremmaksi kieleksi jokin kieli, mikä C:hen verrattuna:

\begin{itemize}
    \item on yhtä nopea tai nopeampi
    \item käyttää saman verran muistia tai vähemmän
    \item on helpompi käyttää
    \item toimii järjestelmissä, joissa C toimii
\end{itemize}

On huomioitava, että lukuisten olemassaolevien C:n kirjastojen, rajapintojen ja
projektien vuoksi yhteensopivuus C:n kanssa tulee olla saumatonta mahdollisten
vaihtoehtoisten ohjelmointikielten osalta, jotta kielen vaihtaminen olisi
mahdollista.

C:n vaihtoehdoiksi tutkitaan seuraavia kieliä: C++~\citep{CPP14},
Go~\citep{golang}, Ada~\citep{ADA12}, D~\citep{D} sekä Rust~\citep{rust}.
Näistä kielistä tutkitaan, mikä tai mitkä ominaisuudet ovat estäneet C:n
korvaamisen ja mitkä ominaisuudet ovat olleet parannuksia C:hen verrattuna.
Lisäksi tutkitaan muista suosituista ohjelmointikielistä ominaisuuksia, jotka
ovat hyödyllisiä matalan tason ohjelmoinnissa ja jotka voi toteuttaa korvaavan
kielen rajoitteissa. Suosituista ohjelmointikielistä tutkitaan seuraavia
kieliä: Python, Java, Haskell ja TypeScript. Erityisesti tutkitaan korvaavan
kielen mahdollisuuksia

\begin{itemize}
    \item tyyppiturvallisuuteen
    \item kääntäjän suorittamaan optimointiin
        (esim. vektorisointi)
    \item käännösaikaiseen koodin varmistamiseen
        (esim. alustamattomat muuttujat)
    \item funktionaaliseen ohjelmointiin
    \item sivukanavahyökkäysten torjumiseen (side-channel attack mitigation)
        (ajoitushyökkäykset, välimuistihyökkäykset)
    \item Alkuperäisestä kielestä uuteen vaihtamiseen (TypeScript)
\end{itemize}

Tutkittavana on myös, mitä optimointeja C:ssä ei voi tehdä helposti johtuen
kielen rajoitteista ja miten tämän voisi korjata. Näitä ominaisuuksia ovat
esimerkiksi sivuvaikutuksettoman ohjelmakoodin merkitseminen,
optimointivinkkien alustariippumaton ilmaiseminen (annotaatiot funktion
parametreihin) ja useat eri funktiot riippuen parametrien arvoista, mikäli ne
voidaan kääntöaikaisesti päätellä.

Tutkielman aloittavassa johdantokappaleessa perustellaan tutkielman tärkeys
sekä kuvaillaan lopun tutkielman sisältö ja tutkimuskysymys.
Toisessa kappaleessa esitetään tutkielman kannalta oleellinen teoria, eli
C-kielen historiaa ja nykypäivää, esitellään C:n hyviä ja huonoja puolia sekä
kerrotaan lyhyesti muista tutkittavista kielistä.
Kolmannessa kappaleessa tutkitaan, minkälaiset ominaisuudet tarvittaisiin
kieleltä, joka voisi korvata C:n kokonaan.
Neljännessä kappaleessa kerrotaan, miksi muut tutkitut kielet eivät täytä näitä
ominaisuuksia.
Viidennessä kappaleessa kuvaillaan kappaleen kolme käsittelemien ominaisuuksien
täyttävä ohjelmointikieli.
Lopettavassa kappaleessa kerrataan tutkielman tavoitteet, vaiheet ja tulokset
sekä pohditaan tutkielman puutteita ja mahdollisia jatkotutkimuskohteita.

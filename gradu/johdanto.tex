\section{Johdanto}

\hl{Johdantoluku, jossa avataan tutkielman aihe eli C-ohjelmointikielen
vertaaminen muihin ohjelmointikieliin sekä uuden ohjelmointikielen luominen
tiettyjen kriteerien ohjaamana.}

C~\citep{C18} on ollut vallitseva ohjelmointikieli järjestelmäohjelmoinnissa
\subsection{Jatkotutkimuskohteet}

C:n alkuajoista lähtien. Useita ohjelmointikieliä on luotu historian saatossa,
joiden oli tarkoitus syrjäyttää C, mutta C on vieläkin johtavana kielenä
varsinkin sulautetuissa järjestelmissä ja UNIX-pohjaisten käyttöjärjestelmien
vallitsevana ohjelmointikielenä. C on myös käytössä
Windows-käyttöjärjestelmäperheen ydinkomponenttien toteutuksessa.

Tutkielmassa selvitetään C:n ominaisuuksia, joiden takia se on ollut
suosituimpien ohjelmointikielten joukossa vuosikymmeniä, kuten myös
ominaisuuksia, joita C:stä voisi kehittää. Vaihtoehtoisista kielistä
selvitetään, mitkä ominaisuudet ovat voineet estää kielen käytön C:n sijaan
uusissa ja olemassa olevissa projekteissa ja mitkä ominaisuudet ovat taas
olleet parannuksia C:n ominaisuuksiin verrattuna. Tutkielmassa suunnitellaan
näiden tulosten pohjalta uusi ohjelmointikieli, Purkka.
%Vertailussa tutkitaan erityisesti mahdollisuuksia olemassa olevien C-kielellä
%kirjoitettujen ohjelmien jatkokehittämistä uudella kielellä ja uusien
%projektien toteuttamista C:n sijaan uudella kielellä.

C:n vaihtoehdoiksi tutkitaan seuraavia tehokkaaseen ohjelmointiin tarkoitettuja
kieliä: Ada~\citep{ADA12}, C++~\citep{CPP17}, D~\citep{D}, Go~\citep{golang}
sekä Rust~\citep{rust}. Näistä kielistä tutkitaan, mikä tai mitkä ominaisuudet
ovat estäneet C:n korvaamisen ja mitkä ominaisuudet ovat olleet parannuksia
C:hen verrattuna. Lisäksi tutkitaan muista suosituista ohjelmointikielistä
korkean tason ominaisuuksia, jotka ovat hyödyllisiä matalan tason
ohjelmoinnissa ja jotka voi toteuttaa korvaavan kielen määrittelyssä luotujen
rajoitusten puitteissa. Vertailun tuloksia käytetään uuden ohjelmointikielen
suunnitteluun, jossa otetaan tavoitteeksi luoda C:tä parempi ohjelmointikieli
tutkielman määrittelyjen puitteissa.

%Tutkittavana on myös, mitä optimointeja C:ssä ei voi tehdä helposti johtuen
%kielen rajoitteista ja miten tämän voisi korjata. Näitä ominaisuuksia ovat
%esimerkiksi sivuvaikutuksettoman ohjelmakoodin merkitseminen,
%optimointivinkkien kääntäjäriippumaton ilmaiseminen (esimerkiksi funktioiden
%koristelulla\defword{function annotation}) ja funktioiden
%yliajaminen\defword{function overloading} riippuen parametrien tyypistä tai
%arvoista, mikäli ne voidaan kääntöaikaisesti päätellä. Koska C:n spesifikaatio
%ei salli useaa samannimistä funktiota, funktioiden yliajaminen on haastavaa
%toteuttaa alustariippumattomasti.

%Tämän tutkielmasuunnitelman toisessa luvussa määritellään seuraavat
%ominaisuudet kielelle, joka voisi korvata C:n kokonaan:
%
%Vaihtoehtoinen kieli aina
%\begin{itemize}[topsep=0pt,itemsep=0pt]
%    \item on yhtä nopea tai nopeampi kuin C,
%    \item käyttää saman verran muistia tai vähemmän kuin C,
%    \item toimii kaikissa järjestelmissä, joissa C toimii,
%    \item toimii saumattomasti C:n kanssa ja
%    \item toimii saumattomasti muiden kielien C-rajapintojen kanssa.
%\end{itemize}
%Jotta jokin kieli olisi absoluuttisesti parempi, vähintään yhden näistä
%kriteereistä tulee olla aidosti parempi.

%Lisäksi kielistä tutkitaan muita suosioon vaikuttavia syitä, kuten
%helppokäyttöisyyttä ja sosiaalisia aspekteja, kuten Googlen tarjoaman tuen
%vaikutusta Go-kielen suosioon.%\hl{Vai tutkitaanko?}

Tämän tutkielman toisessa luvussa esitetään tutkielman kannalta
oleellinen teoria, eli määritetään tutkielman vertailussa käytetyt kielten
ominaisuuksien vertailukohteet, pohditaan kielen suosioon vaikuttavia tekijöitä
sekä kerrotaan lyhyesti muista tutkittavista kielistä. Kolmannessa luvussa
kerrotaan, miksi muut tutkitut kielet eivät täytä näitä ominaisuuksia.
Neljännessä luvussa kerrotaan, mitä ominaisuuksia uudesta ohjelmointikielestä
voisi löytyä, joita ei löydy C:stä. Viidennessä luvussa määritellään
näiden tulosten pohjalta uusi ohjelmointikieli, Purkka, joka voisi tutkielman
määritelmien mukaisesti korvata C:n käytön. Kuudennessa luvussa verrataan tätä
ohjelmointikieltä C:hen ja muihin tutkielmassa käsitellyihin kieliin.
Seitsemännessä luvussa kerrataan tutkimuksen tulokset.


\section{Ohjelmointikielten vertailu}
\label{sec:muut}

\hl{Vertailuluku, jossa käydään läpi toisessa luvussa määriteltyjä kriteerejä
kunkin kielen kohdalla, ja pohditaan, täyttääkö kukin kieli kriteerejä.
Jokaiselle kielelle tulee yksi aliluku, jossa käydään kriteerejä läpi kyseisen
kielen yksilöllisten ominaisuuksien kannalta. Tämän luvun sisältö on hyvin
samankaltainen tutkimussuunnitelman kirjallisuusosan kolmannen luvun kanssa.}

\hl{ Luku 3: metatekstiä luvun alkuun - mitä asioita käsitellään ja miten ne
liittyvät toisiinsa. }

\subsection{Yleisiä vertailtavien ohjelmointikielten ominaisuuksia}
\label{sec:muut:common}

\hl{Kooste verrattavien ohjelmointikielten yhteisistä ominaisuuksista. Koska
verrattavissa ohjelmointikielissä on useita ominaisuuksia, joita C:ssä ei ole,
mutta muissa verrattavissa kielissä on, tällä aliluvulla voidaan merkittävästi
poistaa turhaa toistoa.}

C:hen vertailtavissa ohjelmointikielissä on yleisesti useita ominaisuuksia,
jotka hidastavat ohjelmien suoritusaikaista nopeutta, lisäävät muistinkäyttöä,
vähentävät alustariippumattomuutta tai heikentävät yhteensopivuutta C:n kanssa.

Yleisin näistä on automaattinen muistinhallinta, joka muistin vapauttamisen
automatisoimiseksi seuraa ohjelman käyttämää muistia. Lähes aina automaattinen
muistinhallinta lisää kieleen ''roskien keräämisen''\defword{garbage
collection, GC}, jonka ajaksi ohjelman suoritus pysäytetään. Lisäksi roskien
keräämiseen perustuva automaattinen muistinhallinta lisää muistinkäyttöä, sillä
ohjelmointikieli joutuu suoritusaikaisesti seuraamaan käytössä olevia
muistiosoitteita.

Monissa vertailtavissa kielissä on käytössä nimiruntelu\defword{name mangling},
joka mahdollistaa useat näennäisesti samannimiset funktiot. Tämä kuitenkin
aiheuttaa ohjelmointikielten välille yhteensopivuusongelmia, sillä toisesta
kielestä kutsuttaessa pitää tietää kutsuttavan funktion todellinen nimi.
Esimerkiksi \texttt{int}-tyyppisen arvon palauttava funktion \texttt{foo()}
oikeaksi nimeksi voisi tulla \texttt{\_Z3foov}, kuten
\texttt{g++}-kääntäjä~\citep{gcc} tekee.

Ohjelmointikielen ominaisuudet vaikuttavat siihen, minkälaisia
ohjelmistoarkkitehtuureja kielellä tehdään~\citep{designpatternsdesign}.
Moderneissa ohjelmointikielissä virheiden käsittely on yleensä toteutettu
kahdella tavalla: toinen on poikkeavat paluuarvot ja toinen on poikkeuksien
heittäminen. Yleisesti ottaen kaikki ohjelmointikielet tukevat ensimmäistä ja
suurin osa toista tapaa. Poikkeusten käsittely on hieman hitaampaa ja aiheuttaa
hieman suuremman muistinkäytön, ja tehokkaaseen ohjelmakoodiin pyrkiessä
yleensä vältetään poikkeusten käyttämistä~\citep{exceptioncosts}. Monet
poikkeuksia tukevien ohjelmointikielten vakiokirjastot kuitenkin hallitsevat
virhetilanteita poikkeuksilla, mikä pakottaa ohjelmoijan käyttämään poikkeuksia
ohjelmoidessa.


\subsection{Ada}

Ada on Yhdysvaltain puolustusministeriön kehittämä ohjelmointikieli, joka
suunniteltiin korvaamaan kaikki muut puolustusministeriön käyttämät
ohjelmointikielet~\citep{adahistory}, muun muassa C:n. Toisin kuin monet muut
ohjelmointikielet, Ada on suunniteltu monta vuotta kestäneen prosessin
kautta~\citep[s.~121]{theoryandpractice}. Ada on hyvin moneen taipuva kieli,
sillä se on suunniteltu hallitsemaan monia eri käyttötarkoituksia matalan tason
bittitason ohjelmoinnista korkean tason arkkitehtuureihin.

Ohjelmoinnin helpottamiseksi Adassa on sekä poikkeukset että automaattinen
muistinhallinta. Nämä kuitenkin hidastavat kieltä hieman aikaisemmin todetuista
syistä. Lisäksi C-kielen kutsuminen on työlästä  -- jokainen C-funktio on
yksitellen määritettävä kutsukonvention\defword{calling convention}
kanssa~\citep[s.~471]{ADA12}. Adan alustariippumaton C-tuki on kuitenkin
äärimmäisen kattava, paikoitellen C:n omaa tukea kattavampi (C:n standardi ei
kuvaile esimerkiksi kutsukonventioita\footnote{Esimerkiksi
Windows-käyttöjärjestelmässä käytetään sekaisin kahta erilaista
kutsukonventiota, sillä käyttöjärjestelmän rajapinnat käyttävät
\texttt{stdcall}-kutsukonventiota ja ohjelmat yleensä
\texttt{cdecl}-kutsukonventiota. Näin jokainen C-ohjelma joutuu käyttämään
kunkin kääntäjän standardoimattomia ominaisuuksia.}, vaan ne on jätetty kunkin
kääntäjätoteutuksen päätettäväksi). Ada on myös vertailun ainoa kieli, joka voi
kutsua muilla ohjelmointikielillä kirjoitettuja kirjastorutiineja suoraan ilman
C-rajapintojen käyttöä. Adassa on C:n lisäksi tuki C++:lle\footnote{C++-tuki ei
sisällä nimiruntelun tukemista, vaan kutsuttavista funktioista pitää määrittää
runnellut nimet.}, Fortranille ja Cobolille~\citep[s.~585]{ADA12}. Adassa ei
kuitenkaan ole makrojärjestelmää, eikä Ada tue C:n makrojärjestelmää.

\subsection{C++}

C++ on Bjarne Stroustrupin 1980-luvulta eteenpäin kehittämä kieli, jonka
yhtenä tarkoituksena on yhdistää Simula-kielen ominaisuudet ohjelman
organisointiin yhteen C:n tehokkuuden ja joustavuuden
kanssa~\citep{cpphistory}. C++ on nykypäivänä suosittu tehokkuutensa ja
monipuolisuutensa takia monimutkaisissa ohjelmistoissa, kuten
palvelinohjelmistoissa, kuvankäsittelyohjelmistoissa sekä
peleissä~\citep{cppapps}.

C++ on kehitetty C:n pohjalta, ja siinä onkin erittäin hyvä C-tuki. Koska
C++\hyp{}funktiot nimirunnellaan eikä nimiruntelua ole määritelty tarkasti C++:n
standardissa, C++-koodia on hankalaa kutsua jopa samalla alustalla eri
C++-kääntäjien välillä -- luvussa~\ref{sec:muut:common} on annettu esimerkki
\texttt{g++}-kääntäjän nimiruntelemasta funktiosta. C-koodin
otsikkotiedostoissa\defword{header file} on usein alussa C++-koodia, joka
laittaa nimiruntelun pois päältä. Näin C++-ohjelmat voivat helposti kutsua
C:llä kirjoitettujen kirjastojen funktioita, sillä C++-ohjelmat voivat käyttää
C-kielen otsikkotiedostoja lähes aina ilman muita muokkauksia.

C++:n standardikirjaston virheidenkäsittely on toteutettu poikkeuksilla, jotka
aiheuttavat pienen suoritusaikaisen hidastuksen. Monet vakiokirjaston funktioista ja
metodeista voivat heittää poikkeuksen virhetilanteissa.

C++:ssa on mahdollista käyttää viitemäärälaskettua\defword{reference counted}
muistinhallintaa (esimerkiksi vakiokirjaston \texttt{std::shared\_ptr}), jolla
voidaan käyttää suoritusaikaisesti varattua muistia ilman muistivuotoja. C++:n
\texttt{std::shared\_ptr} ei käytä erillistä roskien keräystä, vaan kun
viimeinen viite olioon poistetaan, myös varattu muisti vapautetaan. Tällöin
ohjelman suorituksen aikana ei tule roskienkeräystaukoja.

C++ tukee geneeristä ohjelmointia\defword{generic programming} luokkien
yhteydessä malliohjelmoinnilla\defword{template programming}. C++:n
toteutuksessa jokaisesta uniikista mallin tyyppiparametrikombinaatiosta luodaan
lopulliseen ohjelmaan kopio mallin ilmentymän\defword{instance} käytetyistä
funktioista, joka kasvattaa ohjelmien kokoa. Tämä mahdollistaa jokaisen luokan
ilmentymän erillisen optimoinnin, mutta yleisesti kasvattaa sekä
kääntämisaikoja että ohjelmien kokoa.

C++ käyttää lähes samaa makrojärjestelmää kuin C. C++:n makrojärjestelmässä on
11 avainsanaa, joita ei voi määrittää uudelleen
esikäsittelijässä~\citep[luku~19.2]{CPP17}\footnote{ Avainsanat ovat
\texttt{and}, \texttt{and\_eq}, \texttt{bitand}, \texttt{bitor},
\texttt{compl}, \texttt{not}, \texttt{not\_eq}, \texttt{or}, \texttt{or\_eq},
\texttt{xor} sekä \texttt{xor\_eq}. }. Tämä tarkoittaa sitä, että C++:n
esikäsittelijä ei hyväksy joitakin C:n esikäsittelijän hyväksymiä makroja,
tosin tämä ei tapahdu käytännössä koskaan. Koska makrojärjestelmä on muuten
sama kuin C:n makrojärjestelmä, se on hyvin rajoittunut \citep[luku~19]{CPP17}.

\subsection{D}

\hl{Harkinnassa: D kokonaan pois tutkielman scopesta koska sitä ei ole
Benchmarks Gamessa -> käsittely on hankalampaa}

D on 2000-luvun alussa Digital Mars -yrityksen julkaisema ohjelmointikieli,
jonka tarkoituksena on mahdollistaa tehokkaiden ohjelmien kirjoittaminen
helposti ja turvallisesti~\citep{dhistory}. D on suunniteltu syntaksiltaan ja
käytökseltään muistuttamaan C:tä ja C++:aa. Vaikka D-kielessä on olemassa
automaattinen muistinhallinta, D:n \emph{BetterC}-tila tekee kielestä
''paremman C:n'' poistamalla suoritusaikaiset ominaisuudet, mukaan lukien
automaattisen muistinhallinnan~\citep{dbetterc}. Tällöin kielestä poistuu
useita ominaisuuksia, mutta esimerkiksi D:n käännösaikaista makrojärjestelmää
voi käyttää.

C-koodin kutsuminen on melko helppoa, mutta ei aivan saumatonta, sillä jokainen
kutsuttava funktio tulee määritellä erikseen -- D ei ymmärrä C:n
otsikkotiedostoja. Tämä kuitenkin onnistuu yhdellä rivillä jokaista C:n
funktiota kohden, sillä D:n tyyppijärjestelmä on hyvin lähellä C:tä. D:lle on
myös olemassa useita työkaluja otsikkotiedostojen automaattiseen muuntamiseen,
kuten \emph{Header to D}-työkalu~\citep{htod}. Työkalut eivät kuitenkaan ole
täydellisiä, sillä useiden D:n ominaisuuksien semantiikka ei ole C:n kanssa
yhteensopiva. D ei sisällä tukea C:n makrojärjestelmälle. D:ssä ei ole C:n
kaltaista esikäsittelijää, sillä sitä ei pidetty tarpeellisena~\citep{pretod}.
D:ssä on kuitenkin käytettävissä C++:n kaltainen mallipohjainen
makrojärjestelmä, jolla voidaan suorittaa koodia
käännösaikaisesti~\citep{Dtemplate}.

\hl{Suorittaa koodia käännösaikaisesti -> pidempi selitys}

\subsection{Go}

Go on Googlen kehittämä ja vuoden 2009 loppupuolella julkaisema
ohjelmointikieli, jonka tarkoituksena on yhdistää käännösaikaisesti tyypitetyn
ohjelmointikielen turvallisuus ja tehokkuus suoritusaikaisesti tyypitettyjen
ohjelmointikielten helppokäyttöisyyteen~\citep{gohistory}. Toisin kuin monissa
moderneissa C-perheen kielissä, Go-kielessä ei ole luokkia vaan pelkkiä
tietueita ja rajapintoja. Go suunniteltiin erityisesti korvaamaan C++:n käyttö
Googlella johtuen C++:n pitkistä käännösajoista.

Go-kielessä ei ole muista vertailtavista kielistä poiketen geneerisiä
tyyppiparametreja. Tämä estää käännösaikaisesti tyyppitarkistetun geneerisen
lähdekoodin kirjoittamisen. Ohjelmat voivat kuitenkin suoritusaikaisesti
reflektion kautta tunnistaa muuttujien konkreettisen tyypin. Tämän
mahdollistaminen kasvattaa ohjelmien suoritusaikaista muistinkäyttöä, sillä
muuttujien mukana on säilytettävä tunniste muuttujan oikeasta tyypistä. Tämä
kuitenkin yksinkertaistaa ohjelmien kirjoittamista, sillä ohjelmoijan ei
tarvitse miettiä kirjoittamishetkellä monimutkaisia
tyyppejä~\citep[esim.][kalvo 8]{gohistory}, kuitenkin mahdollistaen geneerisen
lähdekoodin kirjoittamisen ilman käännösaikaisia geneerisiä malleja.

%Ohjelmassa~\ref{fig:goreflection} käsitellään value-muuttujaa geneerisesti -
%jos muuttuja on tyyppiä \texttt{string}, muuttuja palautetaan sellaisenaan.
%Jos muuttujassa toteuttaa \texttt{Stringer}-rajapinnan, eli siinä on
%\texttt{String()}-metodi, sitä kutsutaan ja palautetaan tulos. Muussa
%tapauksessa palautetaan tyhjä merkkijono.

Vaikka Go-kielessä ei itsessään ole makroja, se sisältää \texttt{go~generate}
-työkalun, jota voidaan käyttää lähdekoodin generointiin~\citep{gogenerate}.
\texttt{go~generate} mahdollistaa minkä tahansa komentorivikomennon suorittamisen
ja on enemmänkin standardoitu tapa suorittaa tiettyjä komentorivikäskyjä osana
ohjelman kääntämistä kuin tyypillinen makrojärjestelmä.
\texttt{\mbox{go~generate}} suoritetaan erillisenä komentona eikä esimerkiksi osana
\texttt{\mbox{go~build}} -komentoa.

%\begin{listing}[ht!]
%    \inputminted{go}{goreflect.go}
%    \caption{Geneerinen funktio Go-kielessä. Jos \texttt{value}-muuttuja on
%    tyyppiä \texttt{string}, muuttuja palautetaan sellaisenaan. Jos muuttujassa
%    toteuttaa \texttt{Stringer}-rajapinnan, eli siinä on
%    \texttt{String()}-metodi, sitä kutsutaan ja palautetaan tulos. Muussa
%    tapauksessa palautetaan tyhjä merkkijono.}
%    \label{fig:goreflection}
%\end{listing}


Go-kielen virheidenkäsittely on toteutettu useissa kohdissa C:n tavoin;
funktioista palautetaan virheellisissä tilanteissa virheellinen arvo. Tämä
tosin tehdään usein palauttamalla erillinen \texttt{Error}-tyyppiä oleva arvo
-- Go mahdollistaa useamman kuin yhden paluuarvon. Go-kielessä on myös
poikkeukset, joita suositellaan käytettävän vain poikkeuksellisissa
tilanteissa, joita koodin kutsuja ei voi korjata
suoritusaikaisesti~\citep{effectivego}. Tällainen tilanne voisi olla esimerkiksi
väärä parametrin tyyppi. Poikkeuksen heittäminen voi myös olla perusteltua,
mikäli kirjastorutiinin suoritus halutaan lopettaa syvällä yksityisissä
funktioissa ja arvon halutaan menevän useiden funktioiden läpi kirjaston
rajapintaan asti, jossa se muutetaan virhearvoksi.

C:n kutsuminen Go-kielestä ei ole aukotonta: koska Go on muistinkäytöltä
turvallinen kieli, erityisesti muistin jakaminen C:n ja Go-kielen välillä on
hankalaa. Lisäksi C:n funktio-osoittimia ei voi kutsua Go-kielen
puolelta~\citep{cgo}. Go mahdollistaa C-otsikkotiedostojen suoran käytön
lähdekoodista, mikä helpottaa C-koodin kutsumista.

\subsection{Rust}

Rust on Mozilla Foundationin kehittämä ohjelmointikieli, joka on suunniteltu
turvalliseksi, rinnakkaiseksi ja käytännölliseksi
järjestelmäohjelmointikieleksi~\citep{rustfaq}. Rustissa on monimutkainen
tyyppijärjestelmä, jolla ohjelmat voivat todistaa esimerkiksi turvallisen
rinnakkaisajon ilman, että ohjelmaan tulee suoritusaikaisia rajoitteita tai
hidastuksia. Rust alkoi Graydon Hoaren henkilökohtaisena sivuprojektina, mutta
on nyt käytössä esimerkiksi osana Gecko-se\-lain\-moot\-to\-rin kehitystä C++:n ja
JavaScriptin ohella.

Kuten Go-kielessä, Rustissa voi myös käyttää poikkeuksia. Rustin
virheidenhallinta on muutenkin lähellä Go-kielen virhehallintaa -- Rustin
ohjekirja opastaa käyttämään mieluummin paluuarvoja kuin
poikkeuksia~\citep{rusterrorhandling}.

Rustin monimutkainen tyyppijärjestelmä kannustaa kirjoittamaan turvallisia
ohjelmia. Tietorakenteiden arvojen muuttaminen on tehty tietoisesti hankalaksi,
sillä monimutkaisissa ohjelmissa holtittomasti muuttuva tila on usean vian
syynä, ja muuttumaton tila tekee monisäikeistettyjen\defword{multithreaded}
ohjelmien toteutuksesta huomattavasti helpompaa~\citep[luku 4, kohta
17]{effectivejava}.

Rustin tyyppijärjestelmän ytimessä on käsitteet arvojen
omistuksesta\defword{ownership}, elinajoista\defword{lifetime} sekä
muuttuvuudesta\defword{mutability}. Kun arvo sijoitetaan muuttujaan, sen
elinajaksi asetetaan muuttujan elinaika, joka päättyy muuttujan poistuessa
näkyvyysalueesta\defword{scope}. Muuttujiin voidaan tehdä viitteitä, mutta
viitteet eivät voi elää arvoja kauempaa, sillä kääntäjän
elinaikatarkistaja\defword{borrow checker} pystyy seuraamaan arvojen
elinaikoja. Kullakin muuttujalla voi olla joko rajaton määrä muuttumattomia
viitteitä\defword{immutable reference} tai yksi muuttuva viite\defword{mutable
reference}, mutta molempia ei voi käyttää yhtä aikaa. \hl{muuttuva -> muuttava?}

\hl{Pidempi selitys Rustista}

Turvallisuudella on kuitenkin hintansa -- Rust-ohjelmat vievät enemmän tilaa
kuin vastaavat C-ohjelmat. Jos Rust-ohjelmista poistaa standardikirjaston ja
käyttää suoraan C:n standardikirjastoa, ohjelmasta saa miltei samankokoisen
kuin vastaavasta C-kielellä kirjoitetusta ohjelmasta~\citep{rustbinarysize}.
Samalla tosin suurin osa Rustin ominaisuuksista jää pois. Rustin turvallisuus
vaatii myös monimutkaisen tyyppijärjestelmän, joka on vaikeampi opetella kuin
yksinkertaisemman kielen tyyppijärjestelmä.

Rust ei pysty suoraan käsittelemään C:n otsikkotiedostoja, mutta D:n tavoin
Rustille on saatavilla työkaluja otsikkotiedostojen automaattiseen
muuntamiseen~\citep{rustbindgen}. Rustin kehittäjät kuitenkin suosittelevat
jokaisen kirjaston kohdalla kirjoittamaan käsin Rust-rajapinnan C-kirjastoille,
sillä C:n tyyppimäärittelyt eivät tarjoa Rustin vaatimaa tarkkuutta funktioiden
turvallisuudesta.

Rust sisältää useita erillisiä kattavia makrojärjestelmiä:
varsinaiset makrot~\citep{rustmacros}, joiden lisäksi Rustista löytyy useita
kokeellisia lähdekoodin generointiin tarkoitettuja
ominaisuuksia~\citep{rustprocmacros, rustplugins}. Rust käsittelee makrot vasta
ohjelman alkioanalyysin jälkeen, eli makroilla ei voi käsitellä mitä tahansa
tekstiä. Rust-kääntäjän tulee tunnistaa kielen tekstialkiot ennen makrojen
suorittamista, eli Rust ei mahdollista uusien operaattoreiden määrittämistä
makrojen avulla. Rust vaatii myös erottimien (sulkeiden, lainausmerkkien ja
heittomerkkien) olevan kielen syntaksin mukaisesti pareina. Rustin
makroprosessori on sekä Turing-täydellinen että hygieeninen~\citep{rustmacros}.

\subsection{Yhteenveto}

\hl{Tässä aliluvussa tehdään yhteenveto ohjelmointikielten vertailusta, ja
todetaan, miksi yksikään verrattavista kielistä ei täytä toisessa luvussa
määriteltyjä vertailukriteerejä.}

Yksikään vertailtavista kielistä ei täytä kaikkia luvussa~\ref{sec:abs}
määriteltyjä rajoitteita. Yksikään kielistä ei täytä muistinkäytön rajoitteita,
jonka lisäksi saumaton C:n käyttö onnistuu vain C++:n kanssa.

Taulukossa~\ref{table:properties} on yhteenveto verrattavista
ohjelmointikielistä sisältäen kunkin kielen nimiruntelun, muistinhallinnan,
yhteensopivuuden C:n vierasfunktiorajapinnan kanssa sekä yhteensopivuuden
muiden kielien vierasfunktiorajapintojen kanssa.

C++, D ja Rust mahdollistavat yhteensopivuuden parantamiseksi nimiruntelun
poistamisen käytöstä, mutta tämä ei toimi esimerkiksi C++:n luokkien
yhteydessä. Ada ja Go mahdollistavat funktioiden nimien valitsemisen
linkittäjää varten, jolloin esimerkiksi \texttt{EsimerkkiFunktio}-nimistä
funktiota voidaan kutsua \texttt{esimFunk}-nimellä C-ohjelmasta. Kaikki
vertailtavat kielet tukevat vierasfunktiorajapintoja C:n mukaisesti, eli kielet
voivat kutsua muita kieliä C:llä toteutettujen rajapintojen läpi.

C++ ja Rust ovat hyvin lähellä C:tä käännettyjen ohjelmien koossa, mutta
häviävät C:lle laajojen ja paljon tilaa vievien vakiokirjastojen takia.
Molemmat ovat myös hyvin monimutkaisia kieliä. Vain C++ ja Go voivat
sisällyttää ohjelmiin C:n otsikkotiedostoja, kun taas Ada, D ja Rust vaativat
jokaisen funktion määrittämistä. Adalle, D:lle ja Rustille on tosin olemassa
työkaluja, joilla tämän määrittämisen voi automatisoida.

D, C++ ja Rust sisältävät mahdollisuuden käännösaikaisiin makroihin. C++:n
makrojärjestelmä on melkein täysin yhteensopiva C:n makrojärjestelmän kanssa.
Rustin makrojärjestelmä on taas huomattavasti ilmaisukykyisempi, mutta
epäyhteensopiva C:n kanssa. Toisin kuin C++:n mallit, D:n mallit voivat ottaa
parametreiksi vain tyyppejä.

\newpage

\hl{Voisiko tähän lisätä myös C:n?}

\hl{Footnote: viitelaskurit}

\begin{table}[ht!]
    \begin{adjustbox}{center}
        \kern-0.7cm
        \begin{tabular}{@{}lllp{6.2cm}@{}} \toprule
        Kieli & Nimiruntelu              & Muistinhallinta & Makrojärjestelmä \\ \midrule
        Ada   & täysin hallittavissa     & automaattinen   & Ei ole \\
        C++   & saa osittain pois päältä & manuaalinen     & Sama esikäsittelijä kuin C:ssä, \mbox{erillinen} kattava malliohjelmointi \\
        D     & saa pois päältä          & molemmat        & Tyyppipohjainen malliohjelmointi \\
        Go    & täysin hallittavissa     & automaattinen   & Ei ole \\
            Rust  & saa pois päältä & automaattinen\footnote{Borrow Checkerin hallitsemana, joka mahdollistaa automaattisen muistinhallinnan ilman sen aiheuttamaa hidastusta. Maininta viitelaskureista?} & Useita erilaisia makrojärjestelmiä \\ \bottomrule
    \end{tabular}
    \end{adjustbox} \\[0.5cm]

    \begin{adjustbox}{center}
        \begin{tabular}{@{}lll@{}} \toprule
            Kieli & Yhteensopivuus C:n kanssa                                                   & Yhteensopivuus muiden kielten kanssa \\ \midrule
            Ada   & Kattava         & C, C++, Fortran, Cobol \\
            C++   & Lähes saumaton                                                              & Vain C \\
            D     & Kattava         & Vain C \\
            Go    & Yhteensopivuusongelmia   & Vain C \\
            Rust  & Kattava         & Vain C \\ \bottomrule
        \end{tabular}
    \end{adjustbox}
    \caption{
        Kielten ominaisuuksien yhteenveto.
    }
    \label{table:properties}
\end{table}

\begin{figure}[ht!]
    \begin{adjustbox}{center}
    \begin{minipage}{1.15\textwidth}
    \begin{minipage}{0.5\textwidth}
        \begin{tikzpicture}[gnuplot]
%% generated with GNUPLOT 5.2p2 (Lua 5.3; terminal rev. 99, script rev. 102)
%% la  1. joulukuuta 2018 09.58.32
\path (0.000,0.000) rectangle (8.000,6.000);
\gpcolor{color=gp lt color border}
\gpsetlinetype{gp lt border}
\gpsetdashtype{gp dt solid}
\gpsetlinewidth{1.00}
\draw[gp path] (1.136,0.616)--(1.316,0.616);
\draw[gp path] (7.447,0.616)--(7.267,0.616);
\node[gp node right] at (0.952,0.616) {$0$};
\draw[gp path] (1.136,1.462)--(1.316,1.462);
\draw[gp path] (7.447,1.462)--(7.267,1.462);
\node[gp node right] at (0.952,1.462) {$5$};
\draw[gp path] (1.136,2.308)--(1.316,2.308);
\draw[gp path] (7.447,2.308)--(7.267,2.308);
\node[gp node right] at (0.952,2.308) {$10$};
\draw[gp path] (1.136,3.154)--(1.316,3.154);
\draw[gp path] (7.447,3.154)--(7.267,3.154);
\node[gp node right] at (0.952,3.154) {$15$};
\draw[gp path] (1.136,3.999)--(1.316,3.999);
\draw[gp path] (7.447,3.999)--(7.267,3.999);
\node[gp node right] at (0.952,3.999) {$20$};
\draw[gp path] (1.136,4.845)--(1.316,4.845);
\draw[gp path] (7.447,4.845)--(7.267,4.845);
\node[gp node right] at (0.952,4.845) {$25$};
\draw[gp path] (1.136,5.691)--(1.316,5.691);
\draw[gp path] (7.447,5.691)--(7.267,5.691);
\node[gp node right] at (0.952,5.691) {$30$};
\draw[gp path] (1.997,0.616)--(1.997,0.796);
\draw[gp path] (1.997,5.691)--(1.997,5.511);
\node[gp node center] at (1.997,0.308) {C};
\draw[gp path] (3.144,0.616)--(3.144,0.796);
\draw[gp path] (3.144,5.691)--(3.144,5.511);
\node[gp node center] at (3.144,0.308) {C++};
\draw[gp path] (4.292,0.616)--(4.292,0.796);
\draw[gp path] (4.292,5.691)--(4.292,5.511);
\node[gp node center] at (4.292,0.308) {Rust};
\draw[gp path] (5.439,0.616)--(5.439,0.796);
\draw[gp path] (5.439,5.691)--(5.439,5.511);
\node[gp node center] at (5.439,0.308) {Ada};
\draw[gp path] (6.586,0.616)--(6.586,0.796);
\draw[gp path] (6.586,5.691)--(6.586,5.511);
\node[gp node center] at (6.586,0.308) {Go};
\draw[gp path] (1.136,5.691)--(1.136,0.616)--(7.447,0.616)--(7.447,5.691)--cycle;
\node[gp node center,rotate=-270] at (0.092,3.153) {sekuntia};
\gpfill{rgb color={1.000,1.000,1.000}} (1.710,0.893)--(2.284,0.893)--(2.284,1.442)--(1.710,1.442)--cycle;
\draw[gp path] (1.710,0.893)--(2.284,0.893)--(2.284,1.442)--(1.710,1.442)--cycle;
\draw[gp path] (1.710,0.932)--(2.284,0.932);
\draw[gp path] (1.997,0.846)--(1.997,0.893);
\draw[gp path] (1.997,1.442)--(1.997,2.201);
\draw[gp path] (1.817,2.201)--(2.177,2.201);
\draw[gp path] (1.817,0.846)--(2.177,0.846);
\gpfill{rgb color={1.000,1.000,1.000}} (2.857,0.926)--(3.431,0.926)--(3.431,1.247)--(2.857,1.247)--cycle;
\draw[gp path] (2.857,0.926)--(3.431,0.926)--(3.431,1.247)--(2.857,1.247)--cycle;
\draw[gp path] (2.857,1.031)--(3.431,1.031);
\draw[gp path] (3.144,0.841)--(3.144,0.926);
\draw[gp path] (2.964,1.247)--(3.324,1.247);
\draw[gp path] (2.964,0.841)--(3.324,0.841);
\gpfill{rgb color={1.000,1.000,1.000}} (4.005,0.910)--(4.579,0.910)--(4.579,1.628)--(4.005,1.628)--cycle;
\draw[gp path] (4.005,0.910)--(4.579,0.910)--(4.579,1.628)--(4.005,1.628)--cycle;
\draw[gp path] (4.005,0.989)--(4.579,0.989);
\draw[gp path] (4.292,0.863)--(4.292,0.910);
\draw[gp path] (4.292,1.628)--(4.292,2.286);
\draw[gp path] (4.112,2.286)--(4.472,2.286);
\draw[gp path] (4.112,0.863)--(4.472,0.863);
\gpfill{rgb color={1.000,1.000,1.000}} (5.152,1.284)--(5.726,1.284)--(5.726,2.182)--(5.152,2.182)--cycle;
\draw[gp path] (5.152,1.284)--(5.726,1.284)--(5.726,2.182)--(5.152,2.182)--cycle;
\draw[gp path] (5.152,1.652)--(5.726,1.652);
\draw[gp path] (5.439,0.912)--(5.439,1.284);
\draw[gp path] (5.439,2.182)--(5.439,2.661);
\draw[gp path] (5.259,2.661)--(5.619,2.661);
\draw[gp path] (5.259,0.912)--(5.619,0.912);
\gpfill{rgb color={1.000,1.000,1.000}} (6.300,1.289)--(6.874,1.289)--(6.874,4.169)--(6.300,4.169)--cycle;
\draw[gp path] (6.300,1.289)--(6.874,1.289)--(6.874,4.169)--(6.300,4.169)--cycle;
\draw[gp path] (6.300,2.143)--(6.874,2.143);
\draw[gp path] (6.587,0.961)--(6.587,1.289);
\draw[gp path] (6.587,4.169)--(6.587,5.520);
\draw[gp path] (6.407,5.520)--(6.767,5.520);
\draw[gp path] (6.407,0.961)--(6.767,0.961);
\draw[gp path] (1.136,5.691)--(1.136,0.616)--(7.447,0.616)--(7.447,5.691)--cycle;
%% coordinates of the plot area
\gpdefrectangularnode{gp plot 1}{\pgfpoint{1.136cm}{0.616cm}}{\pgfpoint{7.447cm}{5.691cm}}
\end{tikzpicture}
%% gnuplot variables

        \vspace*{-0.8cm}
    \end{minipage}
    \begin{minipage}{0.5\textwidth}
        \begin{tikzpicture}[gnuplot]
%% generated with GNUPLOT 5.2p2 (Lua 5.3; terminal rev. 99, script rev. 102)
%% ma 21. tammikuuta 2019 23.24.51
\path (0.000,0.000) rectangle (8.000,6.000);
\gpcolor{color=gp lt color border}
\gpsetlinetype{gp lt border}
\gpsetdashtype{gp dt solid}
\gpsetlinewidth{1.00}
\draw[gp path] (1.320,0.616)--(1.500,0.616);
\draw[gp path] (7.447,0.616)--(7.267,0.616);
\node[gp node right] at (1.136,0.616) {$0$};
\draw[gp path] (1.320,1.341)--(1.500,1.341);
\draw[gp path] (7.447,1.341)--(7.267,1.341);
\node[gp node right] at (1.136,1.341) {$50$};
\draw[gp path] (1.320,2.066)--(1.500,2.066);
\draw[gp path] (7.447,2.066)--(7.267,2.066);
\node[gp node right] at (1.136,2.066) {$100$};
\draw[gp path] (1.320,2.791)--(1.500,2.791);
\draw[gp path] (7.447,2.791)--(7.267,2.791);
\node[gp node right] at (1.136,2.791) {$150$};
\draw[gp path] (1.320,3.516)--(1.500,3.516);
\draw[gp path] (7.447,3.516)--(7.267,3.516);
\node[gp node right] at (1.136,3.516) {$200$};
\draw[gp path] (1.320,4.241)--(1.500,4.241);
\draw[gp path] (7.447,4.241)--(7.267,4.241);
\node[gp node right] at (1.136,4.241) {$250$};
\draw[gp path] (1.320,4.966)--(1.500,4.966);
\draw[gp path] (7.447,4.966)--(7.267,4.966);
\node[gp node right] at (1.136,4.966) {$300$};
\draw[gp path] (1.320,5.691)--(1.500,5.691);
\draw[gp path] (7.447,5.691)--(7.267,5.691);
\node[gp node right] at (1.136,5.691) {$350$};
\draw[gp path] (2.341,0.616)--(2.341,0.796);
\draw[gp path] (2.341,5.691)--(2.341,5.511);
\node[gp node center] at (2.341,0.308) {C++};
\draw[gp path] (3.703,0.616)--(3.703,0.796);
\draw[gp path] (3.703,5.691)--(3.703,5.511);
\node[gp node center] at (3.703,0.308) {Rust};
\draw[gp path] (5.064,0.616)--(5.064,0.796);
\draw[gp path] (5.064,5.691)--(5.064,5.511);
\node[gp node center] at (5.064,0.308) {Ada};
\draw[gp path] (6.426,0.616)--(6.426,0.796);
\draw[gp path] (6.426,5.691)--(6.426,5.511);
\node[gp node center] at (6.426,0.308) {Go};
\draw[gp path] (1.320,5.691)--(1.320,0.616)--(7.447,0.616)--(7.447,5.691)--cycle;
\gpsetdashtype{dash pattern=on 10.00*\gpdashlength off 10.00*\gpdashlength }
\draw[gp path](1.320,2.066)--(7.447,2.066);
\node[gp node center,rotate=-270] at (0.000,3.153) {\shortstack{Muistinkäyttö \\ C:hen verrattuna (\%)}};
\gpfill{rgb color={1.000,1.000,1.000}} (2.001,2.096)--(2.683,2.096)--(2.683,2.951)--(2.001,2.951)--cycle;
\gpsetdashtype{gp dt solid}
\draw[gp path] (2.001,2.096)--(2.683,2.096)--(2.683,2.951)--(2.001,2.951)--cycle;
\draw[gp path] (2.001,2.184)--(2.683,2.184);
\draw[gp path] (2.342,2.065)--(2.342,2.096);
\draw[gp path] (2.342,2.951)--(2.342,3.867);
\draw[gp path] (2.162,3.867)--(2.522,3.867);
\draw[gp path] (2.162,2.065)--(2.522,2.065);
\gpfill{rgb color={1.000,1.000,1.000}} (3.362,2.198)--(4.044,2.198)--(4.044,3.387)--(3.362,3.387)--cycle;
\draw[gp path] (3.362,2.198)--(4.044,2.198)--(4.044,3.387)--(3.362,3.387)--cycle;
\draw[gp path] (3.362,2.321)--(4.044,2.321);
\draw[gp path] (3.703,2.112)--(3.703,2.198);
\draw[gp path] (3.703,3.387)--(3.703,3.499);
\draw[gp path] (3.523,3.499)--(3.883,3.499);
\draw[gp path] (3.523,2.112)--(3.883,2.112);
\gpfill{rgb color={1.000,1.000,1.000}} (4.724,2.117)--(5.406,2.117)--(5.406,4.203)--(4.724,4.203)--cycle;
\draw[gp path] (4.724,2.117)--(5.406,2.117)--(5.406,4.203)--(4.724,4.203)--cycle;
\draw[gp path] (4.724,3.707)--(5.406,3.707);
\draw[gp path] (5.065,0.616)--(5.065,2.117);
\draw[gp path] (5.065,4.203)--(5.065,5.121);
\draw[gp path] (4.885,5.121)--(5.245,5.121);
\draw[gp path] (4.885,0.616)--(5.245,0.616);
\gpfill{rgb color={1.000,1.000,1.000}} (6.085,2.209)--(6.767,2.209)--(6.767,5.158)--(6.085,5.158)--cycle;
\draw[gp path] (6.085,2.209)--(6.767,2.209)--(6.767,5.158)--(6.085,5.158)--cycle;
\draw[gp path] (6.085,3.417)--(6.767,3.417);
\draw[gp path] (6.426,1.298)--(6.426,2.209);
\draw[gp path] (6.426,5.158)--(6.426,5.378);
\draw[gp path] (6.246,5.378)--(6.606,5.378);
\draw[gp path] (6.246,1.298)--(6.606,1.298);
\draw[gp path] (1.320,5.691)--(1.320,0.616)--(7.447,0.616)--(7.447,5.691)--cycle;
%% coordinates of the plot area
\gpdefrectangularnode{gp plot 1}{\pgfpoint{1.320cm}{0.616cm}}{\pgfpoint{7.447cm}{5.691cm}}
\end{tikzpicture}
%% gnuplot variables

        \vspace*{-0.9cm}
    \end{minipage}
    \end{minipage}
    \end{adjustbox}
    \begin{adjustbox}{center}
    \begin{minipage}{1.15\textwidth}\makebox[\textwidth][c]{%
    \begin{minipage}{0.5\textwidth}
        \begin{tikzpicture}[gnuplot]
%% generated with GNUPLOT 5.2p2 (Lua 5.3; terminal rev. 99, script rev. 102)
%% to 10. tammikuuta 2019 17.28.43
\path (0.000,0.000) rectangle (8.000,6.000);
\gpcolor{color=gp lt color border}
\gpsetlinetype{gp lt border}
\gpsetdashtype{gp dt solid}
\gpsetlinewidth{1.00}
\draw[gp path] (1.320,0.616)--(1.500,0.616);
\draw[gp path] (7.447,0.616)--(7.267,0.616);
\node[gp node right] at (1.136,0.616) {$0$};
\draw[gp path] (1.320,1.302)--(1.500,1.302);
\draw[gp path] (7.447,1.302)--(7.267,1.302);
\node[gp node right] at (1.136,1.302) {$50$};
\draw[gp path] (1.320,1.988)--(1.500,1.988);
\draw[gp path] (7.447,1.988)--(7.267,1.988);
\node[gp node right] at (1.136,1.988) {$100$};
\draw[gp path] (1.320,2.673)--(1.500,2.673);
\draw[gp path] (7.447,2.673)--(7.267,2.673);
\node[gp node right] at (1.136,2.673) {$150$};
\draw[gp path] (1.320,3.359)--(1.500,3.359);
\draw[gp path] (7.447,3.359)--(7.267,3.359);
\node[gp node right] at (1.136,3.359) {$200$};
\draw[gp path] (1.320,4.045)--(1.500,4.045);
\draw[gp path] (7.447,4.045)--(7.267,4.045);
\node[gp node right] at (1.136,4.045) {$250$};
\draw[gp path] (1.320,4.731)--(1.500,4.731);
\draw[gp path] (7.447,4.731)--(7.267,4.731);
\node[gp node right] at (1.136,4.731) {$300$};
\draw[gp path] (1.320,5.417)--(1.500,5.417);
\draw[gp path] (7.447,5.417)--(7.267,5.417);
\node[gp node right] at (1.136,5.417) {$350$};
\draw[gp path] (2.341,0.616)--(2.341,0.796);
\draw[gp path] (2.341,5.691)--(2.341,5.511);
\node[gp node center] at (2.341,0.308) {C++};
\draw[gp path] (3.703,0.616)--(3.703,0.796);
\draw[gp path] (3.703,5.691)--(3.703,5.511);
\node[gp node center] at (3.703,0.308) {Rust};
\draw[gp path] (5.064,0.616)--(5.064,0.796);
\draw[gp path] (5.064,5.691)--(5.064,5.511);
\node[gp node center] at (5.064,0.308) {Ada};
\draw[gp path] (6.426,0.616)--(6.426,0.796);
\draw[gp path] (6.426,5.691)--(6.426,5.511);
\node[gp node center] at (6.426,0.308) {Go};
\draw[gp path] (1.320,5.691)--(1.320,0.616)--(7.447,0.616)--(7.447,5.691)--cycle;
\gpsetdashtype{dash pattern=on 10.00*\gpdashlength off 10.00*\gpdashlength }
\draw[gp path](1.320,1.988)--(7.447,1.988);
\node[gp node center,rotate=-270] at (0.000,3.153) {\shortstack{Lähdekoodissa tavuja \\ C:hen verrattuna (\%)}};
\gpfill{rgb color={1.000,1.000,1.000}} (2.001,1.778)--(2.683,1.778)--(2.683,2.235)--(2.001,2.235)--cycle;
\gpsetdashtype{gp dt solid}
\draw[gp path] (2.001,1.778)--(2.683,1.778)--(2.683,2.235)--(2.001,2.235)--cycle;
\draw[gp path] (2.001,1.953)--(2.683,1.953);
\draw[gp path] (2.342,1.423)--(2.342,1.778);
\draw[gp path] (2.342,2.235)--(2.342,2.307);
\draw[gp path] (2.162,2.307)--(2.522,2.307);
\draw[gp path] (2.162,1.423)--(2.522,1.423);
\gpfill{rgb color={1.000,1.000,1.000}} (3.362,1.684)--(4.044,1.684)--(4.044,3.542)--(3.362,3.542)--cycle;
\draw[gp path] (3.362,1.684)--(4.044,1.684)--(4.044,3.542)--(3.362,3.542)--cycle;
\draw[gp path] (3.362,2.858)--(4.044,2.858);
\draw[gp path] (3.703,1.478)--(3.703,1.684);
\draw[gp path] (3.703,3.542)--(3.703,4.922);
\draw[gp path] (3.523,4.922)--(3.883,4.922);
\draw[gp path] (3.523,1.478)--(3.883,1.478);
\gpfill{rgb color={1.000,1.000,1.000}} (4.724,2.508)--(5.406,2.508)--(5.406,5.086)--(4.724,5.086)--cycle;
\draw[gp path] (4.724,2.508)--(5.406,2.508)--(5.406,5.086)--(4.724,5.086)--cycle;
\draw[gp path] (4.724,2.891)--(5.406,2.891);
\draw[gp path] (5.065,2.184)--(5.065,2.508);
\draw[gp path] (5.065,5.086)--(5.065,5.691);
\draw[gp path] (4.885,2.184)--(5.245,2.184);
\gpfill{rgb color={1.000,1.000,1.000}} (6.085,1.627)--(6.767,1.627)--(6.767,2.604)--(6.085,2.604)--cycle;
\draw[gp path] (6.085,1.627)--(6.767,1.627)--(6.767,2.604)--(6.085,2.604)--cycle;
\draw[gp path] (6.085,1.959)--(6.767,1.959);
\draw[gp path] (6.426,1.514)--(6.426,1.627);
\draw[gp path] (6.426,2.604)--(6.426,3.035);
\draw[gp path] (6.246,3.035)--(6.606,3.035);
\draw[gp path] (6.246,1.514)--(6.606,1.514);
\draw[gp path] (1.320,5.691)--(1.320,0.616)--(7.447,0.616)--(7.447,5.691)--cycle;
%% coordinates of the plot area
\gpdefrectangularnode{gp plot 1}{\pgfpoint{1.320cm}{0.616cm}}{\pgfpoint{7.447cm}{5.691cm}}
\end{tikzpicture}
%% gnuplot variables

    \end{minipage}}
    \end{minipage}
    \end{adjustbox}
    \caption{
        Benchmarks Gamen~\citep[tiedot haettu 10.1.2019]{benchmarks} tuloksiin
        perustuvat kuvaajat ohjelmointikielten suorituskyvystä, muistinkäytöstä
        ja ohjelmien koosta verrattuna C:llä kirjoitettujen ohjelmien
        tuloksiin. Benchmarks Game ei sisällä verrattavista kielistä D:tä.}
    \label{fig:benchmarksgame}
\end{figure}

Benchmarks Gamen tulokset myös heijastavat näitä tuloksia -- C++ ja Rust ovat
nopeudeltaan hyvin lähellä C:tä, kun taas Ada ja Go ovat huomattavasti
hitaampia. Kuvassa~\ref{fig:benchmarksgame} verrataan suorituskykyä,
muistinkäyttöä ja ohjelmien lähdekoodin pituutta C:llä kirjoitettuun ohjelmaan.
Lähdekoodimittauksessa mitataan ohjelman kokoa siten, että ohjelmasta
poistetaan kommentit sekä ylimääräiset välimerkit. Tämän jälkeen ohjelma
pakataan \texttt{gzip}-pakkausohjelmalla~\citep{howmeasured} \hl{Miksi?}.
Kuvaajat perustuvat nopeimman kielellä kirjoitetun ohjelman tuloksiin -- lähes
jokaisella verrattavalla kielellä on jokaisessa suorituskykymittauksissa
useampi ohjelma.

C ja C++ ovat hyvin lähellä toisiaan suoritusnopeudessa. Rust on jonkin verran
hitaampi, ja Ada ja Go huomattavasti hitaampia. Kaikki verrattavat kielet
käyttävät enemmän muistia kuin C. C++ ja Rust ovat yksittäisissä
suorituskykymittauksissa C:tä nopeampia. Yhdessä suorituskykymittauksista Go
käyttää noin puolet C:n käyttämästä muistista, mutta on samassa mittauksessa
kaksi kertaa hitaampi. Mielenkiintoisesti Rustilla, Adalla ja Go-kielellä
kirjoitetut ohjelmat ovat C:llä kirjoitettuihin ohjelmiin verrattuna hitaampia
ja vievät enemmän muistia, mutta eivät tarjoa merkittäviä säästöjä lähdekoodin
määrään. \hl{Ymmärrettävyyden ja ylläpidettävyyden suhteen voi olla eroja}

Benchmarks Game ei sisällä D:tä, eli Benchmarks Gamen tuloksia ei voi käyttää
D:n vertaamiseen muiden kielten kanssa.

\FloatBarrier

\null

\newpage

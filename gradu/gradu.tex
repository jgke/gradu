\documentclass[gradu]{tktltiki}
%\documentclass[gradu,draft]{tktltiki}

\usepackage{gradu}

\begin{document}

\title{C-ohjelmointikielen korvaajiksi tehdyt ohjelmointikielet \\
ja miksi C on niistä huolimatta käytössä}
\gdef\formtitle{C-ohjelmointikielen korvaajiksi tehdyt ohjelmointikielet
ja miksi C on niistä huolimatta käytössä}
\author{Jaakko Hannikainen}
\date{\today}
\level{Pro gradu -tutkielma}

\maketitle

\classification{\protect{\ \\
\  \textbf{Software and its engineering
$\rightarrow$ Software notations and tools
$\rightarrow$ General programming languages
$\rightarrow$ Language types
$\rightarrow$ Imperative languages} \\
\  Software and its engineering
$\rightarrow$ Software organization and properties
$\rightarrow$ Operating systems \\
\  Computer systems organization
$\rightarrow$ Embedded and cyber-physical systems
$\rightarrow$ Embedded systems
$\rightarrow$ Embedded software \\
}}

\keywords{C, ohjelmointikielet}

\begin{abstract}
    C on ollut vallitseva ohjelmointikieli järjestelmäohjelmoinnissa C:n
    alkuajoista lähtien. Useita ohjelmointikieliä on luotu historian saatossa,
    joiden oli tarkoitus syrjäyttää C, mutta C on vieläkin johtavana kielenä
    varsinkin sulautetuissa järjestelmissä ja UNIX-pohjaisten
    käyttöjärjestelmien vallitsevana ohjelmointikielenä. C on myös käytössä
    Windows-käyttöjärjestelmäperheen ydinkomponenttien toteutuksessa.

    Tässä opinnäytetyössä selvitetään, miksi vaihtoehdoista huolimatta C on vieläkin
    laajalti käytössä myös uusissa projekteissa ja minkälainen ohjelmointikieli
    voisi syrjäyttää C:n.

    Johtopäätöksinä saatiin, että muut kielet eivät ole parempia.

    Tähän tulee parempi kuvaus graduprosessin loppupuolella, kun on
    tiedossa, mitä tämä gradu oikeasti sisältää.
\end{abstract}

\newpage
\tocbeginshere

\mytableofcontents

\section{Johdanto}

C~\citep{C18} on nykypäivänä yksi eniten käytetyimmistä ohjelmointikielistä. C
on ollut vallitseva ohjelmointikieli järjestelmäohjelmoinnissa kielen
alkuajoista lähtien. Useita ohjelmointikieliä on luotu historian saatossa,
joiden oli tarkoitus syrjäyttää C, mutta C on vieläkin johtavana kielenä
varsinkin sulautetuissa järjestelmissä ja Unix-pohjaisten käyttöjärjestelmien
vallitsevana ohjelmointikielenä. C on myös käytössä
Windows-käyttöjärjestelmäperheen ydinkomponenttien toteutuksessa.

Tutkielmassa selvitetään C:n ominaisuuksia, joiden takia se on ollut
suosituimpien ohjelmointikielten joukossa vuosikymmeniä, kuten myös
ominaisuuksia, joita C:stä voisi kehittää. Vaihtoehtoisista kielistä
selvitetään, mitkä ominaisuudet ovat voineet estää kielen käytön C:n sijaan
uusissa ja olemassa olevissa projekteissa ja mitkä ominaisuudet ovat taas
olleet parannuksia C:n ominaisuuksiin verrattuna. Tutkielmassa suunnitellaan
näiden tulosten pohjalta uusi ohjelmointikieli, Purkka.

C:n vaihtoehdoiksi tutkitaan seuraavia tehokkaaseen ohjelmointiin tarkoitettuja
kieliä: Ada~\citep{ADA12}, C++~\citep{CPP17}, D~\citep{D}, Go~\citep{golang}
sekä Rust~\citep{rust}. Näistä kielistä tutkitaan, mikä tai mitkä ominaisuudet
ovat estäneet C:n korvaamisen ja mitkä ominaisuudet ovat olleet parannuksia
C:hen verrattuna. Vertailun tuloksia käytetään uuden ohjelmointikielen
suunnitteluun, jossa otetaan tavoitteeksi luoda C:tä parempi ohjelmointikieli
tutkielman määrittelyjen puitteissa.

Kaikkia verrattavia kieliä tutkitaan sekä analyyttisesti että
suorituskykymittausten muodossa. Suorituskykymittauksiin käytetään Benchmarks
Gamea~\citep{benchmarks}, johon on toteutettu lukuisia pieniä ohjelmia
suorituskyvyn mittaamiseen. D on ainoa tutkielman käsittelemä kieli, jota
Benchmarks Game ei sisällä. Benchmarks Gamen Purkka-ohjelmat on toteutettu
nopeimpien C-ohjelmien pohjalta, jotta ohjelmien arkkitehtuuriset valinnat
eivät vaikuta vertailuun. 

Tutkielman toisessa luvussa käsitellään C-ohjelmointikieltä. Kielestä
käsitellään perusteiden lisäksi kielen historiaa ja nykypäivää, tärkeimpiä
ominaisuuksia ja kehityskohteita. Historian käsitteleminen mahdollistaa
ymmärryksen siitä, mitkä C:n ominaisuudet ovat tärkeitä kielen nykykäytön
kannalta ja mitkä ovat jäänteitä historiallisista syistä. Tärkeimpien
ominaisuuksien ja kehitettävien ominaisuuksien tunnistamiseen käytetään
ohjelmointikielten tulevaisuutta käsittelevää artikkelia \emph{The Next 7000
Programming Languages}~\citep{next7000}, C-kielen suosiota käsittelevää
artikkelia \emph{Some Were Meant For C}~\citep{somemeantforc} sekä Dennis
Ritchien artikkelissa \emph{The Development of the C Language}~\citep{chistory}
esiin nostettuja C:n ominaisuuksia.

Kolmannessa luvussa määritetään toisen luvun esiin nostamien ominaisuuksien
pohjalta vertailukriteerit kielten vertaamiseen, käsitellään lyhyesti
tutkielmaan valittuja vertailtavia kieliä sekä käsitellään erilaisia yleisiä
ohjelmointikielten valintaan liittyviä tekijöitä. Edellä mainitut kielet
valitaan, sillä ne ovat suosittuja~\citep{tiobe} sekä kunkin kielen historiassa
on ollut tavoitteena korvata C:n käyttö. Ohjelmistojen toteutuskielen
valintaprosessia käsitellään artikkelin \emph{Empirical Analysis of Programming
Language Adoption}~\citep{empiricalpopularity} avulla, jossa tutkitaan
kyselyillä erilaisia syitä ohjelmointikielen valintaan.

Neljännessä luvussa esitellään vertailtavat kielet ja käsitellään näiden
kielten ominaisuuksien tehokkuutta ja yhteensopivuutta C:n kanssa. Kaikki
vertailtavat kielet sisältävät ominaisuuksia, jotka haittaavat yhteensopivuutta
C:n kanssa. Tämän lisäksi suorituskykymittauksista ilmenee, kuinka kaikkien
kielten toteutukset käyttävät enemmän muistia Benchmarks Gamen vertailuissa.

Viidennessä luvussa määritellään uusi ohjelmointikieli, Purkka. Kielen
suunnittelussa painotetaan erityisesti C-yhteensopivuutta esimerkiksi C-kielen
esikäsittelijätuen kautta. Suunnittelussa pyritään parantamaan C:tä erityisesti
syntaksin osalta, mutta myös tarjoamalla vahvempaa tyypitystä. Purkka-kieli
käännetään C-kieleksi, jotta kielen yhteensopivuus olisi mahdollisimman hyvä
nykyisten ohjelmistojen yhteydessä.

Kuudennessa luvussa verrataan Purkka-kielellä toteutettuja Benchmarks Gamen
ohjelmia muihin kieliin. Suorituskykymittauksissa Purkka pysyy yhtä tehokkaana
kuin C, mutta Purkan lähdekooditiedostot ovat noin kuusi prosenttia pienempiä
verrattuna vastaaviin C-tiedostoihin. Kuudennessa luvussa myös arvioidaan
tutkielman oikeellisuutta ja pohditaan jatkotutkimuskohteita.

Seitsemännessä luvussa kerrataan tutkielman tulokset.

\newpage
\section{Taustaa}

\hl{Ohjelmointikielten piirteistä ja vertailukriteereistä}

\hl{Tässä luvussa kerrotaan ohjelmointikielten taustoista ja yleisistä
ominaisuuksista, luodaan määrittelyt vertailuun liittyville kriteereille,
perustelut verrattavien kielten valinnoille sekä pohditaan erilaisia
mahdollisia syitä kielten suosioon. Tämän luvun sisältö on hyvin samankaltainen
tutkimussuunnitelman kirjallisuusosan toisen luvun kanssa.}

\subsection{C-ohjelmointikielen taustaa}
\label{sec:ctaustaa}

\hl{Tässä aliluvussa kerrotaan C-kielen taustasta, alkuperäisistä käyttökohteista,
historiasta, suosion kehityksestä sekä nykytilanteesta. Erityisesti keskitytään
C:n tämänhetkisiin käyttökohteisiin, joista selviää, mitä ohjelmointikielten
ominaisuuksia tulisi käsitellä ohjelmointikielten vertailussa.}

C-kielen taustaa:
Milloin tehtiin, mihin tarkoitukseen? Kuka?

Kehitys:

Nykytilanne

Käyttökohteet

\subsection{Kehitettävissä olevat ominaisuudet C-ohjelmointikielessä}

\hl{Tässä aliluvussa pohditaan, miksi C pitäisi korvata uudella
ohjelmointikielellä, mitä ominaisuuksia C:stä voisi kehittää, mitkä C:n
ominaisuudet voisi jättää pois ja mitä uusia ominaisuuksia voisi tulla.}

\hl{Näitä ominaisuuksia ovat ainakin tarkempi käännösaikainen tyypitys etenkin
tyhjien osoittimien osalta, makrojärjestelmän uusiminen sekä syntaksin
selkeyttäminen.}

Tyyppisyntaksi (muuttujan nimi keskellä tyyppiä)

Tyyppien nurkkatapaukset:
\begin{itemize}
    \item Array onkin pointer jos se on parametrina (paitsi jos structin sisällä)
    \item char vs signed char vs unsigned char
    \item assignment: uint = int == int = uint == "ok"
\end{itemize}

Modernit tyypit:
\begin{itemize}
    \item Non-nullable pointer
    \item Tuple (shorthand structille)
    \item Array parametrina (shorthand structiin)
    \item Tagged union (shorthand struct, jossa union+enum)
\end{itemize}

Tyyppi-inferenssi

Makrot

Importit

Tooling

\subsection{Ohjelmointikielten vertailun kriteerit}
\label{sec:abs}

\hl{Jotta kielten vertailun voi tehdä objektiivisesti, valitaan vertailua
varten jotkin kriteerit. Kriteereiksi valitaan suorituskyky, muistinkäyttö ja
yhteensopivuus C:n ja muiden ohjelmointikielten kanssa. Vertailtavat kielet
eivät saa olla C:tä huonompia millään osa-alueella.}

Verrattavissa ohjelmointikielissä on pyritty parantamaan C:n huonoja puolia
hyvien puolien kustannuksella, usein lisäämällä kieleen turvallisuutta
parantavia ominaisuuksia tai tehden kielestä helppokäyttöisemmän esimerkiksi
automaattisella muistinhallinnalla. Tämä kuitenkin heikentää kielen tehokkuutta
tai alustariippumattomuutta tehden kielten suorasta vertailusta hankalaa.
Määrittelemällä absoluuttiset reunaehdot voidaan vertailla kieliä ehtojen
puitteissa objektiivisesti. Jos yksikin näistä kriteereistä ei pidä, verrattava
kieli ei ole aidosti C:tä parempi, vaan se häviää C:lle joissain osa-alueissa
ja vastaavasti voi olla parempi joissakin toisissa.

Tutkielmassa vertaillaan kolmea osa-aluetta ohjelmointikielistä: suorituskykyä,
muistinkäyttöä sekä yhteensopivuutta C:n ja muiden ohjelmointikielten kanssa.
Tutkielmassa käsitellään myös subjektiivisia kielten ominaisuuksia, kuten
kielen tiiviyttä\defword{terseness, expressiveness}, mutta näitä
ominaisuuksia ei huomioida kielten paremmuusvertailussa.

Ohjelmointikielellä kirjoitetun ohjelman tulee olla suoritusaikaisesti
vähintään yhtä nopea kuin vastaava C:llä kirjoitettu ohjelma. Kieli siis ei saa
vaatia ohjelmoijaa käyttämään mitään kielen ominaisuuksia, jotka voisivat
hidastaa ohjelmien suoritusta C:hen verrattuna. Monet suoritusaikaiset
turvallisuutta lisäävät ominaisuudet, kuten muistialueiden tarkistukset,
hidastavat kielen suoritusaikaista nopeutta.

Ohjelmointikielellä toteutettu ohjelma ei myöskään saa käyttää enempää muistia
niin suoritusaikaisesti kuin talletusvälineelläkään verrattuna vastaavaan
C-ohjelmaan. Tämä koskee myös vakiokirjastoa\defword{standard library} --
yksikin konkreettinen vakiokirjaston funktio linkitettynä ohjelmaan kasvattaa
ohjelman kokoa. Mikäli jokin vakiokirjasto toteutetaan, sen käyttäminen tulee
olla ohjelmoijalle täysin vapaaehtoista. Yksi tapa toteuttaa tämä on liittää
vain käytetyt funktiot osaksi ohjelmaa, jolloin käyttämättömät funktiot eivät
kasvata ohjelman kokoa.

Jos samaa vakiokirjastoa käytetään useassa ohjelmassa, tilankäytön kannalta on
edullisempaa säilöä vakiokirjasto jaettuna kirjastona, mutta tämä heikentää
kääntäjän mahdollisuuksia käännösaikaiseen optimointiin. Jos vakiokirjastoa ei
ole ladattu muistiin ohjelman käynnistyessä, ohjelman käynnistys voi kestää
hieman kauemmin. C:n vakiokirjasto liitetään usein moderneissa
käyttöjärjestelmissä ohjelmiin jaettuna kirjastona, sillä C:tä käytetään lähes
jokaisessa käyttöjärjestelmän ohjelmassa, ja näin jaetun kirjaston käyttäminen
välttää osan jaetun kirjaston huonoista puolista: kirjasto löytyy vain yhtenä
kopiona kiintolevyltä, ja se on valmiina ladattuna välimuistiin.

\newpage

Ohjelmointikielen tulee olla täysin yhteensopiva C:n suoritusympäristön kanssa.
Tämä koskee C-koodin kutsumista C:n vierasfunktiorajapinnan
läpi\defword{Foreign function interface, FFI, \emph{käännös
\citealt[s.~8]{vierasfunktiorajapinta}}} sekä kielen funktioiden kutsumista
muiden ohjelmointikielten C-rajapinnan läpi. Kielen tulee näiden lisäksi toimia
kaikissa ympäristöissä, joissa C toimii. Kielen olisi myös hyvä tukea C:n
esikäsittelijää, jotta C:n käyttäminen verrattavan ohjelmointikielen kanssa
olisi mahdollisimman saumatonta.

\subsection{C:hen verrattavissa olevat ohjelmointikielet}

\hl{Vertailtaviksi kieliksi valitaan Ada, C++, D, Go ja Rust. Kaikki
vertailtavat kielet ovat kohtalaisen tehokkaita, jonka lisäksi kaikilla on
yritetty korvata C:n tai C++:n käyttöä.}

Historian saatossa on tehty useita C:n kilpailijoita, jotka ovat yrittäneet
parantaa C:tä joidenkin C:n hyvien puolien kustannuksella. Muutamat näistä ovat
päätyneet hyvin suosituiksi ohjelmointikieliksi, kuten esimerkiksi C++ ja Go.
Kielten suosion mittaamiseen on tehty useita projekteja, jotka vertailevat
kieliä esimerkiksi hakutulosten tai projektien mukaan. Näitä ovat esimerkiksi
TIOBEn ohjelmointikielten suosion indeksi~\citep{tiobe} ja GitHub-palvelun
julkaisema \mbox{Octoverse}~\citep{octoverse}.
%Valitsemalla vertailuun suosittuja kieliä voidaan tutkia, miksi muuten
%selkeästi täysin käyttökelpoinen kieli ei ole syrjäyttänyt C:tä.

Koska tutkimuskysymyksessä vertaillaan ohjelmointikieliä suorituskyvyn ja
muistinkäytön suhteen, vertailuun kannattaa ottaa mukaan vain tehokkaita kieliä
-- korkeamman tason ohjelmointikielet on tarkoitettu ohjelmointinopeuden
parantamiseen ja turvallisempien ohjelmistojen toteuttamiseen nopeiden
ohjelmien sijaan. Tällöin kielen suorituskyky on heikompi. Yksi kattava
suorituskykyä mittaava vertailu on Benchmarks Game~\citep{benchmarks}, jossa
pyritään kirjoittamaan mahdollisimman nopea ohjelma pysyen silti kielelle
idiomaattisessa lähdekoodissa\footnote{Hyvin monessa kielessä voi kirjoittaa
C:hen verrattavissa olevaa matalan tason ohjelmointia, mutta Benchmarks Gamessa
on tarkoituksena välttää tätä.}.

TIOBEn listasta Ada, C++, D, Go ja Rust nousevat esiin verrattavina kielinä.
Kaikki viisi kieltä ovat tehokkaita. Tämän lisäksi jokaisen kielten historiassa
on ollut tavoitteena korvata C:n tai C++:n käyttö, kuten luvussa~\ref{sec:muut}
kerrotaan.

Viime vuosina on tehty myös useita C:hen käännettäviä ohjelmointikieliä, jotka
ovat jääneet pitkälti ilman mitään näkyvyyttä, kuten LISP/c~\citep{clisp1},
C-Mera~\citep{clisp2}, Carp~\citep{clisp3} ja Nymph~\citep{nymph}. LISP/c,
C-Mera ja Carp ovat LISP-perheeseen kuuluvia C:ksi kääntyviä ohjelmointikieliä,
jotka pyrkivät parantamaan C:n syntaksia korvaamalla sen LISP-perheen
syntaksilla. Nymph taas on olio-ohjelmointikieli. Erityisesti Carp on tämän
tutkielman kannalta kiintoisa ohjelmointikieli, sillä se on C:ksi kääntyvä
ohjelmointikieli, joka on suunniteltu mahdollisimman suorituskykyiseksi.

%\hl{Tässä voisi olla jotain pohdintaa, miksi nuo ovat jääneet huomiotta ja
%miten tämän voisi välttää. Erityisesti Carp tuntuu tätä tutkielmaa vastaavalta
%kieleltä, olisi ikävää jättää se täysin huomiotta. Tosin tällaiset pohdinnat
%jäänee puhtaasti spekulaatioksi, sillä noista ei ole erityisen paljoa dataa
%saatavilla.}

\subsection{Kielten suosioon vaikuttavat tekijät}

\hl{Analyysiä muista syistä kielen suosioon, erityisesti Meyerovichin ja
Rabkinin tekemän tutkimuksen perusteella. Tämä antaa tietoa ominaisuuksista,
jotka ohjelmointikielessä olisi hyvä olla.}

Eräässä tutkimuksessa~\citep{empiricalpopularity} tutkittiin syitä
ohjelmointikielten valintaan. Yhden tutkimuksen järjestämän kyselyn (s. 8,
\mbox{Slashdotissa} julkaistu kysely, n=1679) perusteella kielen valintaan
vaikuttaa avoimen lähdekoodin kirjastojen saatavuus, olemassa olevien ohjelmien
jatkokehitys sekä kielen tunnettavuus ohjelmoijien keskuudessa -- tutkimuksen
mukaan ohjelmoijat siis suosivat jo käytettyjä ohjelmointikieliä uusien kielten
sijaan. Saman kyselyn vastaajat arvioivat suorituskyvyn turvallisuutta
tärkeämmäksi.

Saman tutkimuksen järjestämässä Slashdot-sivustolla julkaistussa kyselyssä noin
40\% vastaajista arvioi tärkeäksi kriteeriksi työkalut. Kyselyn perusteella
kielen olisi siis hyvä tarjota toimivat työkalut, kuten
ohjelmistopakettien~\defword{software package, \emph{käännös
\citealt[s.~18]{ohjelmistopaketti}}} hakemiseen
paketinhallintajärjestelmän\defword{package manager}, nopean ja
käyttäjäystävällisen käännöstyökalun sekä valmiudet olemassa oleviin
kehitysympäristöihin integroitumiselle. Olemassa olevien C-ohjelmistojen
tukeminen on välttämätöntä mutta haastavaa johtuen C:n ekosysteemin
monimuotoisuudesta, erityisesti lukuisista kääntämistyökaluista.

Tutkimuksessa selvitettiin myös suosittuja ominaisuuksia ohjelmointikieliltä
(s. 13, SaaS MOOC -kurssin yhteydessä oleva kysely, n=415). Useita
tutkimuksessa selvitetyistä suosituimmista ominaisuuksista ei ole mahdollista
toteuttaa johtuen luvussa~\ref{sec:abs} määritetyistä rajoitteista, kuten
poikkeuksia ja rajapintoja. Useat muut tutkimuksessa esiin nousseet
ominaisuudet, kuten suorituskyky, ovat taas suoraan rajotteiden mukaisesti osa
verrattavan ohjelmointikielen tavoitteita.

Tutkimuksessa myös verrattiin tiettyjen toteamuksien keskinäistä korrelaatiota.
Kielen tiiviys korreloi eniten (korrelaatiokertoimella 0.76) kielestä pitämisen
kanssa (s. 13, The Hammer Principle -sivustolla julkaistu kysely).

\hl{? (alleviivattu toteamuksien)}

%Eräässä blogikirjoituksessa \citep{microsoftdictperf} vertaillaan C\#:n ja
%C++:n suorituskykyä. Kirjoituksessa C\#:llä kirjoitettu yksinkertainen toteutus
%suoriutuu tehtävästä nopeammin ja yksinkertaisemmin kuin optimoitu C++-ohjelma.
%Vasta usean C++-ohjelma päihitti C\#-toteuden nopeudella. C\#-toteutus
%kuitenkin vei noin neljä kertaa enemmän muistia kuin C++-toteutus.

Tutkimuksen perusteella valmiiksi suosittuja kieliä käytetään enemmän myös
uusissa projekteissa. Olemassa olevien kirjastojen tärkeyttä korostetaan
useassa kohdassa tutkimusta. Täysin C:n kanssa yhteensopiva kieli voi käyttää
C:lle tehtyjä kirjastoja, jolloin kielellä on käytettävissään laaja C:n
ekosysteemi\footnote{Esimerkiksi GitHubista hakusanalla 'library' löytyy yli
26\,000 C:llä kirjoitettua projektia.}.

Uusia ohjelmointikieliä opetellessa ohjelmoijat turvautuvat aikaisemmista
kielistä opittuihin käytäntöihin~\citep{languagelearning}. Uusien
ohjelmointikielten käyttöönottoa helpottaa siis muiden vastaavien kielten
osaaminen, sillä aikaisempi kokemus tukee uuden kielen opiskelua.
Suunnittelemalla C:n korvaajan C:n kanssa samankaltaiseksi kieleksi voidaan
pienentää uuden kielen opettelun kynnystä. Kielen eriävät ominaisuudet olisi
siis hyvä toteuttaa siten, että ne ovat mahdollisimman helppoja oppia C:stä
uuteen ohjelmointikieleen siirtyvälle ohjelmoijalle.

%Helppokäyttöisyys \\
%- turvallisuus (esim. tyypit) \\
%- esim. annotaatiot \\
%- selkeä mitä tekee

%\subsection{Mahdollisia kielen ominaisuuksia}
%
%Kielen ominaisuudet vaikuttavat sillä rakennettujen ohjelmistojen
%arkkitehtuuriin. Tyyppiluokat tai rajapinnat kannustavat vahvan
%tyypityksen kautta tyyppiturvalliseen ohjelmointiin, kun taas dynaamiset kielet
%(esimerkiksi LISP-perheen kielet) kannustavat nopeaan kehitystahtiin staattisen
%turvallisuuden hinnalla. Yksinkertaiset kielet, kuten C ja Go, tarjoavat usein
%vain yhden selkeän tavan toteuttaa yksittäiset funktiot, kun taas
%monimutkaisemmat kielet tarjoavat lukuisia vaihtoehtoja.\citationneeded
%
%BitC-ohjelmointikielen sähköpostilistalla käydyssä keskustelussa pohdittiin
%mahdollisia ongelmia vahvan tyypityksen (etenkin tyyppiluokkien) käytöstä
%matalan tason ohjelmoinnissa~\citep{bitc}. Shapiron sähköpostissa todetaan,
%että tyyppiluokkia ei voi toteuttaa ilman suoritusaikaista tukea.
%
%On huomioitava, että lukuisten olemassa olevien C:n kirjastojen, rajapintojen
%ja projektien vuoksi yhteensopivuus C:n kanssa tulee olla saumatonta
%mahdollisten vaihtoehtoisten ohjelmointikielten osalta, jotta kielen
%vaihtaminen olisi mahdollista. Tämä sisältää myös kirkastorutiinien kutsumisen
%muista ohjelmointikielistä, sillä C on lukuisissa järjestelmissä \emph{lingua
%franca}, jonka avulla ohjelmointikielet pystyvät kommunikoimaan keskenään.
%Esimerkiksi Python-ohjelmointikieli~\citep{python} sisältää tuen C-funktioiden
%kutsumiseen~\citep{pythonffi}, jota voi käyttää muiden ohjelmointikielten
%funktioiden kutsumiseen, mikäli käyttäjä kirjoittaa ''sillan'' C-ohjelmana.
%Käytännössä jokaisesta aktiivisesti käytetystä ohjelmointikielestä on
%mahdollista kutsua C-koodia.
%
%Ohjelmointikielten ekosysteemit koostuvat eri tahojen luomista kirjastoista. On
%tärkeää, että näihin kirjastoihin pääsee käsiksi mahdollisimman helposti.

% Kirjastojen lisenssien tulee myös olla yhteensopivia, jotta kirjastoja voi
% käyttää yhdessä toistensa kanssa. Eri ohjelmointikieliekosysteemeissä on
% käytössä erilaisia lisenssejä -- JavaScript-kirjastot ovat usein
% MIT-yhteensopivia, kun taas Java-kirjastot ovat usein Apache 2 -yhteensopivia.
% Liitteessä~\ref{app:github} on taulukko GitHub-verkkopalvelun sisältämien
% julkisten projektien määrä ryhmiteltynä lisenssin ja kielen mukaan.
%
% Kirjastot voi myös julkaista useammalla kuin yhdellä lisenssillä, jolloin
% käyttäjät voivat päättää, mitä lisenssiä haluaa seurata. On kuitenkin
% filosofinen kysymys, onko esimerkiksi Apache2+GPL parempi lisenssi kuin pelkkä
% GPL, sillä Apache2 ei vaadi tiettyjä oikeuksia loppukäyttäjille, kuten pääsyä
% ohjelmiston lähdekoodiin~\citep{apachetldr, gpl3tldr}. Apache2 siis antaa
% ohjelmoijille enemmän vapauksia GPL3:een verrattuna, mutta käyttäjien vapaus
% esimerkiksi muokata ohjelmistoa kärsii tästä.

\subsection{Makrojärjestelmät}

\hl{Tässä aliluvussa selitetään makrojärjestelmien käsite, kerrotaan
makrojärjestelmien tarpeellisuudesta ja käyttökohteista sekä verrataan
vaihtoehtoisia toteutuksia makrojärjestelmiin.}

Ohjelmointikielten makrojärjestelmillä tarkoitetaan ohjelmointikielen
ominaisuuksia, joita voidaan käyttää käännösaikaiseen lähdekoodin luomiseen ja
muuntamiseen. Lähdekoodin luonti käännösaikaisesti voi pienentää huomattavasti
tarvittavaa lähdekoodin määrää tietyissä tilanteissa aiheuttamatta
suoritusaikaisia haittoja. Ensimmäiset makrojärjestelmät rakennettiin
symbolisen konekielen käsittelyyn, jolloin makrojärjestelmiä käytettiin
poistamaan turhaa toistoa ohjelmien lähdekoodista.

Rust käyttää vakiokirjastonsa lähdekoodissa paljon makroja erityisesti
alkeistyyppien liittyvien määrittelyjen kohdalla. Rustin vakiokirjaston
kokonaislukutyyppejä määrittelevässä moduulissa \texttt{std::num} on käytetty
makroja huomattavan paljon lähdekoodin määrän vähentämiseen -- tiedoston
\texttt{mod.rs} lähdekoodin koko kasvaa noin 5\,000 rivistä yli 20\,000 riviin,
kun makrot laajennetaan. Kaikki makroilla määritellyt funktiot ovat identtisiä
tyyppejä tai tiettyjä vakioita lukuun ottamatta, ja makrojärjestelmän käytöllä
taataan paras mahdollinen suorituskyky. Makrojärjestelmän käyttäminen
määrittelyihin pakottaa yhtenäiset määrittelyt kaikille alkeistyypeille,
jolloin ohjelmointivirheiden määrä pysyy mahdollisimman pienenä.

\hl{? (alleviivattu lause funktioiden identtisyydestä)}

Lähdekoodin luominen käännösaikaisesti voi kasvattaa lopullisen ohjelman kokoa,
sillä makron sisältö kopioidaan makrokutsun paikalle. Toisaalta ohjelmakoodin
luonti voi myös mahdollistaa optimointeja, joita kääntäjä ei olisi pystynyt
tekemään tavalliselle funktiolle. Kääntäjästä riippuen lähdekoodin
sijoittaminen funktiokutsun paikalle voi esimerkiksi mahdollistaa paremman
silmukoiden vektorisoinnin tai kuolleen koodin poistamisen~\citep{cinlining}.
Ohjelmoijalle tulisi siis antaa mahdollisuus päättää, haluaako hän
ohjelmoidessaan käyttää ohjelmointikielen makrojärjestelmää vai perinteisiä
funktioita.

Makroja käyttäessä voi tapahtua ristiriitoja esimerkiksi luodessa väliaikaisia
muuttujia, joilla on sama tunniste kuin aikaisemmin lähdekoodissa
määritellyillä muuttujilla~\citep{macrohygiene}. Makrojen tulokset voivat myös
aiheuttaa syntaksivirheitä sekä muuta odottamatonta käytöstä. Jos
ohjelmointikielen makrojärjestelmä estää tällaiset ristiriidat, sitä kutsutaan
hygieeniseksi makrojärjestelmäksi.

\hl{? (alleviivattu syntaksivirheitä, odottamatonta käytöstä, estää
ristiriidat), pidempi selitys}

Makrojärjestelmiin on runsaasti vaihtoehtoisia toteutustapoja. Yksinkertaisin
vaihtoehto on jättää makrot kokonaan pois ohjelmointikielestä, mikä tosin
rajoittaa ohjelmointikielen mahdollisuuksia käännösaikaiseen
laajennettavuuteen.

Tekstialkioiden korvaamiseen perustuvat makrojärjestelmät ovat erittäin
yksinkertainen ja joustava tapa toteuttaa käännösaikainen makrojärjestelmä.
Tekstialkioiden korvaamisessa makroprosessori etsii syötteestä tiettyjä
tekstialkioita ja korvaa niitä joukoilla tekstialkioita. Tekstialkioiden
korvaaminen toimii yleensä erillisenä osana kääntämistä, eikä makroilla ole
tällöin tietoa varsinaisen ohjelmointikielen tyypeistä tai tunnisteista.

Mallipohjaisissa makrojärjestelmissä asetetaan tyyppejä tai arvoja ennalta
määritettyyn malliin, yleensä luokkiin tai funktioihin. Mallipohjaiset
makrojärjestelmät eivät voi tehdä kaikkea, mitä tekstialkioiden korvaamiseen
perustuvat makroprosessorit voivat tehdä. Mallipohjaiset järjestelmät voivat
käyttää hyväksi ohjelmointikielen tyyppijärjestelmää ja asettaa käännösaikaisia
rajoitteita mallin parametreille, mikä mahdollistaa paremmat käännösaikaiset
tarkistukset ohjelmien oikeellisuudesta. Monet mallipohjaisen ohjelmoinnin
käyttökohteista ovat suunnattu geneeristen funktioiden tai luokkien luomiseen
eikä syntaksin muuttamiseen.

Syntaksipuupohjaiset makrojärjestelmät käsittelevät ohjelmointikielten
syntaksipuita. Syntaksipuiden käsittely pelkkien tekstialkioiden sijaan
mahdollistaa uusien kontrollirakenteiden määrittämisen yksinkertaisesti ja
turvallisesti. Puhtaassa syntaksipuiden käsittelyssä sekä syötteen että makron
paluuarvon on oltava ohjelmointikielen syntaksin mukaisia, eli syntaksipuita
käsittelevä makro ei voi käsitellä mitä tahansa syötettä.

\hl{paluuarvon -> tuloksen ?}

Monikäyttöisimmät makrojärjestelmät ovat proseduraalisia makrojärjestelmiä.
Proseduraalisissa makrojärjestelmissä ohjelmoija voi käyttää samaa
ohjelmointikieltä sekä tavalliseen ohjelmointiin että makrojen toteutukseen.
Proseduraaliset makrojärjestelmät vaativat yleensä enemmän koodia makrojen
toteutukseen verrattuna muihin makrojärjestelmiin, mutta toisaalta
mahdollistavat kaiken, mitä ohjelmointikielellä voisi tehdä. Proseduraalinen
makrojärjestelmä voisi esimerkiksi lukea tiedostoja käännösaikaisesti ja
vaikuttaa lopulliseen lähdekoodiin tiedoston sisällöstä riippuen.

Mikään makrojärjestelmä ei ole täydellinen, vaan jokaisessa makrojärjestelmässä
on sekä hyviä että huonoja puolia. Kun makrojärjestelmän tehokkuutta
kasvatetaan, sen käyttäminen muuttuu monimutkaisemmaksi.

%\hl{Osa tästä seuraavaan lukuun?}
%
%C:n ja C++:n tekstin korvaamiseen perustuvassa makrojärjestelmissä korvataan
%tekstialkioita ja funktion kaltaisia makroja uusilla sekvensseillä alkioita. Ei
%rekursiota yleisessä tapauksessa, ei turing-täydellinen. Yksinkertainen
%yksinkertaisissa tapauksissa.
%
%C++:n ja D:n mallipohjaisissa järjestelmissä ... C++ erittäin hidas kääntää.
%Turing-täydellinen. D:n mallipohjainen ei turing-täydellinen.
%
%Rust syntaksipohjainen, ottaa tyypitettyjä ja pattern match. Oma DSL, joka
%eroaa jonkin verran varsinaisesta kielestä.
%
%D ja Rust myös proseduraalinen (D: string mixin + ctfe), you can do everything!
%
%C:n ja C++:n esikäsittelijä on tekstialkioiden korvaamiseen perustuva
%makroprosessori.
%
%C++:n ja D:n mallit mahdollistavat metaohjelmoinnin, tosin D vaatii kaikkien
%parametrien olevan tyyppejä.
%
%Rustin makrojärjestelmä perustuu syntaksipuiden käsittelyyn.
%Rustin makrojärjestelmä on käännösaikaisesti tyypitetty, ja sisältää tyypit
%esimerkiksi tunnisteille ja lausekkeille.
%
%Rust sisältää syntaksipuupohjaisen makrojärjestelmän lisäksi kokeellisen
%proseduraalisen makrojärjestelmän.
%

%\subsection{Vertailtavien kielten makrojärjestelmät}
%
%\hl{Tarkista kieli}

%Ada, D ja Go eivät sisällä mahdollisuutta käännösaikaisille makroille. C ja C++
%käyttävät samaa makrojärjestelmää, joka on kuitenkin hyvin
%rajoittunut~(\citeauthor{CPP17}, \citeyear{CPP17}, luku~19; \citeauthor{C18},
%\citeyear{C18}, luku~6.10).
%
%C:n ja C++:n makrot ajetaan ennen muuta koodin käsittelyä, eli makroja voi
%sijoittaa mihin tahansa koodia., kun taas C:n ja C++:n ei\footnote{C:n ja
%C++:n makrojärjestelmä ei ole edes primitiivirekursiivinen, sillä makrot eivät
%voi sisältää itseään.}.

\newpage
\section{Ohjelmointikielten vertailu}
\label{sec:muut}

\subsection{Yleisiä vertailtavien ohjelmointikielten ominaisuuksia}

C:hen vertailtavissa ohjelmointikielissä on yleisesti useita ominaisuuksia,
jotka vaikuttavat ohjelmien suoritusaikaiseen nopeuteen hidastavasti, lisäävät
muistinkäyttöä, vähentävät alustariippumattomuutta tai heikentävät
yhteensopivuutta C:n kanssa.

Yleisin näistä on automaattinen muistinhallinta, joka muistin vapauttamisen
automatisoimiseksi seuraa ohjelman käyttämää muistia. Lähes aina automaattinen
muistinhallinta lisää kieleen ''roskien keräämisen''\defword{garbage
collection, GC}, jonka ajaksi ohjelman suoritus pysäytetään. Lisäksi roskien
keräämiseen perustuva automaattinen muistinhallinta lisää muistinkäyttöä, sillä
ohjelmointikieli joutuu suoritusaikaisesti seuraamaan käytössä olevia
muistiosoitteita.

Monissa vertailtavissa kielissä on käytössä nimiruntelu\defword{name mangling},
joka mahdollistaa useat näennäisesti samannimiset funktiot. Tämä kuitenkin
aiheuttaa ohjelmointikielten välille yhteensopivuusongelmia, sillä toisesta
kielestä kutsuttaessa pitää tietää kutsuttavan funktion todellinen nimi.
Esimerkiksi \texttt{int}-tyyppiä palauttava funktion \texttt{foo()} nimeksi
voisi tulla \texttt{\_Z3foov}, kuten \texttt{g++}-kääntäjä tekee.

Ohjelmointikielen ominaisuudet vaikuttavat siihen, minkälaisia
ohjelmistoarkkitehtuureja kielellä tehdään~\citep{designpatternsdesign}.
Moderneissa ohjelmointikielissä virheiden käsittely on yleensä toteutettu
kahdella tavalla: toinen on poikkeavat paluuarvot ja toinen on poikkeuksien
heittäminen. Yleisesti ottaen kaikki ohjelmointikielet tukevat ensimmäistä ja
suurin osa toista tapaa. Poikkeusten käsittely on hieman hitaampaa ja aiheuttaa
hieman suuremman muistinkäytön, ja tehokkaaseen ohjelmakoodiin pyrkiessä
yleensä vältetään poikkeusten käyttämistä~\citep{exceptioncosts}. Monet
poikkeuksia tukevien ohjelmointikielten vakiokirjastot kuitenkin hallitsevat
virhetilanteita poikkeuksilla, mikä pakottaa ohjelmoijan käyttämään poikkeuksia
ohjelmoidessa.

\hl{tarkista kieli}

Ohjelmointikielten makrojärjestelmillä tarkoitetaan ohjelmointikielen
ominaisuuksia, joita voidaan käyttää käännösaikaiseen koodin luomiseen ja
muuntamiseen. Koodin luonti käännösaikaisesti voi pienentää huomattavasti
tarvittavaa koodin määrää tietyissä tilanteissa. Koodia luodessa voi kuitenkin 
tapahtua konflikteja esimerkiksi luodessa väliaikaisia muuttujia, joilla on
sama tunniste kuin aikaisemmin koodissa määritellyillä
muuttujilla~\citep{macrohygiene}. Luotu koodi voi myös aiheuttaa
syntaksivirheitä sekä muuta odottamatonta käytöstä. Jos ohjelmointikielen
makrojärjestelmä estää tällaiset konfliktit, sitä kutsutaan hygieeniseksi
makrojärjestelmäksi.

%\subsection{Vertailtavien kielten makrojärjestelmät}
%
%\hl{Tarkista kieli}

%Ada, D ja Go eivät sisällä mahdollisuutta käännösaikaisille makroille. C ja C++
%käyttävät samaa makrojärjestelmää, joka on kuitenkin hyvin
%rajoittunut~(\citeauthor{CPP17}, \citeyear{CPP17}, luku~19; \citeauthor{C18},
%\citeyear{C18}, luku~6.10).
%
%C:n ja C++:n makrot ajetaan ennen muuta koodin käsittelyä, eli makroja voi
%sijoittaa mihin tahansa koodia., kun taas C:n ja C++:n ei\footnote{C:n ja
%C++:n makrojärjestelmä ei ole edes primitiivirekursiivinen, sillä makrot eivät
%voi sisältää itseään.}. 

\subsection{Ada}

Ada on Yhdysvaltain puolustusministeriön kehittämä ohjelmointikieli, joka
suunniteltiin korvaamaan kaikki muut puolustusministeriön käyttämät
ohjelmointikielet~\citep{adahistory}, muun muassa C:n. Ada on hyvin moneen
taipuva kieli, sillä se on suunniteltu hallitsemaan monia eri
käyttötarkoituksia matalan tason bittitason ohjelmoinnista korkean tason
arkkitehtuureihin.

Ohjelmoinnin helpottamiseksi Adassa on sekä poikkeukset että automaattinen
muistinhallinta. Nämä kuitenkin hidastavat kieltä hieman aikaisemmin todetuista
syistä. Lisäksi C-kielen kanssa yhteensopivuus on kielen taipuvuudesta johtuen
hankalaa -- jokainen kutsuttava C-funktio on yksitellen määritettävä
kutsukonvention\defword{calling convention} kanssa~\citep[s.~471]{ADA12}. Adan
alustariippumaton C-tuki on kuitenkin äärimmäisen kattava, paikoitellen C:n
omaa tukea kattavampi (C:n standardi ei kuvaile esimerkiksi kutsukonventioita,
vaan ne on jätetty kunkin kääntäjätoteutuksen päätettäväksi). Ada on myös
vertailuin ainoa kieli, joka voi kutsua muilla ohjelmointikielillä
kirjoitettuja kirjastorutiineja suoraan ilman C-rajapintojen käyttöä. Adassa on
C:n lisäksi tuki C++:lle\footnote{C++-tuki ei sisällä nimiruntelun tukemista,
vaan kutsuttavista funktioista pitää määrittää runnellut nimet.}, Fortranille
ja Cobolille~\citep[s.~585]{ADA12}. Adassa ei kuitenkaan ole makrojärjestelmää,
eikä Ada tue C:n makrojärjestelmää.

\subsection{C++}

C++ on Bjarne Stroustrupin 1980-luvusta eteenpäin kehittämä kieli, jonka
yhtenä tarkoituksena on yhdistää Simula-kielen ominaisuudet ohjelman
organisointiin yhteen C:n tehokkuuden ja joustavuuden
kanssa~\citep{cpphistory}. C++ on nykypäivänä suosittu tehokkuutensa ja
monipuolisuutensa takia monimutkaisissa ohjelmistoissa, kuten
palvelinohjelmistoissa, kuvankäsittelyohjelmistoissa sekä
peleissä~\citep{cppapps}.

C++ on kehitetty C:n pohjalta, ja siinä onkin erittäin hyvä C-tuki. Koska
C++\hyp{}funktiot nimirunnellaan eikä nimiruntelua ole määritelty tarkasti C++:n
standardissa, C++-koodia on hankalaa kutsua jopa samalla alustalla eri
C++-kääntäjien välillä. C-koodin otsikkotiedostoissa\defword{header file} on
usein alussa C++-koodia, joka laittaa nimiruntelun pois päältä. Näin
C++-ohjelmat voivat helposti kutsua C:llä kirjoitettujen kirjastojen funktioita
-- C++-ohjelmat voivat usein käyttää C-kielen otsikkotiedostoja ilman muita
muokkauksia.

C++:n standardikirjaston virheidenkäsittely on toteutettu poikkeuksilla, jotka
aiheuttavat pienen hidastuksen. C++:ssa on myös käytettävissä
viitemäärälaskettu\defword{reference counting} muistinhallinta (vakiokirjaston
\texttt{std::shared\_ptr}), jolla voidaan käyttää suoritusaikaisesti varattua
muistia ilman muistivuotoja. C++:n \texttt{std::shared\_ptr} ei käytä erillistä
roskien keräystä, vaan kun viimeinen viite olioon poistetaan, myös varattu
muisti vapautetaan. Tällöin ohjelman suorituksen aikana ei tule
roskienkeräystaukoja.

C++ tukee geneeristä ohjelmointia\defword{generic programming} luokkien
yhteydessä malliohjelmoinnilla\defword{template programming}. C++:n
toteutuksessa jokaisesta uniikista mallin tyyppiparametrikombinaatiosta luodaan
lopulliseen ohjelmaan kopio mallin funktioista, joka kasvattaa ohjelmien kokoa.
Tämä mahdollistaa jokaisen luokan ilmentymän\defword{instance} erillisen
optimoinnin, mutta yleisesti kasvattaa sekä kääntämisaikoja että ohjelmien
kokoa.

\hl{tarkista kieli}

C++ käyttää lähes samaa makrojärjestelmää kuin C. C++:n makrojärjestelmässä on
11 avainsanaa, joita ei voi määrittää uudelleen
esikäsittelijässä~\citep[luku~19.2]{CPP17}\footnote{ Avainsanat ovat
\texttt{and}, \texttt{and\_eq}, \texttt{bitand}, \texttt{bitor},
\texttt{compl}, \texttt{not}, \texttt{not\_eq}, \texttt{or}, \texttt{or\_eq},
\texttt{xor} sekä \texttt{xor\_eq}. }. Tämä tarkoittaa sitä, että C++:n
esikäsittelijä ei hyväksy joitakin C:n esikäsittelijän hyväksymiä makroja,
tosin tämä ei tapahdu käytännössä koskaan. Koska makrojärjestelmä on muuten
sama kuin C:n makrojärjestelmä, se on hyvin rajoittunut \citep[luku~19]{CPP17}.

\subsection{D}

D on 2000-luvun alussa Digital Mars -yrityksen julkaisema ohjelmointikieli,
jonka tarkoituksena on mahdollistaa tehokkaiden ohjelmien kirjoittaminen
helposti ja turvallisesti~\citep{dhistory}. D on suunniteltu syntaksiltaan ja
käytökseltään lähelle C:tä ja C++:aa. Vaikka D-kielessä on olemassa
automaattinen muistinhallinta, D:n \emph{BetterC}-tila tekee kielestä
''paremman C:n'' poistamalla suoritusaikaiset ominaisuudet, mukaan lukien
automaattisen muistinhallinnan~\citep{dbetterc}. Tällöin kielestä poistuu
useita ominaisuuksia, mutta esimerkiksi D:n käännösaikaista makrojärjestelmää
voi käyttää.

C-koodin kutsuminen on melko helppoa, mutta ei aivan saumatonta, sillä jokainen
kutsuttava funktio tulee määritellä erikseen -- D ei ymmärrä C:n
otsikkotiedostoja. Tämä kuitenkin onnistuu yhdellä rivillä jokaista C:n
funktiota kohden, sillä D:n tyyppijärjestelmä on hyvin lähellä C:tä. D:lle on
myös olemassa useita työkaluja otsikkotiedostojen automaattiseen muuntamiseen,
kuten \texttt{htod}-työkalu~\citep{htod}. Työkalut eivät kuitenkaan ole
täydellisiä, sillä useiden D:n ominaisuuksien semantiikka ei ole C:n kanssa
yhteensopiva. D ei sisällä tukea C:n makrojärjestelmälle, eikä D:ssä ole
omaa makrojärjestelmää.

\subsection{Go}

Go on Googlen 2000-luvun loppupuolella kehittämä ohjelmointikieli, jonka
tarkoituksena on yhdistää käännösaikaisesti tyypitetyn ohjelmointikielen
turvallisuus ja tehokkuus suoritusaikaisesti tyypitettyjen ohjelmointikielten
helppokäyttöisyyteen~\citep{gohistory}. Toisin kuin monissa moderneissa
C-perheen kielissä, Go-kielessä ei ole luokkia, vaan pelkkiä tietueita ja
rajapintoja. Go suunniteltiin erityisesti korvaamaan C++:n käyttö Googlella
johtuen C++:n pitkistä käännösajoista.

Go-kielessä ei ole muista vertailtavista kielistä poiketen tyyppiparametreja.
Tämä estää käännösaikaisesti tyyppitarkistetun geneerisen koodin
kirjoittamisen. Ohjelmat voivat kuitenkin suoritusaikaisesti reflektion kautta
tunnistaa muuttujien konkreettisen tyypin. Tämän mahdollistaminen kasvattaa
ohjelmien kokoa, sillä rajapinnan mukana on säilytettävä rajapinnan oikeaa
tyyppiä. Tämä kuitenkin yksinkertaistaa ohjelmien kirjoittamista, sillä
ohjelmoijan ei tarvitse miettiä kirjoittamishetkellä monimutkaisia
tyyppejä~\citep[esim.][kalvo 8]{gohistory}, kuitenkin mahdollistaen geneerisen
koodin kirjoittamisen.
%Ohjelmassa~\ref{fig:goreflection} käsitellään
%value-muuttujaa geneerisesti - jos muuttuja on tyyppiä \texttt{string},
%muuttuja palautetaan sellaisenaan. Jos muuttujassa toteuttaa
%\texttt{Stringer}-rajapinnan, eli siinä on \texttt{String()}-metodi, sitä
%kutsutaan ja palautetaan tulos. Muussa tapauksessa palautetaan tyhjä
%merkkijono.

Vaikka Go-kielessä ei itsessään ole makroja, se sisältää \texttt{go~generate}
-työkalun, jota voidaan käyttää koodin generointiin~\citep{gogenerate}.
\texttt{go~generate} mahdollistaa minkä tahansa komentorivikomennon ajamisen,
ja on enemmänkin standardoitu tapa ajaa tiettyjä komentorivikäskyjä osana
ohjelman kääntämistä kuin tyypillinen makrojärjestelmä.
\texttt{\mbox{go~generate}} ajetaan erillisenä komentona,
eikä esimerkiksi osana \texttt{\mbox{go~build}}-komentoa.

%\begin{listing}[ht!]
%    \inputminted{go}{goreflect.go}
%    \caption{Geneerinen funktio Go-kielessä. Jos \texttt{value}-muuttuja on
%    tyyppiä \texttt{string}, muuttuja palautetaan sellaisenaan. Jos muuttujassa
%    toteuttaa \texttt{Stringer}-rajapinnan, eli siinä on
%    \texttt{String()}-metodi, sitä kutsutaan ja palautetaan tulos. Muussa
%    tapauksessa palautetaan tyhjä merkkijono.}
%    \label{fig:goreflection}
%\end{listing}


Go-kielen virheidenkäsittely on toteutettu useissa kohdissa C:n tavoin;
funktioista palautetaan virheellisissä tilanteissa virheellinen arvo. Tämä
tosin tehdään usein palauttamalla erillinen \texttt{Error}-tyyppiä oleva arvo
-- Go mahdollistaa useamman kuin yhden paluuarvon. Go-kielessä on myös
poikkeukset, joita suositellaan käytettävän vain poikkeuksellisissa
tilanteissa~\citep{effectivego}.

C:n kutsuminen Go-kielestä ei ole aukotonta: koska Go on muistinkäytöltä
turvallinen kieli, erityisesti muistin jakaminen C:n ja Go-kielen välillä on
hankalaa. Lisäksi C:n funktio-osoittimia ei voi kutsua Go-kielen
puolelta~\citep{cgo}. Go mahdollistaa C-otsikkotiedostojen suoran käytön
lähdekoodista, mikä helpottaa C-koodin kutsumista.

\subsection{Rust}

Rust on Mozilla Foundationin kehittämä ohjelmointikieli, joka on suunniteltu
turvalliseksi, rinnakkaiseksi ja käytännölliseksi
järjestelmäohjelmointikieleksi~\citep{rustfaq}. Rustissa on monimutkainen
tyyppijärjestelmä, jolla ohjelmat voivat todistaa esimerkiksi turvallisen
rinnakkaisajon ilman, että ohjelmaan tulee suoritusaikaisia rajoitteita tai
hidastuksia. Rust alkoi Graydon Hoaren henkilökohtaisena sivuprojektina, mutta
on nyt käytössä osana Gecko-selainmoottorin kehitystä C++:n ja JavaScriptin
ohella.

Kuten Go-kielessä, Rustissa voi myös käyttää poikkeuksia. Rustin
virheidenhallinta on muutenkin lähellä Go-kielen virhehallintaa -- Rustin
ohjekirja opastaa käyttämään mieluummin paluuarvoja kuin
poikkeuksia~\citep{rusterrorhandling}.

Rustin monimutkainen tyyppijärjestelmä kannustaa kirjoittamaan turvallisia
ohjelmia, Tietorakenteiden mutatoiminen on tehty tietoisesti hankalaksi, sillä
monimutkaisissa ohjelmissa holtittomasti muuttuva tila on usean vian syynä, ja
muuttumaton tila tekee monisäikeistettyjen\defword{multithreaded} ohjelmien
toteutuksesta huomattavasti helpompaa~\citep[luku 4, kohta 17]{effectivejava}.
Kullakin muuttujalla voi olla joko rajaton määrä muuttumattomia
viitteitä\defword{immutable reference} tai yksi muuttuva viite\defword{mutable
reference}, mutta molempia ei voi käyttää yhtä aikaa.

Turvallisuudella on kuitenkin hintansa -- Rust-ohjelmat vievät enemmän tilaa
kuin vastaavat C-ohjelmat. Jos Rust-ohjelmista poistaa standardikirjaston ja
käyttää suoraan C:n standardikirjastoa, ohjelmasta saa miltei samankokoisen
kuin vastaavasta C-kielellä kirjoitetusta ohjelmasta~\citep{rustbinarysize}.
Samalla tosin suurin osa Rustin ominaisuuksista jää pois. Rustin turvallisuus
vaatii myös monimutkaisen tyyppijärjestelmän, joka on vaikeampi opetella kuin
yksinkertaisemman kielen tyyppijärjestelmä.

Rust ei pysty suoraan käsittelemään C:n otsikkotiedostoja, mutta D:n tavoin
Rustille on saatavilla työkaluja otsikkotiedostojen automaattiseen
muuntamiseen~\citep{rustbindgen}. Rust kuitenkin suosittelee jokaisen kirjaston
kohdalla käsin ympäröimään C-kirjaston funktiot, sillä C:n tyyppimäärittelyt
eivät tarjoa Rustin vaatimaa tarkkuutta funktioiden turvallisuudesta.

Vaikka Rust ei tue C:n makroja, siinä on useita erillisiä kattavia
makrojärjestelmiä: varsinaiset makrot~\citep{rustmacros}, joiden lisäksi
Rustista löytyy useita kokeellisia koodin generointiin tarkoitettuja
ominaisuuksia~\citep{rustprocmacros, rustplugins}. Rust taas käsittelee makrot
vasta ohjelman tokenisoinnin jälkeen, eli ohjelman täytyy olla kyseisen
ohjelmointikielen mukaista ennen kuin makrot suoritetaan -- makroilla ei voi
käsitellä mitä tahansa tekstiä. Rust vaatii myös erottimien (sulkeiden,
lainausmerkkien ja heittomerkkien) olevan kielen syntaksin mukaisesti. Lisäksi
Rust-kääntäjän tulee tunnistaa kielen tekstialkiot eli
lekseemit\defword{lexeme} ennen makrojen suorittamista, eli Rust ei mahdollista
uusien operaattoreiden määrittämistä makrojen avulla. Rustin makroprosessori on
Turing-täydellinen~\citep{rustmacros} sekä hygieeninen.

\newpage

\subsection{Yhteenveto}

Yksikään vertailtavista kielistä ei täytä kaikkia luvussa~\ref{sec:abs}
määriteltyjä rajoitteita. Yksikään kielistä ei täytä muistinkäytön rajoitteita,
jonka lisäksi saumaton yhteistyö C:n kanssa onnistuu vain C++:n kanssa.

C++, D ja Rust mahdollistavat yhteensopivuuden parantamiseksi nimiruntelun
poistamisen käytöstä, mutta tämä ei toimi esimerkiksi C++:n luokkien
yhteydessä. Ada ja Go mahdollistavat funktioiden nimien valitsemisen linkkeriä
varten, jolloin esimerkiksi \texttt{EsimerkkiFunktio}-nimistä funktiota voidaan
kutsua \texttt{esimFunk}-nimellä C-ohjelmasta. Kaikki vertailtavat kielet
tukevat vierasfunktiorajapintoja C:n mukaisesti, eli kielet voivat kutsua muita
kieliä C-rajapintojen läpi.

C++ ja Rust ovat hyvin lähellä C:tä käännettyjen ohjelmien koossa, mutta
häviävät C:lle laajojen vakiokirjastojen takia. Molemmat ovat myös hyvin
monimutkaisia kieliä. Vain C++ ja Go voivat sisällyttää ohjelmiin C:n
otsikkotiedostoja, kun taas Ada, D ja Rust vaativat jokaisen funktion
määrittämistä. Adalle, D:lle ja Rustille on tosin olemassa työkaluja, joilla
tämän määrittämisen voi automatisoida.

\begin{table}[ht!]
    \begin{adjustbox}{center}
    \begin{tabular}{@{}lllll@{}} \toprule
        Kieli & Nimiruntelu   & Muistinhallinta                                     & C:n VFR               & muiden kielten VFR \\ \midrule
        Ada   & täysin hallittavissa & automaattinen                                       & Työläs mutta kattava  & C, C++, Fortran, Cobol \\
        C++   & saa pois päältä      & manuaalinen                                         & Lähes saumaton        & C:n läpi \\
        D     & saa pois päältä      & molemmat                                            & Työläs mutta kattava  & C:n läpi \\
        Go    & saa pois päältä      & automaattinen                                       & Epätäydellinen        & C:n läpi \\
        Rust  & saa pois päältä & automaattinen\footnote{Borrow Checkerin hallitsemana, joka mahdollistaa automaattisen muistinhallinnan ilman sen aiheuttamaa hidastusta.} & Työläs mutta kattava  & C:n läpi \\ \bottomrule
    \end{tabular}
    \end{adjustbox}
    \caption{
        Kielten ominaisuuksien yhteenveto.
    }
    \label{table:properties}
\end{table}

Rust ja C++ ovat verrattavista kielistä ainoat, joilla on olemassa varsinainen
makrojärjestelmä. C++:n makrojärjestelmä on melkein täysin yhteensopiva C:n
makrojärjestelmän kanssa, kun taas Rustin makrojärjestelmä on huomattavasti
ilmaisukykyisempi.

\newpage

\begin{figure}[ht!]
    \begin{adjustbox}{center}
    \begin{minipage}{1.15\textwidth}
    \begin{minipage}{0.5\textwidth}
        \begin{tikzpicture}[gnuplot]
%% generated with GNUPLOT 5.2p2 (Lua 5.3; terminal rev. 99, script rev. 102)
%% la  1. joulukuuta 2018 09.58.32
\path (0.000,0.000) rectangle (8.000,6.000);
\gpcolor{color=gp lt color border}
\gpsetlinetype{gp lt border}
\gpsetdashtype{gp dt solid}
\gpsetlinewidth{1.00}
\draw[gp path] (1.136,0.616)--(1.316,0.616);
\draw[gp path] (7.447,0.616)--(7.267,0.616);
\node[gp node right] at (0.952,0.616) {$0$};
\draw[gp path] (1.136,1.462)--(1.316,1.462);
\draw[gp path] (7.447,1.462)--(7.267,1.462);
\node[gp node right] at (0.952,1.462) {$5$};
\draw[gp path] (1.136,2.308)--(1.316,2.308);
\draw[gp path] (7.447,2.308)--(7.267,2.308);
\node[gp node right] at (0.952,2.308) {$10$};
\draw[gp path] (1.136,3.154)--(1.316,3.154);
\draw[gp path] (7.447,3.154)--(7.267,3.154);
\node[gp node right] at (0.952,3.154) {$15$};
\draw[gp path] (1.136,3.999)--(1.316,3.999);
\draw[gp path] (7.447,3.999)--(7.267,3.999);
\node[gp node right] at (0.952,3.999) {$20$};
\draw[gp path] (1.136,4.845)--(1.316,4.845);
\draw[gp path] (7.447,4.845)--(7.267,4.845);
\node[gp node right] at (0.952,4.845) {$25$};
\draw[gp path] (1.136,5.691)--(1.316,5.691);
\draw[gp path] (7.447,5.691)--(7.267,5.691);
\node[gp node right] at (0.952,5.691) {$30$};
\draw[gp path] (1.997,0.616)--(1.997,0.796);
\draw[gp path] (1.997,5.691)--(1.997,5.511);
\node[gp node center] at (1.997,0.308) {C};
\draw[gp path] (3.144,0.616)--(3.144,0.796);
\draw[gp path] (3.144,5.691)--(3.144,5.511);
\node[gp node center] at (3.144,0.308) {C++};
\draw[gp path] (4.292,0.616)--(4.292,0.796);
\draw[gp path] (4.292,5.691)--(4.292,5.511);
\node[gp node center] at (4.292,0.308) {Rust};
\draw[gp path] (5.439,0.616)--(5.439,0.796);
\draw[gp path] (5.439,5.691)--(5.439,5.511);
\node[gp node center] at (5.439,0.308) {Ada};
\draw[gp path] (6.586,0.616)--(6.586,0.796);
\draw[gp path] (6.586,5.691)--(6.586,5.511);
\node[gp node center] at (6.586,0.308) {Go};
\draw[gp path] (1.136,5.691)--(1.136,0.616)--(7.447,0.616)--(7.447,5.691)--cycle;
\node[gp node center,rotate=-270] at (0.092,3.153) {sekuntia};
\gpfill{rgb color={1.000,1.000,1.000}} (1.710,0.893)--(2.284,0.893)--(2.284,1.442)--(1.710,1.442)--cycle;
\draw[gp path] (1.710,0.893)--(2.284,0.893)--(2.284,1.442)--(1.710,1.442)--cycle;
\draw[gp path] (1.710,0.932)--(2.284,0.932);
\draw[gp path] (1.997,0.846)--(1.997,0.893);
\draw[gp path] (1.997,1.442)--(1.997,2.201);
\draw[gp path] (1.817,2.201)--(2.177,2.201);
\draw[gp path] (1.817,0.846)--(2.177,0.846);
\gpfill{rgb color={1.000,1.000,1.000}} (2.857,0.926)--(3.431,0.926)--(3.431,1.247)--(2.857,1.247)--cycle;
\draw[gp path] (2.857,0.926)--(3.431,0.926)--(3.431,1.247)--(2.857,1.247)--cycle;
\draw[gp path] (2.857,1.031)--(3.431,1.031);
\draw[gp path] (3.144,0.841)--(3.144,0.926);
\draw[gp path] (2.964,1.247)--(3.324,1.247);
\draw[gp path] (2.964,0.841)--(3.324,0.841);
\gpfill{rgb color={1.000,1.000,1.000}} (4.005,0.910)--(4.579,0.910)--(4.579,1.628)--(4.005,1.628)--cycle;
\draw[gp path] (4.005,0.910)--(4.579,0.910)--(4.579,1.628)--(4.005,1.628)--cycle;
\draw[gp path] (4.005,0.989)--(4.579,0.989);
\draw[gp path] (4.292,0.863)--(4.292,0.910);
\draw[gp path] (4.292,1.628)--(4.292,2.286);
\draw[gp path] (4.112,2.286)--(4.472,2.286);
\draw[gp path] (4.112,0.863)--(4.472,0.863);
\gpfill{rgb color={1.000,1.000,1.000}} (5.152,1.284)--(5.726,1.284)--(5.726,2.182)--(5.152,2.182)--cycle;
\draw[gp path] (5.152,1.284)--(5.726,1.284)--(5.726,2.182)--(5.152,2.182)--cycle;
\draw[gp path] (5.152,1.652)--(5.726,1.652);
\draw[gp path] (5.439,0.912)--(5.439,1.284);
\draw[gp path] (5.439,2.182)--(5.439,2.661);
\draw[gp path] (5.259,2.661)--(5.619,2.661);
\draw[gp path] (5.259,0.912)--(5.619,0.912);
\gpfill{rgb color={1.000,1.000,1.000}} (6.300,1.289)--(6.874,1.289)--(6.874,4.169)--(6.300,4.169)--cycle;
\draw[gp path] (6.300,1.289)--(6.874,1.289)--(6.874,4.169)--(6.300,4.169)--cycle;
\draw[gp path] (6.300,2.143)--(6.874,2.143);
\draw[gp path] (6.587,0.961)--(6.587,1.289);
\draw[gp path] (6.587,4.169)--(6.587,5.520);
\draw[gp path] (6.407,5.520)--(6.767,5.520);
\draw[gp path] (6.407,0.961)--(6.767,0.961);
\draw[gp path] (1.136,5.691)--(1.136,0.616)--(7.447,0.616)--(7.447,5.691)--cycle;
%% coordinates of the plot area
\gpdefrectangularnode{gp plot 1}{\pgfpoint{1.136cm}{0.616cm}}{\pgfpoint{7.447cm}{5.691cm}}
\end{tikzpicture}
%% gnuplot variables

        \vspace*{-0.8cm}
    \end{minipage}
    \begin{minipage}{0.5\textwidth}
        \begin{tikzpicture}[gnuplot]
%% generated with GNUPLOT 5.2p2 (Lua 5.3; terminal rev. 99, script rev. 102)
%% ma 21. tammikuuta 2019 23.24.51
\path (0.000,0.000) rectangle (8.000,6.000);
\gpcolor{color=gp lt color border}
\gpsetlinetype{gp lt border}
\gpsetdashtype{gp dt solid}
\gpsetlinewidth{1.00}
\draw[gp path] (1.320,0.616)--(1.500,0.616);
\draw[gp path] (7.447,0.616)--(7.267,0.616);
\node[gp node right] at (1.136,0.616) {$0$};
\draw[gp path] (1.320,1.341)--(1.500,1.341);
\draw[gp path] (7.447,1.341)--(7.267,1.341);
\node[gp node right] at (1.136,1.341) {$50$};
\draw[gp path] (1.320,2.066)--(1.500,2.066);
\draw[gp path] (7.447,2.066)--(7.267,2.066);
\node[gp node right] at (1.136,2.066) {$100$};
\draw[gp path] (1.320,2.791)--(1.500,2.791);
\draw[gp path] (7.447,2.791)--(7.267,2.791);
\node[gp node right] at (1.136,2.791) {$150$};
\draw[gp path] (1.320,3.516)--(1.500,3.516);
\draw[gp path] (7.447,3.516)--(7.267,3.516);
\node[gp node right] at (1.136,3.516) {$200$};
\draw[gp path] (1.320,4.241)--(1.500,4.241);
\draw[gp path] (7.447,4.241)--(7.267,4.241);
\node[gp node right] at (1.136,4.241) {$250$};
\draw[gp path] (1.320,4.966)--(1.500,4.966);
\draw[gp path] (7.447,4.966)--(7.267,4.966);
\node[gp node right] at (1.136,4.966) {$300$};
\draw[gp path] (1.320,5.691)--(1.500,5.691);
\draw[gp path] (7.447,5.691)--(7.267,5.691);
\node[gp node right] at (1.136,5.691) {$350$};
\draw[gp path] (2.341,0.616)--(2.341,0.796);
\draw[gp path] (2.341,5.691)--(2.341,5.511);
\node[gp node center] at (2.341,0.308) {C++};
\draw[gp path] (3.703,0.616)--(3.703,0.796);
\draw[gp path] (3.703,5.691)--(3.703,5.511);
\node[gp node center] at (3.703,0.308) {Rust};
\draw[gp path] (5.064,0.616)--(5.064,0.796);
\draw[gp path] (5.064,5.691)--(5.064,5.511);
\node[gp node center] at (5.064,0.308) {Ada};
\draw[gp path] (6.426,0.616)--(6.426,0.796);
\draw[gp path] (6.426,5.691)--(6.426,5.511);
\node[gp node center] at (6.426,0.308) {Go};
\draw[gp path] (1.320,5.691)--(1.320,0.616)--(7.447,0.616)--(7.447,5.691)--cycle;
\gpsetdashtype{dash pattern=on 10.00*\gpdashlength off 10.00*\gpdashlength }
\draw[gp path](1.320,2.066)--(7.447,2.066);
\node[gp node center,rotate=-270] at (0.000,3.153) {\shortstack{Muistinkäyttö \\ C:hen verrattuna (\%)}};
\gpfill{rgb color={1.000,1.000,1.000}} (2.001,2.096)--(2.683,2.096)--(2.683,2.951)--(2.001,2.951)--cycle;
\gpsetdashtype{gp dt solid}
\draw[gp path] (2.001,2.096)--(2.683,2.096)--(2.683,2.951)--(2.001,2.951)--cycle;
\draw[gp path] (2.001,2.184)--(2.683,2.184);
\draw[gp path] (2.342,2.065)--(2.342,2.096);
\draw[gp path] (2.342,2.951)--(2.342,3.867);
\draw[gp path] (2.162,3.867)--(2.522,3.867);
\draw[gp path] (2.162,2.065)--(2.522,2.065);
\gpfill{rgb color={1.000,1.000,1.000}} (3.362,2.198)--(4.044,2.198)--(4.044,3.387)--(3.362,3.387)--cycle;
\draw[gp path] (3.362,2.198)--(4.044,2.198)--(4.044,3.387)--(3.362,3.387)--cycle;
\draw[gp path] (3.362,2.321)--(4.044,2.321);
\draw[gp path] (3.703,2.112)--(3.703,2.198);
\draw[gp path] (3.703,3.387)--(3.703,3.499);
\draw[gp path] (3.523,3.499)--(3.883,3.499);
\draw[gp path] (3.523,2.112)--(3.883,2.112);
\gpfill{rgb color={1.000,1.000,1.000}} (4.724,2.117)--(5.406,2.117)--(5.406,4.203)--(4.724,4.203)--cycle;
\draw[gp path] (4.724,2.117)--(5.406,2.117)--(5.406,4.203)--(4.724,4.203)--cycle;
\draw[gp path] (4.724,3.707)--(5.406,3.707);
\draw[gp path] (5.065,0.616)--(5.065,2.117);
\draw[gp path] (5.065,4.203)--(5.065,5.121);
\draw[gp path] (4.885,5.121)--(5.245,5.121);
\draw[gp path] (4.885,0.616)--(5.245,0.616);
\gpfill{rgb color={1.000,1.000,1.000}} (6.085,2.209)--(6.767,2.209)--(6.767,5.158)--(6.085,5.158)--cycle;
\draw[gp path] (6.085,2.209)--(6.767,2.209)--(6.767,5.158)--(6.085,5.158)--cycle;
\draw[gp path] (6.085,3.417)--(6.767,3.417);
\draw[gp path] (6.426,1.298)--(6.426,2.209);
\draw[gp path] (6.426,5.158)--(6.426,5.378);
\draw[gp path] (6.246,5.378)--(6.606,5.378);
\draw[gp path] (6.246,1.298)--(6.606,1.298);
\draw[gp path] (1.320,5.691)--(1.320,0.616)--(7.447,0.616)--(7.447,5.691)--cycle;
%% coordinates of the plot area
\gpdefrectangularnode{gp plot 1}{\pgfpoint{1.320cm}{0.616cm}}{\pgfpoint{7.447cm}{5.691cm}}
\end{tikzpicture}
%% gnuplot variables

        \vspace*{-0.9cm}
    \end{minipage}
    \end{minipage}
    \end{adjustbox}
    \begin{adjustbox}{center}
    \begin{minipage}{1.15\textwidth}\makebox[\textwidth][c]{%
    \begin{minipage}{0.5\textwidth}
        \begin{tikzpicture}[gnuplot]
%% generated with GNUPLOT 5.2p2 (Lua 5.3; terminal rev. 99, script rev. 102)
%% to 10. tammikuuta 2019 17.28.43
\path (0.000,0.000) rectangle (8.000,6.000);
\gpcolor{color=gp lt color border}
\gpsetlinetype{gp lt border}
\gpsetdashtype{gp dt solid}
\gpsetlinewidth{1.00}
\draw[gp path] (1.320,0.616)--(1.500,0.616);
\draw[gp path] (7.447,0.616)--(7.267,0.616);
\node[gp node right] at (1.136,0.616) {$0$};
\draw[gp path] (1.320,1.302)--(1.500,1.302);
\draw[gp path] (7.447,1.302)--(7.267,1.302);
\node[gp node right] at (1.136,1.302) {$50$};
\draw[gp path] (1.320,1.988)--(1.500,1.988);
\draw[gp path] (7.447,1.988)--(7.267,1.988);
\node[gp node right] at (1.136,1.988) {$100$};
\draw[gp path] (1.320,2.673)--(1.500,2.673);
\draw[gp path] (7.447,2.673)--(7.267,2.673);
\node[gp node right] at (1.136,2.673) {$150$};
\draw[gp path] (1.320,3.359)--(1.500,3.359);
\draw[gp path] (7.447,3.359)--(7.267,3.359);
\node[gp node right] at (1.136,3.359) {$200$};
\draw[gp path] (1.320,4.045)--(1.500,4.045);
\draw[gp path] (7.447,4.045)--(7.267,4.045);
\node[gp node right] at (1.136,4.045) {$250$};
\draw[gp path] (1.320,4.731)--(1.500,4.731);
\draw[gp path] (7.447,4.731)--(7.267,4.731);
\node[gp node right] at (1.136,4.731) {$300$};
\draw[gp path] (1.320,5.417)--(1.500,5.417);
\draw[gp path] (7.447,5.417)--(7.267,5.417);
\node[gp node right] at (1.136,5.417) {$350$};
\draw[gp path] (2.341,0.616)--(2.341,0.796);
\draw[gp path] (2.341,5.691)--(2.341,5.511);
\node[gp node center] at (2.341,0.308) {C++};
\draw[gp path] (3.703,0.616)--(3.703,0.796);
\draw[gp path] (3.703,5.691)--(3.703,5.511);
\node[gp node center] at (3.703,0.308) {Rust};
\draw[gp path] (5.064,0.616)--(5.064,0.796);
\draw[gp path] (5.064,5.691)--(5.064,5.511);
\node[gp node center] at (5.064,0.308) {Ada};
\draw[gp path] (6.426,0.616)--(6.426,0.796);
\draw[gp path] (6.426,5.691)--(6.426,5.511);
\node[gp node center] at (6.426,0.308) {Go};
\draw[gp path] (1.320,5.691)--(1.320,0.616)--(7.447,0.616)--(7.447,5.691)--cycle;
\gpsetdashtype{dash pattern=on 10.00*\gpdashlength off 10.00*\gpdashlength }
\draw[gp path](1.320,1.988)--(7.447,1.988);
\node[gp node center,rotate=-270] at (0.000,3.153) {\shortstack{Lähdekoodissa tavuja \\ C:hen verrattuna (\%)}};
\gpfill{rgb color={1.000,1.000,1.000}} (2.001,1.778)--(2.683,1.778)--(2.683,2.235)--(2.001,2.235)--cycle;
\gpsetdashtype{gp dt solid}
\draw[gp path] (2.001,1.778)--(2.683,1.778)--(2.683,2.235)--(2.001,2.235)--cycle;
\draw[gp path] (2.001,1.953)--(2.683,1.953);
\draw[gp path] (2.342,1.423)--(2.342,1.778);
\draw[gp path] (2.342,2.235)--(2.342,2.307);
\draw[gp path] (2.162,2.307)--(2.522,2.307);
\draw[gp path] (2.162,1.423)--(2.522,1.423);
\gpfill{rgb color={1.000,1.000,1.000}} (3.362,1.684)--(4.044,1.684)--(4.044,3.542)--(3.362,3.542)--cycle;
\draw[gp path] (3.362,1.684)--(4.044,1.684)--(4.044,3.542)--(3.362,3.542)--cycle;
\draw[gp path] (3.362,2.858)--(4.044,2.858);
\draw[gp path] (3.703,1.478)--(3.703,1.684);
\draw[gp path] (3.703,3.542)--(3.703,4.922);
\draw[gp path] (3.523,4.922)--(3.883,4.922);
\draw[gp path] (3.523,1.478)--(3.883,1.478);
\gpfill{rgb color={1.000,1.000,1.000}} (4.724,2.508)--(5.406,2.508)--(5.406,5.086)--(4.724,5.086)--cycle;
\draw[gp path] (4.724,2.508)--(5.406,2.508)--(5.406,5.086)--(4.724,5.086)--cycle;
\draw[gp path] (4.724,2.891)--(5.406,2.891);
\draw[gp path] (5.065,2.184)--(5.065,2.508);
\draw[gp path] (5.065,5.086)--(5.065,5.691);
\draw[gp path] (4.885,2.184)--(5.245,2.184);
\gpfill{rgb color={1.000,1.000,1.000}} (6.085,1.627)--(6.767,1.627)--(6.767,2.604)--(6.085,2.604)--cycle;
\draw[gp path] (6.085,1.627)--(6.767,1.627)--(6.767,2.604)--(6.085,2.604)--cycle;
\draw[gp path] (6.085,1.959)--(6.767,1.959);
\draw[gp path] (6.426,1.514)--(6.426,1.627);
\draw[gp path] (6.426,2.604)--(6.426,3.035);
\draw[gp path] (6.246,3.035)--(6.606,3.035);
\draw[gp path] (6.246,1.514)--(6.606,1.514);
\draw[gp path] (1.320,5.691)--(1.320,0.616)--(7.447,0.616)--(7.447,5.691)--cycle;
%% coordinates of the plot area
\gpdefrectangularnode{gp plot 1}{\pgfpoint{1.320cm}{0.616cm}}{\pgfpoint{7.447cm}{5.691cm}}
\end{tikzpicture}
%% gnuplot variables

        \vspace*{-1cm}
    \end{minipage}}
    \end{minipage}
    \end{adjustbox}
    \caption{
        Benchmarks Gamen~\citep[tiedot haettu 1.1.2019]{benchmarks} tuloksiin
        perustuvat kuvaajat ohjelmointikielten suorituskyvystä, muistinkäytöstä
        ja ohjelmien koosta verrattuna C:llä kirjoitettujen ohjelmien
        tuloksiin.}
    \label{fig:benchmarksgame}
\end{figure}

\FloatBarrier

Benchmarks Gamen tulokset myös heijastavat näitä tuloksia -- C++ ja Rust ovat
nopeudeltaan hyvin lähellä C:tä, kun taas Ada ja Go ovat huomattavasti
hitaampia. Kuvassa~\ref{fig:benchmarksgame} verrataan suorituskykyä,
muistinkäyttöä ja ohjelmien lähdekoodin pituutta C:llä kirjoitettuun ohjelmaan.
Lähdekoodimittauksessa mitataan ohjelman kokoa siten, että ohjelmasta
poistetaan kommentit sekä ylimääräiset välimerkit. Tämän jälkeen ohjelma
pakataan \texttt{gzip}-pakkausohjelmalla~\citep{howmeasured}. Kuvaajat
perustuvat nopeimman kielellä kirjoitetun ohjelman tuloksiin -- lähes kaikilla
kielillä on jokaisessa suorituskykymittauksissa useampi ohjelma.

C ja C++ ovat hyvin lähellä toisiaan ajonopeudessa. Rust on jonkin verran
hitaampi, ja Ada ja Go huomattavasti hitaampia. C++ käyttää jonkin verran
enemmän muistia kuin C, kun taas Rust, Ada ja Go käyttävät selkeästi enemmän
muistia. C++ ja Rust ovat yksittäisissä suorituskykymittauksissa C:tä
nopeampia. Yhdessä suorituskykymittauksista Go käyttää noin puolet C:n
käyttämästä muistista, mutta on samassa mittauksessa kaksi kertaa hitaampi.
Mielenkiintoisesti Rust, Ada ja Go ovat C:hen verrattuna hitaampia ja vievät
enemmän muistia, mutta eivät tarjoa merkittäviä säästöjä lähdekoodin määrään.

\newpage
\section{Uuden kielen mahdolliset ominaisuudet}

\subsection{C:n kehitettävissä olevat ominaisuudet}

C on lähes kaikissa moderneissa järjestelmissä käytetty ohjelmointikieli, jota
käytetään matalan tason ohjelmointiin. C mahdollistaa erityisesti nopeutta tai
pientä muistijalanjälkeä vaativien sovelluksien toteuttamisen. C on hyvin
lähellä konekieltä~--~koodista voi päätellä hyvin suurella tarkkuudella, miksi
konekielen käskyiksi se kääntyy. C on suosittu erityisesti käyttöjärjestelmien
ytimien toteutukseen sekä sulautettujen järjestelmien toteuttamiseen.

C:llä kirjoitetut ohjelmat ovat usein pidempiä kuin modernimmilla
ohjelmointikielillä kirjoitetut vastaavat ohjelmat~\citep{codelength,
qsmcodelength}. Koska C on kielenä hyvin yksinkertainen, siinä on hyvin vähän
rakenteita ohjelman rakenteen hallitsemiseen. Erityisesti C:llä kirjoitetuissa
ohjelmissa hyvin tehty virheidenhallinta vie paljon lähdekoodia, sillä
ohjelmoijan pitää erikseen tunnistaa virhetilanne, jonka jälkeen ohjelmoijan
tulee tilanteesta riippuen joko palauttaa jokin tietty arvo, vapauttaa muistia
tai tehdä jotain muuta.

Ohjelma~\ref{fig:cerrorhandling} esittelee C:n virheidenkäsittelyä. Ohjelmassa
tarkistetaan useita arvoja, jotka vaikuttavat funktion kontrollivuohon.
Ensimmäisessä osassa tarkistetaan ajurin tilaa
\texttt{dev\_priv}-muistiosoittimen läpi, ja palautetaan virhearvo
\texttt{-EINVAL} tai \texttt{-ENODEV} riippuen virheestä. Toisessa osassa
tarkistetaan \texttt{oa\_get\_render\_ctx\_id}-funktion paluuarvo, ja
palautetaan sen palauttama virhekoodi virhetilanteessa. Tämän jälkeen
tarkistetaan \texttt{get\_oa\_config}-funktion paluuarvo, ja virhetilanteessa
hypätään \texttt{err\_config}-osioon, jossa mahdollisesti kutsutaan
\texttt{oa\_put\_render\_ctx\_id}-funktiota, joka vapauttaa
\texttt{oa\_get\_render\_ctx\_id}-funktion varaaman semaforin.

Ohjelmassa on useita tyypillisiä piirteitä C:n virheidenkäsittelystä. Alussa
tarkistetaan ohjelman tilan invariantteja, palauttaen tiettyjä virhearvoja
riippuen arvoista. Myöhemmin käsitellään muiden funktioiden paluuarvoja ja
mahdollisesti vapautetaan ohjelman varaamia resursseja virhetilanteiden
yhteydessä.

C:n makrojärjestelmä on hyvin yksinkertainen. Kun esikääntäjä tunnistaa
lähdekoodissa makron tai makrokutsun, esikääntäjä korvaa sen makron
määritelmällä. C:n makrojärjestelmä on kuitenkin rajoittunut -- rekursiivisten
makrojen rakentaminen on mahdotonta, sillä itsensä sisältäviä makroja ei voi
muodostaa johtuen makrojärjestelmän rajoituksista~\citep[luku 6.10.3.4]{C18}.

\FloatBarrier

\begin{listing}[ht!]
    \inputminted{C}{c-error-handling.c}
    \caption{Linux-kernelin i915-näytönohjainajurin lähdekoodia
    typistettynä~\citep[\texttt{drivers/gpu/drm/i915/i915\_perf.c}, funktio
    \texttt{i915\_oa\_stream\_init}]{i915debugfs}.}
    \label{fig:cerrorhandling}
\end{listing}

\FloatBarrier

\FloatBarrier

\begin{listing}[ht!]
    \inputminted{C}{c-hygiene.c}
    \inputminted{text}{c-hygiene-output.txt}
    \caption{C:n ja C++:n makrot eivät ole hygieenisiä. DOUBLE-makro muuttuu
    käännösvaiheessa muotoon $1+1*2$, joka on laskujärjestyksen takia 3 eikä
    odotettu 4.}
    \label{fig:cmacro}
\end{listing}

\FloatBarrier

C:n ja C++:n makro eivät ole hygieenisiä, kun taas esimerkiksi Rustin makrot
ovat. Ohjelmassa~\ref{fig:cmacro} on esimerkki C:llä kirjoitetusta
epähygieenisestä makrosta. \texttt{DOUBLE}-makro muuttuu käännösvaiheessa
muotoon $1+1*2$, joka on laskujärjestyksen takia 3 eikä odotettu 4.
Lopulliseksi printf-kutsuksi muodostuu siis \texttt{printf("One plus one
doubled is \%d", 1+1*2)}. Ongelman voi kiertää tässä tapauksessa
määrittelemällä \texttt{DOUBLE}-makron \texttt{x*2} sijaan \texttt{((x)*2)},
jolloin sulut säilyttävät oikean laskujärjestyksen.

C:n makrot eivät myöskään pysty muokkaamaan lähdekoodin rakennetta, kun taas
esimerkiksi LISP-kieliperheessä lukuisat ominaisuudet on toteutettu puhtaasti
käännösaikaisilla makroilla~\citep{lispexpressions}. Myös Rustissa oleva
makrojärjestelmä mahdollistaa kielen syntaksipuun muokkaamiseen. Vahva
makrojärjestelmä mahdollistaa LISPin tapauksessa yksinkertaisen kielen, jota
voi muokata kirjastoilla ilman suoritusaikaisia haittoja.

C:n tyyppisyntaksi on uudempien ohjelmointikielten tyyppisyntakseihin
verrattuna hyvin epäselkeä. Erityisesti osoitintyypit ja funktio-osoitintyypit
ovat hyvin epäselkeitä.
%vaikka kielessä onkin sinänsä järkevä logiikka tyyppimäärittelyyn: jos
%tyyppimäärittely on esimerkiksi \texttt{int~(*fp)(int~arg,~char~arg2)},
%kirjoittamalla lähdekoodiin \texttt{(*fp)(int,~char)} saa arvon tyyppiä
%\texttt{int}
Monimutkaisissa tapauksissa on vaikeaa seurata muuttujien oikeaa tyyppiä, ja
uudemmissa ohjelmointikielissä onkin selkeämpiä tapoja merkitä tyyppejä.
Esimerkiksi C:n funktio-osoittimen \texttt{int~(*fp)(int~arg,~char~arg2)}
tyypin voi ilmaista TypeScriptin syntaksilla \texttt{fp:~(int,~char)~=>~int},
jossa tyypin voi lukea suoraviivaisesti vasemmalta oikealle.

%\begin{listing}[ht!]
%    \inputminted{Kotlin}{c-error-handling-maybe.kt}
%    \caption{Vastaava lähdekoodi kuin ohjelmassa~\ref{fig:cerrorhandling},
%    mutta Kotlinin~\citep{kotlin} kaltaisella syntaksilla. Muuttuja \texttt{it}
%    viittaa nimettömiin lambda-parametreihin. Tämän version lähdekoodissa on
%    kolme riviä vähemmän ja noin 10\% vähemmän merkkejä. Lambdapohjainen
%    lähestymistapa ei toimi hyvin matalalla tasolla, sillä sulkeumien
%    alustariippumaton toteutus on haastavaa ilman dynaamista muistinhallintaa.
%    Esimerkin voisi kuitenkin kääntää vastaamaan
%    ohjelman~\ref{fig:cerrorhandling} lähdekoodia.}
%    \vspace*{1cm}
%\end{listing}

\FloatBarrier

\begin{listing}[ht!]
    \inputminted{C}{c-overflow.c}
    \inputminted{text}{c-overflow-output.txt}

    \caption{Kokonaisluvun ylivuoto C-kielessä. C-kielen
    spesifikaatiossa kokonaislukujen ylivuoto on määrittelemätöntä toimintaa,
    ja GCC-kääntäjän eri optimointitasoilla ohjelma käyttäytyy eri tavoin.}
    \label{fig:coverflow}
\end{listing}

\FloatBarrier

C on suunniteltu olemaan mahdollisimman alustariippumaton, mutta jos tämä
aiheuttaa ristiriidan mahdollisimman nopean toteutuksen kanssa, suositaan
nopeutta -- useista C:n operaatioista tulee eri tuloksia riippuen
alustasta~\citep[liite J, luku J.3]{C18}. Tästä yleisin esimerkki on
kokonaislukujen ylivuoto, jonka tulos riippuu kääntäjäoptimoinnista, kuten
ohjelma~\ref{fig:coverflow} näyttää.

C:ssä useat operaatiot voivat aiheuttaa määrittelemätöntä
toimintaa\defword{undefined behavior}. Esimerkiksi ylivuodon käyttäytyminen
C:ssä riippuu kääntäjäoptimoinnin määrästä -- esimerkiksi GCC-kääntäjän version
7.3.0 optimointitasolla~0 lauseketta $x~+~1~<=~x$ ei optimoida pois.
Optimointitasolla~3 kääntäjä taas ''tietää'' että kokonaisluvut eivät ylivuoda,
sillä \mbox{C-kielen} spesifikaatiossa kokonaislukujen ylivuoto on
määrittelemätöntä toimintaa~\citep[liite J, luku J.2]{C18}, ja kääntäjä poistaa
lausekkeen. Luonnollisten lukujen yli- ja alivuoto taas on tarkasti määritelty
-- minkä tahansa laskutoimituksen yli- tai alivuodon tulos on aina $x$ modulo
$n$, missä $x$ on laskutoimituksen tulos ja $n$ on yhtä suurempi kuin suurin
laskutoimituksen tyypin mahdollinen arvo.~\citep[luku 6.2.5]{C18}.

\begin{listing}[ht!]
    \inputminted{C}{openssl_md5.c}

    \caption{OpenSSL-kirjaston~\citep{openssl} MD5-tiivisteen laskevan koodin
    R0-makro. Ylimmässä versiossa on alkuperäinen versio, keskimmäisessä on
    C-versio, josta on laajennettu \texttt{F}- ja \texttt{ROTATE}-makrot ja
    alimmassa versiossa on teoreettista \texttt{<<<}-operaattoria käyttävä
    laajennettu versio.}

    \label{fig:opensslmd5}
\end{listing}

C:stä puuttuu useita matalan tason operaatioita, joista yksi on erityisesti
kryptografiassa käytetty kiertobittisiirto\defword{circular bit shift}, joka
löytyy lähes kaikista moderneista prosessoreista suoraan konekäskynä.
C:n syntaksi ei mahdollista uusien operaattoreiden määrittämistä, vaan
ohjelmoijat joutuvat käyttämään tavallisia funktioita. Tämä pitää kielen
eksplisiittisenä, mutta kasvattaa koodin pituutta.
Ohjelmassa~\ref{fig:opensslmd5} näytetään, miten koodia voisi selkeyttää
käyttämällä käyttäjän määrittelemiä operaattoreita. Ohjelmassa esitetään useita
bittitason operaatioita osana MD5-tiivisteen laskemista. Esimerkissä on kolme
vaihtoehtoa, joista ensimmäisessä on siivottu alkuperäinen makron sisältö, kun
taas toisessa vaihtoehdossa makrojen sisältö on laajennettu loppuun.
Kolmannessa vaihtoehdossa käytetään teoreettista \texttt{<<<}-operaattoria,
joka tarkoittaa esimerkissä kiertobittisiirtoa vasemmalle, ja vastaa
alkuperäisen ohjelman ROTATE-makroa.

C:n standardikirjastossa on suuri määrä merkkijonojen hallintaan tarkoitettuja
funktioita. Nämä eivät kuitenkaan tue monitavuisia merkistöjä\defword{multibyte
character set}, kuten Unicodea~\citep{unicode11}\footnote{Esimekiksi
\texttt{strlen}-funktio palauttaa Unicode-merkkijonon pituuden sijaan tavujen
määrän ennen ensimmäistä tavua, jonka lukuarvo on 0.}, vaan toimivat pelkästään
yksittäisistä tavuista koostuvista merkkijonoilla. Standardikirjastosta löytyy
myös sekä monitavuisia merkkijonoja että monitavuisia merkkejä varten
funktioita, mutta niiden osalta C-standardi on tarkoituksenmukaisesti jätetty
avoimeksi. Joustavampi ratkaisu olisi irrottaa merkkijonojen käsittely kielestä
erilliseksi kirjastoksi, ja mahdollistaa merkkijonojen muodon määrittäminen
kirjastotasolla. Kieli voisi tarjota oletuksena jonkin
merkistökoodauksen\defword{encoding} merkkijonoliteraaleille, kuten vaikka
UTF-8:n~\citep[s. 36]{unicode11}, joka on laajalti käytössä moderneissa
järjestelmissä. UTF-8 on myös taaksepäin yhteensopiva 7-bittisen
ASCII-merkistön kanssa, joka on käytössä C:n kanssa käytännössä jokaisessa
järjestelmässä, jossa C toimii.

\FloatBarrier

\subsection{Mahdollisia uusia ominaisuuksia}

\subsubsection{Tyypit}

Suoritusaikaisen tyyppijärjestelmän tulee olla täysin yhteensopiva C:n kanssa. Kieli
ei siis voi tuottaa esimerkiksi automaattisesti vapautettua muistia, sillä sen
integroiminen olemassa olevaan C-ohjelmaan ei ole saumatonta. Myös rajapintojen
toteuttaminen on hankalaa.

Tyhjä osoitin\defword{null pointer} on erityinen osoitinmuuttuja, joka osoittaa
johonkin olemattomaan muistialueeseen. Tyhjien osoittimien osoittaman arvon
hakeminen tai muuttaminen johtaa C:ssä määrittelemättömään toimintaan, ja
useissa muissa ohjelmointikielissä heittää poikkeuksen. Tyhjiä osoittimia on
kuvailtu ''Miljardin dollarin virheeksi'' niiden keksijän
toimesta~\citep{billiondollars}. Useat modernit ohjelmointikielet ovatkin
poistaneet muuttujilta mahdollisuuden olla tyhjiä osoittimia, kuitenkin
mahdollistaen tyhjät osoittimet esimerkiksi yhdellä merkillä lisää
tyyppimäärittelyssä.

Jos uuteen kieleen otetaan tällainen oletus, C-funktiomäärittelyiden
osoitinmuuttujien tyypin tulee saumattoman yhteistyön takia olla helposti
muutettavissa näiden kahden tilan välillä, sillä C:n syntaksi ei sisällä
lupauksia mahdollisista tyhjistä osoittimista -- tai niiden puutteesta.
Ohjelmoijaa siis ei tule pakottaa tarkistamaan tyhjiä osoittimia jos hän kutsuu
jotain C-funktiota, mutta toisaalta ei myöskään varoittaa ''turhista''
tarkistuksista. Kotlin-ohjelmointikieli~\citep{kotlin} on toteutettu tällä
periaatteella. Kutsuessa Java-koodia Kotlinista mikä tahansa operaatio voi
palauttaa \texttt{null}in, joka voisi väärin käytettynä aiheuttaa poikkeuksia
Kotlinilla kirjoitetussa koodissa. Kotlin olettaa tyypin erityiseksi tyypiksi,
joka sallii tarkistukset tyhjiä osoittimia varten, mutta ei vaadi
niitä~\citep{kotlinnullability}.

%Esimerkiksi Rustin suoritusaikainen tyypitys on lähellä C:tä; Rustin
%\texttt{Option<T>} vastaa C:n osoitinmuuttujia\footnote{Vastaava tietotyyppi on
%Javan \texttt{Optional<T>} ja Haskellin \texttt{Maybe a}.}. \texttt{Option<T>}
%vaatii, että sen sisältö ei ole tyhjä osoitin\defword[~-- onko tähän parempia
%käännöksiä?]{NULL pointer}. Tällöin \texttt{Option}-muuttuja voi ilmaista tyhjää
%arvoa olemalla suoritusaikaisesti identtinen C:n NULL-arvon kanssa, ja on siten
%suoritusaikaisesti muistissa identtinen C-osoitinmuuttujien kanssa. Tällöin
%C-rajapintojen osoitinargumenteiksi voi antaa \texttt{Option<T>} -muuttujia.

Yksi hyvin suosituista ominaisuuksista moderneissa ohjelmointikielissä on
tyyppipäättely. Oikein käytettynä sillä voidaan poistaa lähdekoodista
pääteltävissä olevat tyypit, jolloin ohjelmoija voi keskittyä
ohjelmalogiikkaan. Tyyppipäättely onkin erityisen kätevä lambdafunktioiden
kirjoittamisessa, sillä ohjelmoijan ei tarvitse kirjoittaa täysin pääteltäviä
tyyppejä.

Summatyypit\defword{sum type \emph{tai} tagged union} ovat useasta modernista
ohjelmointikielestä löytyvä ominaisuus. Toisin kuin
tulotyypeillä\defword{product type}, joissa tyypin mahdolliset arvot ovat
tulotyypin sisältämien arvojen tulojoukko eli tyyppien
monikko\footnote{Tyypillisesti tulotyypit määritellään ohjelmointikielissä
tietueina\defword{struct}, joissa jokaista tyyppiä vastaa jokin tunniste.},
summatyyppi voi sisältää vain yksittäisen tyypin arvon kerrallaan. Tällöin
tyypin arvojoukko on siis sen sisältämien tyyppien arvojoukkojen summa. C:llä
summatyypin voi määritellä esimerkiksi yhdistämällä \texttt{enum}- ja
\texttt{union}-tyypin ohjelman~\ref{fig:sumtypec} esittelemällä tavalla.
Useassa modernissa ohjelmointikielessä kääntäjä voi toteuttaa tämän
tyyppiturvallisesti, jolloin summatyypin sisältämä arvo ei voi olla itsensä
kanssa ristiriidassa.

Ohjelmassa~\ref{fig:sumtype} määritetään Rustin syntaksilla summatyyppi, jossa
luetellaan värejä. Summatyyppi \texttt{Color} sisältää parametrittomat
vaihtoehdot \texttt{Red}, \texttt{Green} ja \texttt{Blue}. Näiden lisäksi
\texttt{Color} sisältää yhden parametrin ottavan
\texttt{Grayscale}-vaihtoehdon, kolmen parametrin \texttt{RGB}-vaihtoehdon, ja
neliparametrisen \texttt{RGBA}-vaihtoehdon. Kaikki \texttt{Color}issa
esiintyvät parametrit ovat yhden tavun kokoisia arvoja (\texttt{i8}).
Ohjelmassa~\ref{fig:sumtypec} määritellään vastaava tyyppi C:llä. C-versio on
huomattavasti monimutkaisempi. Tämän lisäksi C-versiossa on mahdollista, että
\texttt{type}-kenttä ei vastaa unionin todellista sisältöä, mikä voi johtaa
määrittelemättömään toimintaan lukiessa unionin arvoa~\citep[luku
6.7.2.1]{C18}. Tämä voisi tapahtua esimerkiksi tilanteessa, jossa
\texttt{type}-kenttä sisältää arvon \texttt{RGBA}, kun \texttt{d}-kentän arvona
on \texttt{empty}-tietue. Tällöin muiden \texttt{d}-kentän arvojen lukeminen on
määrittelemätöntä toimintaa.

\FloatBarrier

\begin{listing}[ht!]
    \inputminted{Rust}{sumtype.rs}
    \caption{Esimerkki Rustissa summatyypin määrittelystä.}
    \label{fig:sumtype}
\end{listing}

\FloatBarrier

\begin{listing}[ht!]
    \inputminted{C}{sumtype.c}
    \caption{Vastaavan tyypin määrittely C:llä. C:ssä on mahdollista, että
    \texttt{type}-kenttä ei vastaa unionin todellista sisältöä, jolloin
    ohjelman tila voi johtaa määrittelemättömään toimintaan.}
    \label{fig:sumtypec}
\end{listing}

\FloatBarrier


\subsubsection{Syntaksi}

Jotta kielestä toiseen vaihtaminen olisi kannattavaa, uuden kielen on oltava
tarpeeksi erilainen: samanlainen kieli ei tarjoa tarpeeksi parannusta
olemassa olevaan, että siirtyminen olisi kannattavaa. C:n syntaksia on hyvin
vuolasta verrattuna moderneihin ohjelmointikieliin, joten syntaksia
typistämällä voisi saada huomattavia parannuksia.

%\begin{listing}[ht!]
%    \inputminted{C}{squaresum.c}
%    \inputminted{Haskell}{squaresum.hs}
%    \caption{Project Eulerin ongelma nro.\ 6~\citep{euler}. Ylempi on
%    C-kieltä, kun taas alempi esimerkki on kirjoitettu Haskellilla.
%    Haskell-esimerkin koodi vie vain kaksi riviä, kun taas C-koodi vie
%    yhdeksän. Molemmat ohjelmat laskevat kaavan
%    $(\sum\limits_{i=1}^n i)^2 - \sum\limits_{i=1}^n i^2$ tuloksen.
%    }
%    \label{fig:strcmp}
%\end{listing}
%
%\FloatBarrier

Syntaksin tulisi kuitenkin olla mahdollisimman lähellä C:tä, jotta siirtyminen
kielestä toiseen olisi mahdollisimman helppoa. Tämä myös helpottaa kielen
toteuttamista ja käyttöä, sillä käännettäessä C:ksi koodin rakenne pysyy
lähestulkoon samanlaisena. C:n kanssa on kuitenkin hankalaa käsitellä
poikkeuksellisia arvoja, jotka ovat kuitenkin hyvin tavallisia C:ssä.
Virhetilanteiden käsittelyssä voisi siis olla parannettavaa.

\newpage

Yksi suosittu ominaisuus on hahmontunnistus\defword{pattern matching}, jolla
voidaan kasvattaa kielen ilmaisuvoimaa. Hahmontunnistus soveltuu erityisen
hyvin monimutkaiseen arvojen käsittelyyn. Esimerkiksi Rustissa
hahmontunnistusta voidaan käyttää erityisesti yhdistettynä summatyyppeihin,
kuten ohjelma~\ref{fig:guards} näyttää. Ohjelmassa Rivien 8--12
\texttt{match}-lauseke käsittelee kolme \texttt{OptionalInt}in mahdollista
tilaa: arvo on olemassa ja on suurempi kuin viisi, arvo on olemassa, ja arvoa
ei ole olemassa. Ohjelma tulostaa lauseen ''Got an int!''. Ohjelman tyyppi
\texttt{OptionalInt} on summatyyppi, jonka mahdolliset arvot ovat
\texttt{Missing} ja \texttt{Value}, joista jälkimmäinen sisältää muuttujan
tyyppiä \texttt{i32}.

\FloatBarrier

\begin{listing}[ht!]
    \inputminted{Rust}{guards.rs}
    \caption{Rust-kirjan esimerkki Rustin
    hahmontunnistuksesta~\citep{rustguards} hieman yksinkertaistettuna. Rivien
    8--12 \texttt{match}-lauseke käsittelee kolme \texttt{OptionalInt}in
    mahdollista tilaa: arvo on olemassa ja on suurempi kuin viisi, arvo on
    olemassa, ja arvoa ei ole olemassa. Ohjelma tulostaa lauseen ''Got an
    int!''. Ohjelman tyyppi \texttt{OptionalInt} on summatyyppi, jonka
    mahdolliset arvot ovat \texttt{Missing} ja \texttt{Value(i32)}.}
    \label{fig:guards}
\end{listing}

\FloatBarrier

\subsubsection{Makrot}

Koska kieli voi erota C:stä käännösaikaisesti, kielen makrojärjestelmä on yksi
harvoista osa-alueista, joita voi muokata melko vapaasti. Uuden
makrojärjestelmän tulisi olla sekä turvallinen että voimakas. Esimerkiksi
Rustin makrojärjestelmä on vahvasti tyypitetty sekä
Turing-täydellinen~\citep{rustmacros}. Tämä mahdollistaa myös monimutkaisten
makrojen kirjoittamisen ja käytön. Rustin makrosyntaksi on kuitenkin
ensisilmäyksellä vaikeaselkoista, kun taas C:n makrot ovat pitkälti
helppolukuisia.

Joidenkin C-ohjelmien otsikkotiedostoissa luodaan makroja, joita
otsikkotiedoston käyttäjät voivat käyttää. Kielen tulee mahdollistaa makrojen
tuottaminen, joita voi käyttää myös C-ohjelmista. Kehittyneempien
makrojärjestelmien kaikkia makroja ei kuitenkaan voi muuntaa C:n makroiksi,
sillä C:n makrot ovat hyvin rajoittuneita. Kielen tulisi kuitenkin pystyä
ymmärtämään C:n makroja.

\subsubsection{Suoritusaikaiset ominaisuudet}

Uuden ohjelmointikielen tulee olla mahdollisimman yhteensopiva C:n kanssa. Tämä
onnistuu parhaiten pitämällä suoritusaikaisten ominaisuuksien määrän mahdollisimman
pienenä. Muistinhallinnan tulee olla manuaalinen, jotta kutsuessa C-funktioita
kielet voivat saumattomasti ja tehokkaasti välittää toisilleen
osoitinmuuttujia.
%Moniajoa ei
%voi alustariippumattomuuden takia tukea suoraan standardikirjastossa, sillä
%kaikki alustat eivät tue rinnakkaisuutta. Standardi voi kuitenkin luoda
%rajapinnan, jonka moniajoa tukevat alustat voivat käyttää.

%C-kääntäjissä on aina alustariippuvaisia ominaisuuksia, ja yksi tärkeimmistä on
%funktioiden kutsuminen. GCC:n~\citep{gcc} kanssa yhteensopivat kääntäjät
%tukevat funktiomäärittelyn kohdalla \texttt{\_\_attribute\_\_()} -määrettä,
%jolla voi määrätä, mikä funktion kutsukonventio on.

C++:n yksi tärkeistä ominaisuuksista on nimiavaruudet, joita C:ssä ei ole. Jos
samassa ohjelmassa on kaksi samannimistä funktiota, ei ole määritelty, kumpaa
niistä kutsutaan. C++:ssa ongelma on osittain ratkaistu nimiruntelulla. Tämä on
kuitenkin aiheuttanut ristiriitoja sekä C-yhteensopivuuden kanssa että eri
C++-kääntäjien yhteensopivuuden kanssa. Nimiruntelu voi myös aiheuttaa pidempiä
nimiä funktioille, jolloin kirjastot voivat viedä useita tavuja enemmän
levytilaa jokaista kirjaston funktiota kohden. Tiedosto- ja projektikohtaisella
nimiruntelun määrittelyllä saisi tehtyä alustariippumattoman nimiruntelun,
jonka määrittäminen ei olisi kielen vastuulla.

Jotta kieli on mahdollisimman kevyt ja alustariippumaton, kielen
standardikirjaston tulee olla mahdollisimman kevyt. Kielen makrojärjestelmän
tulee olla tarpeeksi voimakas, jotta monimutkaiset ominaisuudet voidaan
toteuttaa puhtaasti kirjastotasolla.

%Go-kielen yksi innovatiivisimmista ominaisuuksista on \texttt{defer}-avainsana,
%jolla voidaan viivästyttää laskentaa funktiosta poistumiseen asti. Esimerkiksi
%\emph{A Tour of Go} -sivuston esimerkissä~\citep{gotourdefer} käytetään
%\texttt{defer}iä Hello World -ohjelman toteuttamiseen:
%
%\begin{listing}[ht!]
%    \inputminted{Go}{defer.go}
%    \inputminted{text}{defer-output.txt}
%    \caption{\emph{A Tour of Go} -sivuston esimerkki \texttt{defer} -avainsanan
%    käytöstä. \texttt{defer}-avainsana siirtää world-sanan tulostamisen
%    funktion loppuun, jolloin ohjelma tulostaa sanat ''hello'' ja ''world''.}
%\end{listing}
%
%\FloatBarrier
%
%Ilman automaattista muistinhallintaa kaikissa tilanteissa toimivaa
%\texttt{defer}iä ei voi toteuttaa, mutta yksinkertaisissa tilanteissa myös
%hyvin matalan tason kieli voisi toteuttaa \texttt{defer}-avainsanan.
%Avainsanalle tulisi käyttöä -- esimerkiksi Linux-ytimessä on yleinen
%suunnittelumalli vapauttaa kaikki funktion varaama muisti käänteisessä
%järjestyksessä virhetilanteissa. Vain virhetilanteissa ajettu \texttt{defer}
%mahdollistaisi huomattavaa lähdekoodin siivoamista -- ohjelmoijan ei
%tarvitsisi ajatella muistin vapauttamista, sillä ohjelmointikieli tekisi sen
%ohjelmoijan puolesta.

Tällaisesta ominaisuudesta esimerkki on sulkeumat, joita suositaan erityisesti
funktionaalisen ohjelmoinnin yhteydessä. GCC-kääntäjään on toteutettu sulkeumat
C-kielelle käyttämällä Thomas Breuelin esittelemää
trampoliinimenetelmää~\citep{gccnested, cppclosure}. C ei itsessään tue
sulkeumia, eikä C tarjoa mahdollisuuksia lambdafunktioiden kirjoittamiseen.

%Ohjelmointikieli voi toteuttaa olio-ohjelmoinnin rajapinnat käytännössä kahdella
%tavalla: Joko korvaamalla jokaisen rajapintaviittauksen numerolla jaettuun
%tauluun funktioita (vtable-rajapinnat), tai korvaamalla rajapintaviittaukset
%tietueella, joka sisältää rajapinnan toteutuksen funktioiden osoitteet
%(tietue-rajapinnat). Ensimmäistä käytetään eritoten C++:n rajapintojen
%toteuttamiseen, kun taas jälkimmäistä käytetään esimerkiksi Go-kielessä.
%Jälkimmäistä käytetään myös C:n kanssa; Linux-ytimen laiteajurit on toteutettu
%tietue-rajapintoina, joita ajurit voivat tarvittaessa periä. Molemmissa
%toteutustavoissa on omat hyvät jä huonot puolensa -- vtable-rajapinnat vievät
%vähemmän muistia, kun taas tietue-rajapinnat ovat yksinkertaisempia
%toteuttaa~\citationneeded.
%
%C tukee yksinkertaista poikkeusten
%käsittelyä \texttt{setjmp}- ja
%\texttt{longjmp}-funktioilla\footnote{\label{cspecnote}C90: Luku 4.6. Tämä tarvitsee viitteen
%oikeaan C:n speksin lukuun. Minulla ei ole pääsyä uusimpaan spesifikaatioon,
%eli en pääse (vielä) tarkistamaan, mistä kohtaa tämä löytyy C11:n
%standardista.}, mutta C:n kanssa käytetään lähes poikkeuksetta paluuarvoihin
%perustuvaa virheidenkäsittelyä\citationneeded.

\subsubsection{Kääntäminen ja työkalut}

Jotta kieltä voi käyttää nykyisissä ympäristöissä, sen tulee mahdollistaa C:ksi
kääntäminen. Tällöin ohjelmointikieli toimii kaikissa ympäristöissä, joihin on
olemassa C-standardin mukainen C-kääntäjä. Tämä helpottaa kielen käyttämistä
nykyisten työkalujen kanssa.
%-- esimerkiksi \texttt{GNU Make} -työkalulla~\citep{gnumake} uuden kielen voi
%integroida projektiin kahdella rivillä.

%\inputminted{make}{Makefile.kieli}

%Tämän jälkeen uuden kieli-tiedostoja vastaavat .c-tiedostot voi poistaa, ja
%\texttt{Make} osaa luoda ne .kieli-tiedostojen pohjalta.

Kääntäjän tulee pystyä tuottamaan tarvittaessa ylläpitokelpoisia .c-tiedostoja,
jotta kielestä voi myös tarvittaessa siirtyä pois helposti.  Kielen tulee myös
pystyä lukemaan C:n otsikkotiedostoja, jotta ohjelmoijilta ei mene turhaan
aikaa funktiomäärittelyiden kääntämiseen kieleltä toiselle. Vastaavasti kielen
tulee myös tarvittaessa tuottaa C-kielen otsikkotiedostoja -- näin voi helposti
jatkaa olemassa olevien C-kirjastojen kehittämistä uudella kielellä.

%\begin{listing}[ht!]
%    \inputminted[firstline=4]{C}{vectorization.c}
%
%    \caption{Vektorisointi nopeuttaa fn\_size -funktion noin kolme kertaa
%    nopeammaksi fn\_iter -versioon verrattuna.}
%\end{listing}

Modernissa ohjelmointikielessä on tärkeänä osana kehittäjän hyvinvointi, jota
voidaan ylläpitää hyvillä työkaluilla. Go-kieli kehitettiin yhdessä kattavan
työkalupaketin kanssa, jolloin työkalut pysyivät mukana kielen kehityksessä.
Go-kielen mukana tuleekin nykyään kääntäjän (\texttt{go build}) yhteydessä muun
muassa paketinhallinta \texttt{go get}, koodin automaattinen muotoilija
\texttt{go fmt} sekä ohjelmointivirheiden tarkastaja \texttt{go vet}. Uuden
kielen yhteydessä tulisi kirjoittaa helppokäyttöiset mutta joustavat työkalut.

%Kirjoittamalla kääntäjän ensin C:llä ja vasta myöhemmin funktio kerrallaan
%siirtymällä uuteen kieleen voidaan esitellä kielen mahdollisuutta toimia C:n
%korvaajana. Vaihtoehtoisesti kääntäjän voi kirjoittaa jollain toisella
%kielellä, jolloin voi todistaa uuden kielen C-yhteensopivuutta alkuperäisen
%kielen siltausominaisuuksilla.


% ---------------------

%Liitteissä~\ref{app:grammar-ml},~\ref{app:grammar-lisp}~ja~\ref{app:grammar-c}
%on konsepteja mahdollisista kielen syntakseista.
%Liitteessä~\ref{app:grammar-ml} on ML-perheen kaltainen syntaksi,
%liitteessä~\ref{app:grammar-lisp} LISP-perheen kaltainen syntaksi ja
%liitteessä~\ref{app:grammar-c} on C-perheen kaltainen syntaksi.
%
%ML-perheen syntaksia seuraava esimerkki on tutkielman kannalta kiintoisin,
%sillä se tarjoaa C:stä huomattavasti poikkeavan syntaksin, kuitenkin pitäen
%C:n ominaisuudet. On olemassa useita sekä LISP-perheen kieliä sekä C-perheen
%kieliä, jotka kääntyvät C:ksi, mutta kaikki ML-kielet vaativat automaattisen
%muistinhallinnan. Manuaalisella muistinhallinnalla varustettu ML-kieli on siis
%kiintoisa aihe tutkia, sillä siitä ei ole aikaisempaa tutkimusta.

\newpage
\section{Yhteenveto}

Tutkielmassa tutkittiin mahdollisuuksia parantaa C-ohjelmointikieltä. Tätä
tarkoitusta varten luotiin uusi ohjelmointikieli, Purkka. Tutkimuksessa
verrattiin useita erilaisia ohjelmointikieliä C:hen suorituskyvyn sekä
muistinkäytön perusteella, mutta yksikään verrattavista ohjelmointikielistä ei
ollut C:tä nopeampi kuin yksittäisissä mittauksissa. Purkka tarjoaa pienen
syntaktisen parannuksen C:hen ilman ajoaikaisia haittoja. C:n parantaminen on
siis mahdollista ainakin syntaksin osalta, mutta ainakaan Purkka tuskin korvaa
C:tä johtuen sosiaalisista tekijöistä kielten vertailussa.

\hld{Tässä aliluvussa kerrataan tutkimuksen tulokset, eli kerrotaan lyhyesti
uuden ohjelmointikielen ja muiden verrattavien kielten suhteesta C:hen, sekä
muut olennaiset tutkielman asiat.}

\newpage

%\bibliographystyle{apacite}
\bibliography{references}

\appendixbeginhere
%\appendixsection{Ensimmäisen liitteen otsikko}

\hilight{Liitteiden tyylittely: Oma sivunumerointi, jota ei näytetä tocissa?}

\appendixendhere

\end{document}

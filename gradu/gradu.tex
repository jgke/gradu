\documentclass[gradu]{tktltiki}

\usepackage{gradu}

\makeatletter
\AtBeginDocument{%
    \renewcommand{\bibname}{Lähteet}
    \renewcommand{\BRetrievedFrom}{Saatavilla\ }
    \renewcommand{\BEd}{painos\hbox{}}
    \renewcommand{\BOthers}[1]{ja muut\hbox{}}%
    \renewcommand{\BOthersPeriod}[1]{ja muut\hbox{}}%
    \renewcommand*{\bibfont}{\interlinepenalty 10000\relax}
}
\makeatother

\begin{document}

\title{C-ohjelmointikielen korvaaminen muilla ohjelmointikielillä}
\author{Jaakko Hannikainen}
\date{\today}
\level{Pro gradu -tutkielma}

\hypersetup{pageanchor=false}
\maketitle
\hypersetup{pageanchor=false}

\classification{\protect{\ \\
\  \textbf{Software and its engineering
$\rightarrow$ Software notations and tools
$\rightarrow$ General programming languages
$\rightarrow$ Language types
$\rightarrow$ Imperative languages} \\
\  Professional topics
$\rightarrow$ Management of computing and information systems
$\rightarrow$ Software management
$\rightarrow$ Software selection and adaptation \\
}}

\keywords{C, ohjelmointikielet}

\begin{abstract}
    Ohjelmointikielen valinta on tärkeä osa ohjelmistoprojektien suunnittelua.
    Vaikka ohjelmointikielet uudistuvat nopeaan tahtiin, nykypäivänä on yhä
    tavallista valita ohjelmiston toteutukseen C-ohjelmointikieli, joka on
    standardoitu yli 30 vuotta sitten. Tutkielmassa tutkitaan syitä, miksi C on
    nykypäivänä vieläkin laajassa käytössä uudempien ohjelmointikielten sijaan.

    Tutkielmassa C:hen verrattaviksi ohjelmointikieliksi valitaan Ada, C++, D,
    Go sekä Rust. Kaikki viisi kieltä ovat tehokkaita. Tämän lisäksi jokaisen
    kielten historiassa on ollut tavoitteena korvata C:n käyttö.
    Ohjelmointikieliä verrataan C:hen suorituskyvyn, muistinkäytön sekä
    C-yhteensopivuuden osalta. Tämän lisäksi tutkielmassa selvitetään
    tärkeimpiä C:n ominaisuuksia sekä C:n kehitettävissä olevia ominaisuuksia.
    Tuloksia käytetään uuden Purkka-ohjelmointikielen suunnitteluun.

    Muut ohjelmointikielet todetaan vertailussa C:tä hitaammiksi. Tämän lisäksi
    muiden ohjelmointikielten ominaisuudet, kuten automaattisen
    muistinhallinnan, todetaan aiheuttavan ongelmia C-yhteensopivuudelle.

    C:n tärkeimmiksi ominaisuuksiksi nousevat esiin yksinkertaisuus, tehokkuus
    sekä alustariippumattomuus. Nämä ominaisuudet otetaan huomioon
    Purkka-kielen suunnittelussa, jossa painotetaan näiden lisäksi
    yhteensopivuutta C-ohjelmointikielen kanssa.
    
    Purkka-kieli suunnitellaan C:n kaltaiseksi ohjelmointikieleksi, jossa on
    muutettu C:n syntaksia yksinkertaisemmaksi ja johdonmukaisemmaksi.
    Suorituskykymittauksissa todetaan, että Purkan muutokset C:hen eivät
    aiheuta suoritusaikaisia rasitteita. Koska Purkka-kieli käännetään C:ksi,
    se on mahdollisimman yhteensopiva nykyisten kääntäjien kanssa.
\end{abstract}

\newpage
\tocbeginshere
\addtocontents{toc}{\protect\thispagestyle{empty}}

\mytableofcontents

\section{Johdanto} 

C~\citep{C11} on ollut vallitseva ohjelmointikieli järjestelmäohjelmoinnissa
C:n alkuajoista lähtien. Useita ohjelmointikieliä on luotu historian saatossa,
joiden oli tarkoitus syrjäyttää C, mutta C on vieläkin johtavana kielenä
varsinkin sulautetuissa järjestelmissä ja UNIX-pohjaisten käyttöjärjestelmien
vallitsevana ohjelmointikielenä. C on myös käytössä
Windows-käyttöjärjestelmäperheen ydinkomponenttien toteutuksessa.
Opinnäytetyössä tutkitaan, miksi vaihtoehdoista huolimatta C on vieläkin
laajalti käytössä myös uusissa projekteissa ja minkälainen ohjelmointikieli
voisi syrjäyttää C:n.

C:n vaihtoehdoiksi tutkitaan seuraavia tehokkaaseen ohjelmointiin tarkoitettuja
kieliä: Ada~\citep{ADA12}, C++~\citep{CPP14}, D~\citep{D}, Go~\citep{golang}
sekä Rust~\citep{rust}. Näistä kielistä tutkitaan, mikä tai mitkä ominaisuudet
ovat estäneet C:n korvaamisen ja mitkä ominaisuudet ovat olleet parannuksia
C:hen verrattuna. Lisäksi tutkitaan muista suosituista ohjelmointikielistä
korkean tason ominaisuuksia, jotka ovat hyödyllisiä matalan tason
ohjelmoinnissa ja jotka voi toteuttaa korvaavan kielen määrittelyssä luotujen
rajoitusten puitteissa. \hl{rajoitteissa -> yhteydessä: minusta
''rajoitteissa'' on parempi, sillä kieli speksataan tietyillä rajoitteilla
(muistinkäyttö yms.) - tuo lause voisi tietysti reflektoida tätä
(''...toteuttaa korvaavan kielen määrittelyssä luotujen rajoitusten
puitteissa'' tjsp.) Muutin tämän nyt tuohon muotoon (ennen: ''...toteuttaa
korvaavan kielen rajoitteissa.'')}

Tutkittavana on myös, mitä optimointeja C:ssä ei voi tehdä helposti johtuen
kielen rajoitteista ja miten tämän voisi korjata. Näitä ominaisuuksia ovat
esimerkiksi sivuvaikutuksettoman ohjelmakoodin merkitseminen,
optimointivinkkien alustariippumaton ilmaiseminen (esimerkiksi Rustin
funktioiden annotaation) ja funktioiden yliajaminen\defword{function
overloading} riippuen parametrien tyypistä tai arvoista, mikäli ne voidaan
kääntöaikaisesti päätellä. Koska C:n spesifikaatio ei salli useita samannimisiä
funktioita, kääntäjä ei voi helposti optimoida tällaisia tapauksia.

Tämän tutkielmasuunnitelman toisessa luvussa määritellään, minkälaiset
ominaisuudet tarvitaan kieleltä, joka voisi korvata C:n kokonaan. Toisessa
luvussa esitetään myös tutkielman kannalta oleellinen teoria, eli
ohjelmointikielten nykypäivää sekä kerrotaan lyhyesti muista tutkittavista
kielistä. Kolmannessa luvussa kerrotaan, miksi muut tutkitut kielet eivät täytä
näitä ominaisuuksia, ja miten C:tä voisi parantaa. Neljännessä luvussa
kuvaillaan toisen luvun määrittämien ominaisuuksien täyttävä ohjelmointikieli.
\hl{Ja sitten myöhemmin: Viidennessä luvussa kuvaillaan tutkielman rakenne}

\hl{Miten käännökset tulisi ilmaista? Alaviitteeseen\defword{footnote} vai
sulkuihin (eng. \emph{parens})?}

\newpage
\section{C-ohjelmointikielen taustaa}

C on matalan tason ohjelmointikieli, jossa yhdistyy yksinkertaisuus, tehokkuus
ja alustariippumattomuus. C on yksi maailman käytetyimmistä
ohjelmointikielistä~\citep{tiobe} johtuen kielen ominaisuuksista sekä
historiallisista syistä. Tässä luvussa kerrotaan C-ohjelmointikielestä, sen
historiasta ja nykypäivästä sekä käsitellään mahdollisia muutoksia, joilla C:tä
voisi parantaa.

\subsection{C-ohjelmointikieli lyhyesti}
\label{sec:clyhyesti}


\hl{Lähteitä lisää.}

\begin{listing}[ht!]
    \inputminted{C}{koodi/hello.c}
    \caption{Yksinkertainen hello world -ohjelma toteutettuna C:llä.}
    \label{fig:helloc}
\end{listing}

C on Dennis Ritchien 1970-luvun taitteessa kehittämä yksinkertainen matalan
tason ohjelmointikieli~\citep{chistory}, jota käytetään nykypäivänä erityisesti
järjestelmäohjelmoinnissa sekä sulautetuissa järjestelmissä. C ei aseta
ohjelmoijalle rajoituksia muistinkäsittelyyn, mikä mahdollistaa tehokkaan mutta
turvattoman ohjelmoinnin. C on \citeauthor{tiobe}:n (\citeyear{tiobe}) mukaan
yksi tällä hetkellä käytetyimmistä ohjelmointikielistä. C ei ole vuoden 1989
ANSI-standardin jälkeen muuttunut mainittavasti~\citep{chistory, C18}, vaan
ohjelma~\ref{fig:helloc} näyttää samalta kuin vuonna 1988 julkaistussa
\emph{The C Programming Language} -kirjassa~\citep{krsecond}. Ohjelma tulostaa
näytölle merkkijonon \emph{Hello,~World}.

C kehitettiin B- ja BCPL-ohjelmointikielten pohjalta näiden osoittautuessa
kömpelöiksi vaihtaessa laitteistoarkkitehtuuria, kuten
luvussa~\ref{sec:ctaustaa} kerrotaan. Uudella ohjelmointikielellä tavoiteltiin
B:tä nopeampaa, mutta BCPL:ää yksinkertaisempaa kieltä Unix-käyttöjärjestelmän
kehitykseen~\citep{chistory}.

Toisin kuin monissa nykypäivänä suosituissa kielissä, C ei tarjoa automaattista
muistinhallintaa, vaan ohjelmoijan täytyy itse hallita sekä muistin varaaminen
että sen vapauttaminen. Tämä on pitänyt kielen toteutuksen hyvin tehokkaana ja
yksinkertaisena, mikä on mahdollistanut C:n leviämisen järjestelmästä toiseen.
Kääntöpuolena muistinhallinta jätetään ohjelmoijan vastuulle, monimutkaistaen
ohjelmien toteutusta.

C:n mahdollistama suorituskyky on johtanut kielen suosioon tehokkuutta
vaativissa sovelluksissa, kuten verkkopalvelimissa ja tietokannoissa. C:n
yksinkertainen lähestymistapa muistinkäsittelyyn on taas mahdollistanut tarkkaa
muistinhallintaa vaativien ohjelmistojen toteutuksen, kuten käyttöjärjestelmien
ytimien tai sulautettujen järjestelmien ohjelmien. Tämä yhdistelmä on johtanut
kielen laajaan käyttöön käyttötarkoituksesta riippumatta.

Koska C:llä on helppoa saada aikaan erilaisia muistinkäsittelyyn liittyviä
tietoturvaongelmia, lukuisia työkaluja on kehitetty havaitsemaan ja estämään
näitä~\citep[mm.][]{valgrind,asan}. Myös lukuisia ohjelmointikieliä on luotu
toteuttamaan ''turvallinen C'', usein lisäämällä jokaisen osoittimen käytön
yhteyteen tarkistuksen, osoittaako osoitin ohjelman omistamaan muistiin.

%\hl{Kieliperheistä liittyvään juttuun ei löydy yhtään järkevää lähdettä. Tästä bulkkipaperi-idea?}

Koska C on suunniteltu mahdollisimman yksinkertaiseksi kieleksi, siitä ei löydy
omaa moduulijärjestelmää. Mikäli ohjelmoija haluaa käyttää jonkun toisen
tiedostojen sisältämää funktiota, ohjelmaan pitää sisällyttää käytettyjen
funktioiden määrittelyt eli funktioprototyypit. Yleensä nämä löytyvät
eritysistä otsikkotiedostoista, joiden sisältö kopioidaan ohjelmaan käyttäen
C:n makrojärjestelmää.

C on vaikuttanut lukuisien kielten kehitykseen, ja useiden kielien sanotaankin
olevan osa C-kieliperhettä. C-kieliperheeseen kuuluviin kieliin liittyy usein
muun muassa imperatiivinen ohjelmointityyli, perinteinen merkintätapa (eli
infix) lausekkeiden muodostamiseen, muuttujien näkyvyysalueiden rajoittaminen
kaarisuluilla sekä staattinen tyypitys. Näitä piirteitä näkyy
ohjelmassa~\ref{fig:helloc}: kaarisuluilla rajoitettu \texttt{main}-funktio
kutsuu \texttt{printf}-funktiota, joka tulostaa käyttäjälle merkkijonon
\emph{Hello,~World}. Funktio ottaa parametriksi kokonaisluvun \texttt{argc} ja
osoittimen merkkijonolistaan \texttt{argv}.

Suurin osa nykypäivänä käytetyimmistä ohjelmointikielistä on osa
C-kieliperhettä, kuten TIOBEn indeksissä~(\citeyear{tiobe}) kärkiviisikosta
löytyvät C:n lisäksi Java, C++ ja C\#. Python on ainoa kärkiviisikossa oleva
ohjelmointikieli, joka ei suoraan ole osa C-kieliperhettä, mutta Pythonin
referenssi-implementaatio CPython on nimensä mukaisesti toteutettu
C:llä ja Pythonilla~\citep{cpython}.

\subsection{C-ohjelmointikielen historiaa ja nykypäivää}
\label{sec:ctaustaa}

Dennis~\citet{chistory} käsittelee C-kielen historiaa ja tekemiään
suunnittelupäätöksiä artikkelissaan \emph{The Development of the C Language}.
Artikkelin mukaan C kehitettiin pitkälti 70-luvulla B- ja
BCPL-ohjelmointikielien pohjalta. Ritchie tavoitteli uudella
ohjelmointikielellä B-ohjelmointikielen tehokkuuden parantamista tarkemmalla
tyypityksellä. B-kielen ainoa primitiivityyppi oli sana\defword{word\emph{,
B-kielen tyyppinä \texttt{cell}}}, sillä B oli suunniteltu ajettavaksi
tietokoneilla, joissa muistiosoittimet osoittivat aina yksittäisiin sanoihin.
Tämä kuitenkin käytännössä osoittautui haastavaksi esimerkiksi merkkijonojen
käsittelyyn, sillä jokaiseen sanaan mahtuu useita tavun kokoisia merkkejä. Tämä
tarkoittaa ohjelmoidessa sitä, että ohjelmoija joutuu purkamaan käsin
yksittäisen sanan merkeiksi, käsittelemään näitä yksittäin ja lopuksi
yhdistämään merkit uudelleen sanoiksi. Epäkäytännöllisyyden lisäksi tämä oli
tehotonta, kun B-kieli muutettiin tavupohjaiselle PDP-11 -tietokoneelle
säilyttäen kuitenkin sanapohjaiset muistiosoittimet -- ohjelmoijan oli pakko
käyttää sanarajoilla toimivia muistiosoittimia, vaikka tavupohjainen
muistiosoitin olisi tehokkaampi ratkaisu.

B-kielen taulukot poikkesivat huomattavasti nykyisistä C:n taulukoista. Jos
B\hyp{}koodissa luodaan kymmenen alkion taulukko \texttt{A}, niin ohjelmaa
ajettaessa varattaisiin kymmenen sanan kokoinen taulukko ja \texttt{A}:han
asetettaisiin tämän taulukon osoitin. C:ssä \texttt{A} muutettaisiin
osoittimeksi määrittelyhetken sijaan vasta, kun muuttujaa käytettäisiin
lausekkeessa.

Tämän lisäksi Ritchien mukaan B-ohjelmointikielestä puuttui kokonaan
liukulukujen käsittely. Vaikka PDP-11 ei tukenut liukuluvuilla laskentaa,
valmistaja oli luvannut tuen tälle. BCPL-ohjelmointikieleen lisättiin
liukulukulaskenta olettaen, että liukuluku mahtuisi yhteen sanaan, mikä ei
pitänyt paikkaansa 16-bittisellä PDP-11 -tietokoneella.

Artikkelin mukaan C-ohjelmointikielen kehitys alkoi vahvemmalla tyypityksellä:
C-kielen ensimmäiseen varsinaiseen esiasteeseen oli lisätty \texttt{char}- ja
\texttt{int}-tyypit sekä muistiosoittimet näihin tyyppeihin. Tässä vaiheessa
C-tyylisten taulukoiden sijaan taulukot toimivat kuin B-kielen taulukot. Kun
C:hen lisättiin tietuetyypit, tämä muistimalli ei toiminut enää, vaan
taulukoiden käsittelyä muutettiin vastaamaan nykystä C:n mallia. Nyt taulukot
muunnettiin muistiosoittimiksi vasta, kun taulukkoa käytettiin lausekkeissa, ja
taulukon määrittelyssä alkioiden ja osoittimen sijaan varattiin muistia vain
taulukon alkioille.

Tästä tavasta käsitellä taulukoita seurasi kuitenkin ominaisuus, joka on myös
nykypäivän C:ssä: jos funktio ottaa parametrikseen taulukon, kääntäjä muuttaa
parametrin tyypin muistiosoittimeksi, sillä funktiokutsut ovat
lausekkeita\footnote{Jos funktio ottaa esimerkiksi \texttt{char[2]} -tyyppisen
parametrin, funktio saakin parametrikseen muistiosoittimen taulukon sijaan.
Tässä tapauksessa funktio saa siis moderneilla tietokoneilla kahden tavun
sijaan kahdeksan tavun kokoisen parametrin. Tämän ominaisuuden voi kiertää
säilömällä taulukon tietueen sisään.}.

Tämän lisäksi C-kieleen lisättiin tyyppi funktio-osoittimille.
Määrittelysyntaksin logiikan perusteena toimi lausekkeiden syntaksi. Jos
jostain muuttujasta saa \texttt{int}-arvon, kun lausekkeessa lukee
\texttt{(*muuttuja)()}, niin \texttt{muuttuja} määritellään kirjoittamalla
\texttt{int~(*muuttuja)();}. Monimutkaisissa tapauksissa on kuitenkin hankala
erottaa tyyppejä toisistaan, kuten erot tyyppien ''osoitin taulukkoon
kokonaislukuja'' eli \texttt{int~(*muuttuja)[]} ja ''taulukko osoittimia
kokonaislukuihin'' eli \texttt{int~*muuttuja[]}.

B-yhteensopivuuden tavoittelu ohjasi kielen suunnittelua syntaksin osalta
mahdollisimman B:n kaltaiseksi. B-lause \texttt{if(a~\&~b)} vastaa C-lausetta
\texttt{if(a~\&\&~b)}, mutta B-kielessä \texttt{\&}:tä käytettiin myös
bittioperaatioihin loogisten operaatioiden lisäksi. Koska B-ohjelmien haluttiin
toimivan C-ohjelmien tavoin mahdollisimman pienillä muutoksilla, C-idiomissa
\texttt{(a\&mask)~==~b} lausekkeessa käytetyn \texttt{\&}-bittioperaattorin
ympärille joutuu lisäämään sulut. Tämä johtuu B-kielen \texttt{if(a == b \& c)}
-tyylisistä lausekkeista, joiden haluttiin toimivan ilman muutoksia C:ssä
kielen vaihtamisen helpottamiseksi.

Seuraavaksi alkoi C-kielen esikäsittelijän kehitys. Aluksi esikäsittelijässä
oli vain toiminnot tiedostojen sisällyttämiseen (\texttt{\#include}) ja
yksinkertaiseen korvaamiseen (\texttt{\#define}), mutta hyvin nopeasti
kieleen lisättiin funktiomakrot sekä \texttt{\#if}-lauseet. Aluksi
esikäsittelijää pidettiin vain vapaaehtoisena laajennoksena C:hen, mikä
selittää myös nykypäivänä esikäsittelijän huomattavat erot muuhun C-kieleen
verrattuna. Ensimmäisen C-standardin jälkeen esikäsittelijä on pysynyt lähes
koskemattomana. Ainoa lisätty ominaisuus oli C99-standardin mukana tulleet
funktiomakrot, jotka voivat ottaa mielivaltaisen määrän argumentteja.

Myöhemmin, kun C oli levinnyt usealle eri alustalle, alkoi olla selkeää, että C
tarvitsi standardin. Brian Kernighan, jonka kanssa Ritchie oli kirjoittanut
\emph{The C Programming Language} -kirjan~\citep{krfirst}, kirjoitti Ritchien
kanssa C:n ensimmäisen standardin ANSIn X3J11 -työryhmässä. Kuuden vuoden
jälkeen työryhmä sai valmiiksi nk. C89 -standardin, joka tunnetaan myös ANSI
C:nä\footnote{ISO-järjestö hyväksyi standardin pienillä muutoksilla vuonna
1990, jonka vuoksi standardi tunnetaan myös C90-standardina.}~\citep{C89}.
Samoihin aikoihin valmistui myös toinen painos \emph{The C Programming
Language} -kirjasta, jossa korjattiin lukuisia eroja ensimmäisen version ja
C-standardin välillä~\citep{krsecond}.

Nykypäivänä C:tä käytetään käytännössä jokaisessa tietokoneessa
käyttöjärjestelmästä riippuen joko pelkästään ydinkomponenttien toteutukseen
tai koko käyttöjärjestelmän toteutukseen. Sulautetuissa järjestelmissä C on
yksi suosituimmista kielistä johtuen kielen yksinkertaisuudesta ja
suorituskyvystä. C:n suosion myötä myös kielen huonot puolet nousevat esiin
erilaisissa tietoturvaongelmissa, jotka johtuvat kielen sallimasta
rajoittamattomasta muistinkäsittelystä yhdistettynä ohjelmoijan tekemiin
virheisiin. Esimerkiksi puskuriylivuodoissa C:llä kirjoitettu ohjelma tallentaa
tietoa muualle tai lukee muistia muualta kuin mitä ohjelmoija on tarkoittanut,
kun ohjelmoinja jättää tekemättä kriittisen syötteen oikeellisuustarkistuksen.
C-kääntäjä voi optimoidessa poistaa tällaisia tarkistuksia ja aiheuttaa
tietoturvaongelmia, jos kääntäjä päättelee tarkistusten olevan
''turhia''~\citep{redhatsecurity}.

C on hyvin yksinkertainen kieli ja se on selviytynyt nykypäivään asti lähes
identtisenä ANSI C:hen. Uudemmat standardit~\citep{C99, C11, C18} ovat lähinnä
tehneet pieniä parannuksia kielen tehokkuuteen esimerkiksi lisäämällä
\texttt{restrict}-avainsanan. Erilaiset kääntäjät ovat kuitenkin tuottaneet
omia laajennoksiaan kieleen mahdollistaen tehokkaampien mutta
kääntäjäriippuvaisten C-ohjelmien kirjoituksen. Moderni esimerkki
kääntäjäriippuvaisesta syntaksia muokkaavasta laajennoksista on vektorityypit,
joita esimerkiksi GCC-kääntäjä tukee omalla \texttt{\_\_attribute\_\_(())}
-syntaksillaan. C:stä löytyy myös standardien mukainen tapa käyttää
kääntäjäriippuvaisia ominaisuuksia, \texttt{\#pragma}. Pragmoja käytetään
erityisesti OpenMP-kirjaston~\citep{openmp} yhteydessä.

Nykypäivänä käytännössä jokainen alusta tukee C:tä. C:tä käytetään alustoilla
muun muassa ohjelmointikielten väliseen kommunikaatioon -- jos C\#-ohjelma
haluaa käyttää Java-ohjelman kirjastorutiineja, C\#-ohjelman on helpointa
käyttää Java-ohjelman C-rajapintaa.

C on nykyään käytössä erityisesti matalan tason ohjelmoinnissa, kuten
käyttöjärjestelmien ytimissä, sulautetuissa järjestelmissä, UNIX-työkaluissa,
vapaan lähdekoodin ohjelmistoissa, tietokannoissa ja muissa tehokkuutta
vaativissa ohjelmistoissa.

\subsection{Tärkeimmät C-ohjelmointikielen ominaisuudet}
\label{sec:cominaisuudet}

Artikkelissa \emph{The Next 7000 Programming Languages}~\citep{next7000}
käsitellään ohjelmointikielten kehitystä ja pohditaan mahdollisia ominaisuuksia
tulevissa ohjelmointikielissä, joita nykyiset ohjelmointikielet eivät sisällä.
Artikkelissa selitetään C:n nykyistä suosiota kielen yksinkertaisuudella ja
tehokkuudella. Nämä ominaisuudet ovat mahdollistaneet C:n laajan käytön
käyttöjärjestelmistä työkaluihin huolimatta C:n turvattomuudesta.

Artikkelin mukaan C:n (ja C++:n) korvaaminen lyhyellä tähtäimellä on mahdotonta
johtuen kielen suosiosta. Koska lukuisat työkalut kääntäjistä
virheenjäljittäjiin\defword{debugger} on kirjoitettu yksinomaan C:tä varten,
vastaavien työkalujen luominen muita ohjelmointikieliä varten on huomattava
investointi. C:n yleisyydestä myös seuraa suuri määrä ohjelmoijia, kirjastoja
ja työkaluja, mikä tekee C:stä luonnollisen valinnan myös uusiin projekteihin.
Artikkelissa kuitenkin todetaan, että useat ohjelmointikielet ovat vähentäneet
C:n suosiota tarjoten yksittäisillä osa-alueilla parannuksia. Yksi mainituista
ohjelmointikielistä on Rust, joka mahdollistaa paremman ohjelmien
käännösaikaisen oikeellisuustarkistamisen heikentämättä tehokkuutta C:hen
verrattuna. Toiseksi vaihtoehdoksi tarjotaan ajoaikaisia tarkistuksia
turvallisuuden parantamiseksi.

Artikkelissa puhutaan myös mahdollisuudesta oikeelisuuden tasaiseen
parantamiseen\defword{gradual verification}, joka mahdollistaisi ohjelmien
ensimmäisten versioiden toteuttamisen ilman kattavaa käännösaikaista
oikeellisuustarkistusta, mutta antaen kehityksen jatkuessa työkalut ohjelmiston
oikeellisuuden varmistamiseen. Esimerkiksi
TypeScript~\citep{typescript}-ohjelmointikieli mahdollistaa tyypittämättömien
JavaScript-ohjelmien vaillinaisen tyypityken\defword{gradual typing}, jolloin
ohjelmien oikeellisuutta voi käännösaikaisesti tarkistaa funktio kerrallaan. 

Artikkelissa \emph{Some Were Meant For C} \citet{somemeantforc} myös nostaa
esiin tarpeen yksittäisten funktioiden kerrallaan muuntamisesta. Useissa
kielissä ei ole saumatonta C-yhteensopivuutta, jolloin kielestä toiseen
siirtyminen vaatii joko koko ohjelman uudelleenkirjoituksen tai erillisen
yhteensopivuuskerroksen alkuperäisen ja uuden kielen väliin. Koska lukuisat
työkalut ja kirjastot on toteutettu C:tä varten, artikkelin mukaan C tulee
pysymään jatkossakin tärkeänä osana tietokoneita.

Yhteensopivuuden sijaan \citeauthor{somemeantforc} kuitenkin painottaa C:n
alusta-agnostisuutta kielen tärkeimpänä ominaisuutena. Useilla alustoilla
voi normaalin välimuistin lisäksi käyttää laitteistoa tai tiedostojärjestelmää
suoraan muistiosoitteina, minkä C:n yksinkertainen lähestymistapa
muistinhallintaan mahdollistaa. Artikkelissa esitetään tästä esimerkkinä
iteraation ohjelman omien konekielisen käskyjen yli, mikä muissa kielissä olisi
kohtuuttoman hankalaa, mutta C:ssä triviaalia.

Molemmissa artikkeleissa käsitellään C:n määrittelemätöntä toimintaa kielen
sekä hyvänä että huonona puolena. Lukuisat tietoturvaongelmat ovat johtuneet
C:n turvattomuudesta, mutta toisaalta kielen turvattomuutta voi käyttää hyväksi
mahdollisimman tehokkaiden ohjelmien toteutuksessa.

\subsection{Kehitettävissä olevat ominaisuudet C-ohjelmointikielessä}
\label{sec:ckehitettavat}

\citeauthor{chistory} (\citeyear{chistory}) nostaa esiin kaksi usein
keskustelua herättänyttä C:n ominaisuutta. Toinen näistä on C:n tyyppisyntaksi,
ja toinen on C:n tapa käsitellä taulukoita ja osoittimia keskenään.
Ensimmäiselle näistä voi tehdä verrattaen helposti jotain, sillä syntaksin
muuntaminen käännösaikaisesti on triviaalia. Taulukoiden ja osoittimien välistä
käytöstä on paljon hankalampi muuntaa, sillä nykyiset C-ohjelmat käyttävät
osoittimia ja taulukoita sekaisin. Yksi C:stä puuttuva ominaisuus on
taulukon antaminen funktion parametrina, jonka voi kuitenkin tehdä käärimällä
taulukko tietueen sisään.

Muita syntaktisia parannuksia lausekkeisiin voi tehdä tietueiden kohdalla. Jos
lausekkeessa käyttää tietuetta, niin tietueen \texttt{foo} jäsenen \texttt{bar}
saa lausekkeella \texttt{foo.bar}, mutta jos \texttt{foo} onkin osoitin,
lausekkeen tulee olla \texttt{foo->bar}. Jos \texttt{foo} olisi osoitin
osoittimeen, lauseke olisi \texttt{(*foo)->bar}. Yksinkertaisempi syntaksi
olisi käyttää jokaisessa tapauksessa lauseketta \texttt{foo.bar}, ja jättää
tarvittava muoto kääntäjän pääteltäväksi. Esimerkiksi Rust-ohjelmointikielen
syntaksi tietueiden käsittelyyn on tällainen.

C:n \texttt{static}-avainsana on jaettu käytön mukaisesti kahteen avainsanaan.
Funktiomäärittelyissä ja globaaleissa muuttujissa C:n
\texttt{static}-avainsana vastaa muiden kielten avainsanaa yksityiselle
funktiolle. Globaaleja \texttt{static}-määreellä määritettyjä funktioita ja
muuttujia ei voi käyttää muusta kuin samasta tiedostosta, jossa kyseinen
symboli on määritelty.

Toinen \texttt{static}-määreen käyttö on funktioiden sisällä muuttujien
määrittelyyn. Staattiset muuttujat alustetaan vain kerran, vaikka funktiota
kutsuttaisiin useita kertoja. Ohjelmassa~\ref{fig:cstatic} käytetään tällaista
muuttujaa. Ohjelma tulostaa ensin numeron 0, jonka jälkeen ohjelma tulostaa
numeron 1.

\begin{listing}[ht!]
    \inputminted{C}{koodi/static.c}
    \caption{Staattinen muuttuja C:ssä}
    \label{fig:cstatic}
\end{listing}

C:n tyypitys sallii monia ilmaisuja, jotka voivat johtaa salakavaliin ongelmiin
suoritusaikaisesti. Koska C sallii suurempien kokonaislukutyyppien asettamisen
pienempiin kokonaislukutyyppeihin, nämä arvot voivat suoritusaikaisesti muuttua
ilman käännösaikaisia varoituksia. Mainittavasti lauseet \texttt{int a = 256;
unsigned char b = a;} eivät aiheuta käännösaikaisesti edes varoituksia
GCC-kääntäjällä, vaikka ''kaikki'' kääntäjän varoitukset olisivat päällä
\texttt{-Wall} -komentolipun avulla. Kääntäjälle pitää antaa erillinen
\texttt{-Wconversion} -komentolippu, jotta kääntäjä edes varoittaisi
mahdollisesti yllättävästä käytöksestä. Tämän ominaisuuden muuttaminen
virheeksi tai edes varoitukseksi ei ole ongelmatonta, sillä esimerkiksi
C-makrojen tuottama koodi voi olettaa tällaisten lausekkeiden toimivan.

C:hen voisi myös lisätä useita uusia tyyppejä, erityisesti tyypin ei-tyhjälle
osoittimelle. Tätä tyyppiä voisi käyttää turvallisempien ja nopeampien
ohjelmien kirjoittamiseen. GCC- ja Clang-kääntäjät tukevat ei-tyhjiä osoittimia
\texttt{\_\_attribute\_\_((nonnull))} -määreellä. Kääntäjä voi käyttää määrettä
optimoidessa funktioita -- erityisesti turhat tarkistukset tyhjien muuttujien
varalta optimoidaan pois.

Muita hyödyllisiä tyyppejä ohjelmoinnin helpottamiseen olisivat
monikot\defword{tuple} ja summatyypit\defword{tagged union, sum type}. Nämä
molemmat tyypit löytyvät esimerkiksi Rustista. Ensimmäisen saa käännettyä
triviaalisti tietueeksi, ja toisen saa käännettyä tietueeksi, jossa on sisällä
vaihtoehdoista luetelman\defword{enumeration} ja yhdisteen\defword{union}
yhdistelmä. Ohjelmassa~\ref{fig:csumtype} on Rustin ja C:n syntaksien mukaiset
summatyypit tyypille, jossa on joko operaation onnistuessa jokin kokonaisluku
tai epäonnistumisen yhteydessä epäonnistumisen syy merkkijonona. Erillisen
summatyypin määritteleminen mahdollistaa yhtä aikaa sekä paremman
oikeellisuustarkistuksen että tehokkaampien ohjelmien tuottamisen.
C-esimerkissä \texttt{failure}-kentässä voisi olla arvo, vaikka
\texttt{type}-kentän arvo olisi \texttt{\_SUCCESS}. Tämä voi johtaa
määrittelemättömään toimintaan, jos \texttt{success}-kentän arvoa yritetään
käyttää ja sillä hetkellä alustettu kenttä on \texttt{failure}.

\begin{listing}[ht!]
    \inputminted{Rust}{koodi/rustenum.rs}
    \inputminted[firstline=3]{C}{koodi/csumtype.c}
    \caption{Summatyyppi Rustissa ja C:ssä.}
    \label{fig:csumtype}
\end{listing}

%C:n kaltaisiin kieliin saa toteutettua helposti tyyppipäättelyn, sillä C:n
%tyypit ovat tyyppitasolla yksinkertaisia. Koska C:ssä tyypit eivät ole
%geneerisiä, lausekkeiden tyyppipäättely on lähes triviaalia paria operaatiota
%lukuun ottamatta. Tyyppipäättely helpottaa ohjelmointia, sillä ohjelmoija
%voi keskittyä ohjelmien logiikkaan muuttujien tyyppien kirjoittamisen sijaan.
%\hl{Viittaus C:n dokseihin?}

C:ssä ei ole erillistä moduulijärjestelmää, jonka lisääminen voisi nopeuttaa
kääntämistä ja tehdä ohjelmien ymmärtämisestä yksinkertaisempaa. C-ohjelmat
voivat käyttää kirjastojen funktioita kirjoittamalla kirjastofunktioista
prototyypit, jotka yleisesti ottaen ovat kirjastojen otsikkotiedostoissa. Koska
otsikkotiedostojen sisältö käytännössä ottaen kopioidaan
\texttt{\#include}-kutsun tilalle, kääntäjä joutuu käsittelemään samoja
otsikkotiedostoja lukuisia kertoja käännösprosessin aikana\footnote{Käytännössä
kääntäjät pystyvät optimoimaan tietyllä yleisellä tavalla kirjoitettuja
otsikkotiedostoja ja jättämään jo kertaalleen luetut tiedostot kokonaan pois.}.
Erillisen moduulijärjestelmän lisääminen voisi kääntämisen nopeuttamisen
lisäksi yksinkertaistaa kirjastojen toteuttamista, sillä ohjelmoijien ei
tarvitsisi kirjoittaa kirjastoilleen otsikkotiedostoja. 

C:n makrojärjestelmässä on paljon parantamisen varaa. C:n esikäsittelijä toimii
hyvin yksinkertaisissa tapauksissa, mutta sen rajoitteet tulevat nopeasti
esille monimutkaisempia makroja kirjoittaessa. Voimakkaampi makrojärjestelmä
mahdollistaa lyhyempien ja selkeämpien makrojen kirjoittamisen
monimutkaisemmissa tapauksissa. \citeauthor{cabuse} (\citeyear{cabuse})
esittelee artikkelissaan tapoja hyväksikäyttää C:n esikäsittelijää esimerkiksi
iteraation toteuttamiseen. Moderneissa makroprosessoreissa muun muassa
iteraation toteuttaminen makroilla on suoraviivaisempaa.

%Uutta makrojärjestelmää kirjoittaessa tulee
%kuitenkin pitää mielessä C-yhteensopivuus.
%Monimutkaisempi
%makrojärjestelmä voisi muuntaa voimakkaampaa kieltä C:n esikäsittelijän
%makroiksi käyttäen artikkelin mukaisia C-makroja, jolloin C-ohjelmat voisivat
%käyttää näitä yksinkertaisemmalla kielellä kirjoitettuja makroja. C:n standardi
%on kuitenkin hyvin avoin makrojen käytöksestä rajatapauksissa, joten
%monimutkaiset makrot eivät välttämättä toimi kääntäjäriippumattomasti, vaikka
%ne olisivatkin standardin mukaisia.

%C:n makrojärjestelmä myös hankaloittaa työkalujen kirjoittamista. Koska
%jokainen pieni muutos lähdekoodiin voi vaikuttaa radikaalisti saatavilla
%oleviin symboleihin, esimerkiksi automaattitäydennyksen tarjoavien työkalujen
%kirjoittaminen on hankalaa. Standardoitu moduulijärjelmä voisi myös
%yksinkertaistaa kääntämistyökalujen kirjoittamista.\hl{relevanssi?}

\newpage
\section{Määritelmät} 
\hilight{Tyylittely: Alkaako kappaleet omalta sivultaan?}

\hilight{Tähän enemmän humanismia ennen seuraavaa palautusta}

Määritetään C:tä paremmaksi kieleksi jokin kieli, mikä C:hen verrattuna:

\begin{itemize}
    \item on yhtä nopea tai nopeampi
    \item käyttää saman verran muistia tai vähemmän
    \item on helpompi käyttää
    \item toimii järjestelmissä, joissa C toimii
\end{itemize}

On huomioitava, että lukuisten olemassaolevien C:n kirjastojen, rajapintojen ja
projektien vuoksi yhteensopivuus C:n kanssa tulee olla saumatonta mahdollisten
vaihtoehtoisten ohjelmointikielten osalta, jotta kielen vaihtaminen olisi
mahdollista. Tämä sisältää myös kirkastorutiinien kutsumisen.

Tutkittavana on myös, mitä optimointeja C:ssä ei voi tehdä helposti johtuen
kielen rajoitteista ja miten tämän voisi korjata. Näitä ominaisuuksia ovat
esimerkiksi sivuvaikutuksettoman ohjelmakoodin merkitseminen,
optimointivinkkien alustariippumaton ilmaiseminen (annotaatiot funktion
parametreihin) ja useat eri funktiot riippuen parametrien arvoista, mikäli ne
voidaan kääntöaikaisesti päätellä.

\newpage
\section{Olemassa olevat ohjelmointikielet}

\subsection{Yleisiä vertailtavien ohjelmointikielten ominaisuuksia}

\hl{Miten tarkasti näitä pitää perustella/kaivaa lähteitä?}

Vertailtavissa ohjelmointikielissä on yleisesti useita ominaisuuksia, jotka
vaikuttavat ohjelmien ajoaikaiseen nopeuteen hidastavasti. Yleisin näistä on
automaattinen muistinhallinta, joka lisää ``roskien keräämisen''\footnote{eng.
garbage collection, gc}, jonka ajaksi ohjelman suoritus
pysäytetään\footnote{eng. gc pause}. Lisäksi automaattinen muistinhallinta
lisää muistinkäyttöä, sillä ohjelmointikieli joutuu ajoaikaisesti määrittämään
käytössä olevat muistiosoitteet.

Moderneissa ohjelmointikielissä virheiden käsittely on lähes poikkeuksetta
toteutettu poikkeuksilla. Poikkeusten käsittely vaatii pienen määrän muistia
jokaiselta funktiokutsulta, sillä poikkeuksen tapahtuessa ohjelman pitää löytää
lähin poikkeuskäsittelijä -- tämä tapahtuu kävelemällä pinoa takaisin ylöspäin,
kunnes poikkeuskäsittelijä löytyy. Jos virheiden käsittely on taas toteutettu
esimerkiksi signaaleilla, pinoa ei tarvitse kävellä, sillä vain yksi
signaalinkäsittelijä voi kerrallaan olla käytössä jokaista signaalia kohden.

Ohjelmointikieli voi toteuttaa olio-ohjelmoinnin rajapinnat käytännössä kahdella
tavalla: Joko korvaamalla jokaisen rajapintaviittauksen numerolla jaettuun
tauluun funktioita (vtable-rajapinnat), tai korvaamalla rajapintaviittaukset
tietueella, joka sisältää rajapinnan toteutuksen funktioiden osoitteet
(tietue-rajapinnat).  Ensimmäistä käytetään eritoten C++:n rajapintojen
toteuttamiseen, kun taas jälkimmäistä käytetään esimerkiksi Gossa.
Jälkimmäistä käytetään myös C:n kanssa; Linux-ytimen laiteajurit on toteutettu
tietue-rajapintoina, joita ajurit voivat tarvittaessa periä. Molemmissa
toteutustavoissa on omat hyvät jä huonot puolensa -- vtable-rajapinnat vievät
vähemmän muistia, kun taas tietue-rajapinnat ovat yksinkertaisempia
toteuttaa~\citationneeded.

\hl{Kumpi on suositumpi käännös nykyään, rajapinta vai liittymä?}

\subsection{C}

C on lähes kaikkien järjestelmien ymmärtämä ohjelmointikieli, jota käytetään
nopeutta tai pientä muistijalanjälkeä vaativien sovelluksien toteuttamiseen.

C on suunniteltu olemaan mahdollisimman alustariippumaton, mutta jos tämä
aiheuttaa ristiriidan mahdollisimman nopean toteutuksen kanssa, suositaan
nopeutta. Tästä yleisin esimerkki on kokonaislukujen ylivuoto, jonka tulos
riippuu kääntäjäoptimoinnista:

\begin{figure}[ht!]
    \inputminted{C}{c-overflow.c}
    \inputminted{text}{c-overflow-output}

    \caption{Kokonaisluvun ylivuoto C-kielessä. Ylivuodon käyttäytyminen
    riippuu kääntäjäoptimoinnin määrästä -- kääntäjän optimointitasolla -O0
    lauseketta $x~+~1~<=~x$ ei optimoida pois, mutta optimointitasolla -O3
    kääntäjä 'tietää' että operaation $+1$ jälkeen kokonaisluvut eivät
    ylivuoda, sillä C-kielen spesifikaatiossa kokonaislukujen ylivuoto on
    ``määrittelemätöntä toimintaa'' (eng. undefined behavior). Luonnollisten
    lukujen yli- ja alivuoto taas on määritelty - luonnolliset luvut on
    määritelty renkaaksi.}
\end{figure}

\FloatBarrier

\hl{Miten koodiesimerkit tulisi merkitä? LaTeXin \textbackslash caption antaa
oletuksella käännökseksi 'Kuva', mikä ei nyt pidä paikkaansa kun tuo on
kuitenkin tekstiä :) \\
'Lähdekoodi 1'?

Jossain oli sivulauseessa että koodiesimerkit pitäisi käydä rivi riviltä läpi,
miten rivi riviltä tämä oikeasti on? Selitänkö että 'void
print\_if\_overflow(int x)' määrittelee funktion yms., vai riittääkö tuo
nykyisen tasoinen selitys tällaisissa kohtalaisen yksinkertaisissa asioissa?}

\begin{figure}[ht!]
    \inputminted{C}{c-hygiene.c}
    \inputminted{text}{c-hygiene-output}
    \caption{C:n makrot eivät ole hygieenisiä. DOUBLE-makro muuttuu
    käännösvaiheessa muotoon 1+1*2, mikä on laskujärjestyksen takia 3 eikä
    odotettu 4. Koko printf-kutsuksi muodostuu siis
    '\texttt{printf("one plus one doubled is \%d", 1+1*2)}'
}
\end{figure}

\FloatBarrier

C:n makrot ovat muutenkin rajoittuneita -- esimerkiksi rekursiivisten makrojen
rakentaminen on liki mahdotonta.

\subsection{Ada}

\hl{Aliluvut eivät ole erityisen tasapainoisia keskenään -- c-luku on
huomattavasti pidempi johtuen koodiesimerkeistä

Onko yleislöpinäkappaleet turhia, eli esim. alla oleva kuka teki, miksi,
milloin?}

Ada on Yhdysvaltain puolustusministeriön kehittämä ohjelmointikieli, joka
suunniteltiin korvaamaan kaikki muut puolustusministeriön käyttämät
ohjelmointikielet~\citationneeded.Ada on hyvin moneen taipuva kieli, sillä se
pystyy hallitsemaan monia eri käyttötarkoituksia matalan tason bittitason
ohjelmoinnista korkean tason arkkitehtuureihin~\citationneeded.

Ohjelmoinnin tehostamiseksi Adassa on sekä poikkeukset että automaattinen
muistinhallinta. Nämä kuitenkin hidastavat kieltä hieman. Lisäksi C-kielen
kanssa yhteensopivuus on kielen taipuvuudesta johtuen hankalaa - jokainen
kutsuttava C-funktio on yksitellen määritettävä kutsukonvention\footnote{eng.
calling convention} kanssa, sillä kutsukonventiot eivät ole
alustariippumattomia~\citationneeded.

\subsection{C++}

C++ on Bjarne Stroustrupin 1980-luvusta eteenpäin kehittämä kieli, jonka
tarkoituksena on yhdistää C:n nopeus luokkapohjaisen olio-ohjelmoinnin
helppokäyttöisyyteen~\citationneeded. C++ on nykypäivänä suosittu tehokkuutensa
ja monipuolisuutensa takia monimutkaisissa ohjelmistoissa, kuten
palvelinohjelmistoissa ja peleissä~\citationneeded.

C++ on kehitetty C:n pohjalta, ja siinä onkin erittäin hyvä C-tuki.  Johtuen
nimiruntelusta\footnote{eng. name mangling -- ei kai tämä ole oikeasti se
suositeltu käännös? esim. mankelointi kuulostaa myös hassulta}, joka on C++:ssa
oletuksella päällä, C-koodin otsikkotiedostoissa\footnote{eng. header file --
ei kai tämäkään ole oikein suomennettu? Eihän?} on usein alussa C++-koodia,
joka laittaa nimiruntelun pois päältä. Näin C++-ohjelmat voivat helposti kutsua
C-ohjelmien kirjastokutsuja -- C++-ohjelmat ymmärtävät suoraan C-kielen
otsikkotiedostoja.

Virheidenkääsittely on toteutettu poikkeuksilla, jotka aiheuttavat pienen
hidastuksen. C++:ssa on myös käytettävissä viitemäärälaskettu\footnote{eng.
reference counting} automaattinen muistinhallinta, jolla voidaan käyttää
ajoaikaisesti varattua muistia ilman muistivuotoja. C++:ssa tosin ei ole
roskien keräystä, vaan kun viimeinen viite olioon poistetaan, myös varattu
muisti vapautetaan -- tällöin ohjelman suorituksen aikana ei tule
roskienkeräystaukoja.

\subsection{D}

D on 2000-luvun alussa Digital Mars -yrityksen julkaisema ohjelmointikieli,
jonka tarkoituksena on yksinkertaistaa C++-koodia~\citationneeded. Vaikka
D-kielessä on automaattinen muistinhallinta, D:n 'BetterC' -tila poistaa
käytöstä alustariippuvaiset ominaisuudet~\citep{dbetterc}. Tällöin kielestä
poistuu useita ominaisuuksia, mutta esimerkiksi D:n makrojärjestelmää voi
käyttää.

C-koodin kutsuminen on melko helppoa, mutta ei aivan saumatonta, sillä jokainen
kutsuttava funktio tulee määritellä erikseen -- D ei ymmärrä C:n
otsikkotiedostoja. Tämä kuitenkin onnistuu yhdellä rivillä per funktio, sillä
D:n tyyppijärjestelmä on hyvin lähellä C:tä.

\subsection{Go}

\hl{Taivutus; Go:ssa, Gossa, Go-kielessä, Golangissa...? Ainakin se lautapeli
taivutetaan toisen mukaan -- käytin nyt toista ja kolmatta muotoa sekaisin}

Go on Googlen 2000-luvun loppupuolella kehittämä ohjelmointikieli, jonka
tarkoituksena on D:n tavoin korvata C++-koodia~\citationneeded. Go on
suunniteltu mahdollisimman yksinkertaiseksi käyttää. Toisin kuin monissa
moderneissa C-perheen kielissä, Gossa ei ole luokkia, vaan pelkkiä tietueita ja
rajapintoja.

Gossa ei ole mahdollista kirjoittaa käännösaikaisesti tyyppitarkistettua
geneeristä koodia, mutta ohjelmat voivat ajoaikaisesti peilauksen\footnote{eng.
reflection} kautta tutkia tietueiden rakennetta. Tämän mahdollistaminen
kasvattaa ohjelmakoodin kokoa, sillä tietueiden mukana on säilytettävä tietueen
oikeaa tyyppiä.

Gon virheidenkäsittely on toteutettu useissa kohdissa C:n tavoin; funktioista
palautetaan virheellisissä tilanteissa virheellinen arvo. Tämä tosin tehdään
usein palauttamalla erillinen \texttt{Error}-tyyppiä oleva arvo -- Go
mahdollistaa helposti useamman kuin yhden paluuarvon. Gossa on myös
poikkeukset, joita suositellaan käyttävän vain poikkeuksellisissa
tilanteissa~\citationneeded.

C:n kutsuminen Gosta ei ole aukotonta: koska Go on muistinkäytöltä turvallinen
kieli, erityisesti muistin jakaminen C:n ja Gon välillä on
hankalaa~\citationneeded. C:n funktio-osoittimia ei voi kutsua Gon puolelta.

\subsection{Rust}

Rust on Mozilla Foundationin kehittämä ohjelmointikieli, joka on D:n ja Gon
tavoin suunniteltu korvaamaan C++. Rust on suunniteltu kolmen ydinperiaatten
ympärille; turvallisuuden, nopeuden ja rinnakkaisuuden~\citationneeded.
Rustissa on monimutkainen tyyppijärjestelmä, jolla ohjelmat voivat todistaa
esimerkiksi turvallisen rinnakkaisajon, ilman että ohjelmaan tulee ajoaikaisia
rajoitteita tai hidastuksia.

Rustin virheidenhallinta on lähellä Gon virhehallintaa -- Rustin ohjekirja
opastaa käyttämään mieluummin paluuarvoja kuin poikkeuksia~\citationneeded.
Kuten Gossa, Rustissa voi myös käyttää poikkeuksia.

\hl{Pitäisikö tässä olla vielä pohdintaa kielissä olevista yhteisistä
piirteistä?}

\newpage
\section{Purkka-ohjelmointikieli}
\label{sec:purkka}

Tässä luvussa määritellään Purkka-ohjelmointikieli.

\subsection{Tyypit}

Jotta yhteensopivuus C:n kanssa olisi joustavaa, Purkan tyyppijärjestelmä
muodostetaan mahdollisimman paljon C:n kaltaiseksi. C:n tyyppisyntaksi sisältää
useita erilaisia tapoja ilmaista samaa pohjatyyppiä -- esimerkisi
\texttt{long}, \texttt{signed~long}, \texttt{long~int} ja
\texttt{signed~long~int} ilmaisevat kokonaislukutyyppiä, joka pystyy
sisältämään ainakin luvut $[-(2^{31} - 1), 2^{31}-1]$ \footnote{Käytännössä
modernit implementaatiot käyttävät kahden komplementtia kokonaislukujen
ilmaisemiseen, jolloin 32-bittinen kokonaislukumuuttuja voi sisältää luvut
$[-2^{31}, 2^{31}]$.}\citationneeded. Nämä tyypit ovat kuitenkin standardien
mukaisessa C:ssä keskenään täysin vaihdettavissa, eli kielen ei tarvitse pystyä
kääntymään jokaiseen mahdolliseen vaihtoehtoon. Perustyypit ovat hyvin lähellä
Rustin pohjatyyppejä\citationneeded. Purkka tarvitsee erikseen tyypit
\texttt{char}, \texttt{i8} ja \texttt{u8}, sillä C:n \texttt{char}-tyyppiä ei
ole määritelty tarkemmin joko kokonaisluvuksi tai 
\texttt{signed char} ja \texttt{unsigned char} ovat erillisiä tyyppejä.

Määritetään Purkkaan seuraavat alkeistyypit: \\[0.3cm]
\begin{tabular}{@{}p{5.8cm}p{8.7cm}@{}} \toprule
    Purkka-tyyppi & Purkka-tyyppiä vastaava C-tyyppi \\ \midrule
    \texttt{void} & \texttt{void} \\
    \texttt{char} & \texttt{char} \\
    \texttt{i8}, \texttt{i16}, \texttt{i32}, \texttt{i64} &
    \texttt{int8\_t}, \texttt{int16\_t}, \texttt{int32\_t}, \texttt{int64\_t} \\
    \texttt{u8}, \texttt{u16}, \texttt{u32}, \texttt{u64} &
    \texttt{uint8\_t}, \texttt{uint16\_t}, \texttt{uint32\_t}, \texttt{uint64\_t} \\
    \texttt{float} & \texttt{float} \\
    \texttt{double} & \texttt{double} \\
    [0.3cm]

    \texttt{cbyte}, \texttt{cshort}, \texttt{cint}, \texttt{clong}, \texttt{clonglong}
    & \texttt{signed char}, \texttt{short}, \texttt{int}, \texttt{long}, \texttt{long long} \\
    \noalign{\vspace{0.3cm}}%

    \texttt{cubyte}, \texttt{cushort}, \texttt{cuint}, \texttt{culong}, \texttt{culonglong}
    & \texttt{unsigned char}, \texttt{unsigned short}, \texttt{unsigned int},
      \texttt{unsigned long}, \texttt{unsigned long long} \\
    \bottomrule
\end{tabular} \\

Määritetään näiden alkeistyyppien pohjalta osoitintyypit, taulukkotyypit sekä
yhdistetyypit: \\[0.3cm]
\begin{tabular}{@{}ll@{}} \toprule
    Purkka-tyyppi & Purkka-tyyppiä vastaava C-tyyppi \\ \midrule
    \texttt{\&T, \&?T} & \texttt{T *} \\
    [0.2cm]

    \texttt{[T]} & \texttt{T\textsubscript{declaration specifiers}[]T\textsubscript{direct abstract declarator}} \\
    \texttt{[T;expr]} & \texttt{T\textsubscript{declaration specifiers}[expr]T\textsubscript{direct abstract declarator}} \\
    [0.2cm]

    \texttt{fun t: (T ..) -> R} & \texttt{R\textsubscript{declaration specifiers}~t(T ..)R\textsubscript{direct abstract declarator}} \\
    \texttt{fun (T ..) -> R} & \texttt{R\textsubscript{declaration specifiers}~(*)(T ..)R\textsubscript{direct abstract declarator}} \\
    [0.2cm]

    \texttt{struct \{ k: T ..\}} & \texttt{struct \{ T k ..\}} \\
    \texttt{enum \{ A [= expr] ..\}} & \texttt{enum \{ A [= expr] ..\}} \\
    \texttt{union \{ k: T ..\}} & \texttt{union \{ T k ..\}} \\
    [0.2cm]

    \texttt{(T1 ..)} & \texttt{struct \{ T1 e1; ..\}} \\
    \texttt{enum \{ T(T1 ..) .. \}} & \texttt{struct \{ union \{T1 t1 ..\} v; enum \{T ..\} e;\}} \\
    [0.2cm]

    GCC-laajennokset & \\
    \texttt{attribute(T, ...))} & \texttt{T \_\_attribute\_\_((...))} \\

    \bottomrule
\end{tabular} \\

Koska kieli ei voi erota C:stä suoritusaikaisesti, tyypitys on hyvin lähellä
C:n tyypitystä.

Tyypit:
 C + tagged union + non-nullable pointer

kuinka tagged union käännetään C:ksi? Entä non-nullable ptr?

tyyppi-inferenssi

Mitä optimointeja non-null tarjoaa? Entä tagged union?

\subsection{Tyyppi-inferenssi}



\subsection{Syntaksi}

Syntaksi:

funktiot on fun, koska se tekee ohjelmoinnista hauskempaa

deklaraatiot: tyyppi-inferenssi

tyyppisyntaksi "käänteinen" C:hen verrattuna
C:ssä suurin ongelma on declspec ennen nimeä, mutta directdecl nimen jäklkeem

static:
    funktiodeklaaratiossa/globaaleissa muuttujissa pub => ei static,
    muuttujissa static => static

expressiot:
    yksinkertaistettu kielioppi operaattorijumppaan
    Array literal != Struct literal
    if/ym. ekspressioita
        - miten muunnos?

statementit:
    suurin osa expr

\subsection{Makrot}

Koska Purkan tulee olla myös makrojen osalta C-yhteensopiva, Purkka osaa
laajentaa C-makroja. Purkka-kääntäjä pystyy muuntamaan makrokutsun
C-lähdekoodiksi, laajentamaan sen C-esikäsittelijällä ja muuntamaan makron
laajennettu muoto takaisin Purkka-koodiksi.

Purkassa on myös oma makrojärjestelmä, joka toimii hahmotunnistuksella ja
rekursiolla.\footnote{Referenssikääntäjään (ks. luku \ref{sec:results}) ei ole
toteutettu makroja.} Jos makrot eivät sisällä rekursiota tai hahmontunnistusta,
ne voidaan muuntaa C-makroiksi.

\hl{Teoriassa myös hahmontunnistuksen ja rekursion voisi muuntaa C-makroiksi,
käyttäen hyväksi C-makrojen detaileja.}

\subsection{Suoritusaikaiset ominaisuudet ja laajennokset}

Purkka ei sisällä yhtään suoritusaikaista ominaisuutta, joita ei ole C:ssä.
Syntaksi kuitenkin sisältää \texttt{pragma}-avainsanan, jota voidaan käyttää
vastaavasti kuin C:n esikäsittelijän pragmaa erilaisten laajennosten
käyttämiseen. Esimerkiksi OpenMP-laajennosta käytetään pragmojen läpi.

Useat C-toteutukset sisältävät erilaisia laajennoksia, joiden tarkoituksena on
mahdollistaa eilaisten alustariippuvaisten ominaisuuksien käyttö. Benchmarks
gamessa yksi käytetyimmistä laajennoksista on SIMD-tyypit, joita varten
esimerkiksi GCC-kääntäjällä on oma syntaksinsa. GCC:llä SIMD-tyypit voidaan
määritellä käyttäen \texttt{\_\_attribute\_\_(())} -\hl{määrettä}.
Purkka-kääntäjä osaa tunnistaa useita GCC:n laajennoksia ja välittää ne
eteenpäin.

Sulkeumien toteutus olettaa, että C-kääntäjä tukee laajennoksena sulkeumia.

\subsection{Ohjelmointikielen kääntäminen}

Purkan kääntäminen C-koodiksi on hieman normaalia kääntämistä monimutkaisempaa.
Koska Purkan tulee pystyä muuntamaan Purkka-koodia C-koodiksi, mutta myös
C-koodia purkka-koodiksi makrojen laajentamisen myötä, kääntäjän pitää ymmärtää
kahta ohjelmointikieltä yhden sijaan.

Suurin osa ominaisuuksista kääntyy suoraan vastaaviksi C-ominaisuuksiksi.
Muutamat vaatii työtä, kuten inferenssi, block expr, lambdat. Inferenssialgo
johonkin nurkkaan.

Lambdat käännetään siten, että funktion body nostetaan ulos lambdan paikalta,
ja korvataan funktio-osoittimella.

\hl{Koska kieli ymmärtää C:tä ja Purkkaa, ja muuntaa näiden välillä,
käännösprosessi on monimutkainen. Kuvassa} \ref{fig:compilerarch} \hl{näytetään
mahdollisimman epäselkeästi arkkitehtuuri.}

\begin{figure}[ht!]
        \footnotesize
        \begin{adjustbox}{center}
        \begin{tikzpicture}[-{Latex[length=2mm,width=3mm]},auto, main node/.style={draw,fill=white},align=center]
            \tikzstyle{ann} = [rounded corners,fill=black!4,draw=black]
            \tikzset{
                -|/.style={to path={-| (\tikztotarget)}},
                |-/.style={to path={|- (\tikztotarget)}},
            }

            \node[main node,text width=3cm,fill=black!10] (core) {Tiedosto syötetään kääntäjälle};

            \node[main node,text width=3.3cm] (purkkaparser) [right = of core] {Purkka-jäsentäjä};

            \node[main node] (cpurkka) [right = of purkkaparser,text width=3.5cm] {Muuntaja C-syntaksipuusta Purkka-syntaksipuuksi};
            \node[main node] (coremacro) [above = of cpurkka,text width=3.5cm] {Muuntaja Purkka-lausekkeista C-syntaksipuiksi};
            \node[main node,text width=3.5cm] (purkkac) [below = of cpurkka] {Muuntaja Purkka-syntaksipuusta C-syntaksipuuksi};
            \coordinate[above left = of coremacro] (phantom);

            \node[draw = none] (phantom1) [above = of coremacro, text width=3.5cm] {};
            \node[draw = none] (phantom3) [left = of phantom1, text width=3.3cm] {};
            \node[draw = none] (phantom2) [right = of phantom1, text width=2.3cm] {};

            \node[main node] (preprocessor) [right = of coremacro,text width=2.3cm] {C-esikäsittelijä};
            \node[main node] (cparser) [right = of cpurkka,text width=2.3cm] {C-jäsentäjä};

            \node[main node,text width=3.3cm] (cformat) [left = of purkkac] {C-syntaksipuun muuntaminen lähdekoodiksi};
            \node[main node,fill=black!10] (done) [left = of cformat] {Tiedosto on käännetty};

            \coordinate (phantom4) at ($(purkkaparser)!0.5!(purkkac)$);

            \draw (purkkaparser.170) |- (phantom) node[blur shadow={shadow blur steps=5},ann,above right = 0.2cm and 0cm]  {Purkka-koodi sisällyttää C-tiedoston} -| (preprocessor); 
            \draw (purkkaparser.130) |- node[blur shadow={shadow blur steps=5},ann,text width=2.7cm,above right = 0.2cm and -0.4cm] {Purkka-koodi kutsuu C-makroa} (coremacro);
            \draw (purkkaparser) |- (phantom4) -| (purkkac);

            \path[every node/.style={font=\sffamily\small},-{Latex[length=2mm,width=3mm]}]
            (core) edge (purkkaparser)
            (coremacro) edge (preprocessor)
            (preprocessor) edge (cparser)
            (cparser) edge (cpurkka)
            (cpurkka) edge (purkkaparser)
            (purkkac) edge (cformat)
            (cformat) edge (done)
            ;

            \draw node[above = of phantom1] {\textbf{AST-muuntaja}};
            \draw node[above = of phantom2] {\textbf{C-kääntäjä}};
            \draw node[above = of phantom3] {\textbf{Purkka-kääntäjä}};
            \begin{scope}[on background layer]
                \draw[thick,dotted,fill=black!5] ($(phantom1.north west)+(-0.3,2)$)  rectangle ($(purkkac.south east)+(0.3,-0.4)$);
                \draw[thick,dotted,fill=black!5] ($(phantom2.north west)+(-0.3,2)$)  rectangle ($(cparser.south east)+(0.3,-0.4)$);
                \draw[thick,dotted,fill=black!5] ($(phantom3.north west)+(-0.3,2)$)  rectangle ($(purkkaparser.south east)+(0.3,-0.4)$);
            \end{scope}
        \end{tikzpicture}
        \end{adjustbox}

        \caption{Graafi kääntämisprosessin hienouksista}
        \label{fig:compilerarch}
\end{figure}

\newpage
\section{Uuden ohjelmointikielen vertaaminen C:hen}

Tässä luvussa verrataan uuden ohjelmointikielen suorituskykymittauksia C:hen
sekä muihin tutkielmassa käsiteltyihin kieliin. Suorituskykymittaukset
näyttävät, että Purkan toteutus ei hidasta C:tä. Tämä taas johtuu siitä, että
Purkassa ei ole yhtään ominaisuutta, joita C:ssä ei ole. Tässä luvussa myös
tutkielman oikeellisuutta ja pohditaan jatkotutkimuskohteita.

\subsection{Vertailun tulokset}
\label{sec:results}

Tätä tutkielmaa varten on toteutettu kääntäjä, joka kääntää yksinkertaiset
ohjelmat Purkka-kielestä C-kieleen \citep{purkka}. Kääntäjään ei ole toteutettu
kaikkia luvussa~\ref{sec:purkka} esitellyistä ominaisuuksia, mutta kääntäjä
tukee esimerkiksi C:n esikäsittelijää.

Nopeimmat Benchmarks Gamen C-toteutukset on käännetty Purkka-kielelle. Nämä
ohjelmat on sitten käännetty Purkka-kääntäjällä takaisin C:ksi ja käännetty
sitten GCC-kääntäjällä vastaavilla asetuksilla kuin verrattavat C-ohjelmat.
Kuvassa~\ref{fig:purkkabenchmarksgame} verrataan Purkka-toteutuksia muihin
verrattaviin kieliin ja kuvassa~\ref{fig:purkkabenchmarksgame2} verrataan
Purkka-toteutuksia nopeimpaan C-ohjelmaan. Koska Purkka-toteutukset on
muunnettu nopeimmasta C-toteutuksesta ja käännetty takaisin lähes identtiseen
muotoon, suoritusajan ja muistinkäytön tulisi olla samat verrattuna
C-toteutukseen.

Suorituskykymittaukset mittaavat ainoastaan prosessoriin ja muistinkäyttöön
liittyviä mittauksia, eivätkä esimerkiksi näytönohjaimella suoritettua
laskentaa. Mittaukset tehdään vain yksittäisillä kääntäjäversioilla, joten
mittaukset eivät välttämättä päde muilla saman ohjelmointikielen kääntäjillä.

Suorituskykymittaukset on suoritettu Ubuntu 19.10 -käyttöjärjestelmällä Linux
5.3.0 -kernelillä. Mittaustietokoneessa on neliytiminen Intel i7-8550U
-prosessori Hy\-per\-Thread\-ing-tuella (4.0 GHz) ja 32 gigatavua DDR3-muistia.
C- ja C++-ohjelmat on käännetty GCC-kääntäjän versiolla 9.2.1, Ada-ohjelmat
GNAT-kääntäjän versiolla 8.3.0, Go-ohjelmat Go-kääntäjän versiolla 1.12.10 ja
Rust-ohjelmat Rust-kääntäjän versiolla 1.41.0.

Mittaukset on toteutettu suorittamalla ohjelma suurimmalla syötteellä viisi
kertaa ja mittaamalla kunkin suorituksen viemä aika ja muistinkäyttö.
Muistinkäyttö on mitattu 200 millisekunnin välein \texttt{libgtop2}-kirjastolla
ja kunkin suorituksen muistinkäytöksi on merkitty suurin mitattu muistikäyttö.

Kuvaajista nähdään, että Benchmarks Gamen tarjoamissa esimerkeissä Purkka on
täsmälleen yhtä tehokas kieli kuin C, eikä se vie enempää muistia. Pienet
eroavaisuudet suuntaan ja toiseen on selitettävissä testauksen aiheuttamalla
luonnollisella vaihtelulla. Purkalla toteutettu \texttt{fannkuchredux}-ohjelma
on kuitenkin noin 50\% nopeampi kuin C:llä toteutettu lähes identtinen versio
ja \texttt{nbody}-ohjelma vie vain 80\% muistia verrattuna C-toteutukseen.

Purkka-tiedostot ovat keskimäärin 6\% pienempiä kuin vastaavat C-tiedostot.
Suurimmat kokoerot tiedostoissa tulevat muuttujien määrittelyistä sekä
tyyppipäättelystä. Kuvassa~\ref{fig:declarations} on yksittäisiä
tyyppimäärittelyjä, jotka ovat Purkassa hieman yksinkertaisempia kuin vastaavat
C-määrittelyt.

\begin{figure}[ht!]
    \begin{adjustbox}{center}
        \begin{tabular}{@{} m{0.55\textwidth} m{0.55\textwidth} @{}} \toprule
            Purkka-määrittely ja C-määrittely & Selite \\ \midrule

            \texttt{let a: u32; \newline unsigned int a;} & Epänegatiivinen 32-bittinen kokonaisluku \\
            \noalign{\vspace{0.2cm}}

            \texttt{let a, b: [\&i8;5]; \newline signed char *a[5], *b[5];} & Taulukko viidestä osoitinmuuttujasta \newline 8-bittiseen kokonaislukuun \\
            \noalign{\vspace{0.2cm}}

            \texttt{const a = ["1", "2", "3"]; \newline const char * const a[] = \{"1", "2", "3"\};} & Kolmen merkkijonon taulukon vakiomuuttuja \\
            \noalign{\vspace{0.2cm}}

            \texttt{let a = fun (a, b) => a + b; \newline int a(int a, int b) \{ return a + b; \}} & Funktio, joka laskee argumenttiensa summan \\ \bottomrule
        \end{tabular}
    \end{adjustbox}
    \label{fig:declarations}
    \caption{Purkan muuttujien määrittelyt verrattuna C:n vastaaviin
    määrittelyihin.}
\end{figure}

\begin{figure}[ht!]
    \begin{adjustbox}{center}
    \begin{minipage}{1.15\textwidth}
    \begin{minipage}{0.5\textwidth}
        \input{data/benchmarkscpu3.tex}
        \vspace*{-0.8cm}
    \end{minipage}
    \begin{minipage}{0.5\textwidth}
        \input{data/benchmarksmem3.tex}
        \vspace*{-0.9cm}
    \end{minipage}
    \end{minipage}
    \end{adjustbox}

    \caption{
        Benchmarks Gamen ohjelmiin perustuvat kuvaajat Purkalla kirjoitettujen
        ohjelmien suorituskyvystä, muistinkäytöstä ja ohjelmien koosta
        verrattuna muilla kielillä kirjoitettujen ohjelmien tuloksiin.}
    \label{fig:purkkabenchmarksgame}
\end{figure}

\begin{figure}[ht!]
    \begin{adjustbox}{center}
        \begin{minipage}{1.25\textwidth}
        \input{data/benchmarkscpu2.tex}
        \end{minipage}
    \end{adjustbox}

    \begin{adjustbox}{center}
        \begin{minipage}{1.25\textwidth}
        \input{data/benchmarksmem2.tex}
        \end{minipage}
    \end{adjustbox}

    \begin{adjustbox}{center}
        \begin{minipage}{1.25\textwidth}
        \input{data/benchmarkslines2.tex}
        \end{minipage}
    \end{adjustbox}
    \caption{
        Benchmarks Gamen ohjelmiin perustuvat kuvaajat Purkalla kirjoitettujen ohjelmien
        suorituskyvystä, muistinkäytöstä ja ohjelmien koosta verrattuna C:llä
        kirjoitettujen ohjelmien tuloksiin.}
    \label{fig:purkkabenchmarksgame2}
\end{figure}

\FloatBarrier

\subsection{Johtopäätökset ja vertailun arviointi}

Purkka on vertailun ainoa kieli, joka on yhtä tehokas kuin C. Tämä johtunee
pitkälti siitä, että Purkka käännetään lähes identtisenä C:ksi, eikä tarjoa
suoritusaikaisia ominaisuuksia, jotka hidastaisivat kieltä. Purkka ei kuitenkaan
päihitä C:tä tutkimuksen määrittelyn mukaisesti, vaan on suorituskyvyltään
täsmälleen yhtä tehokas kuin C.

Kaikkien vertailtavien kielten suunnittelutavoitteissa on mainittu C:n lisäksi
myös C++:n käytön korvaaminen. Tämä on todennäköisesti vaikuttanut kielen
suunnitteluun monimutkaistaen kieliä, jotta kieli pystyisi korvaamaan C++:n
monimutkaiset malliohjelmointirakenteet. Muista kielistä poiketen Purkan
tavoitteena on korvata ainoastaan C, mikä on ohjannut suunnittelua tutkielman
määrittelyjen mukaisesti.

Suorituskykymittaukset eivät kuitenkaan vastaa todellista maailmaa, sillä ne
usein mittaavat vain yksittäistä pientä osa-aluetta, kuten yksittäisten
operaatioiden nopeutta monimutkaisten ohjelmistojen sijaan. Lisäksi
suorituskykymittaukset ovat usein epätarkkoja, sillä monimutkaiset
moniajoympäristöt eivät mahdollista deterministisiä mittauksia.

Mittaukset koskevat vain yksittäisiä ohjelmia ajettuna tietyssä ympäristössä
tietyllä kääntäjällä. Ne eivät siis anna kattavaa kuvaa ohjelmointikielistä,
vaan mittaustuloksia yksittäisen kääntäjätoteutuksen kääntämistä ohjelmista.
Kääntäjät voivat käyttää jopa yksittäisten versioiden välillä erilaisia
optimointeja, jotka voivat nopeuttaa tai hidastaa ohjelmaa juuri mitatulla
syötteellä, mikä pienentää tulosten vertailukelpoisuutta. Kääntäjät voivat
myös toimia paremmin tai huonommin erilaisilla tietokonearkkitehtuureilla,
jolloin esimerkiksi prosessorin valinta vaikuttaa mittaustuloksiin.

Purkkaa ei todennäköisesti oteta laajaan käyttöön, sillä se ei tarjoa
merkittäviä parannuksia C:hen, vaan mahdollistaa lähinnä hieman lyhyemmän
lähdekoodin. Ohjelmointikielten valintaan liittyy usein monimutkaisia syitä,
kuten aikaisempi kokemus tai ohjelmoijan henkilökohtainen mieltymys johonkin
kieleen, eikä projekteissa käytettyjä ohjelmointikieliä valita pelkästään
kielen ominaisuuksien perusteella. Tämän lisäksi Purkan toteuttaja on
yksittäinen opiskelija, kun taas esimerkiksi Go-kielen ja Rustin toteuttajat
ovat kokeneita ohjelmointikielten suunnittelijoita, jonka lisäksi kieliä
ylläpidetään aktiivisesti tunnettujen organisaatioiden toimesta. Tämä on
todennäköisesti myös syy sille, miksi esimerkiksi LISP/c~\citep{clisp1},
C-Mera~\citep{clisp2}, Carp~\citep{clisp3} ja Nymph~\citep{nymph} ovat jääneet
ilman laajempaa huomiota.

Tutkimuksesta voidaan kuitenkin päätellä, että C:tä parempi kieli on
todennäköisesti toteutettavissa. Käännösaikaista turvallisuutta voi parantaa
tiukemmalla tyyppijärjestelmällä ilman suoritusaikaisia haittoja. Tarkemmilla
tyyppimäärittelyillä voi myös tehdä optimointeja, esimerkiksi vaatimalla
osoitinargumentit aina ei-tyhjiksi osoittimiksi.

Osan Purkan ominaisuuksista voisi ottaa käyttöön jopa uusissa C:n versioissa.
Tyyppipäättelyn lisääminen esimerkiksi korvaamalla käyttämättömän
\texttt{auto}-avainsanan\footnote{Avainsanaa käytetään tarkkaan ottaen
\texttt{static}-avainsanan vastinparina, eli ei-staattisten muuttujien
luomiseen. Koska kaikki muuttujat ilman \texttt{static}-määrettä ovat
ei-staattisia, \texttt{auto}-avainsana on turha.} tarkoittamaan pääteltyä
tyyppiä voisi auttaa moderneja C-kääntäjiä käyttävien projektien
kirjoittamista. GCC-kääntäjä tukee tyyppipäättelyä
\texttt{\_\_auto\_type}-määreellä. Summatyyppien lisääminen laajennoksena
\texttt{enum}- tai \texttt{union}-tyyppisyntaksiin ei vaikuta tämänhetkisiin
standardien mukaisiin ohjelmiin, mutta mahdollistaisi summatyyppien käytön.
Suurempia syntaktisia muutoksia, kuten tyyppisyntaksin uudelleenkirjoitusta
tuskin pystyy toteuttamaan säilyttäen yhä tuen nykyiselle C:lle.

Purkan jatkokehityksen voisi aloittaa omalla makrojärjestelmällä, sillä
Purkassa ei ole määriteltynä omaa makrojärjestelmää. Purkka tukee C:n
makrojärjestelmää, mutta C:n makrojärjestelmää kattavamman makrojärjestelmä
voisi helpottaa monimutkaisten käännösaikaisten laskujen laskelmista.

Purkkaa varten ei ole kääntäjän lisäksi kehitetty yhtään kehitystyökalua,
vaikka luvussa~\ref{sec:suosio} valmiit työkalut nostetaan tärkeäksi
ominaisuudeksi. Esimerkiksi hyvälaatuinen automaattinen käännösjärjestelmä
helpottaisi kielen käyttämissä uusissa ohjelmissa. Tämän voisi nimetä
ohjelmointikielen nimen mukaisella teemalla Jesariksi, sillä se sitoo
käännettävät ohjelman palaset toisiinsa yhdistettynä mahdollisiin
riippuvuuksiin.

\newpage
\section{Yhteenveto}

\hl{Tässä aliluvussa kerrataan tutkimuksen tulokset, eli kerrotaan lyhyesti
uuden ohjelmointikielen ja muiden verrattavien kielten suhteesta C:hen, sekä
muut olennaiset tutkielman asiat.}

\newpage

\bibliography{references}

\appendixbeginhere
\inputappendix{data/data.tex}
\addtocontents{toc}{\protect\enlargethispage{2\baselineskip}}
\inputappendix{dokumentaatio.tex}
\appendixendhere


\end{document}

\documentclass[gradu]{tktltiki}

\usepackage{gradu}

\makeatletter
\AtBeginDocument{%
    \renewcommand{\BEd}{painos\hbox{}}
    \renewcommand{\BOthers}[1]{ja muut\hbox{}}%
    \renewcommand{\BOthersPeriod}[1]{ja muut\hbox{}}%
}
\makeatother

\begin{document}

\title{C-ohjelmointikielen korvaaminen muilla ohjelmointikielillä}
\author{Jaakko Hannikainen}
\date{\today}
\level{Pro gradu -tutkielma}

\hypersetup{pageanchor=false}
\maketitle
\hypersetup{pageanchor=false}

\classification{\protect{\ \\
\  \textbf{Software and its engineering
$\rightarrow$ Software notations and tools
$\rightarrow$ General programming languages
$\rightarrow$ Language types
$\rightarrow$ Imperative languages} \\
\  Software and its engineering
$\rightarrow$ Software organization and properties
$\rightarrow$ Operating systems \\
\  Computer systems organization
$\rightarrow$ Embedded and cyber-physical systems
$\rightarrow$ Embedded systems
$\rightarrow$ Embedded software \\
}}

\keywords{C, ohjelmointikielet}

\begin{abstract}
    Tähän kohtaan tulee myöhemmin tutkielman tiivistelmä. Tähän tullee ainakin:
    \begin{itemize}
        \item Miksi C pitäisi korvata?
        \item Miten ohjelmointikieliä verrataan?
        \item Miksi muut ohjelmointikielet eivät päihitä C:tä?
        \item Minkälainen uusi kieli syntyi?
        \item Miten uusi kieli pärjää vertailussa?
    \end{itemize}

\end{abstract}

\newpage
\tocbeginshere
\addtocontents{toc}{\protect\thispagestyle{empty}}

\mytableofcontents

\section{Johdanto}

C~\citep{C18} on nykypäivänä yksi eniten käytetyimmistä ohjelmointikielistä. C
on ollut vallitseva ohjelmointikieli järjestelmäohjelmoinnissa kielen
alkuajoista lähtien. Useita ohjelmointikieliä on luotu historian saatossa,
joiden oli tarkoitus syrjäyttää C, mutta C on vieläkin johtavana kielenä
varsinkin sulautetuissa järjestelmissä ja Unix-pohjaisten käyttöjärjestelmien
vallitsevana ohjelmointikielenä. C on myös käytössä
Windows-käyttöjärjestelmäperheen ydinkomponenttien toteutuksessa.

Tutkielmassa selvitetään C:n ominaisuuksia, joiden takia se on ollut
suosituimpien ohjelmointikielten joukossa vuosikymmeniä, kuten myös
ominaisuuksia, joita C:stä voisi kehittää. Vaihtoehtoisista kielistä
selvitetään, mitkä ominaisuudet ovat voineet estää kielen käytön C:n sijaan
uusissa ja olemassa olevissa projekteissa ja mitkä ominaisuudet ovat taas
olleet parannuksia C:n ominaisuuksiin verrattuna. Tutkielmassa suunnitellaan
näiden tulosten pohjalta uusi ohjelmointikieli, Purkka.

C:n vaihtoehdoiksi tutkitaan seuraavia tehokkaaseen ohjelmointiin tarkoitettuja
kieliä: Ada~\citep{ADA12}, C++~\citep{CPP17}, D~\citep{D}, Go~\citep{golang}
sekä Rust~\citep{rust}. Näistä kielistä tutkitaan, mikä tai mitkä ominaisuudet
ovat estäneet C:n korvaamisen ja mitkä ominaisuudet ovat olleet parannuksia
C:hen verrattuna. Vertailun tuloksia käytetään uuden ohjelmointikielen
suunnitteluun, jossa otetaan tavoitteeksi luoda C:tä parempi ohjelmointikieli
tutkielman määrittelyjen puitteissa.

Kaikkia verrattavia kieliä tutkitaan sekä analyyttisesti että
suorituskykymittausten muodossa. Suorituskykymittauksiin käytetään Benchmarks
Gamea~\citep{benchmarks}, johon on toteutettu lukuisia pieniä ohjelmia
suorituskyvyn mittaamiseen. D on ainoa tutkielman käsittelemä kieli, jota
Benchmarks Game ei sisällä. Benchmarks Gamen Purkka-ohjelmat on toteutettu
nopeimpien C-ohjelmien pohjalta, jotta ohjelmien arkkitehtuuriset valinnat
eivät vaikuta vertailuun. 

Tutkielman toisessa luvussa käsitellään C-ohjelmointikieltä. Kielestä
käsitellään perusteiden lisäksi kielen historiaa ja nykypäivää, tärkeimpiä
ominaisuuksia ja kehityskohteita. Historian käsitteleminen mahdollistaa
ymmärryksen siitä, mitkä C:n ominaisuudet ovat tärkeitä kielen nykykäytön
kannalta ja mitkä ovat jäänteitä historiallisista syistä. Tärkeimpien
ominaisuuksien ja kehitettävien ominaisuuksien tunnistamiseen käytetään
ohjelmointikielten tulevaisuutta käsittelevää artikkelia \emph{The Next 7000
Programming Languages}~\citep{next7000}, C-kielen suosiota käsittelevää
artikkelia \emph{Some Were Meant For C}~\citep{somemeantforc} sekä Dennis
Ritchien artikkelissa \emph{The Development of the C Language}~\citep{chistory}
esiin nostettuja C:n ominaisuuksia.

Kolmannessa luvussa määritetään toisen luvun esiin nostamien ominaisuuksien
pohjalta vertailukriteerit kielten vertaamiseen, käsitellään lyhyesti
tutkielmaan valittuja vertailtavia kieliä sekä käsitellään erilaisia yleisiä
ohjelmointikielten valintaan liittyviä tekijöitä. Edellä mainitut kielet
valitaan, sillä ne ovat suosittuja~\citep{tiobe} sekä kunkin kielen historiassa
on ollut tavoitteena korvata C:n käyttö. Ohjelmistojen toteutuskielen
valintaprosessia käsitellään artikkelin \emph{Empirical Analysis of Programming
Language Adoption}~\citep{empiricalpopularity} avulla, jossa tutkitaan
kyselyillä erilaisia syitä ohjelmointikielen valintaan.

Neljännessä luvussa esitellään vertailtavat kielet ja käsitellään näiden
kielten ominaisuuksien tehokkuutta ja yhteensopivuutta C:n kanssa. Kaikki
vertailtavat kielet sisältävät ominaisuuksia, jotka haittaavat yhteensopivuutta
C:n kanssa. Tämän lisäksi suorituskykymittauksista ilmenee, kuinka kaikkien
kielten toteutukset käyttävät enemmän muistia Benchmarks Gamen vertailuissa.

Viidennessä luvussa määritellään uusi ohjelmointikieli, Purkka. Kielen
suunnittelussa painotetaan erityisesti C-yhteensopivuutta esimerkiksi C-kielen
esikäsittelijätuen kautta. Suunnittelussa pyritään parantamaan C:tä erityisesti
syntaksin osalta, mutta myös tarjoamalla vahvempaa tyypitystä. Purkka-kieli
käännetään C-kieleksi, jotta kielen yhteensopivuus olisi mahdollisimman hyvä
nykyisten ohjelmistojen yhteydessä.

Kuudennessa luvussa verrataan Purkka-kielellä toteutettuja Benchmarks Gamen
ohjelmia muihin kieliin. Suorituskykymittauksissa Purkka pysyy yhtä tehokkaana
kuin C, mutta Purkan lähdekooditiedostot ovat noin kuusi prosenttia pienempiä
verrattuna vastaaviin C-tiedostoihin. Kuudennessa luvussa myös arvioidaan
tutkielman oikeellisuutta ja pohditaan jatkotutkimuskohteita.

Seitsemännessä luvussa kerrataan tutkielman tulokset.

\newpage
\section{C-ohjelmointikielen taustaa}

\hl{Miten C on siirtynyt tähän päivään? Miten C:tä voisi parantaa?}

\subsection{C-ohjelmointikieli lyhyesti}
\label{sec:clyhyesti}

\hl{Lähteitä lisää.}

\begin{listing}[ht!]
    \inputminted{C}{koodi/hello.c}
    \caption{Yksinkertainen hello world -ohjelma toteutettuna C:llä.}
    \label{fig:helloc}
\end{listing}

C on Dennis Ritchien 1970-luvun taitteessa kehittämä yksinkertainen matalan
tason ohjelmointikieli~\citep{chistory}, jota käytetään nykypäivänä erityisesti
järjestelmäohjelmoinnissa sekä sulautetuissa järjestelmissä. C ei aseta
ohjelmoijalle rajoituksia muistinkäsittelyyn, mikä mahdollistaa tehokkaan mutta
turvattoman ohjelmoinnin. C on \citeauthor{tiobe}:n (\citeyear{tiobe}) mukaan
yksi tällä hetkellä käytetyimmistä ohjelmointikielistä. C ei ole vuoden 1989
ANSI-standardin jälkeen muuttunut mainittavasti~\citep{chistory, C18}, vaan
ohjelma~\ref{fig:helloc} näyttää samalta kuin vuonna 1988 julkaistussa
\emph{The C Programming Language} -kirjassa~\citep{krsecond}. Ohjelma tulostaa
näytölle merkkijonon \emph{Hello,~World}.

C kehitettiin B- ja BCPL-ohjelmointikielten pohjalta näiden osoittautuessa
kömpelöiksi vaihtaessa laitteistoarkkitehtuuria, kuten
luvussa~\ref{sec:ctaustaa} kerrotaan. Uudella ohjelmointikielellä tavoiteltiin
B:tä nopeampaa, mutta BCPL:ää yksinkertaisempaa kieltä Unix-käyttöjärjestelmän
kehitykseen~\citep{chistory}.

Toisin kuin monissa nykypäivänä suosituissa kielissä, C ei tarjoa automaattista
muistinhallintaa, vaan ohjelmoijan täytyy itse hallita sekä muistin varaaminen
että sen vapauttaminen. Tämä on pitänyt kielen toteutuksen hyvin tehokkaana ja
yksinkertaisena, mikä on mahdollistanut C:n leviämisen järjestelmästä toiseen.
Kääntöpuolena muistinhallinta jätetään ohjelmoijan vastuulle, monimutkaistaen
ohjelmien toteutusta.

C:n mahdollistama suorituskyky on johtanut kielen suosioon tehokkuutta
vaativissa sovelluksissa, kuten verkkopalvelimissa ja tietokannoissa.
Toisaalta C:n yksinkertainen lähestymistapa muistinkäsittelyyn on
mahdollistanut tarkkaa muistinhallintaa vaativien ohjelmistojen toteutuksen,
kuten käyttöjärjestelmien ytimien tai sulautettujen järjestelmien ohjelmien.
Tämä yhdistelmä on johtanut C:n laajaan käyttöön käyttötarkoituksesta
riippumatta.

Koska C:llä on helppoa saada aikaan erilaisia muistinkäsittelyyn liittyviä
tietoturvaongelmia, useita työkaluja on kehitetty havaitsemaan ja estämään
näitä~(mm.~\citet{valgrind,asan}). Myös lukuisia ohjelmointikieliä on luotu
toteuttamaan ''turvallinen C'', usein lisäämällä jokaisen osoittimen käytön
yhteyteen tarkistuksen, osoittaako osoitin ohjelman omistamaan muistiin.

\hl{Kieliperheistä liittyvään juttuun ei löydy yhtään järkevää lähdettä. Tästä
bulkkipaperi-idea?}

Koska C on suunniteltu mahdollisimman yksinkertaiseksi kieleksi, siitä ei löydy
omaa moduulijärjestelmää. Mikäli ohjelmoija haluaa käyttää jonkun toisen
tiedostojen sisältämää funktiota, hänen pitää joko kirjoittaa
funktioprototyypit tai sisällyttää ohjelmaan erityinen otsikkotiedosto käyttäen
C:n makrojärjestelmää.

C on vaikuttanut lukuisien kielten kehitykseen, ja useiden kielien sanotaankin
olevan osa C-kieliperhettä. C-kieliperheeseen kuuluviin kieliin liittyy usein
muun muassa imperatiivinen ohjelmointityyli, perinteinen merkintätapa (eli
infix) lausekkeiden muodostamiseen, muuttujien näkyvyysalueiden rajoittaminen
kaarisuluilla sekä staattinen tyypitys. Näitä piirteitä näkyy
ohjelmassa~\ref{fig:helloc}: kaarisuluilla rajoitettu \texttt{main}-funktio
kutsuu \texttt{printf}-funktiota, joka tulostaa käyttäjälle merkkijonon
\emph{Hello,~World}. Funktio ottaa parametriksi kokonaisluvun \texttt{argc} ja
osoittimen merkkijonolistaan \texttt{argv}.

Suurin osa nykypäivänä käytetyimmistä ohjelmointikielistä on osa
C-kieliperhettä, kuten TIOBEn indeksissä~(\citeyear{tiobe}) kärkiviisikosta
löytyvät C:n lisäksi Java, C++ ja C\#. Python on ainoa kärkiviisikossa oleva
ohjelmointikieli, joka ei suoraan ole osa C-kieliperhettä, mutta Pythonin
referenssi-implementaatio CPython on nimensä mukaisesti toteutettu
C:llä ja Pythonilla~\citep{cpython}.

\subsection{C-ohjelmointikielen historiaa ja nykypäivää}
\label{sec:ctaustaa}

\hl{Tässä aliluvussa kerrotaan C-kielen taustasta, alkuperäisistä
käyttökohteista, historiasta, suosion kehityksestä sekä nykytilanteesta.
Erityisesti keskitytään C:n tämänhetkisiin käyttökohteisiin, joista selviää,
mitä ohjelmointikielten ominaisuuksia tulisi käsitellä ohjelmointikielten
vertailussa.}

\hl{ Luvussa 2.2 ihmetyttää, että ovatko nämä asiat nyt keskeisimmän piirteet
C-kielessä? Miksei muita kielen ominaisuuksia selosteta? }

Dennis~\citet{chistory} käsittelee C-kielen historiaa ja tekemiään
suunnittelupäätöksiä artikkelissaan \emph{The Development of the C Language}.
Artikkelin mukaan C kehitettiin pitkälti 70-luvulla B- ja
BCPL-ohjelmointikielien pohjalta. Ritchie tavoitteli uudella
ohjelmointikielellä B-ohjelmointikielen tehokkuuden parantamista tarkemmalla
tyypityksellä. B-kielen ainoa primitiivityyppi oli sana\defword{word\emph{,
B-kielen tyyppinä \texttt{cell}}}, sillä B oli suunniteltu ajettavaksi
tietokoneilla, joissa muistiosoittimet osoittivat aina yksittäisiin sanoihin.
Tämä kuitenkin käytännössä osoittautui haastavaksi esimerkiksi merkkijonojen
käsittelyyn, sillä jokaiseen sanaan mahtuu useita tavun kokoisia merkkejä. Tämä
tarkoittaa ohjelmoidessa sitä, että ohjelmoija joutuu purkamaan käsin
yksittäisen sanan merkeiksi, käsittelemään näitä yksittäin ja lopuksi
yhdistämään merkit uudelleen sanoiksi. Epäkäytännöllisyyden lisäksi tämä oli
tehotonta, kun B-kieli muutettiin tavupohjaiselle PDP-11 -tietokoneelle
säilyttäen kuitenkin sanapohjaiset muistiosoittimet -- ohjelmoijan oli pakko
käyttää sanarajoilla toimivia muistiosoittimia, vaikka tavupohjainen
muistiosoitin olisi tehokkaampi ratkaisu.

B-kielen taulukot poikkesivat huomattavasti nykyisistä C:n taulukoista. Jos
B-koodissa luodaan kymmenen alkion taulukko \texttt{A}, niin ohjelmaa
ajettaessa varattaisiin kymmenen sanan kokoinen taulukko ja \texttt{A}:han
asetettaisiin tämän taulukon osoitin. C:ssä \texttt{A} muutettaisiin
osoittimeksi määrittelyhetken sijaan vasta, kun muuttujaa käytettäisiin
lausekkeessa.

Tämän lisäksi Ritchien mukaan B-ohjelmointikielestä puuttui kokonaan
liukulukujen käsittely. Vaikka PDP-11 ei tukenut liukuluvuilla laskentaa,
valmistaja oli luvannut tuen tälle. BCPL-ohjelmointikieleen lisättiin
liukulukulaskenta olettaen, että liukuluku mahtuisi yhteen sanaan, mikä ei
pitänyt paikkaansa 16-bittisellä PDP-11 -tietokoneella.

\grayrule

Artikkelin mukaan C-ohjelmointikielen kehitys alkoi vahvemmalla tyypityksellä:
C-kielen ensimmäiseen varsinaiseen esiasteeseen oli lisätty \texttt{char}- ja
\texttt{int}-tyypit sekä muistiosoittimet näihin tyyppeihin. Tässä vaiheessa
C-tyylisten taulukoiden sijaan taulukot toimivat kuin B-kielen taulukot. Kun
C:hen lisättiin tietuetyypit, tämä muistimalli ei toiminut enää, vaan
taulukoiden käsittelyä muutettiin vastaamaan nykystä C:n mallia. Nyt taulukot
muunnettiin muistiosoittimiksi vasta, kun taulukkoa käytettiin lausekkeissa, ja
taulukon määrittelyssä alkioiden ja osoittimen sijaan varattiin muistia vain
taulukon alkioille.

Tästä tavasta käsitellä taulukoita seurasi kuitenkin ominaisuus, joka on myös
nykypäivän C:ssä: jos funktio ottaa parametrikseen taulukon, kääntäjä muuttaa
parametrin tyypin muistiosoittimeksi, sillä funktiokutsut ovat
lausekkeita\footnote{Jos funktio ottaa esimerkiksi \texttt{char[2]} -tyyppisen
parametrin, funktio saakin parametrikseen muistiosoittimen taulukon sijaan.
Tässä tapauksessa funktio saa siis moderneilla tietokoneilla kahden tavun
sijaan kahdeksan tavun kokoisen parametrin. Tämän ominaisuuden voi kiertää
säilömällä taulukon tietueen sisään.}.

Tämän lisäksi C-kieleen lisättiin tyyppi funktio-osoittimille.
Määrittelysyntaksin logiikan perusteena toimi lausekkeiden syntaksi. Jos
jostain muuttujasta saa \texttt{int}-arvon, kun lausekkeessa lukee
\texttt{(*muuttuja)()}, niin \texttt{muuttuja} määritellään kirjoittamalla
\texttt{int~(*muuttuja)();}. Monimutkaisissa tapauksissa on kuitenkin hankala
erottaa tyyppejä toisistaan, kuten erot tyyppien ''osoitin taulukkoon
kokonaislukuja'' eli \texttt{int~(*muuttuja)[]} ja ''taulukko osoittimia
kokonaislukuihin'' eli \texttt{int~*muuttuja[]}.

B-yhteensopivuuden tavoittelu ohjasi kielen suunnittelua syntaksin osalta
mahdollisimman B:n kaltaiseksi. B-lause \texttt{if(a~\&~b)} vastaa C-lausetta
\texttt{if(a~\&\&~b)}, mutta B-kielessä \texttt{\&}:tä käytettiin myös
bittioperaatioihin loogisten operaatioiden lisäksi. Koska B-ohjelmien haluttiin
toimivan C-ohjelmien tavoin mahdollisimman pienillä muutoksilla, C-idiomissa
\texttt{(a\&mask)~==~b} lausekkeessa käytetyn \texttt{\&}-bittioperaattorin
ympärille joutuu lisäämään sulut. Tämä johtuu B-kielen \texttt{if(a == b \& c)}
-tyylisistä lausekkeista, joiden haluttiin toimivan ilman muutoksia C:ssä
kielen vaihtamisen helpottamiseksi.

Seuraavaksi alkoi C-kielen esikäsittelijän kehitys. Aluksi esikäsittelijässä
oli vain toiminnot tiedostojen sisällyttämiseen (\texttt{\#include}) ja
yksinkertaiseen korvaamiseen (\texttt{\#define}), mutta hyvin nopeasti
kieleen lisättiin funktiomakrot sekä \texttt{\#if}-lauseet. Aluksi
esikäsittelijää pidettiin vain vapaaehtoisena laajennoksena C:hen, mikä
selittää myös nykypäivänä esikäsittelijän huomattavat erot muuhun C-kieleen
verrattuna. Ensimmäisen C-standardin jälkeen esikäsittelijä on pysynyt lähes
koskemattomana. Ainoa lisätty ominaisuus oli C99-standardin mukana tulleet
funktiomakrot, jotka voivat ottaa mielivaltaisen määrän argumentteja.

\grayrule

Myöhemmin, kun C oli levinnyt usealle eri alustalle, alkoi olla selkeää, että C
tarvitsi standardin. Brian Kernighan, jonka kanssa Ritchie oli kirjoittanut
\emph{The C Programming Language} -kirjan~\citep{krfirst}, kirjoitti Ritchien
kanssa C:n ensimmäisen standardin ANSIn X3J11 -työryhmässä. Kuuden vuoden
jälkeen työryhmä sai valmiiksi nk. C89 -standardin, joka tunnetaan myös ANSI
C:nä\footnote{ISO-järjestö hyväksyi standardin pienillä muutoksilla vuonna
1990, jonka vuoksi standardi tunnetaan myös C90-standardina.}~\citep{C89}.
Samoihin aikoihin valmistui myös toinen painos \emph{The C Programming
Language} -kirjasta, jossa korjattiin lukuisia eroja ensimmäisen version ja
C-standardin välillä~\citep{krsecond}.

\grayrule

Nykypäivänä C:tä käytetään käytännössä jokaisessa tietokoneessa
käyttöjärjestelmästä riippuen joko pelkästään ydinkomponenttien toteutukseen
tai koko käyttöjärjestelmän toteutukseen. Sulautetuissa järjestelmissä C on
yksi suosituimmista kielistä johtuen kielen yksinkertaisuudesta ja
suorituskyvystä. C:n suosion myötä myös kielen huonot puolet nousevat esiin
erilaisissa tietoturvaongelmissa, jotka johtuvat kielen sallimasta
rajoittamattomasta muistinkäsittelystä yhdistettynä ohjelmoijan tekemiin
virheisiin. Esimerkiksi puskuriylivuodoissa C:llä kirjoitettu ohjelma tallentaa
tietoa muualle tai lukee muistia muualta kuin mitä ohjelmoija on tarkoittanut,
kun ohjelmoinja jättää tekemättä kriittisen syötteen oikeellisuustarkistuksen.
C-kääntäjä voi optimoidessa poistaa tällaisia tarkistuksia ja aiheuttaa
tietoturvaongelmia, jos kääntäjä päättelee tarkistusten olevan
turhia~\citep{redhatsecurity}.

C on hyvin yksinkertainen kieli, ja se on selviytynyt nykypäivään asti lähes
identtisenä ANSI C:hen. Uudemmat standardit~\citep{C99, C11, C18} ovat lähinnä
tehneet pieniä parannuksia kielen tehokkuuteen esimerkiksi lisäämällä
\texttt{restrict}-avainsanan. Erilaiset kääntäjät ovat kuitenkin tuottaneet
omia laajennoksiaan kieleen mahdollistaen tehokkaampien mutta
kääntäjäriippuvaisten C-ohjelmien kirjoituksen. Moderni esimerkki
kääntäjäriippuvaisesta syntaksia muokkaavasta laajennoksista on vektorityypit,
joita esimerkiksi GCC-kääntäjä tukee omalla \texttt{\_\_attribute\_\_(())}
-syntaksillaan. GCC-kääntäjä tukee myös lukuisia muita laajennoksia, kuten
tietueliteraaleja. C:stä löytyy myös standardien mukainen tapa käyttää
kääntäjäriippuvaisia ominaisuuksia, \texttt{\#pragma}. Pragmoja käytetään
erityisesti OpenMP-kirjaston~\citep{openmp} yhteydessä.

Nykypäivänä käytännössä jokainen alusta tukee C:tä. C:tä käytetään alustoilla
muun muassa ohjelmointikielten väliseen kommunikaatioon -- jos C\#-ohjelma
haluaa käyttää Java-ohjelman kirjastorutiineja, C\#-ohjelman on helpointa
käyttää Java-ohjelman C-rajapintaa.

C on nykyään käytössä erityisesti matalan tason ohjelmoinnissa, kuten
käyttöjärjestelmien ytimissä, sulautetuissa järjestelmissä, UNIX-työkaluissa,
vapaan lähdekoodin ohjelmistoissa, tietokannoissa ja muissa tehokkuutta
vaativissa ohjelmistoissa.

\subsection{Tärkeimmät C-ohjelmointikielen ominaisuudet}
\label{sec:cominaisuudet}

Artikkelissa \emph{The Next 7000 Programming Languages}~\citep{next7000}
käsitellään ohjelmointikielten kehitystä ja pohditaan mahdollisia ominaisuuksia
tulevissa ohjelmointikielissä, joita nykyiset ohjelmointikielet eivät sisällä.
Artikkelissa selitetään C:n nykyistä suosiota kielen yksinkertaisuudella ja
tehokkuudella. Nämä ominaisuudet ovat mahdollistaneet C:n laajan käytön
käyttöjärjestelmistä työkaluihin huolimatta C:n turvattomuudesta.

Artikkelin mukaan C:n (ja C++:n) korvaaminen lyhyellä tähtäimellä on mahdotonta
johtuen kielen suosiosta. Koska lukuisat työkalut kääntäjistä
virheenjäljittäjiin\defword{debugger} on kirjoitettu yksinomaan C:tä varten,
vastaavien työkalujen luominen muita ohjelmointikieliä varten on huomattava
investointi. C:n yleisyydestä myös seuraa suuri määrä ohjelmoijia, kirjastoja
ja työkaluja, mikä tekee C:stä luonnollisen valinnan myös uusiin projekteihin.
Artikkelissa kuitenkin todetaan, että useat ohjelmointikielet ovat vhentäneet
C:n suosiota tarjoten yksittäisillä osa-alueilla parannuksia. Yksi mainituista
ohjelmointikielistä on Rust, joka mahdollistaa paremman ohjelmien
käännösaikaisen oikeellisuustarkistamisen heikentämättä tehokkuutta C:hen
verrattuna. Toiseksi vaihtoehdoksi tarjotaan ajoaikaisia tarkistuksia
turvallisuuden parantamiseksi.

Artikkelissa puhutaan myös mahdollisuudesta oikeelisuuden tasaiseen
parantamiseen\defword{gradual verification}, joka mahdollistaisi ohjelmien
ensimmäisten versioiden toteuttamisen ilman kattavaa käännösaikaista
oikeellisuustarkistusta, mutta antaen kehityksen jatkuessa työkalut ohjelmiston
oikeellisuuden varmistamiseen. Esimerkiksi
TypeScript~\citep{typescript}-ohjelmointikieli mahdollistaa tyypittämättömien
JavaScript-ohjelmien vaillinaisen tyypityken\defword{gradual typing}, jolloin
ohjelmien oikeellisuutta voi käännösaikaisesti tarkistaa funktio kerrallaan. 

Artikkelissa \emph{Some Were Meant For C} \citet{somemeantforc} myös nostaa
esiin tarpeen yksittäisten funktioiden kerrallaan muuntamisesta. Useissa
kielissä ei ole saumatonta C-yhteensopivuutta, jolloin kielestä toiseen
siirtyminen vaatii joko koko ohjelman uudelleenkirjoituksen tai erillisen
yhteensopivuuskerroksen alkuperäisen ja uuden kielen väliin. Koska lukuisat
työkalut ja kirjastot on toteutettu C:tä varten, C tulee pysymään jatkossakin
tärkeänä osana tietokoneita.

Yhteensopivuuden sijaan \citeauthor{somemeantforc} kuitenkin painottaa C:n
alusta-agnostisuutta kielen tärkeimpänä ominaisuutena. Useilla alustoilla
voi normaalin välimuistin lisäksi käyttää laitteistoa tai tiedostojärjestelmää
suoraan muistiosoitteina, minkä C:n yksinkertainen lähestymistapa
muistinhallintaan mahdollistaa. Artikkelissa esitetään tästä esimerkkinä
iteraation ohjelman omien konekielisen käskyjen yli, mikä muissa kielissä olisi
kohtuuttoman hankalaa, mutta C:ssä triviaalia.

Molemmissa artikkeleissa käsitellään C:n määrittelemätöntä toimintaa kielen
sekä hyvänä että huonona puolena. Lukuisat tietoturvaongelmat ovat johtuneet
C:n turvattomuudesta, mutta toisaalta kielen turvattomuutta voi käyttää hyväksi
mahdollisimman tehokkaiden ohjelmien toteutuksessa.

\subsection{Kehitettävissä olevat ominaisuudet C-ohjelmointikielessä}
\label{sec:ckehitettavat}

\hl{Tässä aliluvussa pohditaan, miksi C pitäisi korvata uudella
ohjelmointikielellä, mitä ominaisuuksia C:stä voisi kehittää, mitkä C:n
ominaisuudet voisi jättää pois ja mitä uusia ominaisuuksia voisi tulla.}

\hl{Näitä ominaisuuksia ovat ainakin tarkempi käännösaikainen tyypitys etenkin
tyhjien osoittimien osalta, makrojärjestelmän uusiminen sekä syntaksin
selkeyttäminen.}

\citeauthor{chistory} (\citeyear{chistory}) nostaa esiin kaksi usein
keskustelua herättänyttä C:n ominaisuutta. Toinen näistä on C:n tyyppisyntaksi,
ja toinen on C:n tapa käsitellä taulukoita ja osoittimia keskenään.
Ensimmäiselle näistä voi tehdä verrattaen helposti jotain, sillä syntaksin
muuntaminen käännösaikaisesti on triviaalia. Taulukoiden ja osoittimien välistä
käytöstä on paljon hankalampi muuntaa, sillä olemassa olevat C-ohjelmat
käyttävät osoittimia ja taulukoita hyvin sekalaisesti. Yksi C:stä puuttuva
ominaisuus on taulukon antaminen funktion parametrina, jonka voi kuitenkin
tehdä käärimällä taulukko tietueen sisään.

Muita syntaktisia parannuksia lausekkeisiin voi tehdä tietueiden kohdalla. Jos
lausekkeessa käyttää tietuetta, niin tietueen \texttt{foo} jäsenen \texttt{bar}
saa lausekkeella \texttt{foo.bar}, mutta jos \texttt{foo} onkin osoitin,
lausekkeen tulee olla \texttt{foo->bar}. Jos \texttt{foo} olisi osoitin
osoittimeen, lauseke olisi \texttt{(*foo)->bar}. Yksinkertaisempi syntaksi
olisi käyttää jokaisessa tapauksessa lauseketta \texttt{foo.bar}, ja jättää
tarvittava muoto kääntäjän pääteltäväksi. Esimerkiksi Rust-ohjelmointikielen
syntaksi tietueiden käsittelyyn on tällainen.

C:n \texttt{static}-avainsana on jaettu käytön mukaisesti kahteen avainsanaan.
Funktiomäärittelyissä ja globaaleissa muuttujissa C:n
\texttt{static}-avainsana vastaa muiden kielten avainsanaa yksityiselle
funktiolle. Globaaleja \texttt{static}-määreellä määritettyjä funktioita ja
muuttujia ei voi käyttää muusta kuin samasta tiedostosta, jossa kyseinen
symboli on määritelty.

Toinen \texttt{static}-määreen käyttö on funktioiden sisällä muuttujien
määrittelyyn. Staattiset muuttujat alustetaan vain kerran, vaikka funktiota
kutsuttaisiin useita kertoja. Ohjelmassa~\ref{fig:cstatic} käytetään tällaista
muuttujaa. Ohjelma tulostaa ensin numeron 0, jonka jälkeen ohjelma tulostaa
numeron 1.

\begin{listing}[ht!]
    \inputminted{C}{koodi/static.c}
    \caption{Staattinen muuttuja C:ssä}
    \label{fig:cstatic}
\end{listing}

\grayrule

C:n tyypitys sallii monia ilmaisuja, jotka voivat johtaa salakavaliin ongelmiin
suoritusaikaisesti. Koska C sallii suurempien kokonaislukutyyppien asettamisen
pienempiin kokonaislukutyyppeihin, nämä arvot voivat suoritusaikaisesti muuttua
ilman käännösaikaisia varoituksia. Mainittavasti lauseet \texttt{int a = 256;
unsigned char b = a;} eivät aiheuta käännösaikaisesti edes varoituksia
GCC-kääntäjällä, vaikka ''kaikki'' kääntäjän varoitukset olisivat päällä
\texttt{-Wall} -komentolipun avulla. Kääntäjälle pitää antaa erillinen
\texttt{-Wconversion} -komentolippu, jotta kääntäjä edes varoittaisi
mahdollisesti yllättävästä käytöksestä. Tämän ominaisuuden muuttaminen
virheeksi tai edes varoitukseksi ei ole ongelmatonta, sillä esimerkiksi
C-makrojen tuottama koodi voi olettaa tällaisten lausekkeiden toimivan.

C:hen voisi myös lisätä useita uusia tyyppejä, erityisesti tyypin ei-tyhjälle
osoittimelle. Tätä tyyppiä voisi käyttää turvallisempien ja nopeampien
ohjelmien kirjoittamiseen. GCC- ja Clang-kääntäjät tukevat ei-tyhjiä osoittimia
\texttt{\_\_attribute\_\_((nonnull))} -määreellä. Kääntäjä voi käyttää määrettä
optimoidessa funktioita -- erityisesti turhat tarkistukset tyhjien muuttujien
varalta optimoidaan pois.

Muita hyödyllisiä tyyppejä ohjelmoinnin helpottamiseen olisivat
monikot\defword{tuple} ja summatyypit\defword{tagged union, sum type}. Nämä
molemmat tyypit löytyvät esimerkiksi Rustista. Ensimmäisen saa käännettyä
triviaalisti tietueeksi, ja toisen saa käännettyä tietueeksi, jossa on sisällä
vaihtoehdoista luetelman\defword{enumeration} ja yhdisteen\defword{union}
yhdistelmä. Ohjelmassa~\ref{fig:csumtype} on Rustin ja C:n syntaksien mukaiset
summatyypit tyypille, jossa on joko operaation onnistuessa jokin kokonaisluku
tai epäonnistumisen yhteydessä epäonnistumisen syy merkkijonona. Erillisen
summatyypin määritteleminen mahdollistaa yhtä aikaa sekä paremman
oikeellisuustarkistuksen että tehokkaampien ohjelmien tuottamisen.
C-esimerkissä \texttt{failure}-kentässä voisi olla arvo, vaikka
\texttt{type}-kentän arvo olisi \texttt{\_SUCCESS}. Tämä voi johtaa
määrittelemättömään toimintaan, jos \texttt{success}-kentän arvoa yritetään
käyttää ja sillä hetkellä alustettu kenttä on \texttt{failure}.

\begin{listing}[ht!]
    \inputminted{Rust}{koodi/rustenum.rs}
    \inputminted[firstline=3]{C}{koodi/csumtype.c}
    \caption{Summatyyppi Rustissa ja C:ssä.}
    \label{fig:csumtype}
\end{listing}

%C:n kaltaisiin kieliin saa toteutettua helposti tyyppipäättelyn, sillä C:n
%tyypit ovat tyyppitasolla yksinkertaisia. Koska C:ssä tyypit eivät ole
%geneerisiä, lausekkeiden tyyppipäättely on lähes triviaalia paria operaatiota
%lukuun ottamatta. Tyyppipäättely helpottaa ohjelmointia, sillä ohjelmoija
%voi keskittyä ohjelmien logiikkaan muuttujien tyyppien kirjoittamisen sijaan.
%\hl{Viittaus C:n dokseihin?}

C:ssä ei ole erillistä moduulijärjestelmää, jonka lisääminen voisi nopeuttaa
kääntämistä ja tehdä ohjelmien ymmärtämisestä yksinkertaisempaa. C-ohjelmat
voivat käyttää kirjastojen funktioita kirjoittamalla kirjastofunktioista
prototyypit, jotka yleisesti ottaen ovat kirjastojen otsikkotiedostoissa. Koska
otsikkotiedostojen sisältö käytännössä ottaen kopioidaan
\texttt{\#include}-kutsun tilalle, kääntäjä joutuu käsittelemään samoja
otsikkotiedostoja lukuisia kertoja käännösprosessin aikana\footnote{Käytännössä
kääntäjät pystyvät optimoimaan tietyllä yleisellä tavalla kirjoitettuja
otsikkotiedostoja ja jättämään jo kertaalleen luetut tiedostot kokonaan pois.}.
Erillisen moduulijärjestelmän lisääminen voisi kääntämisen nopeuttamisen
lisäksi yksinkertaistaa kirjastojen toteuttamista, sillä ohjelmoijien ei
tarvitsisi kirjoittaa kirjastoilleen otsikkotiedostoja. 

C:n makrojärjestelmässä on paljon parantamisen varaa. C:n esikäsittelijä toimii
hyvin yksinkertaisissa tapauksissa, mutta sen rajoitteet tulevat nopeasti
esille monimutkaisempia makroja kirjoittaessa. Voimakkaampi makrojärjestelmä
mahdollistaa lyhyempien ja selkeämpien makrojen kirjoittamisen
monimutkaisemmissa tapauksissa.
%Uutta makrojärjestelmää kirjoittaessa tulee
%kuitenkin pitää mielessä C-yhteensopivuus. \citeauthor{cabuse}
%(\citeyear{cabuse}) esittelee artikkelissaan tapoja hyväksikäyttää C:n
%esikäsittelijää esimerkiksi iteraation toteuttamiseen. Monimutkaisempi
%makrojärjestelmä voisi muuntaa voimakkaampaa kieltä C:n esikäsittelijän
%makroiksi käyttäen artikkelin mukaisia C-makroja, jolloin C-ohjelmat voisivat
%käyttää näitä yksinkertaisemmalla kielellä kirjoitettuja makroja. C:n standardi
%on kuitenkin hyvin avoin makrojen käytöksestä rajatapauksissa, joten
%monimutkaiset makrot eivät välttämättä toimi kääntäjäriippumattomasti, vaikka
%ne olisivatkin standardin mukaisia.


C:n makrojärjestelmä myös hankaloittaa työkalujen kirjoittamista. Koska
jokainen pieni muutos lähdekoodiin voi vaikuttaa radikaalisti saatavilla
oleviin symboleihin, esimerkiksi automaattitäydennyksen tarjoavien työkalujen
kirjoittaminen on hankalaa. Standardoitu moduulijärjelmä voisi myös
yksinkertaistaa kääntämistyökalujen kirjoittamista.\hl{relevanssi?}

\newpage
\section{Taustaa}

\hl{Ohjelmointikielten piirteistä ja vertailukriteereistä}

\hl{Tässä luvussa kerrotaan ohjelmointikielten taustoista ja yleisistä
ominaisuuksista, luodaan määrittelyt vertailuun liittyville kriteereille,
perustelut verrattavien kielten valinnoille sekä pohditaan erilaisia
mahdollisia syitä kielten suosioon. Tämän luvun sisältö on hyvin samankaltainen
tutkimussuunnitelman kirjallisuusosan toisen luvun kanssa.}

\subsection{C-ohjelmointikielen taustaa}
\label{sec:ctaustaa}

\hl{Tässä aliluvussa kerrotaan C-kielen taustasta, alkuperäisistä käyttökohteista,
historiasta, suosion kehityksestä sekä nykytilanteesta. Erityisesti keskitytään
C:n tämänhetkisiin käyttökohteisiin, joista selviää, mitä ohjelmointikielten
ominaisuuksia tulisi käsitellä ohjelmointikielten vertailussa.}

C-kielen taustaa:
Milloin tehtiin, mihin tarkoitukseen? Kuka?

Kehitys:

Nykytilanne

Käyttökohteet

\subsection{Kehitettävissä olevat ominaisuudet C-ohjelmointikielessä}

\hl{Tässä aliluvussa pohditaan, miksi C pitäisi korvata uudella
ohjelmointikielellä, mitä ominaisuuksia C:stä voisi kehittää, mitkä C:n
ominaisuudet voisi jättää pois ja mitä uusia ominaisuuksia voisi tulla.}

\hl{Näitä ominaisuuksia ovat ainakin tarkempi käännösaikainen tyypitys etenkin
tyhjien osoittimien osalta, makrojärjestelmän uusiminen sekä syntaksin
selkeyttäminen.}

Tyyppisyntaksi (muuttujan nimi keskellä tyyppiä)

Tyyppien nurkkatapaukset:
\begin{itemize}
    \item Array onkin pointer jos se on parametrina (paitsi jos structin sisällä)
    \item char vs signed char vs unsigned char
    \item assignment: uint = int == int = uint == "ok"
\end{itemize}

Modernit tyypit:
\begin{itemize}
    \item Non-nullable pointer
    \item Tuple (shorthand structille)
    \item Array parametrina (shorthand structiin)
    \item Tagged union (shorthand struct, jossa union+enum)
\end{itemize}

Tyyppi-inferenssi

Makrot

Importit

Tooling

\subsection{Ohjelmointikielten vertailun kriteerit}
\label{sec:abs}

\hl{Jotta kielten vertailun voi tehdä objektiivisesti, valitaan vertailua
varten jotkin kriteerit. Kriteereiksi valitaan suorituskyky, muistinkäyttö ja
yhteensopivuus C:n ja muiden ohjelmointikielten kanssa. Vertailtavat kielet
eivät saa olla C:tä huonompia millään osa-alueella.}

Verrattavissa ohjelmointikielissä on pyritty parantamaan C:n huonoja puolia
hyvien puolien kustannuksella, usein lisäämällä kieleen turvallisuutta
parantavia ominaisuuksia tai tehden kielestä helppokäyttöisemmän esimerkiksi
automaattisella muistinhallinnalla. Tämä kuitenkin heikentää kielen tehokkuutta
tai alustariippumattomuutta tehden kielten suorasta vertailusta hankalaa.
Määrittelemällä absoluuttiset reunaehdot voidaan vertailla kieliä ehtojen
puitteissa objektiivisesti. Jos yksikin näistä kriteereistä ei pidä, verrattava
kieli ei ole aidosti C:tä parempi, vaan se häviää C:lle joissain osa-alueissa
ja vastaavasti voi olla parempi joissakin toisissa.

Tutkielmassa vertaillaan kolmea osa-aluetta ohjelmointikielistä: suorituskykyä,
muistinkäyttöä sekä yhteensopivuutta C:n ja muiden ohjelmointikielten kanssa.
Tutkielmassa käsitellään myös subjektiivisia kielten ominaisuuksia, kuten
kielen tiiviyttä\defword{terseness, expressiveness}, mutta näitä
ominaisuuksia ei huomioida kielten paremmuusvertailussa.

Ohjelmointikielellä kirjoitetun ohjelman tulee olla suoritusaikaisesti
vähintään yhtä nopea kuin vastaava C:llä kirjoitettu ohjelma. Kieli siis ei saa
vaatia ohjelmoijaa käyttämään mitään kielen ominaisuuksia, jotka voisivat
hidastaa ohjelmien suoritusta C:hen verrattuna. Monet suoritusaikaiset
turvallisuutta lisäävät ominaisuudet, kuten muistialueiden tarkistukset,
hidastavat kielen suoritusaikaista nopeutta.

Ohjelmointikielellä toteutettu ohjelma ei myöskään saa käyttää enempää muistia
niin suoritusaikaisesti kuin talletusvälineelläkään verrattuna vastaavaan
C-ohjelmaan. Tämä koskee myös vakiokirjastoa\defword{standard library} --
yksikin konkreettinen vakiokirjaston funktio linkitettynä ohjelmaan kasvattaa
ohjelman kokoa. Mikäli jokin vakiokirjasto toteutetaan, sen käyttäminen tulee
olla ohjelmoijalle täysin vapaaehtoista. Yksi tapa toteuttaa tämä on liittää
vain käytetyt funktiot osaksi ohjelmaa, jolloin käyttämättömät funktiot eivät
kasvata ohjelman kokoa.

Jos samaa vakiokirjastoa käytetään useassa ohjelmassa, tilankäytön kannalta on
edullisempaa säilöä vakiokirjasto jaettuna kirjastona, mutta tämä heikentää
kääntäjän mahdollisuuksia käännösaikaiseen optimointiin. Jos vakiokirjastoa ei
ole ladattu muistiin ohjelman käynnistyessä, ohjelman käynnistys voi kestää
hieman kauemmin. C:n vakiokirjasto liitetään usein moderneissa
käyttöjärjestelmissä ohjelmiin jaettuna kirjastona, sillä C:tä käytetään lähes
jokaisessa käyttöjärjestelmän ohjelmassa, ja näin jaetun kirjaston käyttäminen
välttää osan jaetun kirjaston huonoista puolista: kirjasto löytyy vain yhtenä
kopiona kiintolevyltä, ja se on valmiina ladattuna välimuistiin.

\newpage

Ohjelmointikielen tulee olla täysin yhteensopiva C:n suoritusympäristön kanssa.
Tämä koskee C-koodin kutsumista C:n vierasfunktiorajapinnan
läpi\defword{Foreign function interface, FFI, \emph{käännös
\citealt[s.~8]{vierasfunktiorajapinta}}} sekä kielen funktioiden kutsumista
muiden ohjelmointikielten C-rajapinnan läpi. Kielen tulee näiden lisäksi toimia
kaikissa ympäristöissä, joissa C toimii. Kielen olisi myös hyvä tukea C:n
esikäsittelijää, jotta C:n käyttäminen verrattavan ohjelmointikielen kanssa
olisi mahdollisimman saumatonta.

\subsection{C:hen verrattavissa olevat ohjelmointikielet}

\hl{Vertailtaviksi kieliksi valitaan Ada, C++, D, Go ja Rust. Kaikki
vertailtavat kielet ovat kohtalaisen tehokkaita, jonka lisäksi kaikilla on
yritetty korvata C:n tai C++:n käyttöä.}

Historian saatossa on tehty useita C:n kilpailijoita, jotka ovat yrittäneet
parantaa C:tä joidenkin C:n hyvien puolien kustannuksella. Muutamat näistä ovat
päätyneet hyvin suosituiksi ohjelmointikieliksi, kuten esimerkiksi C++ ja Go.
Kielten suosion mittaamiseen on tehty useita projekteja, jotka vertailevat
kieliä esimerkiksi hakutulosten tai projektien mukaan. Näitä ovat esimerkiksi
TIOBEn ohjelmointikielten suosion indeksi~\citep{tiobe} ja GitHub-palvelun
julkaisema \mbox{Octoverse}~\citep{octoverse}.
%Valitsemalla vertailuun suosittuja kieliä voidaan tutkia, miksi muuten
%selkeästi täysin käyttökelpoinen kieli ei ole syrjäyttänyt C:tä.

Koska tutkimuskysymyksessä vertaillaan ohjelmointikieliä suorituskyvyn ja
muistinkäytön suhteen, vertailuun kannattaa ottaa mukaan vain tehokkaita kieliä
-- korkeamman tason ohjelmointikielet on tarkoitettu ohjelmointinopeuden
parantamiseen ja turvallisempien ohjelmistojen toteuttamiseen nopeiden
ohjelmien sijaan. Tällöin kielen suorituskyky on heikompi. Yksi kattava
suorituskykyä mittaava vertailu on Benchmarks Game~\citep{benchmarks}, jossa
pyritään kirjoittamaan mahdollisimman nopea ohjelma pysyen silti kielelle
idiomaattisessa lähdekoodissa\footnote{Hyvin monessa kielessä voi kirjoittaa
C:hen verrattavissa olevaa matalan tason ohjelmointia, mutta Benchmarks Gamessa
on tarkoituksena välttää tätä.}.

TIOBEn listasta Ada, C++, D, Go ja Rust nousevat esiin verrattavina kielinä.
Kaikki viisi kieltä ovat tehokkaita. Tämän lisäksi jokaisen kielten historiassa
on ollut tavoitteena korvata C:n tai C++:n käyttö, kuten luvussa~\ref{sec:muut}
kerrotaan.

Viime vuosina on tehty myös useita C:hen käännettäviä ohjelmointikieliä, jotka
ovat jääneet pitkälti ilman mitään näkyvyyttä, kuten LISP/c~\citep{clisp1},
C-Mera~\citep{clisp2}, Carp~\citep{clisp3} ja Nymph~\citep{nymph}. LISP/c,
C-Mera ja Carp ovat LISP-perheeseen kuuluvia C:ksi kääntyviä ohjelmointikieliä,
jotka pyrkivät parantamaan C:n syntaksia korvaamalla sen LISP-perheen
syntaksilla. Nymph taas on olio-ohjelmointikieli. Erityisesti Carp on tämän
tutkielman kannalta kiintoisa ohjelmointikieli, sillä se on C:ksi kääntyvä
ohjelmointikieli, joka on suunniteltu mahdollisimman suorituskykyiseksi.

%\hl{Tässä voisi olla jotain pohdintaa, miksi nuo ovat jääneet huomiotta ja
%miten tämän voisi välttää. Erityisesti Carp tuntuu tätä tutkielmaa vastaavalta
%kieleltä, olisi ikävää jättää se täysin huomiotta. Tosin tällaiset pohdinnat
%jäänee puhtaasti spekulaatioksi, sillä noista ei ole erityisen paljoa dataa
%saatavilla.}

\subsection{Kielten suosioon vaikuttavat tekijät}

\hl{Analyysiä muista syistä kielen suosioon, erityisesti Meyerovichin ja
Rabkinin tekemän tutkimuksen perusteella. Tämä antaa tietoa ominaisuuksista,
jotka ohjelmointikielessä olisi hyvä olla.}

Eräässä tutkimuksessa~\citep{empiricalpopularity} tutkittiin syitä
ohjelmointikielten valintaan. Yhden tutkimuksen järjestämän kyselyn (s. 8,
\mbox{Slashdotissa} julkaistu kysely, n=1679) perusteella kielen valintaan
vaikuttaa avoimen lähdekoodin kirjastojen saatavuus, olemassa olevien ohjelmien
jatkokehitys sekä kielen tunnettavuus ohjelmoijien keskuudessa -- tutkimuksen
mukaan ohjelmoijat siis suosivat jo käytettyjä ohjelmointikieliä uusien kielten
sijaan. Saman kyselyn vastaajat arvioivat suorituskyvyn turvallisuutta
tärkeämmäksi.

Saman tutkimuksen järjestämässä Slashdot-sivustolla julkaistussa kyselyssä noin
40\% vastaajista arvioi tärkeäksi kriteeriksi työkalut. Kyselyn perusteella
kielen olisi siis hyvä tarjota toimivat työkalut, kuten
ohjelmistopakettien~\defword{software package, \emph{käännös
\citealt[s.~18]{ohjelmistopaketti}}} hakemiseen
paketinhallintajärjestelmän\defword{package manager}, nopean ja
käyttäjäystävällisen käännöstyökalun sekä valmiudet olemassa oleviin
kehitysympäristöihin integroitumiselle. Olemassa olevien C-ohjelmistojen
tukeminen on välttämätöntä mutta haastavaa johtuen C:n ekosysteemin
monimuotoisuudesta, erityisesti lukuisista kääntämistyökaluista.

Tutkimuksessa selvitettiin myös suosittuja ominaisuuksia ohjelmointikieliltä
(s. 13, SaaS MOOC -kurssin yhteydessä oleva kysely, n=415). Useita
tutkimuksessa selvitetyistä suosituimmista ominaisuuksista ei ole mahdollista
toteuttaa johtuen luvussa~\ref{sec:abs} määritetyistä rajoitteista, kuten
poikkeuksia ja rajapintoja. Useat muut tutkimuksessa esiin nousseet
ominaisuudet, kuten suorituskyky, ovat taas suoraan rajotteiden mukaisesti osa
verrattavan ohjelmointikielen tavoitteita.

Tutkimuksessa myös verrattiin tiettyjen toteamuksien keskinäistä korrelaatiota.
Kielen tiiviys korreloi eniten (korrelaatiokertoimella 0.76) kielestä pitämisen
kanssa (s. 13, The Hammer Principle -sivustolla julkaistu kysely).

\hl{? (alleviivattu toteamuksien)}

%Eräässä blogikirjoituksessa \citep{microsoftdictperf} vertaillaan C\#:n ja
%C++:n suorituskykyä. Kirjoituksessa C\#:llä kirjoitettu yksinkertainen toteutus
%suoriutuu tehtävästä nopeammin ja yksinkertaisemmin kuin optimoitu C++-ohjelma.
%Vasta usean C++-ohjelma päihitti C\#-toteuden nopeudella. C\#-toteutus
%kuitenkin vei noin neljä kertaa enemmän muistia kuin C++-toteutus.

Tutkimuksen perusteella valmiiksi suosittuja kieliä käytetään enemmän myös
uusissa projekteissa. Olemassa olevien kirjastojen tärkeyttä korostetaan
useassa kohdassa tutkimusta. Täysin C:n kanssa yhteensopiva kieli voi käyttää
C:lle tehtyjä kirjastoja, jolloin kielellä on käytettävissään laaja C:n
ekosysteemi\footnote{Esimerkiksi GitHubista hakusanalla 'library' löytyy yli
26\,000 C:llä kirjoitettua projektia.}.

Uusia ohjelmointikieliä opetellessa ohjelmoijat turvautuvat aikaisemmista
kielistä opittuihin käytäntöihin~\citep{languagelearning}. Uusien
ohjelmointikielten käyttöönottoa helpottaa siis muiden vastaavien kielten
osaaminen, sillä aikaisempi kokemus tukee uuden kielen opiskelua.
Suunnittelemalla C:n korvaajan C:n kanssa samankaltaiseksi kieleksi voidaan
pienentää uuden kielen opettelun kynnystä. Kielen eriävät ominaisuudet olisi
siis hyvä toteuttaa siten, että ne ovat mahdollisimman helppoja oppia C:stä
uuteen ohjelmointikieleen siirtyvälle ohjelmoijalle.

%Helppokäyttöisyys \\
%- turvallisuus (esim. tyypit) \\
%- esim. annotaatiot \\
%- selkeä mitä tekee

%\subsection{Mahdollisia kielen ominaisuuksia}
%
%Kielen ominaisuudet vaikuttavat sillä rakennettujen ohjelmistojen
%arkkitehtuuriin. Tyyppiluokat tai rajapinnat kannustavat vahvan
%tyypityksen kautta tyyppiturvalliseen ohjelmointiin, kun taas dynaamiset kielet
%(esimerkiksi LISP-perheen kielet) kannustavat nopeaan kehitystahtiin staattisen
%turvallisuuden hinnalla. Yksinkertaiset kielet, kuten C ja Go, tarjoavat usein
%vain yhden selkeän tavan toteuttaa yksittäiset funktiot, kun taas
%monimutkaisemmat kielet tarjoavat lukuisia vaihtoehtoja.\citationneeded
%
%BitC-ohjelmointikielen sähköpostilistalla käydyssä keskustelussa pohdittiin
%mahdollisia ongelmia vahvan tyypityksen (etenkin tyyppiluokkien) käytöstä
%matalan tason ohjelmoinnissa~\citep{bitc}. Shapiron sähköpostissa todetaan,
%että tyyppiluokkia ei voi toteuttaa ilman suoritusaikaista tukea.
%
%On huomioitava, että lukuisten olemassa olevien C:n kirjastojen, rajapintojen
%ja projektien vuoksi yhteensopivuus C:n kanssa tulee olla saumatonta
%mahdollisten vaihtoehtoisten ohjelmointikielten osalta, jotta kielen
%vaihtaminen olisi mahdollista. Tämä sisältää myös kirkastorutiinien kutsumisen
%muista ohjelmointikielistä, sillä C on lukuisissa järjestelmissä \emph{lingua
%franca}, jonka avulla ohjelmointikielet pystyvät kommunikoimaan keskenään.
%Esimerkiksi Python-ohjelmointikieli~\citep{python} sisältää tuen C-funktioiden
%kutsumiseen~\citep{pythonffi}, jota voi käyttää muiden ohjelmointikielten
%funktioiden kutsumiseen, mikäli käyttäjä kirjoittaa ''sillan'' C-ohjelmana.
%Käytännössä jokaisesta aktiivisesti käytetystä ohjelmointikielestä on
%mahdollista kutsua C-koodia.
%
%Ohjelmointikielten ekosysteemit koostuvat eri tahojen luomista kirjastoista. On
%tärkeää, että näihin kirjastoihin pääsee käsiksi mahdollisimman helposti.

% Kirjastojen lisenssien tulee myös olla yhteensopivia, jotta kirjastoja voi
% käyttää yhdessä toistensa kanssa. Eri ohjelmointikieliekosysteemeissä on
% käytössä erilaisia lisenssejä -- JavaScript-kirjastot ovat usein
% MIT-yhteensopivia, kun taas Java-kirjastot ovat usein Apache 2 -yhteensopivia.
% Liitteessä~\ref{app:github} on taulukko GitHub-verkkopalvelun sisältämien
% julkisten projektien määrä ryhmiteltynä lisenssin ja kielen mukaan.
%
% Kirjastot voi myös julkaista useammalla kuin yhdellä lisenssillä, jolloin
% käyttäjät voivat päättää, mitä lisenssiä haluaa seurata. On kuitenkin
% filosofinen kysymys, onko esimerkiksi Apache2+GPL parempi lisenssi kuin pelkkä
% GPL, sillä Apache2 ei vaadi tiettyjä oikeuksia loppukäyttäjille, kuten pääsyä
% ohjelmiston lähdekoodiin~\citep{apachetldr, gpl3tldr}. Apache2 siis antaa
% ohjelmoijille enemmän vapauksia GPL3:een verrattuna, mutta käyttäjien vapaus
% esimerkiksi muokata ohjelmistoa kärsii tästä.

\subsection{Makrojärjestelmät}

\hl{Tässä aliluvussa selitetään makrojärjestelmien käsite, kerrotaan
makrojärjestelmien tarpeellisuudesta ja käyttökohteista sekä verrataan
vaihtoehtoisia toteutuksia makrojärjestelmiin.}

Ohjelmointikielten makrojärjestelmillä tarkoitetaan ohjelmointikielen
ominaisuuksia, joita voidaan käyttää käännösaikaiseen lähdekoodin luomiseen ja
muuntamiseen. Lähdekoodin luonti käännösaikaisesti voi pienentää huomattavasti
tarvittavaa lähdekoodin määrää tietyissä tilanteissa aiheuttamatta
suoritusaikaisia haittoja. Ensimmäiset makrojärjestelmät rakennettiin
symbolisen konekielen käsittelyyn, jolloin makrojärjestelmiä käytettiin
poistamaan turhaa toistoa ohjelmien lähdekoodista.

Rust käyttää vakiokirjastonsa lähdekoodissa paljon makroja erityisesti
alkeistyyppien liittyvien määrittelyjen kohdalla. Rustin vakiokirjaston
kokonaislukutyyppejä määrittelevässä moduulissa \texttt{std::num} on käytetty
makroja huomattavan paljon lähdekoodin määrän vähentämiseen -- tiedoston
\texttt{mod.rs} lähdekoodin koko kasvaa noin 5\,000 rivistä yli 20\,000 riviin,
kun makrot laajennetaan. Kaikki makroilla määritellyt funktiot ovat identtisiä
tyyppejä tai tiettyjä vakioita lukuun ottamatta, ja makrojärjestelmän käytöllä
taataan paras mahdollinen suorituskyky. Makrojärjestelmän käyttäminen
määrittelyihin pakottaa yhtenäiset määrittelyt kaikille alkeistyypeille,
jolloin ohjelmointivirheiden määrä pysyy mahdollisimman pienenä.

\hl{? (alleviivattu lause funktioiden identtisyydestä)}

Lähdekoodin luominen käännösaikaisesti voi kasvattaa lopullisen ohjelman kokoa,
sillä makron sisältö kopioidaan makrokutsun paikalle. Toisaalta ohjelmakoodin
luonti voi myös mahdollistaa optimointeja, joita kääntäjä ei olisi pystynyt
tekemään tavalliselle funktiolle. Kääntäjästä riippuen lähdekoodin
sijoittaminen funktiokutsun paikalle voi esimerkiksi mahdollistaa paremman
silmukoiden vektorisoinnin tai kuolleen koodin poistamisen~\citep{cinlining}.
Ohjelmoijalle tulisi siis antaa mahdollisuus päättää, haluaako hän
ohjelmoidessaan käyttää ohjelmointikielen makrojärjestelmää vai perinteisiä
funktioita.

Makroja käyttäessä voi tapahtua ristiriitoja esimerkiksi luodessa väliaikaisia
muuttujia, joilla on sama tunniste kuin aikaisemmin lähdekoodissa
määritellyillä muuttujilla~\citep{macrohygiene}. Makrojen tulokset voivat myös
aiheuttaa syntaksivirheitä sekä muuta odottamatonta käytöstä. Jos
ohjelmointikielen makrojärjestelmä estää tällaiset ristiriidat, sitä kutsutaan
hygieeniseksi makrojärjestelmäksi.

\hl{? (alleviivattu syntaksivirheitä, odottamatonta käytöstä, estää
ristiriidat), pidempi selitys}

Makrojärjestelmiin on runsaasti vaihtoehtoisia toteutustapoja. Yksinkertaisin
vaihtoehto on jättää makrot kokonaan pois ohjelmointikielestä, mikä tosin
rajoittaa ohjelmointikielen mahdollisuuksia käännösaikaiseen
laajennettavuuteen.

Tekstialkioiden korvaamiseen perustuvat makrojärjestelmät ovat erittäin
yksinkertainen ja joustava tapa toteuttaa käännösaikainen makrojärjestelmä.
Tekstialkioiden korvaamisessa makroprosessori etsii syötteestä tiettyjä
tekstialkioita ja korvaa niitä joukoilla tekstialkioita. Tekstialkioiden
korvaaminen toimii yleensä erillisenä osana kääntämistä, eikä makroilla ole
tällöin tietoa varsinaisen ohjelmointikielen tyypeistä tai tunnisteista.

Mallipohjaisissa makrojärjestelmissä asetetaan tyyppejä tai arvoja ennalta
määritettyyn malliin, yleensä luokkiin tai funktioihin. Mallipohjaiset
makrojärjestelmät eivät voi tehdä kaikkea, mitä tekstialkioiden korvaamiseen
perustuvat makroprosessorit voivat tehdä. Mallipohjaiset järjestelmät voivat
käyttää hyväksi ohjelmointikielen tyyppijärjestelmää ja asettaa käännösaikaisia
rajoitteita mallin parametreille, mikä mahdollistaa paremmat käännösaikaiset
tarkistukset ohjelmien oikeellisuudesta. Monet mallipohjaisen ohjelmoinnin
käyttökohteista ovat suunnattu geneeristen funktioiden tai luokkien luomiseen
eikä syntaksin muuttamiseen.

Syntaksipuupohjaiset makrojärjestelmät käsittelevät ohjelmointikielten
syntaksipuita. Syntaksipuiden käsittely pelkkien tekstialkioiden sijaan
mahdollistaa uusien kontrollirakenteiden määrittämisen yksinkertaisesti ja
turvallisesti. Puhtaassa syntaksipuiden käsittelyssä sekä syötteen että makron
paluuarvon on oltava ohjelmointikielen syntaksin mukaisia, eli syntaksipuita
käsittelevä makro ei voi käsitellä mitä tahansa syötettä.

\hl{paluuarvon -> tuloksen ?}

Monikäyttöisimmät makrojärjestelmät ovat proseduraalisia makrojärjestelmiä.
Proseduraalisissa makrojärjestelmissä ohjelmoija voi käyttää samaa
ohjelmointikieltä sekä tavalliseen ohjelmointiin että makrojen toteutukseen.
Proseduraaliset makrojärjestelmät vaativat yleensä enemmän koodia makrojen
toteutukseen verrattuna muihin makrojärjestelmiin, mutta toisaalta
mahdollistavat kaiken, mitä ohjelmointikielellä voisi tehdä. Proseduraalinen
makrojärjestelmä voisi esimerkiksi lukea tiedostoja käännösaikaisesti ja
vaikuttaa lopulliseen lähdekoodiin tiedoston sisällöstä riippuen.

Mikään makrojärjestelmä ei ole täydellinen, vaan jokaisessa makrojärjestelmässä
on sekä hyviä että huonoja puolia. Kun makrojärjestelmän tehokkuutta
kasvatetaan, sen käyttäminen muuttuu monimutkaisemmaksi.

%\hl{Osa tästä seuraavaan lukuun?}
%
%C:n ja C++:n tekstin korvaamiseen perustuvassa makrojärjestelmissä korvataan
%tekstialkioita ja funktion kaltaisia makroja uusilla sekvensseillä alkioita. Ei
%rekursiota yleisessä tapauksessa, ei turing-täydellinen. Yksinkertainen
%yksinkertaisissa tapauksissa.
%
%C++:n ja D:n mallipohjaisissa järjestelmissä ... C++ erittäin hidas kääntää.
%Turing-täydellinen. D:n mallipohjainen ei turing-täydellinen.
%
%Rust syntaksipohjainen, ottaa tyypitettyjä ja pattern match. Oma DSL, joka
%eroaa jonkin verran varsinaisesta kielestä.
%
%D ja Rust myös proseduraalinen (D: string mixin + ctfe), you can do everything!
%
%C:n ja C++:n esikäsittelijä on tekstialkioiden korvaamiseen perustuva
%makroprosessori.
%
%C++:n ja D:n mallit mahdollistavat metaohjelmoinnin, tosin D vaatii kaikkien
%parametrien olevan tyyppejä.
%
%Rustin makrojärjestelmä perustuu syntaksipuiden käsittelyyn.
%Rustin makrojärjestelmä on käännösaikaisesti tyypitetty, ja sisältää tyypit
%esimerkiksi tunnisteille ja lausekkeille.
%
%Rust sisältää syntaksipuupohjaisen makrojärjestelmän lisäksi kokeellisen
%proseduraalisen makrojärjestelmän.
%

%\subsection{Vertailtavien kielten makrojärjestelmät}
%
%\hl{Tarkista kieli}

%Ada, D ja Go eivät sisällä mahdollisuutta käännösaikaisille makroille. C ja C++
%käyttävät samaa makrojärjestelmää, joka on kuitenkin hyvin
%rajoittunut~(\citeauthor{CPP17}, \citeyear{CPP17}, luku~19; \citeauthor{C18},
%\citeyear{C18}, luku~6.10).
%
%C:n ja C++:n makrot ajetaan ennen muuta koodin käsittelyä, eli makroja voi
%sijoittaa mihin tahansa koodia., kun taas C:n ja C++:n ei\footnote{C:n ja
%C++:n makrojärjestelmä ei ole edes primitiivirekursiivinen, sillä makrot eivät
%voi sisältää itseään.}.

\newpage
\section{Ohjelmointikielten vertailu}
\label{sec:muut}

\subsection{Yleisiä vertailtavien ohjelmointikielten ominaisuuksia}

C:hen vertailtavissa ohjelmointikielissä on yleisesti useita ominaisuuksia,
jotka vaikuttavat ohjelmien suoritusaikaiseen nopeuteen hidastavasti, lisäävät
muistinkäyttöä, vähentävät alustariippumattomuutta tai heikentävät
yhteensopivuutta C:n kanssa.

Yleisin näistä on automaattinen muistinhallinta, joka muistin vapauttamisen
automatisoimiseksi seuraa ohjelman käyttämää muistia. Lähes aina automaattinen
muistinhallinta lisää kieleen ''roskien keräämisen''\defword{garbage
collection, GC}, jonka ajaksi ohjelman suoritus pysäytetään. Lisäksi roskien
keräämiseen perustuva automaattinen muistinhallinta lisää muistinkäyttöä, sillä
ohjelmointikieli joutuu suoritusaikaisesti seuraamaan käytössä olevia
muistiosoitteita.

Monissa vertailtavissa kielissä on käytössä nimiruntelu\defword{name mangling},
joka mahdollistaa useat näennäisesti samannimiset funktiot. Tämä kuitenkin
aiheuttaa ohjelmointikielten välille yhteensopivuusongelmia, sillä toisesta
kielestä kutsuttaessa pitää tietää kutsuttavan funktion todellinen nimi.
Esimerkiksi \texttt{int}-tyyppiä palauttava funktion \texttt{foo()} nimeksi
voisi tulla \texttt{\_Z3foov}, kuten \texttt{g++}-kääntäjä tekee.

Ohjelmointikielen ominaisuudet vaikuttavat siihen, minkälaisia
ohjelmistoarkkitehtuureja kielellä tehdään~\citep{designpatternsdesign}.
Moderneissa ohjelmointikielissä virheiden käsittely on yleensä toteutettu
kahdella tavalla: toinen on poikkeavat paluuarvot ja toinen on poikkeuksien
heittäminen. Yleisesti ottaen kaikki ohjelmointikielet tukevat ensimmäistä ja
suurin osa toista tapaa. Poikkeusten käsittely on hieman hitaampaa ja aiheuttaa
hieman suuremman muistinkäytön, ja tehokkaaseen ohjelmakoodiin pyrkiessä
yleensä vältetään poikkeusten käyttämistä~\citep{exceptioncosts}. Monet
poikkeuksia tukevien ohjelmointikielten vakiokirjastot kuitenkin hallitsevat
virhetilanteita poikkeuksilla, mikä pakottaa ohjelmoijan käyttämään poikkeuksia
ohjelmoidessa.

\hl{tarkista kieli}

Ohjelmointikielten makrojärjestelmillä tarkoitetaan ohjelmointikielen
ominaisuuksia, joita voidaan käyttää käännösaikaiseen koodin luomiseen ja
muuntamiseen. Koodin luonti käännösaikaisesti voi pienentää huomattavasti
tarvittavaa koodin määrää tietyissä tilanteissa. Koodia luodessa voi kuitenkin 
tapahtua konflikteja esimerkiksi luodessa väliaikaisia muuttujia, joilla on
sama tunniste kuin aikaisemmin koodissa määritellyillä
muuttujilla~\citep{macrohygiene}. Luotu koodi voi myös aiheuttaa
syntaksivirheitä sekä muuta odottamatonta käytöstä. Jos ohjelmointikielen
makrojärjestelmä estää tällaiset konfliktit, sitä kutsutaan hygieeniseksi
makrojärjestelmäksi.

%\subsection{Vertailtavien kielten makrojärjestelmät}
%
%\hl{Tarkista kieli}

%Ada, D ja Go eivät sisällä mahdollisuutta käännösaikaisille makroille. C ja C++
%käyttävät samaa makrojärjestelmää, joka on kuitenkin hyvin
%rajoittunut~(\citeauthor{CPP17}, \citeyear{CPP17}, luku~19; \citeauthor{C18},
%\citeyear{C18}, luku~6.10).
%
%C:n ja C++:n makrot ajetaan ennen muuta koodin käsittelyä, eli makroja voi
%sijoittaa mihin tahansa koodia., kun taas C:n ja C++:n ei\footnote{C:n ja
%C++:n makrojärjestelmä ei ole edes primitiivirekursiivinen, sillä makrot eivät
%voi sisältää itseään.}. 

\subsection{Ada}

Ada on Yhdysvaltain puolustusministeriön kehittämä ohjelmointikieli, joka
suunniteltiin korvaamaan kaikki muut puolustusministeriön käyttämät
ohjelmointikielet~\citep{adahistory}, muun muassa C:n. Ada on hyvin moneen
taipuva kieli, sillä se on suunniteltu hallitsemaan monia eri
käyttötarkoituksia matalan tason bittitason ohjelmoinnista korkean tason
arkkitehtuureihin.

Ohjelmoinnin helpottamiseksi Adassa on sekä poikkeukset että automaattinen
muistinhallinta. Nämä kuitenkin hidastavat kieltä hieman aikaisemmin todetuista
syistä. Lisäksi C-kielen kanssa yhteensopivuus on kielen taipuvuudesta johtuen
hankalaa -- jokainen kutsuttava C-funktio on yksitellen määritettävä
kutsukonvention\defword{calling convention} kanssa~\citep[s.~471]{ADA12}. Adan
alustariippumaton C-tuki on kuitenkin äärimmäisen kattava, paikoitellen C:n
omaa tukea kattavampi (C:n standardi ei kuvaile esimerkiksi kutsukonventioita,
vaan ne on jätetty kunkin kääntäjätoteutuksen päätettäväksi). Ada on myös
vertailuin ainoa kieli, joka voi kutsua muilla ohjelmointikielillä
kirjoitettuja kirjastorutiineja suoraan ilman C-rajapintojen käyttöä. Adassa on
C:n lisäksi tuki C++:lle\footnote{C++-tuki ei sisällä nimiruntelun tukemista,
vaan kutsuttavista funktioista pitää määrittää runnellut nimet.}, Fortranille
ja Cobolille~\citep[s.~585]{ADA12}. Adassa ei kuitenkaan ole makrojärjestelmää,
eikä Ada tue C:n makrojärjestelmää.

\subsection{C++}

C++ on Bjarne Stroustrupin 1980-luvusta eteenpäin kehittämä kieli, jonka
yhtenä tarkoituksena on yhdistää Simula-kielen ominaisuudet ohjelman
organisointiin yhteen C:n tehokkuuden ja joustavuuden
kanssa~\citep{cpphistory}. C++ on nykypäivänä suosittu tehokkuutensa ja
monipuolisuutensa takia monimutkaisissa ohjelmistoissa, kuten
palvelinohjelmistoissa, kuvankäsittelyohjelmistoissa sekä
peleissä~\citep{cppapps}.

C++ on kehitetty C:n pohjalta, ja siinä onkin erittäin hyvä C-tuki. Koska
C++\hyp{}funktiot nimirunnellaan eikä nimiruntelua ole määritelty tarkasti C++:n
standardissa, C++-koodia on hankalaa kutsua jopa samalla alustalla eri
C++-kääntäjien välillä. C-koodin otsikkotiedostoissa\defword{header file} on
usein alussa C++-koodia, joka laittaa nimiruntelun pois päältä. Näin
C++-ohjelmat voivat helposti kutsua C:llä kirjoitettujen kirjastojen funktioita
-- C++-ohjelmat voivat usein käyttää C-kielen otsikkotiedostoja ilman muita
muokkauksia.

C++:n standardikirjaston virheidenkäsittely on toteutettu poikkeuksilla, jotka
aiheuttavat pienen hidastuksen. C++:ssa on myös käytettävissä
viitemäärälaskettu\defword{reference counting} muistinhallinta (vakiokirjaston
\texttt{std::shared\_ptr}), jolla voidaan käyttää suoritusaikaisesti varattua
muistia ilman muistivuotoja. C++:n \texttt{std::shared\_ptr} ei käytä erillistä
roskien keräystä, vaan kun viimeinen viite olioon poistetaan, myös varattu
muisti vapautetaan. Tällöin ohjelman suorituksen aikana ei tule
roskienkeräystaukoja.

C++ tukee geneeristä ohjelmointia\defword{generic programming} luokkien
yhteydessä malliohjelmoinnilla\defword{template programming}. C++:n
toteutuksessa jokaisesta uniikista mallin tyyppiparametrikombinaatiosta luodaan
lopulliseen ohjelmaan kopio mallin funktioista, joka kasvattaa ohjelmien kokoa.
Tämä mahdollistaa jokaisen luokan ilmentymän\defword{instance} erillisen
optimoinnin, mutta yleisesti kasvattaa sekä kääntämisaikoja että ohjelmien
kokoa.

\hl{tarkista kieli}

C++ käyttää lähes samaa makrojärjestelmää kuin C. C++:n makrojärjestelmässä on
11 avainsanaa, joita ei voi määrittää uudelleen
esikäsittelijässä~\citep[luku~19.2]{CPP17}\footnote{ Avainsanat ovat
\texttt{and}, \texttt{and\_eq}, \texttt{bitand}, \texttt{bitor},
\texttt{compl}, \texttt{not}, \texttt{not\_eq}, \texttt{or}, \texttt{or\_eq},
\texttt{xor} sekä \texttt{xor\_eq}. }. Tämä tarkoittaa sitä, että C++:n
esikäsittelijä ei hyväksy joitakin C:n esikäsittelijän hyväksymiä makroja,
tosin tämä ei tapahdu käytännössä koskaan. Koska makrojärjestelmä on muuten
sama kuin C:n makrojärjestelmä, se on hyvin rajoittunut \citep[luku~19]{CPP17}.

\subsection{D}

D on 2000-luvun alussa Digital Mars -yrityksen julkaisema ohjelmointikieli,
jonka tarkoituksena on mahdollistaa tehokkaiden ohjelmien kirjoittaminen
helposti ja turvallisesti~\citep{dhistory}. D on suunniteltu syntaksiltaan ja
käytökseltään lähelle C:tä ja C++:aa. Vaikka D-kielessä on olemassa
automaattinen muistinhallinta, D:n \emph{BetterC}-tila tekee kielestä
''paremman C:n'' poistamalla suoritusaikaiset ominaisuudet, mukaan lukien
automaattisen muistinhallinnan~\citep{dbetterc}. Tällöin kielestä poistuu
useita ominaisuuksia, mutta esimerkiksi D:n käännösaikaista makrojärjestelmää
voi käyttää.

C-koodin kutsuminen on melko helppoa, mutta ei aivan saumatonta, sillä jokainen
kutsuttava funktio tulee määritellä erikseen -- D ei ymmärrä C:n
otsikkotiedostoja. Tämä kuitenkin onnistuu yhdellä rivillä jokaista C:n
funktiota kohden, sillä D:n tyyppijärjestelmä on hyvin lähellä C:tä. D:lle on
myös olemassa useita työkaluja otsikkotiedostojen automaattiseen muuntamiseen,
kuten \texttt{htod}-työkalu~\citep{htod}. Työkalut eivät kuitenkaan ole
täydellisiä, sillä useiden D:n ominaisuuksien semantiikka ei ole C:n kanssa
yhteensopiva. D ei sisällä tukea C:n makrojärjestelmälle, eikä D:ssä ole
omaa makrojärjestelmää.

\subsection{Go}

Go on Googlen 2000-luvun loppupuolella kehittämä ohjelmointikieli, jonka
tarkoituksena on yhdistää käännösaikaisesti tyypitetyn ohjelmointikielen
turvallisuus ja tehokkuus suoritusaikaisesti tyypitettyjen ohjelmointikielten
helppokäyttöisyyteen~\citep{gohistory}. Toisin kuin monissa moderneissa
C-perheen kielissä, Go-kielessä ei ole luokkia, vaan pelkkiä tietueita ja
rajapintoja. Go suunniteltiin erityisesti korvaamaan C++:n käyttö Googlella
johtuen C++:n pitkistä käännösajoista.

Go-kielessä ei ole muista vertailtavista kielistä poiketen tyyppiparametreja.
Tämä estää käännösaikaisesti tyyppitarkistetun geneerisen koodin
kirjoittamisen. Ohjelmat voivat kuitenkin suoritusaikaisesti reflektion kautta
tunnistaa muuttujien konkreettisen tyypin. Tämän mahdollistaminen kasvattaa
ohjelmien kokoa, sillä rajapinnan mukana on säilytettävä rajapinnan oikeaa
tyyppiä. Tämä kuitenkin yksinkertaistaa ohjelmien kirjoittamista, sillä
ohjelmoijan ei tarvitse miettiä kirjoittamishetkellä monimutkaisia
tyyppejä~\citep[esim.][kalvo 8]{gohistory}, kuitenkin mahdollistaen geneerisen
koodin kirjoittamisen.
%Ohjelmassa~\ref{fig:goreflection} käsitellään
%value-muuttujaa geneerisesti - jos muuttuja on tyyppiä \texttt{string},
%muuttuja palautetaan sellaisenaan. Jos muuttujassa toteuttaa
%\texttt{Stringer}-rajapinnan, eli siinä on \texttt{String()}-metodi, sitä
%kutsutaan ja palautetaan tulos. Muussa tapauksessa palautetaan tyhjä
%merkkijono.

Vaikka Go-kielessä ei itsessään ole makroja, se sisältää \texttt{go~generate}
-työkalun, jota voidaan käyttää koodin generointiin~\citep{gogenerate}.
\texttt{go~generate} mahdollistaa minkä tahansa komentorivikomennon ajamisen,
ja on enemmänkin standardoitu tapa ajaa tiettyjä komentorivikäskyjä osana
ohjelman kääntämistä kuin tyypillinen makrojärjestelmä.
\texttt{\mbox{go~generate}} ajetaan erillisenä komentona,
eikä esimerkiksi osana \texttt{\mbox{go~build}}-komentoa.

%\begin{listing}[ht!]
%    \inputminted{go}{goreflect.go}
%    \caption{Geneerinen funktio Go-kielessä. Jos \texttt{value}-muuttuja on
%    tyyppiä \texttt{string}, muuttuja palautetaan sellaisenaan. Jos muuttujassa
%    toteuttaa \texttt{Stringer}-rajapinnan, eli siinä on
%    \texttt{String()}-metodi, sitä kutsutaan ja palautetaan tulos. Muussa
%    tapauksessa palautetaan tyhjä merkkijono.}
%    \label{fig:goreflection}
%\end{listing}


Go-kielen virheidenkäsittely on toteutettu useissa kohdissa C:n tavoin;
funktioista palautetaan virheellisissä tilanteissa virheellinen arvo. Tämä
tosin tehdään usein palauttamalla erillinen \texttt{Error}-tyyppiä oleva arvo
-- Go mahdollistaa useamman kuin yhden paluuarvon. Go-kielessä on myös
poikkeukset, joita suositellaan käytettävän vain poikkeuksellisissa
tilanteissa~\citep{effectivego}.

C:n kutsuminen Go-kielestä ei ole aukotonta: koska Go on muistinkäytöltä
turvallinen kieli, erityisesti muistin jakaminen C:n ja Go-kielen välillä on
hankalaa. Lisäksi C:n funktio-osoittimia ei voi kutsua Go-kielen
puolelta~\citep{cgo}. Go mahdollistaa C-otsikkotiedostojen suoran käytön
lähdekoodista, mikä helpottaa C-koodin kutsumista.

\subsection{Rust}

Rust on Mozilla Foundationin kehittämä ohjelmointikieli, joka on suunniteltu
turvalliseksi, rinnakkaiseksi ja käytännölliseksi
järjestelmäohjelmointikieleksi~\citep{rustfaq}. Rustissa on monimutkainen
tyyppijärjestelmä, jolla ohjelmat voivat todistaa esimerkiksi turvallisen
rinnakkaisajon ilman, että ohjelmaan tulee suoritusaikaisia rajoitteita tai
hidastuksia. Rust alkoi Graydon Hoaren henkilökohtaisena sivuprojektina, mutta
on nyt käytössä osana Gecko-selainmoottorin kehitystä C++:n ja JavaScriptin
ohella.

Kuten Go-kielessä, Rustissa voi myös käyttää poikkeuksia. Rustin
virheidenhallinta on muutenkin lähellä Go-kielen virhehallintaa -- Rustin
ohjekirja opastaa käyttämään mieluummin paluuarvoja kuin
poikkeuksia~\citep{rusterrorhandling}.

Rustin monimutkainen tyyppijärjestelmä kannustaa kirjoittamaan turvallisia
ohjelmia, Tietorakenteiden mutatoiminen on tehty tietoisesti hankalaksi, sillä
monimutkaisissa ohjelmissa holtittomasti muuttuva tila on usean vian syynä, ja
muuttumaton tila tekee monisäikeistettyjen\defword{multithreaded} ohjelmien
toteutuksesta huomattavasti helpompaa~\citep[luku 4, kohta 17]{effectivejava}.
Kullakin muuttujalla voi olla joko rajaton määrä muuttumattomia
viitteitä\defword{immutable reference} tai yksi muuttuva viite\defword{mutable
reference}, mutta molempia ei voi käyttää yhtä aikaa.

Turvallisuudella on kuitenkin hintansa -- Rust-ohjelmat vievät enemmän tilaa
kuin vastaavat C-ohjelmat. Jos Rust-ohjelmista poistaa standardikirjaston ja
käyttää suoraan C:n standardikirjastoa, ohjelmasta saa miltei samankokoisen
kuin vastaavasta C-kielellä kirjoitetusta ohjelmasta~\citep{rustbinarysize}.
Samalla tosin suurin osa Rustin ominaisuuksista jää pois. Rustin turvallisuus
vaatii myös monimutkaisen tyyppijärjestelmän, joka on vaikeampi opetella kuin
yksinkertaisemman kielen tyyppijärjestelmä.

Rust ei pysty suoraan käsittelemään C:n otsikkotiedostoja, mutta D:n tavoin
Rustille on saatavilla työkaluja otsikkotiedostojen automaattiseen
muuntamiseen~\citep{rustbindgen}. Rust kuitenkin suosittelee jokaisen kirjaston
kohdalla käsin ympäröimään C-kirjaston funktiot, sillä C:n tyyppimäärittelyt
eivät tarjoa Rustin vaatimaa tarkkuutta funktioiden turvallisuudesta.

Vaikka Rust ei tue C:n makroja, siinä on useita erillisiä kattavia
makrojärjestelmiä: varsinaiset makrot~\citep{rustmacros}, joiden lisäksi
Rustista löytyy useita kokeellisia koodin generointiin tarkoitettuja
ominaisuuksia~\citep{rustprocmacros, rustplugins}. Rust taas käsittelee makrot
vasta ohjelman tokenisoinnin jälkeen, eli ohjelman täytyy olla kyseisen
ohjelmointikielen mukaista ennen kuin makrot suoritetaan -- makroilla ei voi
käsitellä mitä tahansa tekstiä. Rust vaatii myös erottimien (sulkeiden,
lainausmerkkien ja heittomerkkien) olevan kielen syntaksin mukaisesti. Lisäksi
Rust-kääntäjän tulee tunnistaa kielen tekstialkiot eli
lekseemit\defword{lexeme} ennen makrojen suorittamista, eli Rust ei mahdollista
uusien operaattoreiden määrittämistä makrojen avulla. Rustin makroprosessori on
Turing-täydellinen~\citep{rustmacros} sekä hygieeninen.

\newpage

\subsection{Yhteenveto}

Yksikään vertailtavista kielistä ei täytä kaikkia luvussa~\ref{sec:abs}
määriteltyjä rajoitteita. Yksikään kielistä ei täytä muistinkäytön rajoitteita,
jonka lisäksi saumaton yhteistyö C:n kanssa onnistuu vain C++:n kanssa.

C++, D ja Rust mahdollistavat yhteensopivuuden parantamiseksi nimiruntelun
poistamisen käytöstä, mutta tämä ei toimi esimerkiksi C++:n luokkien
yhteydessä. Ada ja Go mahdollistavat funktioiden nimien valitsemisen linkkeriä
varten, jolloin esimerkiksi \texttt{EsimerkkiFunktio}-nimistä funktiota voidaan
kutsua \texttt{esimFunk}-nimellä C-ohjelmasta. Kaikki vertailtavat kielet
tukevat vierasfunktiorajapintoja C:n mukaisesti, eli kielet voivat kutsua muita
kieliä C-rajapintojen läpi.

C++ ja Rust ovat hyvin lähellä C:tä käännettyjen ohjelmien koossa, mutta
häviävät C:lle laajojen vakiokirjastojen takia. Molemmat ovat myös hyvin
monimutkaisia kieliä. Vain C++ ja Go voivat sisällyttää ohjelmiin C:n
otsikkotiedostoja, kun taas Ada, D ja Rust vaativat jokaisen funktion
määrittämistä. Adalle, D:lle ja Rustille on tosin olemassa työkaluja, joilla
tämän määrittämisen voi automatisoida.

\begin{table}[ht!]
    \begin{adjustbox}{center}
    \begin{tabular}{@{}lllll@{}} \toprule
        Kieli & Nimiruntelu   & Muistinhallinta                                     & C:n VFR               & muiden kielten VFR \\ \midrule
        Ada   & täysin hallittavissa & automaattinen                                       & Työläs mutta kattava  & C, C++, Fortran, Cobol \\
        C++   & saa pois päältä      & manuaalinen                                         & Lähes saumaton        & C:n läpi \\
        D     & saa pois päältä      & molemmat                                            & Työläs mutta kattava  & C:n läpi \\
        Go    & saa pois päältä      & automaattinen                                       & Epätäydellinen        & C:n läpi \\
        Rust  & saa pois päältä & automaattinen\footnote{Borrow Checkerin hallitsemana, joka mahdollistaa automaattisen muistinhallinnan ilman sen aiheuttamaa hidastusta.} & Työläs mutta kattava  & C:n läpi \\ \bottomrule
    \end{tabular}
    \end{adjustbox}
    \caption{
        Kielten ominaisuuksien yhteenveto.
    }
    \label{table:properties}
\end{table}

Rust ja C++ ovat verrattavista kielistä ainoat, joilla on olemassa varsinainen
makrojärjestelmä. C++:n makrojärjestelmä on melkein täysin yhteensopiva C:n
makrojärjestelmän kanssa, kun taas Rustin makrojärjestelmä on huomattavasti
ilmaisukykyisempi.

\newpage

\begin{figure}[ht!]
    \begin{adjustbox}{center}
    \begin{minipage}{1.15\textwidth}
    \begin{minipage}{0.5\textwidth}
        \begin{tikzpicture}[gnuplot]
%% generated with GNUPLOT 5.2p2 (Lua 5.3; terminal rev. 99, script rev. 102)
%% la  1. joulukuuta 2018 09.58.32
\path (0.000,0.000) rectangle (8.000,6.000);
\gpcolor{color=gp lt color border}
\gpsetlinetype{gp lt border}
\gpsetdashtype{gp dt solid}
\gpsetlinewidth{1.00}
\draw[gp path] (1.136,0.616)--(1.316,0.616);
\draw[gp path] (7.447,0.616)--(7.267,0.616);
\node[gp node right] at (0.952,0.616) {$0$};
\draw[gp path] (1.136,1.462)--(1.316,1.462);
\draw[gp path] (7.447,1.462)--(7.267,1.462);
\node[gp node right] at (0.952,1.462) {$5$};
\draw[gp path] (1.136,2.308)--(1.316,2.308);
\draw[gp path] (7.447,2.308)--(7.267,2.308);
\node[gp node right] at (0.952,2.308) {$10$};
\draw[gp path] (1.136,3.154)--(1.316,3.154);
\draw[gp path] (7.447,3.154)--(7.267,3.154);
\node[gp node right] at (0.952,3.154) {$15$};
\draw[gp path] (1.136,3.999)--(1.316,3.999);
\draw[gp path] (7.447,3.999)--(7.267,3.999);
\node[gp node right] at (0.952,3.999) {$20$};
\draw[gp path] (1.136,4.845)--(1.316,4.845);
\draw[gp path] (7.447,4.845)--(7.267,4.845);
\node[gp node right] at (0.952,4.845) {$25$};
\draw[gp path] (1.136,5.691)--(1.316,5.691);
\draw[gp path] (7.447,5.691)--(7.267,5.691);
\node[gp node right] at (0.952,5.691) {$30$};
\draw[gp path] (1.997,0.616)--(1.997,0.796);
\draw[gp path] (1.997,5.691)--(1.997,5.511);
\node[gp node center] at (1.997,0.308) {C};
\draw[gp path] (3.144,0.616)--(3.144,0.796);
\draw[gp path] (3.144,5.691)--(3.144,5.511);
\node[gp node center] at (3.144,0.308) {C++};
\draw[gp path] (4.292,0.616)--(4.292,0.796);
\draw[gp path] (4.292,5.691)--(4.292,5.511);
\node[gp node center] at (4.292,0.308) {Rust};
\draw[gp path] (5.439,0.616)--(5.439,0.796);
\draw[gp path] (5.439,5.691)--(5.439,5.511);
\node[gp node center] at (5.439,0.308) {Ada};
\draw[gp path] (6.586,0.616)--(6.586,0.796);
\draw[gp path] (6.586,5.691)--(6.586,5.511);
\node[gp node center] at (6.586,0.308) {Go};
\draw[gp path] (1.136,5.691)--(1.136,0.616)--(7.447,0.616)--(7.447,5.691)--cycle;
\node[gp node center,rotate=-270] at (0.092,3.153) {sekuntia};
\gpfill{rgb color={1.000,1.000,1.000}} (1.710,0.893)--(2.284,0.893)--(2.284,1.442)--(1.710,1.442)--cycle;
\draw[gp path] (1.710,0.893)--(2.284,0.893)--(2.284,1.442)--(1.710,1.442)--cycle;
\draw[gp path] (1.710,0.932)--(2.284,0.932);
\draw[gp path] (1.997,0.846)--(1.997,0.893);
\draw[gp path] (1.997,1.442)--(1.997,2.201);
\draw[gp path] (1.817,2.201)--(2.177,2.201);
\draw[gp path] (1.817,0.846)--(2.177,0.846);
\gpfill{rgb color={1.000,1.000,1.000}} (2.857,0.926)--(3.431,0.926)--(3.431,1.247)--(2.857,1.247)--cycle;
\draw[gp path] (2.857,0.926)--(3.431,0.926)--(3.431,1.247)--(2.857,1.247)--cycle;
\draw[gp path] (2.857,1.031)--(3.431,1.031);
\draw[gp path] (3.144,0.841)--(3.144,0.926);
\draw[gp path] (2.964,1.247)--(3.324,1.247);
\draw[gp path] (2.964,0.841)--(3.324,0.841);
\gpfill{rgb color={1.000,1.000,1.000}} (4.005,0.910)--(4.579,0.910)--(4.579,1.628)--(4.005,1.628)--cycle;
\draw[gp path] (4.005,0.910)--(4.579,0.910)--(4.579,1.628)--(4.005,1.628)--cycle;
\draw[gp path] (4.005,0.989)--(4.579,0.989);
\draw[gp path] (4.292,0.863)--(4.292,0.910);
\draw[gp path] (4.292,1.628)--(4.292,2.286);
\draw[gp path] (4.112,2.286)--(4.472,2.286);
\draw[gp path] (4.112,0.863)--(4.472,0.863);
\gpfill{rgb color={1.000,1.000,1.000}} (5.152,1.284)--(5.726,1.284)--(5.726,2.182)--(5.152,2.182)--cycle;
\draw[gp path] (5.152,1.284)--(5.726,1.284)--(5.726,2.182)--(5.152,2.182)--cycle;
\draw[gp path] (5.152,1.652)--(5.726,1.652);
\draw[gp path] (5.439,0.912)--(5.439,1.284);
\draw[gp path] (5.439,2.182)--(5.439,2.661);
\draw[gp path] (5.259,2.661)--(5.619,2.661);
\draw[gp path] (5.259,0.912)--(5.619,0.912);
\gpfill{rgb color={1.000,1.000,1.000}} (6.300,1.289)--(6.874,1.289)--(6.874,4.169)--(6.300,4.169)--cycle;
\draw[gp path] (6.300,1.289)--(6.874,1.289)--(6.874,4.169)--(6.300,4.169)--cycle;
\draw[gp path] (6.300,2.143)--(6.874,2.143);
\draw[gp path] (6.587,0.961)--(6.587,1.289);
\draw[gp path] (6.587,4.169)--(6.587,5.520);
\draw[gp path] (6.407,5.520)--(6.767,5.520);
\draw[gp path] (6.407,0.961)--(6.767,0.961);
\draw[gp path] (1.136,5.691)--(1.136,0.616)--(7.447,0.616)--(7.447,5.691)--cycle;
%% coordinates of the plot area
\gpdefrectangularnode{gp plot 1}{\pgfpoint{1.136cm}{0.616cm}}{\pgfpoint{7.447cm}{5.691cm}}
\end{tikzpicture}
%% gnuplot variables

        \vspace*{-0.8cm}
    \end{minipage}
    \begin{minipage}{0.5\textwidth}
        \begin{tikzpicture}[gnuplot]
%% generated with GNUPLOT 5.2p2 (Lua 5.3; terminal rev. 99, script rev. 102)
%% ma 21. tammikuuta 2019 23.24.51
\path (0.000,0.000) rectangle (8.000,6.000);
\gpcolor{color=gp lt color border}
\gpsetlinetype{gp lt border}
\gpsetdashtype{gp dt solid}
\gpsetlinewidth{1.00}
\draw[gp path] (1.320,0.616)--(1.500,0.616);
\draw[gp path] (7.447,0.616)--(7.267,0.616);
\node[gp node right] at (1.136,0.616) {$0$};
\draw[gp path] (1.320,1.341)--(1.500,1.341);
\draw[gp path] (7.447,1.341)--(7.267,1.341);
\node[gp node right] at (1.136,1.341) {$50$};
\draw[gp path] (1.320,2.066)--(1.500,2.066);
\draw[gp path] (7.447,2.066)--(7.267,2.066);
\node[gp node right] at (1.136,2.066) {$100$};
\draw[gp path] (1.320,2.791)--(1.500,2.791);
\draw[gp path] (7.447,2.791)--(7.267,2.791);
\node[gp node right] at (1.136,2.791) {$150$};
\draw[gp path] (1.320,3.516)--(1.500,3.516);
\draw[gp path] (7.447,3.516)--(7.267,3.516);
\node[gp node right] at (1.136,3.516) {$200$};
\draw[gp path] (1.320,4.241)--(1.500,4.241);
\draw[gp path] (7.447,4.241)--(7.267,4.241);
\node[gp node right] at (1.136,4.241) {$250$};
\draw[gp path] (1.320,4.966)--(1.500,4.966);
\draw[gp path] (7.447,4.966)--(7.267,4.966);
\node[gp node right] at (1.136,4.966) {$300$};
\draw[gp path] (1.320,5.691)--(1.500,5.691);
\draw[gp path] (7.447,5.691)--(7.267,5.691);
\node[gp node right] at (1.136,5.691) {$350$};
\draw[gp path] (2.341,0.616)--(2.341,0.796);
\draw[gp path] (2.341,5.691)--(2.341,5.511);
\node[gp node center] at (2.341,0.308) {C++};
\draw[gp path] (3.703,0.616)--(3.703,0.796);
\draw[gp path] (3.703,5.691)--(3.703,5.511);
\node[gp node center] at (3.703,0.308) {Rust};
\draw[gp path] (5.064,0.616)--(5.064,0.796);
\draw[gp path] (5.064,5.691)--(5.064,5.511);
\node[gp node center] at (5.064,0.308) {Ada};
\draw[gp path] (6.426,0.616)--(6.426,0.796);
\draw[gp path] (6.426,5.691)--(6.426,5.511);
\node[gp node center] at (6.426,0.308) {Go};
\draw[gp path] (1.320,5.691)--(1.320,0.616)--(7.447,0.616)--(7.447,5.691)--cycle;
\gpsetdashtype{dash pattern=on 10.00*\gpdashlength off 10.00*\gpdashlength }
\draw[gp path](1.320,2.066)--(7.447,2.066);
\node[gp node center,rotate=-270] at (0.000,3.153) {\shortstack{Muistinkäyttö \\ C:hen verrattuna (\%)}};
\gpfill{rgb color={1.000,1.000,1.000}} (2.001,2.096)--(2.683,2.096)--(2.683,2.951)--(2.001,2.951)--cycle;
\gpsetdashtype{gp dt solid}
\draw[gp path] (2.001,2.096)--(2.683,2.096)--(2.683,2.951)--(2.001,2.951)--cycle;
\draw[gp path] (2.001,2.184)--(2.683,2.184);
\draw[gp path] (2.342,2.065)--(2.342,2.096);
\draw[gp path] (2.342,2.951)--(2.342,3.867);
\draw[gp path] (2.162,3.867)--(2.522,3.867);
\draw[gp path] (2.162,2.065)--(2.522,2.065);
\gpfill{rgb color={1.000,1.000,1.000}} (3.362,2.198)--(4.044,2.198)--(4.044,3.387)--(3.362,3.387)--cycle;
\draw[gp path] (3.362,2.198)--(4.044,2.198)--(4.044,3.387)--(3.362,3.387)--cycle;
\draw[gp path] (3.362,2.321)--(4.044,2.321);
\draw[gp path] (3.703,2.112)--(3.703,2.198);
\draw[gp path] (3.703,3.387)--(3.703,3.499);
\draw[gp path] (3.523,3.499)--(3.883,3.499);
\draw[gp path] (3.523,2.112)--(3.883,2.112);
\gpfill{rgb color={1.000,1.000,1.000}} (4.724,2.117)--(5.406,2.117)--(5.406,4.203)--(4.724,4.203)--cycle;
\draw[gp path] (4.724,2.117)--(5.406,2.117)--(5.406,4.203)--(4.724,4.203)--cycle;
\draw[gp path] (4.724,3.707)--(5.406,3.707);
\draw[gp path] (5.065,0.616)--(5.065,2.117);
\draw[gp path] (5.065,4.203)--(5.065,5.121);
\draw[gp path] (4.885,5.121)--(5.245,5.121);
\draw[gp path] (4.885,0.616)--(5.245,0.616);
\gpfill{rgb color={1.000,1.000,1.000}} (6.085,2.209)--(6.767,2.209)--(6.767,5.158)--(6.085,5.158)--cycle;
\draw[gp path] (6.085,2.209)--(6.767,2.209)--(6.767,5.158)--(6.085,5.158)--cycle;
\draw[gp path] (6.085,3.417)--(6.767,3.417);
\draw[gp path] (6.426,1.298)--(6.426,2.209);
\draw[gp path] (6.426,5.158)--(6.426,5.378);
\draw[gp path] (6.246,5.378)--(6.606,5.378);
\draw[gp path] (6.246,1.298)--(6.606,1.298);
\draw[gp path] (1.320,5.691)--(1.320,0.616)--(7.447,0.616)--(7.447,5.691)--cycle;
%% coordinates of the plot area
\gpdefrectangularnode{gp plot 1}{\pgfpoint{1.320cm}{0.616cm}}{\pgfpoint{7.447cm}{5.691cm}}
\end{tikzpicture}
%% gnuplot variables

        \vspace*{-0.9cm}
    \end{minipage}
    \end{minipage}
    \end{adjustbox}
    \begin{adjustbox}{center}
    \begin{minipage}{1.15\textwidth}\makebox[\textwidth][c]{%
    \begin{minipage}{0.5\textwidth}
        \begin{tikzpicture}[gnuplot]
%% generated with GNUPLOT 5.2p2 (Lua 5.3; terminal rev. 99, script rev. 102)
%% to 10. tammikuuta 2019 17.28.43
\path (0.000,0.000) rectangle (8.000,6.000);
\gpcolor{color=gp lt color border}
\gpsetlinetype{gp lt border}
\gpsetdashtype{gp dt solid}
\gpsetlinewidth{1.00}
\draw[gp path] (1.320,0.616)--(1.500,0.616);
\draw[gp path] (7.447,0.616)--(7.267,0.616);
\node[gp node right] at (1.136,0.616) {$0$};
\draw[gp path] (1.320,1.302)--(1.500,1.302);
\draw[gp path] (7.447,1.302)--(7.267,1.302);
\node[gp node right] at (1.136,1.302) {$50$};
\draw[gp path] (1.320,1.988)--(1.500,1.988);
\draw[gp path] (7.447,1.988)--(7.267,1.988);
\node[gp node right] at (1.136,1.988) {$100$};
\draw[gp path] (1.320,2.673)--(1.500,2.673);
\draw[gp path] (7.447,2.673)--(7.267,2.673);
\node[gp node right] at (1.136,2.673) {$150$};
\draw[gp path] (1.320,3.359)--(1.500,3.359);
\draw[gp path] (7.447,3.359)--(7.267,3.359);
\node[gp node right] at (1.136,3.359) {$200$};
\draw[gp path] (1.320,4.045)--(1.500,4.045);
\draw[gp path] (7.447,4.045)--(7.267,4.045);
\node[gp node right] at (1.136,4.045) {$250$};
\draw[gp path] (1.320,4.731)--(1.500,4.731);
\draw[gp path] (7.447,4.731)--(7.267,4.731);
\node[gp node right] at (1.136,4.731) {$300$};
\draw[gp path] (1.320,5.417)--(1.500,5.417);
\draw[gp path] (7.447,5.417)--(7.267,5.417);
\node[gp node right] at (1.136,5.417) {$350$};
\draw[gp path] (2.341,0.616)--(2.341,0.796);
\draw[gp path] (2.341,5.691)--(2.341,5.511);
\node[gp node center] at (2.341,0.308) {C++};
\draw[gp path] (3.703,0.616)--(3.703,0.796);
\draw[gp path] (3.703,5.691)--(3.703,5.511);
\node[gp node center] at (3.703,0.308) {Rust};
\draw[gp path] (5.064,0.616)--(5.064,0.796);
\draw[gp path] (5.064,5.691)--(5.064,5.511);
\node[gp node center] at (5.064,0.308) {Ada};
\draw[gp path] (6.426,0.616)--(6.426,0.796);
\draw[gp path] (6.426,5.691)--(6.426,5.511);
\node[gp node center] at (6.426,0.308) {Go};
\draw[gp path] (1.320,5.691)--(1.320,0.616)--(7.447,0.616)--(7.447,5.691)--cycle;
\gpsetdashtype{dash pattern=on 10.00*\gpdashlength off 10.00*\gpdashlength }
\draw[gp path](1.320,1.988)--(7.447,1.988);
\node[gp node center,rotate=-270] at (0.000,3.153) {\shortstack{Lähdekoodissa tavuja \\ C:hen verrattuna (\%)}};
\gpfill{rgb color={1.000,1.000,1.000}} (2.001,1.778)--(2.683,1.778)--(2.683,2.235)--(2.001,2.235)--cycle;
\gpsetdashtype{gp dt solid}
\draw[gp path] (2.001,1.778)--(2.683,1.778)--(2.683,2.235)--(2.001,2.235)--cycle;
\draw[gp path] (2.001,1.953)--(2.683,1.953);
\draw[gp path] (2.342,1.423)--(2.342,1.778);
\draw[gp path] (2.342,2.235)--(2.342,2.307);
\draw[gp path] (2.162,2.307)--(2.522,2.307);
\draw[gp path] (2.162,1.423)--(2.522,1.423);
\gpfill{rgb color={1.000,1.000,1.000}} (3.362,1.684)--(4.044,1.684)--(4.044,3.542)--(3.362,3.542)--cycle;
\draw[gp path] (3.362,1.684)--(4.044,1.684)--(4.044,3.542)--(3.362,3.542)--cycle;
\draw[gp path] (3.362,2.858)--(4.044,2.858);
\draw[gp path] (3.703,1.478)--(3.703,1.684);
\draw[gp path] (3.703,3.542)--(3.703,4.922);
\draw[gp path] (3.523,4.922)--(3.883,4.922);
\draw[gp path] (3.523,1.478)--(3.883,1.478);
\gpfill{rgb color={1.000,1.000,1.000}} (4.724,2.508)--(5.406,2.508)--(5.406,5.086)--(4.724,5.086)--cycle;
\draw[gp path] (4.724,2.508)--(5.406,2.508)--(5.406,5.086)--(4.724,5.086)--cycle;
\draw[gp path] (4.724,2.891)--(5.406,2.891);
\draw[gp path] (5.065,2.184)--(5.065,2.508);
\draw[gp path] (5.065,5.086)--(5.065,5.691);
\draw[gp path] (4.885,2.184)--(5.245,2.184);
\gpfill{rgb color={1.000,1.000,1.000}} (6.085,1.627)--(6.767,1.627)--(6.767,2.604)--(6.085,2.604)--cycle;
\draw[gp path] (6.085,1.627)--(6.767,1.627)--(6.767,2.604)--(6.085,2.604)--cycle;
\draw[gp path] (6.085,1.959)--(6.767,1.959);
\draw[gp path] (6.426,1.514)--(6.426,1.627);
\draw[gp path] (6.426,2.604)--(6.426,3.035);
\draw[gp path] (6.246,3.035)--(6.606,3.035);
\draw[gp path] (6.246,1.514)--(6.606,1.514);
\draw[gp path] (1.320,5.691)--(1.320,0.616)--(7.447,0.616)--(7.447,5.691)--cycle;
%% coordinates of the plot area
\gpdefrectangularnode{gp plot 1}{\pgfpoint{1.320cm}{0.616cm}}{\pgfpoint{7.447cm}{5.691cm}}
\end{tikzpicture}
%% gnuplot variables

        \vspace*{-1cm}
    \end{minipage}}
    \end{minipage}
    \end{adjustbox}
    \caption{
        Benchmarks Gamen~\citep[tiedot haettu 1.1.2019]{benchmarks} tuloksiin
        perustuvat kuvaajat ohjelmointikielten suorituskyvystä, muistinkäytöstä
        ja ohjelmien koosta verrattuna C:llä kirjoitettujen ohjelmien
        tuloksiin.}
    \label{fig:benchmarksgame}
\end{figure}

\FloatBarrier

Benchmarks Gamen tulokset myös heijastavat näitä tuloksia -- C++ ja Rust ovat
nopeudeltaan hyvin lähellä C:tä, kun taas Ada ja Go ovat huomattavasti
hitaampia. Kuvassa~\ref{fig:benchmarksgame} verrataan suorituskykyä,
muistinkäyttöä ja ohjelmien lähdekoodin pituutta C:llä kirjoitettuun ohjelmaan.
Lähdekoodimittauksessa mitataan ohjelman kokoa siten, että ohjelmasta
poistetaan kommentit sekä ylimääräiset välimerkit. Tämän jälkeen ohjelma
pakataan \texttt{gzip}-pakkausohjelmalla~\citep{howmeasured}. Kuvaajat
perustuvat nopeimman kielellä kirjoitetun ohjelman tuloksiin -- lähes kaikilla
kielillä on jokaisessa suorituskykymittauksissa useampi ohjelma.

C ja C++ ovat hyvin lähellä toisiaan ajonopeudessa. Rust on jonkin verran
hitaampi, ja Ada ja Go huomattavasti hitaampia. C++ käyttää jonkin verran
enemmän muistia kuin C, kun taas Rust, Ada ja Go käyttävät selkeästi enemmän
muistia. C++ ja Rust ovat yksittäisissä suorituskykymittauksissa C:tä
nopeampia. Yhdessä suorituskykymittauksista Go käyttää noin puolet C:n
käyttämästä muistista, mutta on samassa mittauksessa kaksi kertaa hitaampi.
Mielenkiintoisesti Rust, Ada ja Go ovat C:hen verrattuna hitaampia ja vievät
enemmän muistia, mutta eivät tarjoa merkittäviä säästöjä lähdekoodin määrään.

\newpage
\section{Purkka-ohjelmointikieli}
\label{sec:purkka}

Jotta uudesta ohjelmointikielestä saisi tutkielman vertailukriteerien
mukaisesti C:tä paremman kielen, uuden kielen suunnittelussa tulee seurata
tarkasti vertailukriteerejä. Erityisesti C-yhteensopivuus on tärkeä ominaisuus,
jota muissa ohjelmointikielissä ei ole pidetty tärkeänä ohjelmointikielen
suunnittelussa. Tässä luvussa ensin käsitellään periaatteet, joilla Purkasta
voisi tulla C:tä parempi ohjelmointikieli tutkielman kontekstissa, jonka
jälkeen käsitellään tarkemmin Purkan tyypitystä, syntaksia sekä
C-yhteensopivuutta.

\subsection{Purkan suunnitteluperiaatteet}

Luvut~\ref{sec:clyhyesti},~\ref{sec:ctaustaa}~ja~\ref{sec:cominaisuudet}
nostavat tärkeinä C:n ominaisuuksina kielen yksinkertaisuuden ja tehokkuuden,
jotka ovat mahdollistaneet kielen leviämisen järjestelmästä toiseen.
Luvussa~\ref{sec:cominaisuudet} nostetaan tämän lisäksi esiin vaatimus
C-ohjelmien kirjoittamisesta funktio kerrallaan uudella kielellä, jotta
kielestä toiseen siirtyminen olisi ylipäätään mahdollista ilman kohtuutonta
investointia. Luvussa~\ref{sec:ckehitettavat} esitellään lukuisia C:n
syntaktisia ongelmia, jotka voi muuttaa tehden kielestä helppolukuisemman,
kuten tyyppipäättelyn lisääminen ohjelmointikieleen.

Käännösaikaista varmennettavuutta voi parantaa lisäämällä esimerkiksi
kieleen summatyypit, jotka löytyvät esimerkiksi Rustista. Yhteensopivuus nousee
myös esiin luvussa~\ref{sec:suosio}, jossa kirjastojen saatavuus näytetään
tärkeäksi ohjelmointikielen valintakriteeriksi.

Uuden kielen määrittelyssä tulee pitää luvun~\ref{sec:abs} kriteerit, jotta
ominaisuuksia päättäessä ei muodostu esimerkiksi yhteensopivuusongelmia C:n
kanssa. Tämä rajoittaakin suurta osaa luvussa~\ref{sec:muut} esiteltyjä
ominaisuuksia, kuten automaattista muistinhallintaa ja poikkeuksia. Monet
ominaisuudet myös monimutkaistaisivat kieltä tarjoamatta kuitenkaan
tehokkuusparannuksia. Yksittäisiä ominaisuuksia kuitenkin pystyy lisäämään,
kuten luvuissa~\ref{sec:go}~ja~\ref{sec:rust} esiintyvät useat paluuarvot sekä
luvussa~\ref{sec:rust} esitellyt summatyypit, sillä näiden sisällyttäminen
kieleen mahdollistaa paremman käännösaikaisen todentamisen ilman
suoritusaikaisia haittoja. Summatyypit voivat myös parantaa
kääntäjäoptimointia, jos kääntäjä pystyy poistamaan ohjelmakoodia päättelemällä
tietyt summatyypin arvot mahdottomiksi.

Uusi kieli tulisi kääntää C:ksi, jotta sen käyttäminen on mahdollista kaikissa
järjestelmissä, joissa C:tä käytetään. C:ksi kääntäminen myös mahdollistaa
C-yhteensopivien kirjastojen käyttämisen ilman erillistä
yhteensopivuuskerrosta. Kielen tulee myös ymmärtää C:llä kirjoitettuja ohjelmia
sisältäen myös C:llä kirjoitetut makrot ja otsikkotiedostot. Esimerkiksi
POSIX-C:n \texttt{errno}-muuttuja voi olla määritelty makrona ja
\texttt{errno}-muuttujan käyttäminen ilman esikäsittelijää esimerkiksi
viittaamalla suoraan \texttt{errno}-nimiseen muuttujaan on määrittelemätöntä
toimintaa~\citep[s. 234]{POSIX}, joten \texttt{errno}-arvon lukeminen vaatii
tuen C:n esikäsittelijälle.

\subsection{Tyypit}

Jotta yhteensopivuus C:n kanssa olisi joustavaa, Purkan tyyppijärjestelmä
muodostetaan mahdollisimman paljon C:n kaltaiseksi. C:n tyyppisyntaksi sisältää
useita erilaisia tapoja ilmaista samaa pohjatyyppiä -- esimerkiksi
\texttt{long}, \texttt{signed~long}, \texttt{long~int} ja
\texttt{signed~long~int} ilmaisevat kokonaislukutyyppiä, joka pystyy
sisältämään ainakin luvut $[-(2^{31} - 1), 2^{31}-1]$\footnote{Käytännössä
modernit toteutukset käyttävät kahden komplementtia kokonaislukujen
ilmaisemiseen, jolloin 32-bittinen kokonaislukumuuttuja voi sisältää luvut
$[-2^{31}, 2^{31} - 1]$.}~\citep{C18}. Tyypit kuten \texttt{long} ja
\texttt{signed long} ovat kuitenkin standardien mukaisessa C:ssä keskenään
täysin vaihdettavissa, eli Purkan ei tarvitse pystyä kääntymään jokaiseen
mahdolliseen vaihtoehtoon, vaan tyyppijärjestelmässä voi olla yksi tyyppi joka
vastaa kaikkia \texttt{long}-tyypin vaihtoehtoisia kirjoitusasuja.

C:n kokonaislukutyypeistä (\texttt{char}, \texttt{short}, \texttt{int},
\texttt{long}) \texttt{char}-tyyppi on ainoa, jolla \texttt{char},
\texttt{signed char} ja \texttt{unsigned char} ovat kolme erillistä
tyyppiä~\citep{C18}, kun muilla alkeistyypillä esimerkiksi \texttt{signed int}
on sama tyyppi kuin \texttt{int}. Merkkijonot koostuvat
\texttt{char}-tyyppisistä alkioista, kun taas \texttt{signed char} ja
\texttt{unsigned char} vastaavat tavun kokoista kokonaislukua ja
epänegatiivista kokonaislukua. Purkka-kielessä C:n \texttt{char}-tyyppiä
vastaa \texttt{char}, kun taas \texttt{signed char} ja \texttt{unsigned char}
ovat \texttt{i8} ja \texttt{u8}.

Tyyppien kääntäminen C:stä Purkaksi ja takaisin on yksinkertaista: jokaista
C-tyyppiä vastaa täsmälleen yksi Purkka-tyyppi. Jokaiselle Purkan
alkeistyypille taas on määritelty yksi C-tyyppi, johon kyseinen Purkka-tyyppi
käännetään. Liitteessä~\ref{app:purkka} on taulukko C:n ja Purkan tyypeistä. 

C:n alkeistyyppien, osoittimien, taulukoiden ja yhdistetyyppien lisäksi
määritellään ohjelmoinnin tehostamiseksi sekä käännösaikaisen optimoinnin ja
oikeellisuuden parantamiseksi yksittäisiä lisätyyppejä. Nämä tyypit ovat
epätyhjät osoittimet sekä summatyypit.

Ensimmäiseksi määritellään epätyhjä osoitin, joka on osoitin, joka ei salli
arvokseen C:n \texttt{NULL}-arvoa. Tyyppi kääntyy Purkasta standardien
mukaiseksi C:ksi yksinkertaisesti vastaavaan C:n osoitintyyppiin. Epätyhjä
osoitin mahdollistaa kääntäjäoptimointeja, jotka eivät muuten olisi
mahdollisia. GCC-kääntäjä mahdollistaa epätyhjät osoittimet
\texttt{\_\_attribute\_\_((nonnull))}-määreellä, jota Purkka-kääntäjä voi
käyttää ohjelmoijan niin pyytäessä.

\begin{listing}[ht!]
    \inputminted{Rust}{koodi/sumtype.prk}
    \caption{Summatyyppi Purkka-kielessä ja sama summatyyppi käännettynä C-kielelle.}
    \label{fig:purkkatree}
\end{listing}

Toiseksi määritellään summatyyppi. Summatyyppi on yhdistelmä C:n
\texttt{struct}, \texttt{union} ja \texttt{enum} -tyypeistä, ja se mahdollistaa
käännösaikaisesti varmennetun \texttt{union}-tyyppien käytön. Summatyyppi kääntyy
\texttt{struct}-tyypiksi, jossa on \texttt{enum}-tyyppinen muuttuja summatyypin
diskriminanttina ja \texttt{union}-tyyppi, joka sisältää kaikki summatyypin
variantit.

Ohjelmassa~\ref{fig:purkkatree} on Purkka-kielellä kirjoitettu summatyyppi
\texttt{Puu}, jossa on \texttt{Lehti}- ja \texttt{Haara}-variantit.
\texttt{Lehti}-variantti sisältää 32-bittisen kokonaisluvun, kun taas
\texttt{Haara}-variantti sisältää osoittimet kahteen alipuuhun. Kääntäessä
tietorakenne muutetaan liitteen~\ref{app:purkka}
ohjelman~\ref{fig:purkkatreecompile} kaltaiseksi lähdekoodiksi. Tässä
erikoistapauksessa kääntäjä voisi optimoida tietorakennetta
liitteen~\ref{app:purkka} ohjelman~\ref{fig:purkkatreecompile2} mukaisesti
säilömällä \texttt{\_Puu\_d} -tyypin tiedon \texttt{\_Puu\_Lehti}- ja
\texttt{\_Puu\_Haara}-tyyppien ensimmäiseen alkioon: jos osoitin on tyhjä, on
kyseessä \texttt{Lehti}-variantti, muulloin \texttt{Haara}-variantti.

\subsection{Syntaksi}

Purkan syntaksi on hyvin samankaltainen C:n syntaksiin verrattuna. Tämä
helpottaa kielen oppimista aikaisemman tutkimuksen
mukaisesti~\citep{languagelearning}. Samankaltainen syntaksi myös helpottaa
nykyisten ohjelmien uudelleenkirjoitusta, sillä ohjelmien rakennetta ei
tarvitse muuntaa. Suurimmat erot C:hen liittyvät tyyppien kirjoittamiseen,
jossa syntaksia on muutettu modernien ohjelmointikielten syntaksin mukaiseksi.

Funktiomäärittelyt alkavat \texttt{fun}-avainsanalla. Tämä noudattaa usean muun
modernin ohjelmointikielen tapaa aloittaa funktiomäärittely avainsanalla.
Vertailun vuoksi Rust käyttää avainsanaa \texttt{fn}, Go käyttää avainsanaa
\texttt{func} ja Kotlin käyttää avainsanaa \texttt{fun}. Avainsanan
käyttäminen yksinkertaistaa kielen kielioppia ja jäsentäjän toteutusta.

Purkka-kieli sisältää tyyppipäättelyn, joka helpottaa koodin kirjoittamista,
kun kääntäjä pystyy päättelemään muuttujien tyypit. C-kielen mahdollisuuksia
automaattiseen tyyppipäättelyä on tutkittu
aikaisemminkin~\citep[mm.][]{ctypeinference}. Koska Purkan syntaksi on
yksiselitteinen, monet artikkelissa esitetyt C:n ongelmat eivät ole Purkassa
haittana. Toisaalta esimerkiksi yhteenlasku toimii samoin kuin C:ssä, joten sen
tyyppipäättely ei ole triviaalia: jos C:ssä tai Purkassa toinen
yhteenlaskettavista on osoitin, summan tyyppi on osoitin, muutoin tyyppi
muodostuu C:n ''tavallisten'' lukujen muunnossääntöjen mukaisesti.

Koska tyyppipäättely toteutetaan puhtaasti käännösaikaisesti, se ei voi
aiheuttaa suoritusaikaisia haittoja. Tyyppisyntaksi itsessään muistuttaa paljon
Rustin tyyppisyntaksia.

C:n \texttt{static}-avainsana on jaettu käytön mukaisesti kahteen avainsanaan.
Funktiomäärittelyissä ja globaaleissa muuttujissa C:n
\texttt{static}-avainsanaa vastaa Purkan \texttt{pub}-avainsana (''public'' eli
julkinen), mutta käänteisellä merkityksellä: jos \texttt{pub}-avainsanaa ei ole
käytetty, koodi käyttäytyy kuin siihen olisi lisätty C:n
\texttt{static}-avainsana. Funktioiden sisällä Purkka tukee myös
\texttt{static}-avainsanaa, joka toimii samalla tavalla kuin C:n
\texttt{static} käytettynä funktioiden sisällä.

C käsittelee tietueiden ja taulukoiden alustamista identtisellä syntaksilla.
Purkka erottaa nämä kaksi syntaksia erikseen ohjelman~\ref{fig:structinit}
tavalla. Ohjelmassa alustetaan \texttt{Esimerkki}-tietue sekä kolmen
kokonaisluvun taulukko. Tietueen tyypin nimen pitäminen mukana
tietueliteraaleissa pitää tyypin selkeästi näkyvissä myös tyyppipäättelyä
käytettäessä. Yksinkertaisemmissa tapauksissa tämä ei vaikuta lähdekoodin
pituuteen, mutta monimutkaisten tietueiden alustuksessa Purkka-koodi on hieman
pidempää kuin vastaava C-koodi.

\begin{listing}[ht!]
    \inputminted{Rust}{koodi/structinit.prk}
    \inputminted{C}{koodi/structinit.c}
    \caption{Tietueen ja taulukon alustaminen Purkassa ja C:ssä.}
    \label{fig:structinit}
\end{listing}

Suurin osa C:n lauseista on Purkassa lausekkeita. Esimerkiksi \texttt{if-else}
lausepari on Purkassa lauseke, joka mahdollistaa selkeämmän koodin
kirjoittamisen. C:ssä vastaavia lausekkeita voi kirjoittaa käyttämällä
välimuuttujaa tai yksinkertaisissa tapauksissa \texttt{?:}-operaattorilla.

\subsection{C-yhteensopivuus}

Koska Purkan tulee olla myös makrojen osalta C-yhteensopiva, Purkka osaa
laajentaa C-makroja. Purkka-kääntäjä pystyy muuntamaan makrokutsun
C-lähdekoodiksi, laajentamaan sen C-esikäsittelijällä ja muuntamaan makron
laajennetun muodon takaisin Purkka-koodiksi.

%Purkassa on myös oma makrojärjestelmä, joka toimii hahmotunnistuksella ja
%rekursiolla.\footnote{Referenssikääntäjään (ks. luku \ref{sec:results}) ei ole
%toteutettu makroja.} Jos makrot eivät sisällä rekursiota tai hahmontunnistusta,
%ne voidaan muuntaa C-makroiksi.

Purkka ei sisällä yhtään suoritusaikaista ominaisuutta, joita ei ole C:ssä.
Tämä pitää kielen yksinkertaisena ja C-yhteensopivana. Tämä toisaalta tekee
kielellä ohjelmoinnista työläämpää verrattuna muihin moderneihin
ohjelmointikieliin.

Useat C-toteutukset sisältävät erilaisia laajennoksia, joiden tarkoituksena on
mahdollistaa erilaisten alustariippuvaisten ominaisuuksien käyttö. Benchmarks
Gamessa yksi käytetyimmistä laajennoksista on SIMD-tyypit, joita varten
esimerkiksi GCC-kääntäjällä on oma syntaksinsa. GCC:llä SIMD-tyypit voidaan
määritellä käyttämällä \texttt{\_\_attribute\_\_((vector\_size))} -määrettä.
Purkka tukee useita GCC:n laajennoksia, muun muassa vektorityyppejä.

Syntaksi sisältää myös \texttt{pragma}-avainsanan, jota voidaan käyttää
vastaavasti kuin C:n esikäsittelijän \texttt{pragma}-direktiiviä erilaisten
laajennosten käyttämiseen. Esimerkiksi OpenMP-laajennosta käytetään pragmojen
läpi.

Jotta Purkkaa voisi käyttää mahdollisimman helposti nykyisten järjestelmien
kanssa, Purkka käännetään C-koodiksi. Tämä mahdollistaa olemassa olevien
C-kääntäjien käyttöä Purkan kääntämiseen kaikille mahdollisille alustoille,
joille on olemassa standardien mukainen C-kääntäjä. Esimerkiksi
\texttt{make}-työkalua käyttäville projekteille riittää yksi Makefile-sääntö
Purkka-ohjelmien kääntäminen C-ohjelmiksi. Tämä sääntö on esitetty
ohjelmassa~\ref{fig:makefile}. Muissa käännösautomaatiosovelluksissa Purkan
integrointi osa kääntämistä on luultavasti yhtä helppoa.

\begin{listing}[ht!]
    \inputminted{Makefile}{koodi/Makefile.kieli}
    \caption{Kaksi riviä Makefile-syntaksia riittää Purkan integroimiseen
    Make-ohjelmaa käyttäviin projekteihin.}
    \label{fig:makefile}
\end{listing}

\newpage
\section{Uuden ohjelmointikielen vertaaminen C:hen}

Tässä luvussa verrataan uuden ohjelmointikielen suorituskykymittauksia C:hen
sekä muihin tutkielmassa käsiteltyihin kieliin. Suorituskykymittaukset
näyttävät, että Purkan toteutus ei hidasta C:tä. Tämä taas johtuu siitä, että
Purkassa ei ole yhtään ominaisuutta, joita C:ssä ei ole. Tässä luvussa myös
tutkielman oikeellisuutta ja pohditaan jatkotutkimuskohteita.

\subsection{Vertailun tulokset}
\label{sec:results}

Tätä tutkielmaa varten on toteutettu kääntäjä, joka kääntää ohjelmat
Purkka-kielestä C-kieleen \citep{purkka}. Nopeimmat Benchmarks Gamen
C-toteutukset on käännetty Purkka-kielelle. Nämä ohjelmat on sitten käännetty
Purkka-kääntäjällä takaisin C:ksi, ja käännetty sitten GCC-kääntäjällä
vastaavilla asetuksilla kuin verrattavat C-ohjelmat.
Kuvassa~\ref{fig:purkkabenchmarksgame} verrataan Purkka-toteutuksia muihin
verrattaviin kieliin, ja kuvassa~\ref{fig:purkkabenchmarksgame2} verrataan
Purkka-toteutuksia nopeimpaan C-ohjelmaan. Koska Purkka-toteutukset on
muunnettu nopeimmasta C-toteutuksesta ja käännetty takaisin lähes identtiseen
muotoon, suoritusajan ja muistinkäytön tulisi olla täsmälleen samat kuin
alkuperäisen C-toteutuksen.

Suorituskykymittaukset mittaavat ainoastaan prosessoriin ja muistinkäyttöön
liittyiviä mittauksia, eivätkä esimerkiksi näytönohjaimella suoritettua
laskentaa. Mittaukset tehdään vain yksittäisillä kääntäjäversioilla, joten
mittaukset eivät välttämättä päde muilla saman ohjelmointikielen kääntäjillä.

Suorituskykymittaukset on suoritettu Ubuntu 19.10 -käyttöjärjestelmällä Linux
5.3.0 -kernelillä. Mittaustietokoneessa oli neliytiminen Intel i7-8550U
-prosessori Hy\-per\-Thread\-ing-tuella (4.0 GHz) ja 32 gigatavua DDR3-muistia.
C- ja C++-ohjelmat on käännetty GCC-kääntäjän versiolla 9.2.1, Ada-ohjelmat
GNAT-kääntäjän versiolla 8.3.0 ja Go-ohjelmat go-kääntäjän versiolla 1.12.10.

Mittaukset on toteutettu suorittamalla ohjelma suurimmalla syötteellä viisi
kertaa ja mittaamalla kunkin suorituksen viemä aika ja muistinkäyttö.
Muistinkäyttö on mitattu 200 millisekunnin välein \texttt{libgtop2}-kirjastolla
ja kunkin suorituksen muistinkäytöksi on merkitty suurin mitattu muistikäyttö.

Kuvaajista nähdään, että Benchmarks Gamen tarjoamissa esimerkeissä Purkka on
täsmälleen yhtä tehokas kieli kuin C, eikä se vie enempää muistia. Pienet
eroavaisuudet suuntaan ja toiseen on selitettävissä testauksen aihuttamalla
luonnollisella vaihtelulla. Purkalla toteutettu \texttt{fannkuchredux}-koe on
kuitenkin noin 50\% nopeampi kuin C:llä toteutettu lähes identtinen versio, ja
\texttt{nbody}-koe vie vie vain 80\% muistia verrattuna C-toteutukseen.

Purkka-tiedostot ovat keskimäärin 6\% pienempiä kuin vastaavat C-tiedostot.
Suurimmat kokoerot tiedostoissa tulevat muuttujien määrittelyistä sekä
tyyppi\hyp{}inferenssistä. Kuvassa~\ref{fig:declarations} on yksittäisiä
tyyppimäärittelyjä, jotka ovat Purkassa hieman yksinkertaisempia kuin vastaavat
C-määrittelyt.

\begin{figure}[ht!]
    \begin{adjustbox}{center}
        \begin{tabular}{@{} m{0.55\textwidth} m{0.55\textwidth} @{}} \toprule
            Purkka-määrittely ja C-määrittely & Selite \\ \midrule

            \texttt{let a: u32; \newline unsigned int a;} & Epänegatiivinen 32-bittinen kokonaisluku \\
            \noalign{\vspace{0.3cm}}

            \texttt{let a, b: [\&i8;5]; \newline signed char *a[5], *b[5];} & Taulukko viidestä osoitinmuuttujasta \newline 8-bittiseen kokonaislukuun \\
            \noalign{\vspace{0.3cm}}

            \texttt{const a = ["1", "2", "3"]; \newline const char * const a[] = \{"1", "2", "3"\};} & Kolmen merkkijonon taulukon vakiomuuttuja \\
            \noalign{\vspace{0.3cm}}

            \texttt{let a = fun (a, b) => a + b; \newline int a(int a, int b) \{ return a + b; \}} & Funktio, joka laskee argumenttiensa summan \\ \bottomrule
        \end{tabular}
    \end{adjustbox}
    \label{fig:declarations}
    \caption{Purkan muuttujien määrittelyt verrattuna C:n vastaaviin
    määrittelyihin.}
\end{figure}

\begin{figure}[ht!]
    \begin{adjustbox}{center}
    \begin{minipage}{1.15\textwidth}
    \begin{minipage}{0.5\textwidth}
        \input{data/benchmarkscpu3.tex}
        \vspace*{-0.8cm}
    \end{minipage}
    \begin{minipage}{0.5\textwidth}
        \input{data/benchmarksmem3.tex}
        \vspace*{-0.9cm}
    \end{minipage}
    \end{minipage}
    \end{adjustbox}

    \caption{
        Benchmarks Gamen ohjelmiin perustuvat kuvaajat Purkalla kirjoitettujen
        ohjelmien suorituskyvystä, muistinkäytöstä ja ohjelmien koosta
        verrattuna muilla kielillä kirjoitettujen ohjelmien tuloksiin.}
    \label{fig:purkkabenchmarksgame}
\end{figure}

\begin{figure}[ht!]
    \begin{adjustbox}{center}
        \begin{minipage}{1.25\textwidth}
        \input{data/benchmarkscpu2.tex}
        \end{minipage}
    \end{adjustbox}

    \begin{adjustbox}{center}
        \begin{minipage}{1.25\textwidth}
        \input{data/benchmarksmem2.tex}
        \end{minipage}
    \end{adjustbox}

    \begin{adjustbox}{center}
        \begin{minipage}{1.25\textwidth}
        \input{data/benchmarkslines2.tex}
        \end{minipage}
    \end{adjustbox}
    \caption{
        Benchmarks Gamen ohjelmiin perustuvat kuvaajat Purkalla kirjoitettujen ohjelmien
        suorituskyvystä, muistinkäytöstä ja ohjelmien koosta verrattuna C:llä
        kirjoitettujen ohjelmien tuloksiin.}
    \label{fig:purkkabenchmarksgame2}
\end{figure}

\FloatBarrier

\subsection{Johtopäätökset ja vertailun arviointi}

Purkka on vertailun ainoa kieli, joka on yhtä tehokas kuin C. Tämä johtunee
pitkälti siitä, että Purkka käännetään lähes identtisenä C:ksi, eikä tarjoa
suoritusaikaisia ominaisuuksia, jotka hidastaisivat kieltä. Purkka ei kuitenkaan
päihitä C:tä tutkimuksen määrittelyn mukaisesti, vaan on suorituskyvyltään
täsmälleen yhtä tehokas kuin C.

Kaikkien vertailtavien kielten suunnittelutavoitteissa on mainittu C:n lisäksi
myös C++:n käytön korvaaminen. Tämä on todennäköisesti vaikuttanut kielen
suunnitteluun monimutkaistaen kieliä, jotta kieli pystyisi korvaamaan C++:n
monimutkaiset malliohjelmointirakenteet. Muista kielistä poiketen Purkan
tavoitteena on korvata ainoastaan C, mikä on ohjannut suunnittelua tutkielman
määrittelyjen mukaisesti.

Suorituskykymittaukset eivät kuitenkaan vastaa todellista maailmaa, sillä ne
usein mittaavat vain yksittäistä pientä osa-aluetta, kuten yksittäisten
operaatioiden nopeutta monimutkaisten ohjelmistojen sijaan. Lisäksi
suorituskykymittaukset ovat usein epätarkkoja, sillä monimutkaiset
moniajoympäristöt eivät mahdollista deterministisiä mittauksia.

Mittaukset koskevat vain yksittäisiä ohjelmia ajettuna tietyssä ympäristössä
tietyllä kääntäjällä. Ne eivät siis anna kattavaa kuvaa ohjelmointikielistä,
vaan mittaustuloksia yksittäisen kääntäjätoteutuksen kääntämistä ohjelmista.
Kääntäjät voivat käyttää jopa yksittäisten versioiden välillä erilaisia
optimointeja, jotka voivat nopeuttaa tai hidastaa ohjelmaa juuri mitatulla
syötteellä, mikä pienentää tulosten vertailukelpoisuutta. Kääntäjät voivat
myös toimia paremmin tai huonommin erilaisilla tietokonearkkitehtuureilla,
jolloin esimerkiksi prosessorin valinta vaikuttaa mittaustuloksiin.

Purkkaa ei todennäköisesti oteta laajaan käyttöön, sillä se ei tarjoa
merkittäviä parannuksia C:hen, vaan mahdollistaa lähinnä hieman lyhyemmän
lähdekoodin. Ohjelmointikielten valintaan liittyy usein monimutkaisia syitä,
kuten aikaisempi kokemus tai ohjelmoijan henkilökohtainen mieltymys johonkin
kieleen, eikä projekteissa käytettyjä ohjelmointikieliä valita pelkästään
kielen ominaisuuksien perusteella. Tämän lisäksi Purkan toteuttaja on
yksittäinen opiskelija, kun taas esimerkiksi Go-kielen ja Rustin toteuttajat
ovat kokeneita ohjelmointikielten suunnittelijoita, jonka lisäksi kieliä
ylläpidetään aktiivisesti tunnettujen organisaatioiden toimesta. Tämä on
todennäköisesti myös syy sille, miksi esimerkiksi LISP/c~\citep{clisp1},
C-Mera~\citep{clisp2}, Carp~\citep{clisp3} ja Nymph~\citep{nymph} ovat jääneet
ilman laajempaa huomiota.

Tutkimuksesta voidaan kuitenkin päätellä, että C:tä parempi kieli on
todennäköisesti toteutettavissa. Käännösaikaista turvallisuutta voi parantaa
tiukemmalla tyyppijärjestelmällä ilman suoritusaikaisia haittoja. Tarkemmilla
tyyppimäärittelyillä voi myös tehdä optimointeja, esimerkiksi vaatimalla
osoitinargumentit aina ei-tyhjiksi osoittimiksi.

Osan Purkan ominaisuuksista voisi ottaa käyttöön jopa uusissa C:n versioissa.
Tyyppipäättelyn lisääminen esimerkiksi korvaamalla käyttämättömän
\texttt{auto}-avainsanan\footnote{Avainsanaa käytetään tarkkaan ottaen
\texttt{static}-avainsanan vastinparina, eli ei-staattisten muuttujien
luomiseen. Koska kaikki muuttujat ilman \texttt{static}-määrettä ovat
ei-staattisia, \texttt{auto}-avainsana on turha.} tarkoittamaan pääteltyä
tyyppiä voisi auttaa moderneja C-kääntäjiä käyttävien projektien
kirjoittamista. \texttt{gcc}-käyttäjä tukee tyyppi-inferenssiä
\texttt{\_\_auto\_type}-määreellä. Summatyyppien lisääminen laajennoksena
\texttt{enum}- tai \texttt{union}-tyyppisyntaksiin ei vaikuta tämänhetkisiin
standardien mukaisiin ohjelmiin, mutta mahdollistaisi summatyyppien käytön.
Suurempia syntaktisia muutoksia, kuten tyyppisyntaksin uudelleenkirjoitusta
tuskin pystyy toteuttamaan säilyttäen yhä tuen nykyiselle C:lle.

Purkan jatkokehityksen voisi aloittaa omalla makrojärjestelmällä, sillä
Purkassa ei ole määriteltynä omaa makrojärjestelmää. Purkka tukee C:n
makrojärjestelmää, mutta C:n makrojärjestelmää kattavamman makrojärjestelmä
voisi helpottaa monimutkaisten käännösaikaisten laskujen laskelmista.

Purkkaa varten ei ole kääntäjän lisäksi kehitetty yhtään kehitystyökalua,
vaikka luvussa~\ref{sec:suosio} valmiit työkalut nostetaan tärkeäksi
ominaisuudeksi. Esimerkiksi hyvälaatuinen automaattinen käännösjärjestelmä
helpottaisi kielen käyttämissä uusissa ohjelmissa. Tämän voisi nimetä
ohjelmointikielen nimen mukaisella teemalla Jesariksi, sillä se sitoo
käännettävät ohjelman palaset toisiinsa yhdistettynä mahdollisiin
riippuvuuksiin.

\newpage
\section{Yhteenveto}

Tutkielmassa tutkittiin mahdollisuuksia parantaa C-ohjelmointikieltä. Tätä
tarkoitusta varten luotiin uusi ohjelmointikieli, Purkka. Tutkimuksessa
verrattiin useita erilaisia ohjelmointikieliä C:hen suorituskyvyn sekä
muistinkäytön perusteella, mutta yksikään verrattavista ohjelmointikielistä ei
ollut C:tä nopeampi kuin yksittäisissä mittauksissa. Purkka tarjoaa pienen
syntaktisen parannuksen C:hen ilman ajoaikaisia haittoja. C:n parantaminen on
siis mahdollista ainakin syntaksin osalta, mutta ainakaan Purkka tuskin korvaa
C:tä johtuen sosiaalisista tekijöistä kielten vertailussa.

\hld{Tässä aliluvussa kerrataan tutkimuksen tulokset, eli kerrotaan lyhyesti
uuden ohjelmointikielen ja muiden verrattavien kielten suhteesta C:hen, sekä
muut olennaiset tutkielman asiat.}

\newpage

\renewcommand{\bibname}{Lähteet}
\renewcommand{\BRetrievedFrom}{Saatavilla\ }
\renewcommand{\BOthers}[1]{ja muut\hbox{}}
\renewcommand{\BOthersPeriod}[1]{ja muut\hbox{}}
\renewcommand*{\bibfont}{\interlinepenalty 10000\relax}

\bibliography{references}

\appendixbeginhere
\inputappendix{data/data.tex}
\inputappendix{dokumentaatio.tex}
\appendixendhere


\end{document}

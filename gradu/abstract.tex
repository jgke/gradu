Ohjelmointikielen valinta on tärkeä osa ohjelmistoprojektien suunnittelua.
Vaikka ohjelmointikielet uudistuvat nopeaan tahtiin, nykypäivänä on yhä
tavallista valita ohjelmiston toteutukseen C-ohjelmointikieli, joka on
standardoitu yli 30 vuotta sitten. Tutkielmassa tutkitaan syitä, miksi C on
nykypäivänä vieläkin laajassa käytössä uudempien ohjelmointikielten sijaan.

Tutkielmassa C:hen verrattaviksi ohjelmointikieliksi valitaan Ada, C++, D,
Go sekä Rust. Kaikki viisi kieltä ovat tehokkaita. Tämän lisäksi jokaisen
kielten historiassa on ollut tavoitteena korvata C:n käyttö.
Ohjelmointikieliä verrataan C:hen suorituskyvyn, muistinkäytön sekä
C-yhteensopivuuden osalta. Tämän lisäksi tutkielmassa selvitetään
tärkeimpiä C:n ominaisuuksia sekä C:n kehitettävissä olevia ominaisuuksia.
Tuloksia käytetään uuden Purkka-ohjelmointikielen suunnitteluun.

Muut ohjelmointikielet todetaan vertailussa C:tä hitaammiksi. Tämän lisäksi
muiden ohjelmointikielten ominaisuudet, kuten automaattisen
muistinhallinnan, todetaan aiheuttavan ongelmia C-yhteensopivuudelle.

C:n tärkeimmiksi ominaisuuksiksi nousevat esiin yksinkertaisuus, tehokkuus
sekä alustariippumattomuus. Nämä ominaisuudet otetaan huomioon
Purkka-kielen suunnittelussa, jossa painotetaan näiden lisäksi
yhteensopivuutta C-ohjelmointikielen kanssa.

Tutkielmaa varten kehitetty Purkka-kieli on suunniteltu C:n kaltaiseksi
ohjelmointikieleksi, jossa on muutettu C:n syntaksia yksinkertaisemmaksi ja
johdonmukaisemmaksi. Suorituskykymittauksissa todetaan, että Purkan muutokset
C:hen eivät aiheuta suoritusaikaisia rasitteita. Koska Purkka-kieli käännetään
C:ksi, se on mahdollisimman yhteensopiva nykyisten kääntäjien kanssa.

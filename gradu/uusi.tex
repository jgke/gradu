\section{Uuden ohjelmointikielen määrittäminen}

\hl{Tässä luvussa määritetään uusi ohjelmointikieli toisen ja kolmannen luvun
tulosten ohjaamana.}

\subsection{Tyypit}

\hl{Koska kieli ei voi erota ratkaisevasti C:stä suoritusaikaisesti, tyypitys
on todennäköisesti hyvin lähellä C:n tyypitystä. Tutkielmassa tutkitaan, voiko
yleisiä käyttötapauksia helpottaa, kuten virhearvoja, geneerisiä
tietorakenteita sekä summatyyppejä. Näistä geneeriset tietorakenteet ovat
luultavasti liian monimutkainen asia sisällytettäväksi yksinkertaisena
pidettävään kieleen johtuen niiden vaatimasta geneerisestä ohjelmoinnista.
Esimerkiksi summatyypit taas voidaan toteuttaa helposti myös C:llä.}

\subsection{Syntaksi}

\hl{Kielestä tulee todennäköisesti myös syntaksin kannalta C:n sukulaiskieli.
Tämä helpottaa kielen omaksumista, mikäli ohjelmoija osaa C:tä. Erityisesti
virheidenhallinnassa on kuitenkin mahdollisuuksia tehdä huomattavia parannuksia
C:hen verrattuna.}

\subsection{Makrot}

\hl{Koska makrot eivät aiheuta suoritusaikaisia haittoja, uuden kielen
makrojärjestelmässä voi olla hyvinkin suuria eroavaisuuksia C:n
makrojärjestelmään verrattuna. Yhteensopivuuden parantamiseksi
makrojärjestelmän tulisi kuitenkin ymmärtää C:n makroja sekä mahdollistaa
C-makrojen tuottaminen. Jos makrojärjestelmästä tekee Turing-täydellisen,
kaikkia makroja ei pysty muuntamaan C:n makroiksi johtuen C:n makrojärjestelmän
rajoituksista. Tämän lisäksi hygieeninen makrojärjestelmä ei välttämättä tee
C:n makrojärjestelmän tukemisesta yksinkertaista.}

\hl{Yksi mahdollinen tapa tuottaa kattava makrojärjestelmä olisi ottaa Rustin
makrojärjestelmästä mallia, mutta tämä jättäisi avoimeksi kysymykseksi
yhteensopivuuden C:n makrojärjestelmän kanssa.}

\subsection{Suoritusaikaiset ominaisuudet}

\hl{Koska kielen tulee olla täysin yhteensopiva suoritusaikaisesti C:n kanssa,
kielessä ei välttämättä ole yhtään C:stä eroavia suoritusaikaisia
ominaisuuksia. Kieli kuitenkin voi mahdollistaa vapaaehtoisia ominaisuuksia,
joita standardien mukaisesta C:stä ei löydy, kunhan ohjelmoijaa ei pakoteta
käyttämään näitä.}

\subsection{Ohjelmointikielen kääntäminen}

\hl{Yhteensopivuuden C:n kanssa saa helposti toteutettua kääntämällä
ohjelmointikielen C:ksi. Tällöin ohjelmointikieli toimii kaikissa
järjestelmissä, joissa C toimii. Tämä mahdollistaa myös uuden ohjelmointikielen
kokeilemisen olemassa olevan C-projektin kanssa ilman, että projektin tulisi
muuten käyttää uutta ohjelmointikieltä.}

\section{Tutkielman rakenne} 

Tutkielman aloittavassa johdantokappaleessa perustellaan tutkielman tärkeys
sekä kuvaillaan lopun tutkielman sisältö ja tutkimuskysymys.

Toisessa kappaleessa esitetään tutkielman kannalta oleellinen teoria, eli
C-kielen historiaa ja nykypäivää sekä pohditaan, minkälaiset ominaisuudet
tarvittaisiin kieleltä, joka voisi korvata C:n kokonaan.

Kolmannessa kappaleessa kerrotaan, miksi muut tutkitut kielet eivät täytä näitä
ominaisuuksia.

Neljännessä kappaleessa kuvaillaan kappaleen kolme käsittelemien ominaisuuksien
täyttävä ohjelmointikieli.

Lopettavassa kappaleessa kerrataan tutkielman tavoitteet, vaiheet ja tulokset
sekä pohditaan tutkielman puutteita ja mahdollisia jatkotutkimuskohteita.

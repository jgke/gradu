\appendixsection{Hahmotelma kielen syntaksista: LISP-perhe}
\appendixlabel{app:grammar-lisp}

\bnfdescription.
Syntaksi on perustettu LISP-perheen pohjalta puukieleksi.

Syntaksin hyvät puolet ovat LISPin yleisesti tunnettu yksinkertainen, tiivis
syntaksi sekä valmiudet voimakkaaseen makroprosessointiin. LISP-tyylisiä C:hen
kääntyviä kieliä on kuitenkin tehty jo useita, esim.~\citet{clisp3}, jonka
lisäksi LISPin kirjoittaminen on lähes mahdotonta ilman automaattista
muistinhallintaa.

\inputminted{\bnflexer}{lisp-base.bnf}

Hello World -ohjelma:

\inputminted{lisp}{hello.lisp}

\newpage

Tree-ohjelma:

\inputminted{lisp}{tree.lisp}

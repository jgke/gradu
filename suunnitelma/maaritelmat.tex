\section{Määritelmät} 

Verrattavissa ohjelmointikielissä on yritetty parantaa C:n huonoja puolia
hyvien puolien kustannuksella, usein lisäämällä kieleen turvallisuutta
parantavia ominaisuuksia. Tämä kuitenkin heikentää kielen tehokkuutta tai
alustayhteensopivuutta tehden kielten suorasta vertailusta mahdotonta. Jos
määritellään absoluuttiset reunaehdot, voidaan näiden puitteissa vertailla
kieliä.

Määritellään C:tä paremmaksi kieleksi kieli, joka C:hen verrattuna aina:

\begin{itemize}
    \item on yhtä nopea tai nopeampi
    \item käyttää saman verran muistia tai vähemmän
    \item toimii kaikissa järjestelmissä, joissa C toimii
    \item on helpompi käyttää
\end{itemize}

Jos yksikin näistä kriteereistä ei pidä, verrattava kieli ei ole absoluuttisesti
C:tä parempi.

On huomioitava, että lukuisten olemassa olevien C:n kirjastojen, rajapintojen
ja projektien vuoksi yhteensopivuus C:n kanssa tulee olla saumatonta
mahdollisten vaihtoehtoisten ohjelmointikielten osalta, jotta kielen
vaihtaminen olisi mahdollista. Tämä sisältää myös kirkastorutiinien kutsumisen
muista ohjelmointikielistä, sillä C on lukuisissa järjestelmissä \emph{lingua
franca}, jonka avulla ohjelmointikielet pystyvät kommunikoimaan keskenään.
Esimerkiksi Python-ohjelmointikieli~\citep{python} sisältää tuen C-funktioiden
kutsumiseen~\citep{pythonffi}, jota voi käyttää muiden ohjelmointikielten
funktioiden kutsumiseen, mikäli käyttäjä kirjoittaa 'sillan' C-ohjelmana.

Ohjelmointikielten ekosysteemit koostuvat eri tahojen luomista kirjastoista. On
tärkeää, että näihin kirjastoihin pääsee käsiksi mahdollisimman helposti. Tämä
vaatii toimiakseen hyvät työkalut, kuten
paketinhallintajärjestelmän\footnote{eng. package manager}, hyvän
käännöstyökalun sekä kehitysympäristöt. \hl{paketinhallinta -> pakkaushallinta:
'paketinhallinta' on yleisesti käytössä oleva termi, ks. esim} \\
\url{https://www.debian.org/index.fi.html} \\
\url{https://fi.wikipedia.org/wiki/Paketinhallintaj\%C3\%A4rjestelm\%C3\%A4} \\
\hl{Onhan se tietenkin hassu käännöksenä, kun pakkaus olisi 'oikeampi' käännös
englannin package-sanasta, mutta luonnolliset kielet eivät toimi loogisesti.}

Kirjastojen lisenssien tulee myös olla yhteensopivia, jotta kirjastoja voi
käyttää yhdessä toistensa kanssa. Eri ohjelmointikieliekosysteemeissä on
käytössä erilaisia lisenssejä -- JavaScript-kirjastot ovat usein
MIT-yhteensopivia, kun taas Java-kirjastot ovat usein Apache 2 -yhteensopivia.
Liitteessä~\ref{app:github} on taulukko GitHub-verkkopalvelun sisältämien
julkisten projektien määrä ryhmiteltynä lisenssin ja kielen mukaan.

Kirjastot voi myös julkaista useammalla kuin yhdellä lisenssillä, jolloin
käyttäjät voivat päättää, mitä lisenssiä haluaa seurata. On kuitenkin
filosofinen kysymys, onko esimerkiksi Apache2+GPL parempi lisenssi kuin pelkkä
GPL, sillä Apache2 ei vaadi tiettyjä oikeuksia loppukäyttäjille, kuten pääsyä
ohjelmiston lähdekoodiin~\citep{apachetldr, gpl3tldr}.

\subsection{C:hen verrattavissa olevat ohjelmointikielet}

C:n suosiosta johtuen historian saatossa on tehty useita C:n kilpailijoita,
jotka ovat yrittäneet parantaa C:tä joidenkin C:n hyvien puolien
kustannuksella. Muutamat näistä ovat päätyneet hyvin suosituiksi
ohjelmointikieliksi, kuten esimerkiksi C++ ja Go. Kielten suosion mittaamiseen
on tehty useita projekteja, jotka vertailevat kieliä esimerkiksi hakutulosten
tai projektien mukaan. Näitä ovat esimerkiksi TIOBEn ohjelmointikielten suosion
indeksi~\citep{tiobe} ja GitHubin julkaisema Octoverse~\citep{octoverse}.
Valitsemalla vertailuun suosittuja kieliä voidaan tutkia, miksi muuten selkeästi
täysin käyttökelpoinen kieli ei ole syrjäyttänyt C:tä.

Koska C on yksi suorituskyvyltään nopeimmista ohjelmointikielistä, on myös
tärkeää rajata vertailuun vain muita nopeita kieliä -- korkeamman tason
ohjelmointikielet on tarkoitettu ohjelmointinopeuden parantamiseen. Tällöin
suorituskyky on jäänyt vähemmälle, koska on usein halvempaa optimoida
kirjoittamiseen käytettyä aikaa kuin nopeuttaa suorituskykyä muutamalla
prosentilla~\citationneeded. Yksi kattava suorituskykyä mittaava vertailu on
Benchmarks game~\citep{benchmarks}, jossa pyritään kirjoittamaan mahdollisimman
nopea ohjelma pysyen silti kielelle idiomaattisessa ohjelmakoodissa.

Ada, C++, D, Go ja Rust nousevat esiin näistä kolmesta vertailusta. Kaikki
viisi kieltä ovat sekä suosittuja että tehokkaita.

Lähivuosina on tehty myös useita C:hen käännettäviä ohjelmointikieliä, jotka
ovat jääneet pitkälti ilman mitään näkyvyyttä, kuten LISP/c~\citep{clisp1},
C-Mera~\citep{clisp2} ja Nymph~\citep{nymph}. LISP/c ja C-Mera ovat
LISP-perheeseen kuuluvia C:ksi kääntyviä ohjelmointikieliä, jotka on
suunniteltu etupäässä parantamaan C:n syntaksia rikkaammalla
makrojärjestelmällä. Nymph taas on olio-ohjelmointikieli.

\hl{Tässä voisi olla jotain pohdintaa, miksi nuo ovat jääneet huomiotta ja
miten tämän voisi välttää... Tai sitten poistaa tuon viimeisen kappaleen
kokonaan.}


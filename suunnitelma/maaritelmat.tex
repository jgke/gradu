\section{Määritelmät} 

\subsection{Aidosti paremman kielen rajoitteet}
\label{sec:abs}

Verrattavissa ohjelmointikielissä on yritetty parantaa C:n huonoja puolia
hyvien puolien kustannuksella, usein lisäämällä kieleen turvallisuutta
parantavia ominaisuuksia tai tehden kielestä helppokäyttöisemmän esimerkiksi
automaattisella muistinhallinnalla. Tämä kuitenkin heikentää kielen tehokkuutta
tai alustayhteensopivuutta tehden kielten suorasta vertailusta hankalaa. 
Määrittelemällä absoluuttiset reunaehdot voidaan vertailla kieliä
objektiivisesti. Jos yksikin näistä kriteereistä ei pidä, verrattava kieli ei
ole aidosti C:tä parempi, vaan se häviää C:lle joissain osa-alueissa ja
vastaavasti on parempi joissakin toisissa.

Tässä tutkielmassa vertaillaan kolmea osa-aluetta ohjelmointikielistä:
suorituskykyä, muistinkäyttöä sekä yhteensopivuutta C:n kanssa.

Ohjelmointikielellä kirjoitetun tulee olla ajoaikaisesti vähintään yhtä nopea
kuin vastaava C:llä kirjoitettu ohjelma. Kieli siis ei saa vaatia ohjelmoijaa
käyttämään mitään kielen ominaisuuksia, jotka voisivat hidastaa ohjelmien
suoritusta C:hen verrattuna. Monet ajoaikaiset turvallisuutta lisäävät
ominaisuudet, kuten muistialueiden tarkistukset, hidastavat kielen ajoaikaista
nopeutta.

Ohjelmointikielellä toteutettu ohjelma ei myöskään saa käyttää enempää muistia
kuin vastaava C-ohjelma. Tämä koskee myös vakiokirjastoa\defword{standard
library} -- yksikin konkreettinen vakiokirjaston funktio linkitettynä ohjelmaan
kasvattaa ohjelman kokoa. Mikäli jokin vakiokirjasto toteutetaan, sen
käyttäminen tulee olla ohjelmoijalle täysin vapaaehtoista. Yksi tapa toteuttaa
tämä on liittää vain käytetyt funktiot osaksi ohjelmaa, jolloin käyttämättömät
funktiot eivät kasvata ohjelman kokoa\footnote{Jos kieltä käytetään tarpeeksi
monessa ohjelmassa, tilankäytön kannalta on edullisempaa säilöä vakiokirjasto
jaetussa tiedostossa. Tällöin tosin käännösaikainen optimointi on vaikeampaa,
ja ohjelman käynnistäminen on hieman hitaampaa vakiokirjaston funktioiden
dynaamisesta lataamisesta johtuen.}.

Ohjelmointikielen tulee olla täysin yhteensopiva C:n kanssa. Tämä koskee
C-koodin kutsumista (C\hyp{}vierasfunktiorajapinta, C-VFR\defword{Foreign
function interface, FFI}) sekä muiden ohjelmointikielten C-rajapinnan läpi
kutsumista (VFR). Kielen tulee näiden lisäksi toimia kaikissa ympäristöissä,
joissa C toimii.

\subsection{Kielen suosioon vaikuttavat tekijät}

Eräässä tutkimuksessa~\citep{empiricalpopularity} tutkittiin syitä
ohjelmointkielten valintaan. Yhden tutkimuksen järjestämän kyselyn (s. 8,
Slashdotissa julkaistu kysely, n=1679) perusteella kielen valintaan vaikuttaa
avoimen lähdekoodin kirjastojen saatavuus, olemassaolevien ohjelmien
jatkokehitys sekä kielen tunnettavuus ohjelmoijien keskuudessa -- ohjelmoijat
siis suosivat jo käytettyjä ohjelmointikieliä uusien kielten sijaan. Saman
kyselyn vastaajat arvioivat suorituskyvyn turvallisuutta tärkeämmäksi.

Slashdot-kyselyssä noin 40\% vastaajista arvioi tärkeäksi kriteeriksi työkalut.
Kielen tulisi siis tarjota toimivat työkalut, kuten
paketinhallintajärjestelmän\defword{package manager}, käännöstyökalun sekä
valmiudet olemassaoleviin kehitysympäristöihin integroitumiselle.
Olemassaolevien C-ohjelmistojen tukeminen on välttämätöntä, mutta haastavaa
johtuen C:n ekosysteemin monimuotoisuudesta, erityisesti kääntämisen
helpottamiseksi tehdyt työkalut.

Kyselyssä selvitettiin myös suosittuja ominaisuuksia ohjelmointikieleltä (s.
13, SaaS MOOC -kurssin yhteydessä oleva kysely, n=415). Useita tutkimuksessa
selvitetyistä suosituimmista ominaisuuksista ei ole mahdollista toteuttaa
johtuen luvussa~\ref{sec:abs} määritetyistä rajoitteista, kuten poikkeuksia ja
rajapintoja. Useat muut tutkimuksessa esiin nousseet ominaisuudet, kuten
suorituskyky, dokumentaatio ja yksinkertaisuus, ovat täysin mahdollisia
toteuttaa.

Tutkimuksessa myös verrattiin tiettyjen toteamuksien keskenäistä korrelaatiota.
Kielen tiiviys\defword{terseness, expressiveness} korreloi eniten
(korrelaatiokertoimella 0.76) kielestä pitämisen kanssa (s. 13, The Hammer
Principle -sivustolla julkaistu kysely).

%Eräässä blogikirjoituksessa \citep{microsoftdictperf} vertaillaan C\#:n ja
%C++:n suorituskykyä. Kirjoituksessa C\#:llä kirjoitettu yksinkertainen toteutus
%suoriutuu tehtävästä nopeammin ja yksinkertaisemmin kuin optimoitu C++-ohjelma.
%Vasta usean C++-ohjelma päihitti C\#-toteuden nopeudella. C\#-toteutus
%kuitenkin vei noin neljä kertaa enemmän muistia kuin C++-toteutus.

Tutkimuksen perusteella valmiiksi suosittuja kieliä käytetään enemmän myös
uusissa projekteissa. Olemassaolevien kirjastojen tärkeyttä korostetaan
useassa kohdassa tutkimusta. Täysin C:n kanssa yhteensopiva kieli voi käyttää
C:lle tehtyjä kirjastoja, jolloin kielellä on käytettävissään laaja C:n
ekosysteemi\footnote{Esimerkiksi GitHubista hakusanalla 'library' löytyy yli
26\,000 C:llä kirjoitettua projektia.}.

Kuten luonnollisissa kielissä, saman kieliperheen jäsenten opettelu muuttuu
helpommaksi sitä mukaa, mitä useampaa kieltä tuntee\citationneeded. Uusien
ohjelmointikielten käyttöönottoa helpottaa siis muiden vastaavien kielten
osaaminen. Suunnittelemalla C:n korvaajan C:n kanssa samankaltaiseksi kieleksi
voidaan pienentää uuden kielen opettelun kynnystä. Kielen eriävät ominaisuudet
tulee siis toteuttaa siten, että ne ovat mahdollisimman helppoja oppia C:stä
kieleen siirtyvälle ohjelmoijalle.

%Helppokäyttöisyys \\
%- turvallisuus (esim. tyypit) \\
%- esim. annotaatiot \\
%- selkeä mitä tekee

%\subsection{Mahdollisia kielen ominaisuuksia}
%
%Kielen ominaisuudet vaikuttavat sillä rakennettujen ohjelmistojen
%arkkitehtuuriin. Tyyppiluokat tai rajapinnat kannustavat vahvan
%tyypityksen kautta tyyppiturvalliseen ohjelmointiin, kun taas dynaamiset kielet
%(esimerkiksi LISP-perheen kielet) kannustavat nopeaan kehitystahtiin staattisen
%turvallisuuden hinnalla. Yksinkertaiset kielet, kuten C ja Go, tarjoavat usein
%vain yhden selkeän tavan toteuttaa yksittäiset funktiot, kun taas
%monimutkaisemmat kielet tarjoavat lukuisia vaihtoehtoja.\citationneeded
%
%BitC-ohjelmointikielen sähköpostilistalla käydyssä keskustelussa pohdittiin
%mahdollisia ongelmia vahvan tyypityksen (etenkin tyyppiluokkien) käytöstä
%matalan tason ohjelmoinnissa~\citep{bitc}. Shapiron sähköpostissa todetaan,
%että tyyppiluokkia ei voi toteuttaa ilman ajoaikaista tukea.
%
%On huomioitava, että lukuisten olemassa olevien C:n kirjastojen, rajapintojen
%ja projektien vuoksi yhteensopivuus C:n kanssa tulee olla saumatonta
%mahdollisten vaihtoehtoisten ohjelmointikielten osalta, jotta kielen
%vaihtaminen olisi mahdollista. Tämä sisältää myös kirkastorutiinien kutsumisen
%muista ohjelmointikielistä, sillä C on lukuisissa järjestelmissä \emph{lingua
%franca}, jonka avulla ohjelmointikielet pystyvät kommunikoimaan keskenään.
%Esimerkiksi Python-ohjelmointikieli~\citep{python} sisältää tuen C-funktioiden
%kutsumiseen~\citep{pythonffi}, jota voi käyttää muiden ohjelmointikielten
%funktioiden kutsumiseen, mikäli käyttäjä kirjoittaa ''sillan'' C-ohjelmana.
%Käytännössä jokaisesta aktiivisesti käytetystä ohjelmointikielestä on
%mahdollista kutsua C-koodia.
%
%Ohjelmointikielten ekosysteemit koostuvat eri tahojen luomista kirjastoista. On
%tärkeää, että näihin kirjastoihin pääsee käsiksi mahdollisimman helposti.

% Kirjastojen lisenssien tulee myös olla yhteensopivia, jotta kirjastoja voi
% käyttää yhdessä toistensa kanssa. Eri ohjelmointikieliekosysteemeissä on
% käytössä erilaisia lisenssejä -- JavaScript-kirjastot ovat usein
% MIT-yhteensopivia, kun taas Java-kirjastot ovat usein Apache 2 -yhteensopivia.
% Liitteessä~\ref{app:github} on taulukko GitHub-verkkopalvelun sisältämien
% julkisten projektien määrä ryhmiteltynä lisenssin ja kielen mukaan.
% 
% Kirjastot voi myös julkaista useammalla kuin yhdellä lisenssillä, jolloin
% käyttäjät voivat päättää, mitä lisenssiä haluaa seurata. On kuitenkin
% filosofinen kysymys, onko esimerkiksi Apache2+GPL parempi lisenssi kuin pelkkä
% GPL, sillä Apache2 ei vaadi tiettyjä oikeuksia loppukäyttäjille, kuten pääsyä
% ohjelmiston lähdekoodiin~\citep{apachetldr, gpl3tldr}. Apache2 siis antaa
% ohjelmoijille enemmän vapauksia GPL3:een verrattuna, mutta käyttäjien vapaus
% esimerkiksi muokata ohjelmistoa kärsii tästä.

\subsection{C:hen verrattavissa olevat ohjelmointikielet}

Historian saatossa on tehty useita C:n kilpailijoita, jotka ovat yrittäneet
parantaa C:tä joidenkin C:n hyvien puolien kustannuksella. Muutamat näistä ovat
päätyneet hyvin suosituiksi ohjelmointikieliksi, kuten esimerkiksi C++ ja Go.
Kielten suosion mittaamiseen on tehty useita projekteja, jotka vertailevat
kieliä esimerkiksi hakutulosten tai projektien mukaan. Näitä ovat esimerkiksi
TIOBEn ohjelmointikielten suosion indeksi~\citep{tiobe} ja GitHubin julkaisema
Octoverse~\citep{octoverse}. Valitsemalla vertailuun suosittuja kieliä voidaan
tutkia, miksi muuten selkeästi täysin käyttökelpoinen kieli ei ole syrjäyttänyt
C:tä.

\hl{... Lisäksi olisi hyvä selittää, millä tavoin vertailu ylipäätään voisi
selvittää syitä C:n syrjäyttämättä jäämiseen.}

Koska C on yksi suorituskyvyltään nopeimmista ohjelmointikielistä, on
tärkeää rajata vertailuun vain muita tehokkaita kieliä -- korkeamman tason
ohjelmointikielet on tarkoitettu ohjelmointinopeuden parantamiseen ja
turvallisempien ohjelmistojen toteuttamiseen nopeiden ohjelmien sijaan. Tällöin
suorituskyky on jäänyt vähemmälle, koska on usein halvempaa optimoida
kirjoittamiseen käytettyä aikaa kuin nopeuttaa suorituskykyä\citationneeded.
Yksi kattava suorituskykyä mittaava vertailu on Benchmarks
game~\citep{benchmarks}, jossa pyritään kirjoittamaan mahdollisimman nopea
ohjelma pysyen silti kielelle idiomaattisessa ohjelmakoodissa.

Ada, C++, D, Go ja Rust nousevat esiin näistä kolmesta vertailusta. Kaikki
viisi kieltä ovat sekä suosittuja että tehokkaita.

Lähivuosina on tehty myös useita C:hen käännettäviä ohjelmointikieliä, jotka
ovat jääneet pitkälti ilman mitään näkyvyyttä, kuten LISP/c~\citep{clisp1},
C-Mera~\citep{clisp2}, Carp~\citep{clisp3} ja Nymph~\citep{nymph}. LISP/c,
C-Mera ja Carp ovat LISP-perheeseen kuuluvia C:ksi kääntyviä ohjelmointikieliä,
jotka pyrkivät parantamaan C:n syntaksia korvaamalla sen LISP-perheen
syntaksilla. Nymph taas on olio-ohjelmointikieli. Erityisesti Carp on tämän
tutkielman kannalta kiintoisa ohjelmointikieli, sillä se on C:ksi kääntyvä
ohjelmointikieli, joka on suunniteltu mahdollisimman suorituskykyiseksi.

\hl{Tässä voisi olla jotain pohdintaa, miksi nuo ovat jääneet huomiotta ja
miten tämän voisi välttää. Erityisesti Carp tuntuu tätä tutkielmaa vastaavalta
kieleltä, olisi ikävää jättää se täysin huomiotta.}

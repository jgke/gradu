\section[Ohjelmointikielten vertailuun liittyvät määritelmät]
{Ohjelmointikielten vertailuun liittyvät \\ määritelmät}

\subsection{Aidosti paremman kielen rajoitteet}
\label{sec:abs}

Verrattavissa ohjelmointikielissä on yritetty parantaa C:n huonoja puolia
hyvien puolien kustannuksella, usein lisäämällä kieleen turvallisuutta
parantavia ominaisuuksia tai tehden kielestä helppokäyttöisemmän esimerkiksi
automaattisella muistinhallinnalla. Tämä kuitenkin heikentää kielen tehokkuutta
tai alustariippumattomuutta tehden kielten suorasta vertailusta hankalaa.
Määrittelemällä absoluuttiset reunaehdot voidaan vertailla kieliä
ehtojen puitteissa objektiivisesti. Jos yksikin näistä kriteereistä ei pidä,
verrattava kieli ei ole aidosti C:tä parempi, vaan se häviää C:lle joissain
osa-alueissa ja vastaavasti voi olla parempi joissakin toisissa.

Tutkielmassa vertaillaan kolmea osa-aluetta ohjelmointikielistä: suorituskykyä,
muistinkäyttöä sekä yhteensopivuutta C:n ja muiden ohjelmointikielten kanssa.
Tutkielmassa käsitellään myös subjektiivisia kielten ominaisuuksia, kuten
kielen tiiviyttä\defword{terseness, expressiveness}, mutta näitä
ominaisuuksia ei huomioida kielten paremmuusvertailussa.

Ohjelmointikielellä kirjoitetun tulee olla suoritusaikaisesti vähintään yhtä nopea
kuin vastaava C:llä kirjoitettu ohjelma. Kieli siis ei saa vaatia ohjelmoijaa
käyttämään mitään kielen ominaisuuksia, jotka voisivat hidastaa ohjelmien
suoritusta C:hen verrattuna. Monet suoritusaikaiset turvallisuutta lisäävät
ominaisuudet, kuten muistialueiden tarkistukset, hidastavat kielen suoritusaikaista
nopeutta.

Ohjelmointikielellä toteutettu ohjelma ei myöskään saa käyttää enempää muistia
niin suoritusaikaisesti kuin talletusvälineelläkään kuin vastaava C-ohjelma. Tämä
koskee myös vakiokirjastoa\defword{standard library} -- yksikin konkreettinen
vakiokirjaston funktio linkitettynä ohjelmaan kasvattaa ohjelman kokoa. Mikäli
jokin vakiokirjasto toteutetaan, sen käyttäminen tulee olla ohjelmoijalle
täysin vapaaehtoista. Yksi tapa toteuttaa tämä on liittää vain käytetyt
funktiot osaksi ohjelmaa, jolloin käyttämättömät funktiot eivät kasvata
ohjelman kokoa\footnote{Jos kieltä käytetään useassa ohjelmassa, tilankäytön
kannalta on edullisempaa säilöä vakiokirjasto jaettuna kirjastona. Tämä voi
myös nopeuttaa ohjelmien käynnistysnopeuksia, tosin tällöin ohjelmien
käännösaikainen optimointi on vaikeampaa. Jos vakiokirjastoa ei ole ladattu
muistiin ohjelman käynnistyessä, ohjelman käynnistys voi kestää hieman
kauemmin.}.

Ohjelmointikielen tulee olla täysin yhteensopiva C:n kanssa. Tämä koskee
C-koodin kutsumista (C\hyp{}vierasfunktiorajapinta, C-VFR\defword{Foreign
function interface, FFI}) sekä kielen funktioiden kutsumista muiden
ohjelmointikielten C-rajapinnan läpi kutsumista (VFR). Kielen tulee näiden
lisäksi toimia kaikissa ympäristöissä, joissa C toimii.

\subsection{Kielen suosioon vaikuttavat tekijät}

Eräässä tutkimuksessa~\citep{empiricalpopularity} tutkittiin syitä
ohjelmointkielten valintaan. Yhden tutkimuksen järjestämän kyselyn (s. 8,
\mbox{Slashdotissa} julkaistu kysely, n=1679) perusteella kielen valintaan
vaikuttaa avoimen lähdekoodin kirjastojen saatavuus, olemassa olevien ohjelmien
jatkokehitys sekä kielen tunnettavuus ohjelmoijien keskuudessa -- ohjelmoijat
siis suosivat jo käytettyjä ohjelmointikieliä uusien kielten sijaan. Saman
kyselyn vastaajat arvioivat suorituskyvyn turvallisuutta tärkeämmäksi.

Saman tutkimuksen järjestämässä Slashdot-sivustolla julkaistussa kyselyssä noin
40\% vastaajista arvioi tärkeäksi kriteeriksi työkalut. Kyselyn perusteella
kielen olisi siis hyvä tarjota toimivat työkalut, kuten kirjastopakettien
hakemiseen paketinhallintajärjestelmän\defword{package manager}, nopean ja
käyttäjäystävällisen käännöstyökalun sekä valmiudet olemassa oleviin
kehitysympäristöihin integroitumiselle. Olemassaolevien C-ohjelmistojen
tukeminen on välttämätöntä, mutta haastavaa johtuen C:n ekosysteemin
monimuotoisuudesta, erityisesti lukuisista kääntämistyökaluista.

Tutkimuksessa selvitettiin myös suosittuja ominaisuuksia ohjelmointikieliltä
(s. 13, SaaS MOOC -kurssin yhteydessä oleva kysely, n=415). Useita
tutkimuksessa selvitetyistä suosituimmista ominaisuuksista ei ole mahdollista
toteuttaa johtuen luvussa~\ref{sec:abs} määritetyistä rajoitteista, kuten
poikkeuksia ja rajapintoja. Useat muut tutkimuksessa esiin nousseet
ominaisuudet, kuten suorituskyky, dokumentaatio ja yksinkertaisuus, ovat täysin
mahdollisia toteuttaa.

Tutkimuksessa myös verrattiin tiettyjen toteamuksien keskenäistä korrelaatiota.
Kielen tiiviys korreloi eniten (korrelaatiokertoimella 0.76) kielestä pitämisen
kanssa (s. 13, The Hammer Principle -sivustolla julkaistu kysely).

%Eräässä blogikirjoituksessa \citep{microsoftdictperf} vertaillaan C\#:n ja
%C++:n suorituskykyä. Kirjoituksessa C\#:llä kirjoitettu yksinkertainen toteutus
%suoriutuu tehtävästä nopeammin ja yksinkertaisemmin kuin optimoitu C++-ohjelma.
%Vasta usean C++-ohjelma päihitti C\#-toteuden nopeudella. C\#-toteutus
%kuitenkin vei noin neljä kertaa enemmän muistia kuin C++-toteutus.

Tutkimuksen perusteella valmiiksi suosittuja kieliä käytetään enemmän myös
uusissa projekteissa. Olemassaolevien kirjastojen tärkeyttä korostetaan
useassa kohdassa tutkimusta. Täysin C:n kanssa yhteensopiva kieli voi käyttää
C:lle tehtyjä kirjastoja, jolloin kielellä on käytettävissään laaja C:n
ekosysteemi\footnote{Esimerkiksi GitHubista hakusanalla 'library' löytyy yli
26\,000 C:llä kirjoitettua projektia.}.

Uusia ohjelmointikieliä opetellessa ohjelmoijat turvautuvat aikaisemmista
kielistä opittuihin käytäntöihin~\citep{languagelearning}. Uusien
ohjelmointikielten käyttöönottoa helpottaa siis muiden vastaavien kielten
osaaminen, sillä aikaisempi kokemus tukee vastaavan kielen opiskelua.
Suunnittelemalla C:n korvaajan C:n kanssa samankaltaiseksi kieleksi voidaan
pienentää uuden kielen opettelun kynnystä. Kielen eriävät ominaisuudet olisi
siis hyvä toteuttaa siten, että ne ovat mahdollisimman helppoja oppia C:stä
kieleen siirtyvälle ohjelmoijalle.

%Helppokäyttöisyys \\
%- turvallisuus (esim. tyypit) \\
%- esim. annotaatiot \\
%- selkeä mitä tekee

%\subsection{Mahdollisia kielen ominaisuuksia}
%
%Kielen ominaisuudet vaikuttavat sillä rakennettujen ohjelmistojen
%arkkitehtuuriin. Tyyppiluokat tai rajapinnat kannustavat vahvan
%tyypityksen kautta tyyppiturvalliseen ohjelmointiin, kun taas dynaamiset kielet
%(esimerkiksi LISP-perheen kielet) kannustavat nopeaan kehitystahtiin staattisen
%turvallisuuden hinnalla. Yksinkertaiset kielet, kuten C ja Go, tarjoavat usein
%vain yhden selkeän tavan toteuttaa yksittäiset funktiot, kun taas
%monimutkaisemmat kielet tarjoavat lukuisia vaihtoehtoja.\citationneeded
%
%BitC-ohjelmointikielen sähköpostilistalla käydyssä keskustelussa pohdittiin
%mahdollisia ongelmia vahvan tyypityksen (etenkin tyyppiluokkien) käytöstä
%matalan tason ohjelmoinnissa~\citep{bitc}. Shapiron sähköpostissa todetaan,
%että tyyppiluokkia ei voi toteuttaa ilman suoritusaikaista tukea.
%
%On huomioitava, että lukuisten olemassa olevien C:n kirjastojen, rajapintojen
%ja projektien vuoksi yhteensopivuus C:n kanssa tulee olla saumatonta
%mahdollisten vaihtoehtoisten ohjelmointikielten osalta, jotta kielen
%vaihtaminen olisi mahdollista. Tämä sisältää myös kirkastorutiinien kutsumisen
%muista ohjelmointikielistä, sillä C on lukuisissa järjestelmissä \emph{lingua
%franca}, jonka avulla ohjelmointikielet pystyvät kommunikoimaan keskenään.
%Esimerkiksi Python-ohjelmointikieli~\citep{python} sisältää tuen C-funktioiden
%kutsumiseen~\citep{pythonffi}, jota voi käyttää muiden ohjelmointikielten
%funktioiden kutsumiseen, mikäli käyttäjä kirjoittaa ''sillan'' C-ohjelmana.
%Käytännössä jokaisesta aktiivisesti käytetystä ohjelmointikielestä on
%mahdollista kutsua C-koodia.
%
%Ohjelmointikielten ekosysteemit koostuvat eri tahojen luomista kirjastoista. On
%tärkeää, että näihin kirjastoihin pääsee käsiksi mahdollisimman helposti.

% Kirjastojen lisenssien tulee myös olla yhteensopivia, jotta kirjastoja voi
% käyttää yhdessä toistensa kanssa. Eri ohjelmointikieliekosysteemeissä on
% käytössä erilaisia lisenssejä -- JavaScript-kirjastot ovat usein
% MIT-yhteensopivia, kun taas Java-kirjastot ovat usein Apache 2 -yhteensopivia.
% Liitteessä~\ref{app:github} on taulukko GitHub-verkkopalvelun sisältämien
% julkisten projektien määrä ryhmiteltynä lisenssin ja kielen mukaan.
% 
% Kirjastot voi myös julkaista useammalla kuin yhdellä lisenssillä, jolloin
% käyttäjät voivat päättää, mitä lisenssiä haluaa seurata. On kuitenkin
% filosofinen kysymys, onko esimerkiksi Apache2+GPL parempi lisenssi kuin pelkkä
% GPL, sillä Apache2 ei vaadi tiettyjä oikeuksia loppukäyttäjille, kuten pääsyä
% ohjelmiston lähdekoodiin~\citep{apachetldr, gpl3tldr}. Apache2 siis antaa
% ohjelmoijille enemmän vapauksia GPL3:een verrattuna, mutta käyttäjien vapaus
% esimerkiksi muokata ohjelmistoa kärsii tästä.

\subsection{C:hen verrattavissa olevat ohjelmointikielet}

Historian saatossa on tehty useita C:n kilpailijoita, jotka ovat yrittäneet
parantaa C:tä joidenkin C:n hyvien puolien kustannuksella. Muutamat näistä ovat
päätyneet hyvin suosituiksi ohjelmointikieliksi, kuten esimerkiksi C++ ja Go.
Kielten suosion mittaamiseen on tehty useita projekteja, jotka vertailevat
kieliä esimerkiksi hakutulosten tai projektien mukaan. Näitä ovat esimerkiksi
TIOBEn ohjelmointikielten suosion indeksi~\citep{tiobe} ja GitHubin julkaisema
Octoverse~\citep{octoverse}.%Valitsemalla vertailuun suosittuja kieliä voidaan tutkia, miksi muuten selkeästi täysin käyttökelpoinen kieli ei ole syrjäyttänyt C:tä.

Koska tutkimuskysymys on rajattu suorituskyvyn ja muistinkäytön mukaisesti,
vertailuun kannattaa ottaa mukaan vain tehokkaita kieliä -- korkeamman tason
ohjelmointikielet on tarkoitettu ohjelmointinopeuden parantamiseen ja
turvallisempien ohjelmistojen toteuttamiseen nopeiden ohjelmien sijaan. Tällöin
kielen suorituskyky on heikompi. Yksi kattava suorituskykyä mittaava vertailu
on Benchmarks Game~\citep{benchmarks}, jossa pyritään kirjoittamaan
mahdollisimman nopea ohjelma pysyen silti kielelle idiomaattisessa
lähdekoodissa\footnote{Hyvin monessa kielessä voi kirjoittaa C:hen
verrattavissa olevaa matalan tason ohjelmointia, mutta Benchmarks Gamessa on
tarkoituksena välttää tätä.}.

TIOBEn listasta Ada, C++, D, Go ja Rust nousevat esiin verrattavina kielinä.
Kaikki viisi kieltä ovat tehokkaita. Tämän lisäksi jokaisen kielten historiassa
on ollut tavoitteena korvata C:n tai C++:n käyttö, kuten luvussa~\ref{sec:muut}
kerrotaan.

Viime vuosina on tehty myös useita C:hen käännettäviä ohjelmointikieliä, jotka
ovat jääneet pitkälti ilman mitään näkyvyyttä, kuten LISP/c~\citep{clisp1},
C-Mera~\citep{clisp2}, Carp~\citep{clisp3} ja Nymph~\citep{nymph}. LISP/c,
C-Mera ja Carp ovat LISP-perheeseen kuuluvia C:ksi kääntyviä ohjelmointikieliä,
jotka pyrkivät parantamaan C:n syntaksia korvaamalla sen LISP-perheen
syntaksilla. Nymph taas on olio-ohjelmointikieli. Erityisesti Carp on tämän
tutkielman kannalta kiintoisa ohjelmointikieli, sillä se on C:ksi kääntyvä
ohjelmointikieli, joka on suunniteltu mahdollisimman suorituskykyiseksi.

%\hl{Tässä voisi olla jotain pohdintaa, miksi nuo ovat jääneet huomiotta ja
%miten tämän voisi välttää. Erityisesti Carp tuntuu tätä tutkielmaa vastaavalta
%kieleltä, olisi ikävää jättää se täysin huomiotta. Tosin tällaiset pohdinnat
%jäänee puhtaasti spekulaatioksi, sillä noista ei ole erityisen paljoa dataa
%saatavilla.}

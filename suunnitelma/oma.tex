\section{Uuden kielen mahdolliset ominaisuudet}

\subsection{C:n kehitettävissä olevat ominaisuudet}

C on lähes kaikissa moderneissa järjestelmissä käytetty ohjelmointikieli, jota
käytetään matalan tason ohjelmointiin. C mahdollistaa erityisesti nopeutta tai
pientä muistijalanjälkeä vaativien sovelluksien toteuttamisen. C on hyvin
lähellä konekieltä~--~lähdekoodista voi usein päätellä hyvin suurella
tarkkuudella, miksi konekielen käskyiksi se kääntyy. C on suosittu erityisesti
käyttöjärjestelmien ytimien toteutukseen sekä sulautettujen järjestelmien
toteuttamiseen.

C:llä kirjoitetut ohjelmat ovat usein pidempiä kuin modernimmilla
ohjelmointikielillä kirjoitetut vastaavat ohjelmat~\citep{codelength,
qsmcodelength}. Koska C on kielenä hyvin yksinkertainen, siinä on hyvin vähän
valmiita rakenteita ohjelman rakenteen hallitsemiseen. Erityisesti C:llä
kirjoitetuissa ohjelmissa hyvin tehty virheidenhallinta vie paljon lähdekoodia,
sillä ohjelmoijan pitää erikseen tunnistaa virhetilanne, jonka jälkeen
ohjelmoijan tulee tilanteesta riippuen joko palauttaa jokin tietty arvo,
vapauttaa muistia tai tehdä jotain muuta.

Ohjelma~\ref{fig:cerrorhandling} esittelee C:n virheidenkäsittelyä. Ohjelmassa
tarkistetaan useita arvoja, jotka vaikuttavat funktion kontrollivuohon.
Ensimmäisessä osassa tarkistetaan ajurin tilaa
\texttt{dev\_priv}-muistiosoittimen läpi, ja palautetaan virhearvo
\texttt{-EINVAL} tai \texttt{-ENODEV} riippuen virheestä. Toisessa osassa
tarkistetaan \texttt{oa\_get\_render\_ctx\_id}-funktion paluuarvo, ja
palautetaan sen palauttama virhekoodi virhetilanteessa. Tämän jälkeen
tarkistetaan \texttt{get\_oa\_config}-funktion paluuarvo, ja virhetilanteessa
hypätään \texttt{err\_config}-osioon, jossa mahdollisesti kutsutaan
\texttt{oa\_put\_render\_ctx\_id}-funk\-ti\-o\-ta, joka vapauttaa
\texttt{oa\_get\_render\_ctx\_id}-funktion varaaman semaforin.

Ohjelmassa on useita tyypillisiä piirteitä C:n virheidenkäsittelystä. Alussa
tarkistetaan ohjelman tilan invariantteja palauttaen tiettyjä virhearvoja
riippuen arvoista. Myöhemmin käsitellään muiden funktioiden paluuarvoja ja
mahdollisesti vapautetaan ohjelman varaamia resursseja virhetilanteiden
yhteydessä.

C:n makrojärjestelmä on hyvin yksinkertainen. Kun esikääntäjä tunnistaa
lähdekoodissa makron tai makrokutsun, esikääntäjä korvaa sen makron
määritelmällä. C:n makrojärjestelmä on kuitenkin rajoittunut -- rekursiivisten
makrojen rakentaminen on mahdotonta, sillä itsensä sisältäviä makroja ei voi
muodostaa johtuen makrojärjestelmän rajoituksista~\citep[luku 6.10.3.4]{C18}.

\FloatBarrier

\begin{listing}[ht!]
    \inputminted{C}{c-error-handling.c}
    \caption{Linux-kernelin i915-näytönohjainajurin lähdekoodia
    typistettynä~\citep[\texttt{drivers/gpu/drm/i915/i915\_perf.c}, funktio
    \texttt{i915\_oa\_stream\_init}]{i915debugfs}.}
    \label{fig:cerrorhandling}
\end{listing}

\FloatBarrier

\FloatBarrier

\begin{listing}[ht!]
    \inputminted{C}{c-hygiene.c}
    \inputminted{text}{c-hygiene-output.txt}
    \caption{C:n ja C++:n makrot eivät ole hygieenisiä. DOUBLE-makro muuttuu
    käännösvaiheessa muotoon $1+1*2$, joka on laskujärjestyksen takia 3 eikä
    odotettu 4.}
    \label{fig:cmacro}
\end{listing}

\FloatBarrier

C:n ja C++:n makro eivät ole hygieenisiä, kun taas esimerkiksi Rustin makrot
ovat. Ohjelmassa~\ref{fig:cmacro} on esimerkki C:llä kirjoitetusta
epähygieenisestä makrosta. \texttt{DOUBLE}-makro muuttuu käännösvaiheessa
muotoon $1+1*2$, joka on laskujärjestyksen takia 3 eikä odotettu 4.
Lopulliseksi printf-kutsuksi muodostuu siis \texttt{printf("One plus one
doubled is \%d", 1+1*2)}. Ongelman voi kiertää tässä tapauksessa
määrittelemällä \texttt{DOUBLE}-makron \texttt{x*2} sijaan \texttt{((x)*2)},
jolloin sulut säilyttävät oikean laskujärjestyksen. C:n makrot eivät myöskään
pysty muokkaamaan lähdekoodin rakennetta. Vahva makrojärjestelmä mahdollistaisi
yksinkertaisen kielen, jota voi muokata kirjastoilla ilman suoritusaikaisia
haittoja.

C:n tyyppisyntaksi on uudempien ohjelmointikielten tyyppisyntakseihin
verrattuna hyvin sekava, sillä muuttujan tunniste on usein syntaktisesti osa
tyyppimäärittelyä. Erityisesti osoitintyypit ja funktio-osoitintyypit ovat
hyvin epäselviä. Esimerkiksi funktio-osoitinmuuttujamäärittely
\texttt{int~(*fp)(int~arg,~char~arg2);} määrittelee osoitinmuuttujan
\texttt{fp}, joka ottaa argumenteiksi kaksi \texttt{int}-muuttujaa, ja
palauttaa arvon tyyppiä \texttt{int}.

Monimutkaisissa tapauksissa on vaikeaa seurata muuttujien oikeaa tyyppiä, ja
uudemmissa ohjelmointikielissä onkin selkeämpiä tapoja merkitä tyyppejä.
Edellisen esimerkin funktio-osoittimen määrittelyn voi ilmaista TypeScriptin
kaltaisella syntaksilla \texttt{fp:~(int,~char)~=>~int}, jossa tyypin voi lukea
suoraviivaisesti vasemmalta oikealle, aloittaen muuttujan nimestä.

%\begin{listing}[ht!]
%    \inputminted{Kotlin}{c-error-handling-maybe.kt}
%    \caption{Vastaava lähdekoodi kuin ohjelmassa~\ref{fig:cerrorhandling},
%    mutta Kotlinin~\citep{kotlin} kaltaisella syntaksilla. Muuttuja \texttt{it}
%    viittaa nimettömiin lambda-parametreihin. Tämän version lähdekoodissa on
%    kolme riviä vähemmän ja noin 10\% vähemmän merkkejä. Lambdapohjainen
%    lähestymistapa ei toimi hyvin matalalla tasolla, sillä sulkeumien
%    alustariippumaton toteutus on haastavaa ilman dynaamista muistinhallintaa.
%    Esimerkin voisi kuitenkin kääntää vastaamaan
%    ohjelman~\ref{fig:cerrorhandling} lähdekoodia.}
%    \vspace*{1cm}
%\end{listing}

\FloatBarrier

\begin{listing}[ht!]
    \inputminted{C}{c-overflow.c}
    \inputminted{text}{c-overflow-output.txt}

    \caption{Kokonaisluvun ylivuoto C-kielessä. C-kielen
    spesifikaatiossa kokonaislukujen ylivuoto on määrittelemätöntä toimintaa,
    ja GCC-kääntäjän eri optimointitasoilla ohjelma käyttäytyy eri tavoin.}
    \label{fig:coverflow}
\end{listing}

\FloatBarrier

C on suunniteltu olemaan mahdollisimman alustariippumaton, mutta jos tämä
aiheuttaa ristiriidan mahdollisimman nopean toteutuksen kanssa, suositaan
nopeutta -- useista C:n operaatioista tulee eri tuloksia riippuen
alustasta~\citep[liite J, luku J.3]{C18}. Tästä yleisin esimerkki on
kokonaislukujen ylivuoto, jonka tulos riippuu kääntäjäoptimoinnista, kuten
ohjelma~\ref{fig:coverflow} näyttää.

C:ssä useat operaatiot voivat aiheuttaa määrittelemätöntä
toimintaa\defword{undefined behavior}. Esimerkiksi ylivuodon käyttäytyminen
C:ssä riippuu kääntäjäoptimoinnin määrästä -- esimerkiksi GCC-kääntäjän version
7.3.0 optimointitasolla~0 lauseketta $x~+~1~<=~x$ ei optimoida pois.
Optimointitasolla~3 kääntäjä käyttää hyväksi sitä, että \mbox{C-kielen}
spesifikaatiossa kokonaislukujen ylivuoto on määrittelemätöntä
toimintaa~\citep[liite J, luku J.2]{C18} ja poistaa lausekkeen. Luonnollisten
lukujen yli- ja alivuoto taas on tarkasti määritelty -- minkä tahansa
laskutoimituksen yli- tai alivuodon tulos on aina $x$ modulo $n$, missä $x$ on
laskutoimituksen tulos ja $n$ on yhtä suurempi kuin suurin laskutoimituksen
tyypin mahdollinen arvo~\citep[luku 6.2.5]{C18}.

\begin{listing}[ht!]
    \inputminted{C}{openssl_md5.c}

    \caption{OpenSSL-kirjaston~\citep{openssl} MD5-tiivisteen laskevan ohjelman
    R0-makro. Ylimmässä versiossa on alkuperäinen versio, keskimmäisessä on
    C-versio, josta on laajennettu \texttt{F}- ja \texttt{ROTATE}-makrot ja
    alimmassa versiossa on teoreettista \texttt{<<<}-operaattoria käyttävä
    laajennettu versio.}

    \label{fig:opensslmd5}
\end{listing}

C:stä puuttuu useita matalan tason operaatioita, joista yksi on erityisesti
kryptografiassa käytetty kiertobittisiirto\defword{circular bit shift}, joka
löytyy lähes kaikista moderneista prosessoreista suoraan konekäskynä.
C:n syntaksi ei mahdollista uusien operaattoreiden määrittämistä, vaan
ohjelmoijat joutuvat käyttämään tavallisia funktioita. Tämä pitää kielen
eksplisiittisenä, mutta kasvattaa lähdekoodin pituutta.
Ohjelmassa~\ref{fig:opensslmd5} näytetään, miten lähdekoodia voisi selkeyttää
käyttämällä käyttäjän määrittelemiä operaattoreita. Ohjelmassa esitetään useita
bittitason operaatioita osana MD5-tiivisteen laskemista. Esimerkissä on kolme
vaihtoehtoa, joista ensimmäisessä on siivottu alkuperäinen makron sisältö, kun
taas toisessa vaihtoehdossa makrojen sisältö on laajennettu loppuun.
Kolmannessa vaihtoehdossa käytetään teoreettista \texttt{<<<}-operaattoria,
joka tarkoittaa esimerkissä kiertobittisiirtoa vasemmalle ja vastaa
alkuperäisen ohjelman ROTATE-makroa.

C:n standardikirjastossa on suuri määrä merkkijonojen hallintaan tarkoitettuja
funktioita. Nämä eivät kuitenkaan tue monitavuisia merkistöjä\defword{multibyte
character set}, kuten Unicodea~\citep{unicode11}\footnote{Esimekiksi
\texttt{strlen}-funktio palauttaa Unicode-merkkijonon pituuden sijaan tavujen
määrän ennen ensimmäistä tavua, jonka lukuarvo on 0.}, vaan toimivat pelkästään
yksittäisistä tavuista koostuvista merkkijonoilla. Standardikirjastosta löytyy
myös sekä monitavuisia merkkijonoja että monitavuisia merkkejä varten
funktioita, mutta niiden osalta C-standardi on tarkoituksenmukaisesti jätetty
avoimeksi. Joustavampi ratkaisu olisi irrottaa merkkijonojen käsittely kielestä
erilliseksi kirjastoksi, ja mahdollistaa merkkijonojen muodon määrittäminen
kirjastotasolla. Kieli voisi tarjota oletuksena jonkin
merkistökoodauksen\defword{encoding} merkkijonoliteraaleille, kuten vaikka
UTF-8:n~\citep[s. 36]{unicode11}, joka on laajalti käytössä moderneissa
järjestelmissä. UTF-8 on myös taaksepäin yhteensopiva 7-bittisen
ASCII-merkistön kanssa, joka on käytössä C:n kanssa käytännössä jokaisessa
järjestelmässä, jossa C toimii.

\FloatBarrier

\subsection{Mahdollisia uusia ominaisuuksia}

\subsubsection{Tyypit}

Suoritusaikaisen tyyppijärjestelmän tulee olla täysin yhteensopiva C:n kanssa.
Kieli ei siis voi sisältää esimerkiksi automaattista muistinhallintaa, sillä
sen integroiminen olemassa olevaan C-ohjelmaan ei ole saumatonta. Myös
tiettyjen muiden ominaisuuksien, kuten rajapintojen, toteuttaminen on hankalaa.

Tyhjä osoitin\defword{null pointer} on erityinen osoitinmuuttuja, joka osoittaa
johonkin olemattomaan muistialueeseen. Tyhjien osoittimien osoittaman arvon
hakeminen tai muuttaminen johtaa C:ssä määrittelemättömään toimintaan, ja
useissa muissa ohjelmointikielissä heittää poikkeuksen. Tyhjiä osoittimia on
kuvailtu ''Miljardin dollarin virheeksi'' niiden keksijän
toimesta~\citep{billiondollars}. Useat modernit ohjelmointikielet ovatkin
poistaneet muuttujilta mahdollisuuden olla tyhjiä osoittimia, kuitenkin
mahdollistaen tyhjät osoittimet esimerkiksi yhdellä merkillä lisää
tyyppimäärittelyssä.

Jos uuteen kieleen otetaan tällainen oletus, C-funktiomäärittelyiden
osoitinmuuttujien tyypin tulee saumattoman yhteistyön takia olla helposti
muutettavissa näiden kahden tilan välillä, sillä C:n syntaksi ei sisällä
lupauksia mahdollisista tyhjistä osoittimista -- tai niiden puutteesta.
Ohjelmoijaa siis ei tule pakottaa tarkistamaan tyhjiä osoittimia jos hän kutsuu
jotain C-funktiota, mutta toisaalta ei myöskään varoittaa ''turhista''
tarkistuksista. Kotlin-ohjelmointikieli~\citep{kotlin} on toteutettu tällä
periaatteella. Mikä tahansa Kotlinista kutsuttu Java-funktio voi palauttaa
\texttt{null}in, joka voisi väärin käytettynä aiheuttaa poikkeuksia Kotlinilla
kirjoitetussa ohjelmissa. Kotlin olettaa tyypin erityiseksi tyypiksi, joka
sallii tarkistukset tyhjiä osoittimia varten, mutta ei vaadi
niitä~\citep{kotlinnullability}.

%Esimerkiksi Rustin suoritusaikainen tyypitys on lähellä C:tä; Rustin
%\texttt{Option<T>} vastaa C:n osoitinmuuttujia\footnote{Vastaava tietotyyppi
%on Javan \texttt{Optional<T>} ja Haskellin \texttt{Maybe a}.}.
%\texttt{Option<T>} vaatii, että sen sisältö ei ole tyhjä osoitin\defword[~--
%onko tähän parempia käännöksiä?]{NULL pointer}. Tällöin
%\texttt{Option}-muuttuja voi ilmaista tyhjää arvoa olemalla suoritusaikaisesti
%identtinen C:n NULL-arvon kanssa, ja on siten suoritusaikaisesti muistissa
%identtinen C-osoitinmuuttujien kanssa. Tällöin C-rajapintojen
%osoitinargumenteiksi voi antaa \texttt{Option<T>} -muuttujia.

Yksi hyvin suosituista ominaisuuksista moderneissa ohjelmointikielissä on
tyyppipäättely. Oikein käytettynä sillä voidaan poistaa lähdekoodista
pääteltävissä olevat tyypit, jolloin ohjelmoija voi keskittyä
ohjelmalogiikkaan. Tyyppipäättely onkin erityisen kätevä lambdafunktioiden
kirjoittamisessa, sillä ohjelmoijan ei tarvitse kirjoittaa tyyppejä, jotka
kääntäjä voi päätellä.

Summatyypit\defword{sum type \emph{tai} tagged union} ovat useasta modernista
ohjelmointikielestä löytyvä ominaisuus. Toisin kuin
tulotyypeillä\defword{product type}, joissa tyypin mahdolliset arvot ovat
tulotyypin sisältämien arvojen tulojoukko eli tyyppien
monikko\footnote{Tyypillisesti tulotyypit määritellään ohjelmointikielissä
tietueina\defword{struct}, joissa jokaista tyyppiä vastaa jokin tunniste.},
summatyyppi voi sisältää vain yksittäisen tyypin arvon kerrallaan. Tällöin
tyypin arvojoukko on siis sen sisältämien tyyppien arvojoukkojen summa. C:llä
summatyypin voi määritellä esimerkiksi yhdistämällä \texttt{enum}- ja
\texttt{union}-tyypin ohjelman~\ref{fig:sumtypec} esittelemällä tavalla.
Useassa modernissa ohjelmointikielessä kääntäjä voi toteuttaa tämän
tyyppiturvallisesti, jolloin summatyypin sisältämä arvo ei voi olla itsensä
kanssa ristiriidassa.

Ohjelmassa~\ref{fig:sumtype} määritetään Rustin syntaksilla summatyyppi, jossa
luetellaan värejä. Summatyyppi \texttt{Color} sisältää parametrittomat
vaihtoehdot \texttt{Red}, \texttt{Green} ja \texttt{Blue}. Näiden lisäksi
\texttt{Color} sisältää yhden parametrin ottavan
\texttt{Grayscale}-vaihtoehdon, kolmen parametrin \texttt{RGB}-vaihtoehdon ja
neliparametrisen \texttt{RGBA}-vaihtoehdon. Kaikki \texttt{Color}issa
esiintyvät parametrit ovat yhden tavun kokoisia arvoja (\texttt{i8}).
Ohjelmassa~\ref{fig:sumtypec} määritellään vastaava tyyppi C:llä. C-versio on
huomattavasti monimutkaisempi. Tämän lisäksi C-versiossa on mahdollista, että
\texttt{type}-kenttä ei vastaa unionin todellista sisältöä, mikä voi johtaa
määrittelemättömään toimintaan luettaessa unionin arvoa~\citep[luku
6.7.2.1]{C18}. Tämä voisi tapahtua esimerkiksi tilanteessa, jossa
\texttt{type}-kenttä sisältää arvon \texttt{RGBA}, kun \texttt{d}-kentän arvona
on \texttt{empty}-tietue. Tällöin muiden \texttt{d}-kentän arvojen lukeminen on
määrittelemätöntä toimintaa.

\FloatBarrier

\begin{listing}[ht!]
    \inputminted{Rust}{sumtype.rs}
    \caption{Esimerkki Rustissa summatyypin määrittelystä.}
    \label{fig:sumtype}
\end{listing}

\FloatBarrier

\begin{listing}[ht!]
    \inputminted{C}{sumtype.c}
    \caption{Vastaavan tyypin määrittely C:llä. C:ssä on mahdollista, että
    \texttt{type}-kenttä ei vastaa unionin todellista sisältöä, jolloin
    ohjelman tila voi johtaa määrittelemättömään toimintaan.}
    \label{fig:sumtypec}
\end{listing}

\FloatBarrier


\subsubsection{Syntaksi}

Jotta kielestä toiseen vaihtaminen olisi kannattavaa, uuden kielen on oltava
tarpeeksi erilainen: samanlainen kieli ei tarjoa tarpeeksi parannusta
olemassa olevaan, että siirtyminen olisi kannattavaa. C:n syntaksia on hyvin
vuolasta verrattuna moderneihin ohjelmointikieliin, joten syntaksia
typistämällä voisi saada huomattavia parannuksia.

%\begin{listing}[ht!]
%    \inputminted{C}{squaresum.c}
%    \inputminted{Haskell}{squaresum.hs}
%    \caption{Project Eulerin ongelma nro.\ 6~\citep{euler}. Ylempi on
%    C-kieltä, kun taas alempi esimerkki on kirjoitettu Haskellilla.
%    Haskell-esimerkin koodi vie vain kaksi riviä, kun taas C-koodi vie
%    yhdeksän. Molemmat ohjelmat laskevat kaavan
%    $(\sum\limits_{i=1}^n i)^2 - \sum\limits_{i=1}^n i^2$ tuloksen.
%    }
%    \label{fig:strcmp}
%\end{listing}
%
%\FloatBarrier

Syntaksin tulisi kuitenkin olla mahdollisimman lähellä C:tä, jotta siirtyminen
kielestä toiseen olisi mahdollisimman helppoa. Tämä myös helpottaa kielen
toteuttamista ja käyttöä, sillä käännettäessä C:ksi lähdekoodin rakenne pysyy
lähes samanlaisena. C:n kanssa on kuitenkin hankalaa käsitellä poikkeuksellisia
arvoja, jotka ovat kuitenkin hyvin tavallisia C:ssä. Virhetilanteiden
käsittelyssä voisi siis olla parannettavaa. Yksi suosittu ominaisuus
moderneista ohjelmointikielistä on hahmontunnistus\defword{pattern matching},
jolla voisi parantaa C:n virheidenkäsittelyä ilman ajoaikaisia haittoja.

%\newpage
%
%Yksi suosittu ominaisuus on hahmontunnistus\defword{pattern matching}, jolla
%voidaan kasvattaa kielen ilmaisuvoimaa. Hahmontunnistus soveltuu erityisen
%hyvin monimutkaiseen arvojen käsittelyyn. Esimerkiksi Rustissa
%hahmontunnistusta voidaan käyttää erityisesti yhdistettynä summatyyppeihin,
%kuten ohjelma~\ref{fig:guards} näyttää. Ohjelmassa Rivien 8--12
%\texttt{match}-lauseke käsittelee kolme \texttt{OptionalInt}in mahdollista
%tilaa: arvo on olemassa ja on suurempi kuin viisi, arvo on olemassa ja arvoa
%ei ole olemassa. Ohjelma tulostaa lauseen ''Got an int!''. Ohjelman tyyppi
%\texttt{OptionalInt} on summatyyppi, jonka mahdolliset arvot ovat
%\texttt{Missing} ja \texttt{Value}, joista jälkimmäinen sisältää muuttujan
%tyyppiä \texttt{i32}.
%
%\FloatBarrier
%
%\begin{listing}[ht!]
%    \inputminted{Rust}{guards.rs}
%    \caption{Rust-kirjan esimerkki Rustin
%    hahmontunnistuksesta~\citep{rustguards} hieman yksinkertaistettuna. Rivien
%    8--12 \texttt{match}-lauseke käsittelee kolme \texttt{OptionalInt}in
%    mahdollista tilaa: arvo on olemassa ja on suurempi kuin viisi, arvo on
%    olemassa, ja arvoa ei ole olemassa. Ohjelma tulostaa lauseen ''Got an
%    int!''. Ohjelman tyyppi \texttt{OptionalInt} on summatyyppi, jonka
%    mahdolliset arvot ovat \texttt{Missing} ja \texttt{Value(i32)}.}
%    \label{fig:guards}
%\end{listing}
%
%\FloatBarrier
%
%\hl{Tässä tuntuu, että käsittely rönsyilee turhan paljon}

\subsubsection{Makrot}

Koska kieli voi erota C:stä käännösaikaisesti, kielen makrojärjestelmä on yksi
harvoista osa-alueista, joita voi muokata melko vapaasti. Uuden
makrojärjestelmän tulisi olla sekä turvallinen että voimakas. Esimerkiksi
Rustin makrojärjestelmä on vahvasti tyypitetty sekä
Turing-täydellinen~\citep{rustmacros}. Tämä mahdollistaa myös monimutkaisten
makrojen kirjoittamisen ja käytön. Rustin makrosyntaksi on kuitenkin
ensisilmäyksellä vaikeaselkoista, kun taas C:n makrot ovat pitkälti
helppolukuisia.

Joidenkin C-ohjelmien otsikkotiedostoissa luodaan makroja, joita
otsikkotiedoston käyttäjät voivat käyttää. Kielen tulee mahdollistaa makrojen
tuottaminen, joita voi käyttää myös C-ohjelmista. Kehittyneempien
makrojärjestelmien kaikkia makroja ei kuitenkaan voi muuntaa C:n makroiksi,
sillä C:n makrot ovat hyvin rajoittuneita. Kielen tulisi kuitenkin pystyä
ymmärtämään C:n makroja.

\subsubsection{Suoritusaikaiset ominaisuudet}

Uuden ohjelmointikielen tulee olla mahdollisimman yhteensopiva C:n kanssa. Tämä
onnistuu parhaiten pitämällä suoritusaikaisten ominaisuuksien määrän mahdollisimman
pienenä. Muistinhallinnan tulee olla manuaalinen, jotta kutsuessa C-funktioita
kielet voivat saumattomasti ja tehokkaasti välittää toisilleen
osoitinmuuttujia.
%Moniajoa ei
%voi alustariippumattomuuden takia tukea suoraan standardikirjastossa, sillä
%kaikki alustat eivät tue rinnakkaisuutta. Standardi voi kuitenkin luoda
%rajapinnan, jonka moniajoa tukevat alustat voivat käyttää.

%C-kääntäjissä on aina alustariippuvaisia ominaisuuksia, ja yksi tärkeimmistä on
%funktioiden kutsuminen. GCC:n~\citep{gcc} kanssa yhteensopivat kääntäjät
%tukevat funktiomäärittelyn kohdalla \texttt{\_\_attribute\_\_()} -määrettä,
%jolla voi määrätä, mikä funktion kutsukonventio on.

C++:n yksi tärkeistä ominaisuuksista on nimiavaruudet, joita C:ssä ei ole. Jos
samassa C-ohjelmassa on kaksi samannimistä funktiota, ei ole määritelty, kumpaa
niistä kutsutaan. C++:ssa ongelma on osittain ratkaistu nimiavaruuksilla, jotka
on toteutettu nimiruntelulla. Tämä on kuitenkin aiheuttanut ristiriitoja sekä
C-yhteensopivuuden kanssa että eri C++-kääntäjien yhteensopivuuden kanssa.
Nimiruntelu voi myös aiheuttaa pidempiä nimiä funktioille, jolloin kirjastot
voivat viedä useita tavuja enemmän levytilaa jokaista kirjaston funktiota
kohden. Tiedosto- ja projektikohtaisella nimiruntelun määrittelyllä saisi
tehtyä alustariippumattoman nimiruntelun, jonka määrittäminen ei olisi kielen
vastuulla. Useassa C-projektissa on projektin sisäinen nimeämiskonventio, joten
kielen sisältämä tuki määriteltävälle nimiruntelulle voi helpottaa funktioiden
nimeämistä.

Jotta kieli on mahdollisimman kevyt ja alustariippumaton, kielen
standardikirjaston tulee olla mahdollisimman kevyt. Kielen makrojärjestelmän
tulee olla tarpeeksi voimakas, jotta monimutkaiset ominaisuudet voidaan
toteuttaa puhtaasti kirjastotasolla. Tällöin käyttämättömät kielen tai
standardikirjaston ominaisuudet eivät aiheuta ajoaikaisia haittoja.

%Go-kielen yksi innovatiivisimmista ominaisuuksista on \texttt{defer}-avainsana,
%jolla voidaan viivästyttää laskentaa funktiosta poistumiseen asti. Esimerkiksi
%\emph{A Tour of Go} -sivuston esimerkissä~\citep{gotourdefer} käytetään
%\texttt{defer}iä Hello World -ohjelman toteuttamiseen:
%
%\begin{listing}[ht!]
%    \inputminted{Go}{defer.go}
%    \inputminted{text}{defer-output.txt}
%    \caption{\emph{A Tour of Go} -sivuston esimerkki \texttt{defer} -avainsanan
%    käytöstä. \texttt{defer}-avainsana siirtää world-sanan tulostamisen
%    funktion loppuun, jolloin ohjelma tulostaa sanat ''hello'' ja ''world''.}
%\end{listing}
%
%\FloatBarrier
%
%Ilman automaattista muistinhallintaa kaikissa tilanteissa toimivaa
%\texttt{defer}iä ei voi toteuttaa, mutta yksinkertaisissa tilanteissa myös
%hyvin matalan tason kieli voisi toteuttaa \texttt{defer}-avainsanan.
%Avainsanalle tulisi käyttöä -- esimerkiksi Linux-ytimessä on yleinen
%suunnittelumalli vapauttaa kaikki funktion varaama muisti käänteisessä
%järjestyksessä virhetilanteissa. Vain virhetilanteissa ajettu \texttt{defer}
%mahdollistaisi huomattavaa lähdekoodin siivoamista -- ohjelmoijan ei
%tarvitsisi ajatella muistin vapauttamista, sillä ohjelmointikieli tekisi sen
%ohjelmoijan puolesta.

Tällaisesta ominaisuudesta esimerkki on lambdalausekkeet, joita käytetään
erityisesti funktionaalisen ohjelmoinnin yhteydessä. C tukee funktioita
parametreina funktio-osoittimien avulla, eli yksinkertaisella
syntaksimuunnoksella voisi tukea lambdalausekkeita (mutta ei sulkeumia)
heikentämättä C-yhteensopivuutta. Tuki sulkeumille \texttt{gcc}-kääntäjään on
toteutettu käyttämällä Thomas Breuelin esittelemää
trampoliinimenetelmää~\citep{gccnested, cppclosure}. Tämä ei kuitenkaan ole
standardien mukaista C:tä, vaan \texttt{gcc}-kääntäjän lisäominaisuus.

%Ohjelmointikieli voi toteuttaa olio-ohjelmoinnin rajapinnat käytännössä kahdella
%tavalla: Joko korvaamalla jokaisen rajapintaviittauksen numerolla jaettuun
%tauluun funktioita (vtable-rajapinnat), tai korvaamalla rajapintaviittaukset
%tietueella, joka sisältää rajapinnan toteutuksen funktioiden osoitteet
%(tietue-rajapinnat). Ensimmäistä käytetään eritoten C++:n rajapintojen
%toteuttamiseen, kun taas jälkimmäistä käytetään esimerkiksi Go-kielessä.
%Jälkimmäistä käytetään myös C:n kanssa; Linux-ytimen laiteajurit on toteutettu
%tietue-rajapintoina, joita ajurit voivat tarvittaessa periä. Molemmissa
%toteutustavoissa on omat hyvät jä huonot puolensa -- vtable-rajapinnat vievät
%vähemmän muistia, kun taas tietue-rajapinnat ovat yksinkertaisempia
%toteuttaa~\citationneeded.
%
%C tukee yksinkertaista poikkeusten
%käsittelyä \texttt{setjmp}- ja
%\texttt{longjmp}-funktioilla\footnote{\label{cspecnote}C90: Luku 4.6. Tämä tarvitsee viitteen
%oikeaan C:n speksin lukuun. Minulla ei ole pääsyä uusimpaan spesifikaatioon,
%eli en pääse (vielä) tarkistamaan, mistä kohtaa tämä löytyy C11:n
%standardista.}, mutta C:n kanssa käytetään lähes poikkeuksetta paluuarvoihin
%perustuvaa virheidenkäsittelyä\citationneeded.

\subsubsection{Kääntäminen ja työkalut}

Jotta kieltä voi käyttää nykyisissä ympäristöissä, sen tulee mahdollistaa C:ksi
kääntäminen. Tällöin ohjelmointikieli toimii kaikissa ympäristöissä, joihin on
olemassa C-standardin mukainen C-kääntäjä. Tämä helpottaa kielen käyttämistä
nykyisten työkalujen kanssa.
%-- esimerkiksi \texttt{GNU Make} -työkalulla~\citep{gnumake} uuden kielen voi
%integroida projektiin kahdella rivillä.

%\inputminted{make}{Makefile.kieli}

%Tämän jälkeen uuden kieli-tiedostoja vastaavat .c-tiedostot voi poistaa, ja
%\texttt{Make} osaa luoda ne .kieli-tiedostojen pohjalta.

Kääntäjän tulee pystyä tuottamaan tarvittaessa ylläpitokelpoisia .c-tiedostoja,
jotta kielestä voi myös tarvittaessa siirtyä pois helposti.  Kielen tulee myös
pystyä lukemaan C:n otsikkotiedostoja, jotta ohjelmoijilta ei mene turhaan
aikaa funktiomäärittelyiden kääntämiseen kieleltä toiselle. Vastaavasti kielen
tulee myös tarvittaessa tuottaa C-kielen otsikkotiedostoja -- näin voi helposti
jatkaa olemassa olevien C-kirjastojen kehittämistä uudella kielellä.

%\begin{listing}[ht!]
%    \inputminted[firstline=4]{C}{vectorization.c}
%
%    \caption{Vektorisointi nopeuttaa fn\_size -funktion noin kolme kertaa
%    nopeammaksi fn\_iter -versioon verrattuna.}
%\end{listing}

Modernissa ohjelmointikielessä on tärkeänä osana kehittäjän hyvinvointi, jota
voidaan ylläpitää hyvillä työkaluilla. Go-kieli kehitettiin yhdessä kattavan
työkalupaketin kanssa, jolloin työkalut pysyivät mukana kielen kehityksessä.
Go-kielen kääntäjän tulee yhteydessä muun muassa paketinhallinta, lähdekoodin
automaattinen muotoilija sekä ohjelmointivirheiden tarkastaja. Uuden kielen
yhteydessä tulisi kirjoittaa helppokäyttöiset mutta joustavat työkalut.

%Kirjoittamalla kääntäjän ensin C:llä ja vasta myöhemmin funktio kerrallaan
%siirtymällä uuteen kieleen voidaan esitellä kielen mahdollisuutta toimia C:n
%korvaajana. Vaihtoehtoisesti kääntäjän voi kirjoittaa jollain toisella
%kielellä, jolloin voi todistaa uuden kielen C-yhteensopivuutta alkuperäisen
%kielen siltausominaisuuksilla.


% ---------------------

%Liitteissä~\ref{app:grammar-ml},~\ref{app:grammar-lisp}~ja~\ref{app:grammar-c}
%on konsepteja mahdollisista kielen syntakseista.
%Liitteessä~\ref{app:grammar-ml} on ML-perheen kaltainen syntaksi,
%liitteessä~\ref{app:grammar-lisp} LISP-perheen kaltainen syntaksi ja
%liitteessä~\ref{app:grammar-c} on C-perheen kaltainen syntaksi.
%
%ML-perheen syntaksia seuraava esimerkki on tutkielman kannalta kiintoisin,
%sillä se tarjoaa C:stä huomattavasti poikkeavan syntaksin, kuitenkin pitäen
%C:n ominaisuudet. On olemassa useita sekä LISP-perheen kieliä sekä C-perheen
%kieliä, jotka kääntyvät C:ksi, mutta kaikki ML-kielet vaativat automaattisen
%muistinhallinnan. Manuaalisella muistinhallinnalla varustettu ML-kieli on siis
%kiintoisa aihe tutkia, sillä siitä ei ole aikaisempaa tutkimusta.

\section{Johdanto} 

C~\citep{C11} on ollut vallitseva ohjelmointikieli järjestelmäohjelmoinnissa
C:n alkuajoista lähtien. Useita ohjelmointikieliä on luotu historian saatossa,
joiden oli tarkoitus syrjäyttää C, mutta C on vieläkin johtavana kielenä
varsinkin sulautetuissa järjestelmissä ja UNIX-pohjaisten käyttöjärjestelmien
vallitsevana ohjelmointikielenä. C on myös käytössä
Windows-käyttöjärjestelmäperheen ydinkomponenttien toteutuksessa.
Opinnäytetyössä tutkitaan, miksi vaihtoehdoista huolimatta C on vieläkin
laajalti käytössä myös uusissa projekteissa ja minkälainen ohjelmointikieli
voisi syrjäyttää C:n.

C:n vaihtoehdoiksi tutkitaan seuraavia tehokkaaseen ohjelmointiin tarkoitettuja
kieliä: Ada~\citep{ADA12}, C++~\citep{CPP14}, D~\citep{D}, Go~\citep{golang}
sekä Rust~\citep{rust}. Näistä kielistä tutkitaan, mikä tai mitkä ominaisuudet
ovat estäneet C:n korvaamisen ja mitkä ominaisuudet ovat olleet parannuksia
C:hen verrattuna. Lisäksi tutkitaan muista suosituista ohjelmointikielistä
korkean tason ominaisuuksia, jotka ovat hyödyllisiä matalan tason
ohjelmoinnissa ja jotka voi toteuttaa korvaavan kielen määrittelyssä luotujen
rajoitusten puitteissa. \hl{rajoitteissa -> yhteydessä: minusta
''rajoitteissa'' on parempi, sillä kieli speksataan tietyillä rajoitteilla
(muistinkäyttö yms.) - tuo lause voisi tietysti reflektoida tätä
(''...toteuttaa korvaavan kielen määrittelyssä luotujen rajoitusten
puitteissa'' tjsp.) Muutin tämän nyt tuohon muotoon (ennen: ''...toteuttaa
korvaavan kielen rajoitteissa.'')}

Tutkittavana on myös, mitä optimointeja C:ssä ei voi tehdä helposti johtuen
kielen rajoitteista ja miten tämän voisi korjata. Näitä ominaisuuksia ovat
esimerkiksi sivuvaikutuksettoman ohjelmakoodin merkitseminen,
optimointivinkkien alustariippumaton ilmaiseminen (esimerkiksi Rustin
funktioiden annotaation) ja funktioiden yliajaminen\defword{function
overloading} riippuen parametrien tyypistä tai arvoista, mikäli ne voidaan
kääntöaikaisesti päätellä. Koska C:n spesifikaatio ei salli useita samannimisiä
funktioita, kääntäjä ei voi helposti optimoida tällaisia tapauksia.

Tämän tutkielmasuunnitelman toisessa luvussa määritellään, minkälaiset
ominaisuudet tarvitaan kieleltä, joka voisi korvata C:n kokonaan. Toisessa
luvussa esitetään myös tutkielman kannalta oleellinen teoria, eli
ohjelmointikielten nykypäivää sekä kerrotaan lyhyesti muista tutkittavista
kielistä. Kolmannessa luvussa kerrotaan, miksi muut tutkitut kielet eivät täytä
näitä ominaisuuksia, ja miten C:tä voisi parantaa. Neljännessä luvussa
kuvaillaan toisen luvun määrittämien ominaisuuksien täyttävä ohjelmointikieli.
\hl{Ja sitten myöhemmin: Viidennessä luvussa kuvaillaan tutkielman rakenne}

\hl{Miten käännökset tulisi ilmaista? Alaviitteeseen\defword{footnote} vai
sulkuihin (eng. \emph{parens})?}

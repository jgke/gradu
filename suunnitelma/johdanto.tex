\section{Johdanto} 

C~\citep{C11} on ollut vallitseva ohjelmointikieli järjestelmäohjelmoinnissa
C:n alkuajoista lähtien. Useita ohjelmointikieliä on luotu historian saatossa,
joiden oli tarkoitus syrjäyttää C, mutta C on vieläkin johtavana kielenä
varsinkin sulautetuissa järjestelmissä ja UNIX-pohjaisten käyttöjärjestelmien
vallitsevana ohjelmointikielenä. C on myös käytössä
Windows-käyttöjärjestelmäperheen ydinkomponenttien toteutuksessa.
Opinnäytetyössä tutkitaan, miksi vaihtoehdoista huolimatta C on vieläkin
laajalti käytössä myös uusissa projekteissa ja minkälainen ohjelmointikieli
voisi syrjäyttää C:n. Syrjäyttämisessä tutkitaan erityisesti mahdollisuuksia
olemassaolevien C-kielellä kirjoitettujen ohjelmien jatkokehittämistä uudella
ohjelmointikielellä ja uusien projektien toteuttamista C:n sijaan uudella
ohjelmointikielellä.

C:n vaihtoehdoiksi tutkitaan seuraavia tehokkaaseen ohjelmointiin tarkoitettuja
kieliä: Ada~\citep{ADA12}, C++~\citep{CPP14}, D~\citep{D}, Go~\citep{golang}
sekä Rust~\citep{rust}. Näistä kielistä tutkitaan, mikä tai mitkä ominaisuudet
ovat estäneet C:n korvaamisen ja mitkä ominaisuudet ovat olleet parannuksia
C:hen verrattuna. Lisäksi tutkitaan muista suosituista ohjelmointikielistä
korkean tason ominaisuuksia, jotka ovat hyödyllisiä matalan tason
ohjelmoinnissa ja jotka voi toteuttaa korvaavan kielen määrittelyssä luotujen
rajoitusten puitteissa.

Tutkittavana on myös, mitä optimointeja C:ssä ei voi tehdä helposti johtuen
kielen rajoitteista ja miten tämän voisi korjata. Näitä ominaisuuksia ovat
esimerkiksi sivuvaikutuksettoman ohjelmakoodin merkitseminen,
optimointivinkkien kääntäjäriippumaton ilmaiseminen (esimerkiksi funktioiden
koristelulla\defword{function annotation}) ja funktioiden
yliajaminen\defword{function overloading} riippuen parametrien tyypistä tai
arvoista, mikäli ne voidaan kääntöaikaisesti päätellä. Koska C:n spesifikaatio
ei salli useaa samannimistä funktiota, funktioiden yliajaminen on haastavaa
toteuttaa alustariippumattomasti.

Tämän tutkielmasuunnitelman toisessa luvussa määritellään seuraavat
ominaisuudet kielelle, joka voisi korvata C:n kokonaan:

Vaihtoehtoinen kieli aina
\begin{itemize}[topsep=0pt]
    \item on yhtä nopea tai nopeampi kuin C,
    \item käyttää saman verran muistia tai vähemmän kuin C,
    \item toimii kaikissa järjestelmissä, joissa C toimii,
    \item toimii saumattomasti C:n kanssa ja
    \item toimii saumattomasti muiden kielien C-rajapintojen kanssa.
\end{itemize}
Jotta jokin kieli olisi absoluuttisesti parempi, vähintään yhden näistä
kriteereistä tulee olla aidosti parempi.

Lisäksi kielistä tutkitaan muita suosioon vaikuttavia syitä, kuten
helppokäyttöisyyttä sekä kielen luojan aiheuttamaa suosiota (esimerkiksi
Googlen tarjoama tuki Go-kielelle).

Toisessa luvussa esitetään tutkielman kannalta oleellinen teoria, eli
määritetään tutkielman kontekstissa paremman kielen rajoitteet, pohditaan
kielen suosioon vaikuttavia tekijöitä sekä kerrotaan lyhyesti muista
tutkittavista kielistä. Kolmannessa luvussa kerrotaan, miksi muut tutkitut
kielet eivät täytä näitä ominaisuuksia. Neljännessä luvussa kerrotaan, miten
C:tä voisi parantaa.
%kuvaillaan toisen luvun määrittämien ominaisuuksien täyttävä ohjelmointikieli.

\hl{Miten käännökset tulisi ilmaista? Alaviitteeseen\footnote{eng. \emph{footnote}} vai
sulkuihin (eng. \emph{parens})?}

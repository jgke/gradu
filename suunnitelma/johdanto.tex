\section{Johdanto} 

\hl{Johdantoluvun perusteella ei edellenkään ymmärrä, mikä tarkkaan ottaen on
gradun tavoite. Koko suunnitelman luettua se jollain tapaa selviää. Minusta
näyttää, että tavoitteena on suunnitella (jollain tasolla) uusi
ohjelmointikieli, joka voisi korvata C-kielen systeemiohjelmoinnissa, eikö?}

\hl{Motivaatio tälle jää kyllä vähän epäselväksi (miksi tarvittaisiin jokin
uusi kieli). Rivien välistä lukemalla se jotenkin aukeaa, mutta varsinaiseen
tutkielmaan tutkielman tavoitteet ja motivaatio pitää hioa kirkkaammiksi.
Vaikka sitten reverse-engineeraamalla tuloksista (missä on riskinsä).}

\hl{Gradusta puuttuu C-kielen esittely ja analyysi. C-kieli mainitaan
tutkielman otsikossa ja C on tietysti erityisen olennainen asia koko käsittelyn
kannalta. Mikä on kielen tausta, tavoitteet, kehitys ja nykytilanne? Mitkä
olivat kielen kehittämisen alkuperäiset syyt ja mistä sen saama suosio johtui?
C-kielestä tulee tutkielman alussa olla oma pääluku. Samalla tulee määriteltyä
ja käsiteltyä aihepiirin käsitteistöä ja termejä. Suunnitelmaan ei minusta
tarvitse lisätä C:n kuvausta.}

\hl{Suunnitelmassa ja itse tutkielmassa perusteellisemmin: miksi C pitäisi tai
halutaan korvata? Mitkä ovat C:n ongelmat ja puutteet?}

\hl{C:n tärkeimpänä kilpailijana voi pitää C++-kieltä, joten tarvitaan
vertailua C++:aan. C++ on usein käytännössä korvannut C-kielen toteutuskielenä.
Mihin tämä perustuu? Ja miksei C++ C-kielen suorana laajennoksena riittäisi sen
korvaajaksi?}

\hl{Johdannon tulee olla aiheeseen johdatteleva: taustaa, motivointia,
rajaukset ja kysymyksenasettelut. Varsinainen asian käsittely  eli määritelmät,
tekniset asiat jne. alkavat vasta toisessa luvussa. Tämä koskee sekä
suunnitelmaa että tutkielman johdantolukua.}

\hl{Yhteenvedossa ei saa olla alilukuja. Yhteenveto on koko tutkielman sisällön
läpikäynti eikä siinä saa olla uutta asiaa. Kaikki tulokset, analyysi ja
merkittävyyden arviointi esitetään sitä edeltävässä luvussa.}

\hl{Kuvat, koodiesimerkit ja taulukot yms. numeroidaan tutkielmassa
päälukukohtaisesti. Samoin voi tehdä jo suunnitelmassa. Kaikkiin kuviin tulee
viitata tekstissä.}

\hl{Tekstissä esiintyy turhaa it-slangia, ainakin seuraavat:  tokenisointi  po.
alkioanalyysi, linkkeri  po. linkittäjä,   paketti (packet) po. pakkaus
(package), koodi po. (lähde)- ohjelma tai ohjelmisto.}

\hrule

\hl{Johdannon luonne!}

C~\citep{C18} on ollut vallitseva ohjelmointikieli järjestelmäohjelmoinnissa
C:n alkuajoista lähtien. Useita ohjelmointikieliä on luotu historian saatossa,
joiden oli tarkoitus syrjäyttää C, mutta C on vieläkin johtavana kielenä
varsinkin sulautetuissa järjestelmissä ja UNIX-pohjaisten käyttöjärjestelmien
vallitsevana ohjelmointikielenä. C on myös käytössä
Windows-käyttöjärjestelmäperheen ydinkomponenttien toteutuksessa.
Tutkielmassa selvitetään, mitkä vaihtoehtoisten kielten ominaisuudet ovat
voineet estää kielen käytön C:n sijaan uusissa ja olemassa olevissa projekteissa
ja minkälaista ohjelmointikieltä voisi aina ominaisuuksiensa puolesta käyttää
C:n sijaan. Vertailussa tutkitaan erityisesti mahdollisuuksia olemassa olevien
C-kielellä kirjoitettujen ohjelmien jatkokehittämistä uudella kielellä ja
uusien projektien toteuttamista C:n sijaan uudella kielellä.

\hl{uusi kieli}

C:n vaihtoehdoiksi tutkitaan seuraavia tehokkaaseen ohjelmointiin tarkoitettuja
kieliä: Ada~\citep{ADA12}, C++~\citep{CPP17}, D~\citep{D}, Go~\citep{golang}
sekä Rust~\citep{rust}. Näistä kielistä tutkitaan, mikä tai mitkä ominaisuudet
ovat estäneet C:n korvaamisen ja mitkä ominaisuudet ovat olleet parannuksia
C:hen verrattuna. Lisäksi tutkitaan muista suosituista ohjelmointikielistä
korkean tason ominaisuuksia, jotka ovat hyödyllisiä matalan tason
ohjelmoinnissa ja jotka voi toteuttaa korvaavan kielen määrittelyssä luotujen
rajoitusten puitteissa.

%Tutkittavana on myös, mitä optimointeja C:ssä ei voi tehdä helposti johtuen
%kielen rajoitteista ja miten tämän voisi korjata. Näitä ominaisuuksia ovat
%esimerkiksi sivuvaikutuksettoman ohjelmakoodin merkitseminen,
%optimointivinkkien kääntäjäriippumaton ilmaiseminen (esimerkiksi funktioiden
%koristelulla\defword{function annotation}) ja funktioiden
%yliajaminen\defword{function overloading} riippuen parametrien tyypistä tai
%arvoista, mikäli ne voidaan kääntöaikaisesti päätellä. Koska C:n spesifikaatio
%ei salli useaa samannimistä funktiota, funktioiden yliajaminen on haastavaa
%toteuttaa alustariippumattomasti.

Tämän tutkielmasuunnitelman toisessa luvussa määritellään seuraavat
ominaisuudet kielelle, joka voisi korvata C:n kokonaan:

Vaihtoehtoinen kieli aina
\begin{itemize}[topsep=0pt,itemsep=0pt]
    \item on yhtä nopea tai nopeampi kuin C,
    \item käyttää saman verran muistia tai vähemmän kuin C,
    \item toimii kaikissa järjestelmissä, joissa C toimii,
    \item toimii saumattomasti C:n kanssa ja
    \item toimii saumattomasti muiden kielien C-rajapintojen kanssa.
\end{itemize}
Jotta jokin kieli olisi absoluuttisesti parempi, vähintään yhden näistä
kriteereistä tulee olla aidosti parempi.

%Lisäksi kielistä tutkitaan muita suosioon vaikuttavia syitä, kuten
%helppokäyttöisyyttä ja sosiaalisia aspekteja, kuten Googlen tarjoaman tuen
%vaikutusta Go-kielen suosioon.%\hl{Vai tutkitaanko?}

\hl{kieli (alleviivattu 'kontekstissa' ja 'miten C:tä voisi parantaa')}

Tämän tutkimussuunnitelman toisessa luvussa esitetään tutkielman kannalta
oleellinen teoria, eli määritetään tutkielman kontekstissa paremman kielen
rajoitteet, pohditaan kielen suosioon vaikuttavia tekijöitä sekä kerrotaan
lyhyesti muista tutkittavista kielistä. Kolmannessa luvussa kerrotaan, miksi
muut tutkitut kielet eivät täytä näitä ominaisuuksia. Neljännessä luvussa
kerrotaan, miten C:tä voisi parantaa.

\hl{Suunnitelmasta ja ehdotetusta tutkielmasta puuttuu kokonaan C:n esittely ja tausta (kuka, milloin, miksi, kehitys, nykytilanne)}

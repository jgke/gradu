\section{Gradun sisällysluettelo}

\subsection{Johdanto}

Johdantoluku, jossa avataan tutkielman aihe eli C-ohjelmointikielen vertaaminen
muihin ohjelmointikieliin sekä uuden ohjelmointikielen luominen tiettyjen
kriteerien ohjaamana.

\subsection{Taustaa}

Tässä luvussa kerrotaan ohjelmointikielten taustoista ja yleisistä
ominaisuuksista, luodaan määrittelyt vertailuun liittyville kriteereille,
perustelut verrattavien kielten valinnoille sekä pohditaan erilaisia
mahdollisia syitä kielten suosioon. Tämän luvun sisältö on hyvin samankaltainen
tutkimussuunnitelman kirjallisuusosan toisen luvun kanssa.

\subsubsection{C-ohjelmointikielen taustaa}

Tässä aliluvussa kerrotaan C-kielen taustasta, alkuperäisistä käyttökohteista,
historiasta, suosion kehityksestä sekä nykytilanteesta. Erityisesti keskitytään
C:n tämänhetkisiin käyttökohteisiin, joista selviää, mitä ohjelmointikielten
ominaisuuksia tulisi käsitellä ohjelmointikielten vertailussa.

\subsubsection{Kehitettävissä olevat ominaisuudet C-ohjelmointikielessä}

Tässä aliluvussa pohditaan, miksi C pitäisi korvata uudella
ohjelmointikielellä, mitä ominaisuuksia C:stä voisi kehittää, mitkä C:n
ominaisuudet voisi jättää pois ja mitä uusia ominaisuuksia voisi tulla.

Näitä ominaisuuksia ovat ainakin tarkempi käännösaikainen tyypitys etenkin
tyhjien osoittimien osalta, makrojärjestelmän uusiminen sekä syntaksin
selkeyttäminen.

\subsubsection{Ohjelmointikielten vertailun kriteerit}

Jotta kielten vertailun voi tehdä objektiivisesti, valitaan vertailua varten
jotkin kriteerit. Kriteereiksi valitaan suorituskyky, muistinkäyttö ja
yhteensopivuus C:n ja muiden ohjelmointikielten kanssa. Vertailtavat kielet
eivät saa olla C:tä huonompia millään osa-alueella.

\subsubsection{C:hen verrattavissa olevat ohjelmointikielet}

Vertailtaviksi kieliksi valitaan Ada, C++, D, Go ja Rust. Kaikki vertailtavat
kielet ovat kohtalaisen tehokkaita, jonka lisäksi kaikilla on yritetty korvata
C:n tai C++:n käyttöä.

\subsubsection{Kielten suosioon vaikuttavat tekijät}

Analyysiä muista syistä kielen suosioon, erityisesti Meyerovichin ja Rabkinin
(\citeyear{empiricalpopularity}) tekemän tutkimuksen perusteella. Tämä antaa
tietoa ominaisuuksista, jotka ohjelmointikielessä olisi hyvä olla.

%\hl{Gradusta puuttuu C-kielen esittely ja analyysi. C-kieli mainitaan
%tutkielman otsikossa ja C on tietysti erityisen olennainen asia koko käsittelyn
%kannalta. Mikä on kielen tausta, tavoitteet, kehitys ja nykytilanne? Mitkä
%olivat kielen kehittämisen alkuperäiset syyt ja mistä sen saama suosio johtui?
%C-kielestä tulee tutkielman alussa olla oma pääluku. Samalla tulee määriteltyä
%ja käsiteltyä aihepiirin käsitteistöä ja termejä. Suunnitelmaan ei minusta
%tarvitse lisätä C:n kuvausta.}

\subsubsection{Makrojärjestelmät}

Tässä aliluvussa selitetään makrojärjestelmien käsite, kerrotaan
makrojärjestelmien tarpeellisuudesta ja käyttökohteista sekä verrataan
vaihtoehtoisia toteutuksia makrojärjestelmiin.

\subsection{Ohjelmointikielten vertailu}

Vertailuluku, jossa käydään läpi toisessa luvussa määriteltyjä kriteerejä
kunkin kielen kohdalla, ja pohditaan, täyttääkö kukin kieli kriteerejä.
Jokaiselle kielelle tulee yksi aliluku, jossa käydään kriteerejä läpi kyseisen
kielen yksilöllisten ominaisuuksien kannalta. Tämän luvun sisältö on hyvin
samankaltainen tutkimussuunnitelman kirjallisuusosan kolmannen luvun kanssa.

\subsubsection{Yleisiä vertailtavien ohjelmointikielten ominaisuuksia}

Kooste verrattavien ohjelmointikielten yhteisistä ominaisuuksista. Koska
verrattavissa ohjelmointikielissä on useita ominaisuuksia, joita C:ssä ei ole,
mutta muissa verrattavissa kielissä on, tällä aliluvulla voidaan merkittävästi
poistaa turhaa toistoa.

\subsubsection{Jokaiselle verrattavalle kielelle oma aliluku}

Jokaiselle kielelle tulee oma aliluku, jossa kerrotaan kyseisen
ohjelmointikielen uniikeista ominaisuuksista.

\subsubsection{Yhteenveto}

Tässä aliluvussa tehdään yhteenveto ohjelmointikielten vertailusta, ja
todetaan, miksi yksikään verrattavista kielistä ei täytä toisessa luvussa
määriteltyjä vertailukriteerejä.

\subsection{Uuden ohjelmointikielen määrittäminen}

Tässä luvussa määritetään uusi ohjelmointikieli toisen ja kolmannen luvun
tulosten ohjaamana.

\subsubsection{Tyypit}

Koska kieli ei voi erota ratkaisevasti C:stä suoritusaikaisesti, tyypitys on
todennäköisesti hyvin lähellä C:n tyypitystä. Tutkielmassa tutkitaan, voiko
yleisiä käyttötapauksia helpottaa, kuten virhearvoja, geneerisiä
tietorakenteita sekä summatyyppejä. Näistä geneeriset tietorakenteet ovat
luultavasti liian monimutkainen asia sisällytettäväksi yksinkertaisena
pidettävään kieleen johtuen niiden vaatimasta geneerisestä ohjelmoinnista.
Esimerkiksi summatyypit taas voidaan toteuttaa helposti myös C:llä.

\subsubsection{Syntaksi}

Kielestä tulee todennäköisesti myös syntaksin kannalta C:n sukulaiskieli. Tämä
helpottaa kielen omaksumista, mikäli ohjelmoija osaa
C:tä~\citep{languagelearning}. Erityisesti virheidenhallinnassa on kuitenkin
mahdollisuuksia tehdä huomattavia parannuksia C:hen verrattuna.

\subsubsection{Makrot}

Koska makrot eivät aiheuta suoritusaikaisia haittoja, uuden kielen
makrojärjestelmässä voi olla hyvinkin suuria eroavaisuuksia C:n
makrojärjestelmään verrattuna. Yhteensopivuuden parantamiseksi
makrojärjestelmän tulisi kuitenkin ymmärtää C:n makroja sekä mahdollistaa
C-makrojen tuottaminen. Jos makrojärjestelmästä tekee Turing-täydellisen,
kaikkia makroja ei pysty muuntamaan C:n makroiksi johtuen C:n makrojärjestelmän
rajoituksista. Tämän lisäksi hygieeninen makrojärjestelmä ei välttämättä tee
C:n makrojärjestelmän tukemisesta yksinkertaista.

Yksi mahdollinen tapa tuottaa kattava makrojärjestelmä olisi ottaa Rustin
makrojärjestelmästä mallia, mutta tämä jättäisi avoimeksi kysymykseksi
yhteensopivuuden C:n makrojärjestelmän kanssa.

\subsubsection{Suoritusaikaiset ominaisuudet}

Koska kielen tulee olla täysin yhteensopiva suoritusaikaisesti C:n kanssa,
kielessä ei välttämättä ole yhtään C:stä eroavia suoritusaikaisia
ominaisuuksia. Kieli kuitenkin voi mahdollistaa vapaaehtoisia ominaisuuksia,
joita standardien mukaisesta C:stä ei löydy, kunhan ohjelmoijaa ei pakoteta
käyttämään näitä.

\subsubsection{Ohjelmointikielen kääntäminen}

Yhteensopivuuden C:n kanssa saa helposti toteutettua kääntämällä
ohjelmointikielen C:ksi. Tällöin ohjelmointikieli toimii kaikissa
järjestelmissä, joissa C toimii. Tämä mahdollistaa myös uuden ohjelmointikielen
kokeilemisen olemassa olevan C-projektin kanssa ilman, että projektin tulisi
muuten käyttää uutta ohjelmointikieltä.

\subsection{Tulokset}

\subsubsection[Uuden ohjelmointikielen vertaaminen muihin ohjelmointikieliin]
{Uuden ohjelmointikielen vertaaminen muihin \\ ohjelmointikieliin}

Tässä luvussa uutta ohjelmointikieltä vertaillaan muihin tutkielmassa
käsiteltyihin ohjelmointikieliin, ja kerrotaan, miksi tai miksei se päihitä
C:tä vertailuehtojen mukaisesti.

\subsubsection{Tutkielman puutteet}

Jos tutkielmassa on mahdollisia puutteita, tai sen tulokset eivät välttämättä
tutkielman rajoituksista johtuen päde tutkielman ulkopuolella, tässä aliluvussa
perustellaan syitä tälle.

Erityisesti tutkimusaiheen rajaus ei välttämättä yleisty tutkimuksen
ulkopuolelle, sillä ohjelmointikielten valintaan liittyy usein monimutkaisia
syitä, kuten aikaisempi kokemus tai ohjelmoijan henkilökohtainen mieltymys
johonkin kieleen, eikä projekteissa käytettyjä ohjelmointikieliä valita
pelkästään kielen ominaisuuksien perusteella.

\subsubsection{Jatkotutkimuskohteet}

Jos tutkielmassa ilmenee mahdollisia jatkotutkimuskohteita, niistä kerrotaan
tässä aliluvussa.

\subsection{Yhteenveto}

Tässä aliluvussa kerrataan tutkimuksen tulokset, eli kerrotaan lyhyesti uuden
ohjelmointikielen ja muiden verrattavien kielten suhteesta C:hen.

%\renewcommand{\thesubsubsection}{Liite \arabic{subsubsection}. }
%\subsection{Liitteet}
%
%\subsubsection{Kielen syntaksi}
%
%Jos tutkimuksen tuloksiin sisältyy uuden ohjelmointikielen formaali syntaksi,
%se on tässä liitteessä.

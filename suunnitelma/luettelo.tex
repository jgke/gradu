\section{Gradun sisällysluettelo}

\subsection{Johdanto}

Johdantoluku, jossa avataan tutkielman aihe, eli C-ohjelmointikielen
vertaaminen muihin ohjelmointikieliin, sekä uuden ohjelmointikielen luominen
tiettyjen kriteerien ohjaamana.

\subsection{Ohjelmointikielten vertailuun liittyvät määritelmät}

Määrittelyt vertailuun liittyville kriteereille, perustelut verrattavien
kielten valinnoille sekä pohdintaa erilaisista mahdollisista syistä kielten
suosioon. Tämän luvun sisältö on hyvin samankaltainen tutkimussuunnitelman
kirjallisuusosan toisen luvun kanssa.

\subsubsection{Ohjelmointikielten vertailun kriteerit}

Jotta kielten vertailun voi tehdä objektiivisesti, valitaan vertailua varten
jotkin kriteerit. Kriteereiksi valitaan suorituskyky, muistinkäyttö ja
yhteensopivuus C:n ja muiden ohjelmointikielten kanssa. Vertailtavat kielet
eivät saa olla C:tä huonompia millään osa-alueella.

\subsubsection{Kielten suosioon vaikuttavat tekijät}

Analyysiä muista syistä kielen suosioon, erityisesti Meyerovichin ja Rabkinin
(\citeyear{empiricalpopularity}) tekemän tutkimuksen perusteella. Tämä antaa
mahdollista tietoa ominaisuuksista, jotka ohjelmointikielessä tulisi olla.

\subsubsection{C:hen verrattavissa olevat ohjelmointikielet}

Vertailtaviksi kieliksi valitaan Ada, C++, D, Go ja Rust. Kaikki vertailtavat
kielet ovat kohtalaisen tehokkaita, jonka lisäksi kaikilla on yritetty korvata
C:n tai C++:n käyttöä.

\subsection{Ohjelmointikielten vertailu}

Vertailuluku, jossa käydään läpi toisessa luvussa määriteltyjä kriteerejä
kunkin kielen kohdalla, ja pohditaan, täyttääkö kukin kieli kriteerejä.
Jokaiselle kielelle tulee yksi aliluku, jossa käydään kriteerejä läpi kyseisen
kielen kohdalla. Tämän luvun sisältö on hyvin samankaltainen
tutkimussuunnitelman kirjallisuusosan kolmannen luvun kanssa.

\subsubsection{Yleisiä vertailtavien ohjelmointikielten ominaisuuksia}

Kooste verrattavien ohjelmointikielten yhteisistä ominaisuuksista. Koska
verrattavissa ohjelmointikielissä on useita ominaisuuksia, joita C:ssä ei ole,
mutta muissa verrattavissa kielissä on, tällä aliluvulla voidaan merkittävästi
poistaa turhaa toistoa.

\subsubsection{Jokaiselle verrattavalle kielelle oma aliluku}

Jokaiselle kielelle tulee oma aliluku, jossa kerrotaan kyseisen
ohjelmointikielen uniikeista ominaisuuksista.

\subsubsection{Yhteenveto}

Tässä aliluvussa tehdään yhteenveto ohjelmointikielten vertailussa, ja
todetaan, miksi yksikään verrattavista kielistä ei täytä toisessa luvussa
määriteltyjä vertailukriteerejä.

\subsection{Kehitettävissä olevat ominaisuudet C-ohjelmointikielessä}

Analyysiä C:n ominaisuuksista, ja mitä näistä voisi kehittää toisen luvun
kriteerien mukaan. Luku sisältää myös pohdintaa ominaisuuksista, joita C:hen
voisi lisätä ilman, että kieli olisi toisen luvun kriteerien mukaan C:tä
huonompi.

\subsection{Uuden ohjelmointikielen määrittäminen}

Määritetään uusi ohjelmointikieli lukujen 2--4 tulosten ohjaamana.

\subsubsection{Tyypit}

Koska kieli ei voi erota ratkaisevasti C:stä suoritusaikaisesti, tyypitys on
todennäköisesti hyvin lähellä C:n tyypitystä. Tästä huolimatta tutkielmassa
tutkitaan mahdollisuuksia helpottaa yleisiä käyttötapauksia, kuten virhearvoja,
tietorakenteita sekä summatyyppejä. Näistä tietorakenteet ovat luultavasti
liian monimutkainen asia sisällytettäväksi yksinkertaisena pidettävään kieleen
johtuen niiden vaatimasta geneerisestä ohjelmoinnista. Esimerkiksi summatyypit
taas voidaan toteuttaa helposti myös C:llä.

\subsubsection{Syntaksi}

Kielestä  tulee todennäköisesti myös syntaksin kannalta C:n sukulaiskieli. Tämä
helpottaa kielen omaksumista, mikäli ohjelmoija osaa aikaisemmin
C:tä~\citep{languagelearning}. Erityisesti virheidenhallinnassa on kuitenkin
mahdollisuuksia tehdä huomattavia parannuksia C:hen verrattuna.

\subsubsection{Makrot}

Koska makrot eivät aiheuta suoritusaikaisia haittoja, tässä osiossa voi tehdä
hyvinkin suuria eroavaisuuksia C:n makrojärjestelmään. Yhteensopivuuden
parantamiseksi makrojärjestelmän tulisi kuitenkin ymmärtää C:n makroja, sekä
mahdollistaa C-makrojen tuottaminen.

\subsubsection{Suoritusaikaiset ominaisuudet}

Koska kielen tulee olla täysin yhteensopiva suoritusaikaisesti C:n kanssa,
kielessä ei välttämättä ole yhtään suoritusaikaisia ominaisuuksia. Kieli
kuitenkin voi mahdollistaa vapaaehtoisia ominaisuuksia, joita standardien
mukaisesta C:stä ei löydy, kuten koodin ajamisen ennen ohjelman suorituksen
aloittamista. Esimerkiksi \texttt{gcc}-kääntäjä tukee tätä
ominaisuutta~\citep{gccattributes}.

\subsubsection{Ohjelmointikielen kääntäminen}

Yhteensopivuuden C:n kanssa saa helposti toteutettua kääntämällä
ohjelmointikielen lähdekoodin C:ksi.

\subsection[Uuden ohjelmointikielen vertaaminen muihin ohjelmointikieliin]
{Uuden ohjelmointikielen vertaaminen muihin \\ ohjelmointikieliin}

Tässä luvussa uutta ohjelmointikieltä vertaillaan muihin tutkielmassa
käsiteltyihin ohjelmointikieliin, ja kerrotaan, miksi tai miksei se päihitä
C:tä vertailuehtojen mukaisesti.

\subsection{Yhteenveto}

Yhteenveto tutkimuksen lopputuloksista, pohdintaa mahdollisista tutkimusaiheen
rajoittamisen aiheuttamista ongelmista sekä mahdollisista
jatkotutkimuskohteista.

\subsubsection{Tutkimuksen tulokset}

Tässä aliluvussa kerrotaan tutkimuksen tuloksista, eli kerrotaan lyhyesti uuden
ohjelmointikielen ja muiden verrattavien kielten suhteesta C:hen.

\subsubsection{Tutkimusalueen rajoittamisen ongelmat}

Tässä aliluvussa kerrotaan mahdollisista tutkimusalueen rajoittamisesta
aiheuttamista ongelmista tutkielman tuloksien kannalta.

Erityisesti tutkimusaiheen rajaus ei välttämättä päde tutkimuksen ulkopuolella,
sillä ohjelmointikielten valintaan liittyy usein monimutkaisia syitä, kuten
aikaisempi kokemus tai ohjelmoijan henkilökohtainen mieltymys johonkin kieleen,
eikä projekteissa käytettyjä ohjelmointikieliä valita pelkästään kielen
ominaisuuksien perusteella.

\subsubsection{Jatkotutkimuskohteet}

Jos tutkielmassa ilmenee mahdollisia jatkotutkimuskohteita, niistä kerrotaan
tässä aliluvussa.

\renewcommand{\thesubsubsection}{Liite \arabic{subsubsection}. }

\subsection{Liitteet}

\subsubsection{Kielen syntaksi}

Jos tutkimuksen tuloksiin sisältyy kielen formaali syntaksi, se on tässä
liitteessä.

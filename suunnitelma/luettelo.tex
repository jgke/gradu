\section{Gradun sisällysluettelo}

\begin{itemize}
    \item ehdotettu jäsentely perustuu ja on ymmärrettävissä kirjallisuusosassa esitetyn
taustamateriaalin perusteella

    \item tutkielman numeroiduista pääluvuista ja niiden aliluvuista kustakin
        yksi tekstikappale kuvaamaan luvun

        \begin{itemize}
            \item sisältö
            \item tarkoitus/tavoite,
            \item käsittelytapa (kirjallisuuteen perustuva, analyysi/vertailu,
                esimerkin käsittely)
        \end{itemize}


    \item tässä jäsentelyosassa esitetään vapaamuotoisesti ongelmia kysymyksiä,
        hypoteeseja ja pohdiskeluja (toisin kuin kirjallisuusosassa)  eli sisältö
        on kuvailevaa "metatekstiä"

    \item kuvauksessa viitataan lähteisiin ja edeltävään suunnitelman
        kirjallisuusosuuteen (jossa osin valmista/täydennettävää tekstiä)

    \item luettele myös liitteet (koodiesimerkit \& tekniset kuvaukset)

    \item tutkielman jäsentelyn kuvauksen pituus on n. 5 sivua

\end{itemize}

\subsection{Johdanto}

Johdantoluku, jossa avataan tutkielman aihe, eli C-ohjelmointikielen
vertaaminen muihin ohjelmointikieliin, sekä uuden ohjelmointikielen luominen
tiettyjen kriteerien ohjaamana.

\subsection{Ohjelmointikielten vertailuun liittyvät määritelmät}

\subsubsection{jotain}

Määrittelyt vertailuun liittyville kriteereille, perustelut verrattavien
kielten valinnoille sekä pohdintaa erilaisista mahdollisista syistä kielten
suosioon.

Kriteereiksi valitaan suorituskyky, muistinkäyttö ja yhteensopivuus C:n ja
muiden ohjelmointikielten kanssa. Vertailtavat kielet eivät saa olla C:tä
huonompia millään osa-alueella.

Vertailtaviksi kieliksi valitaan Ada, C++, D, Go ja Rust. Kaikki vertailtavat
kielet ovat kohtalaisen tehokkaita, jonka lisäksi kaikilla on yritetty korvata
C:n tai C++:n käyttöä.

\subsection{Ohjelmointikielten vertailu}

Vertailuluku, jossa käydään läpi toisessa luvussa määriteltyjä kriteerejä
kunkin kielen kohdalla, ja pohditaan, täyttääkö kukin kieli kriteerejä.
Jokaiselle kielelle tulee yksi aliluku, jossa käydään kriteerejä läpi

\subsection{Kehitettävissä olevat ominaisuudet C-ohjelmointikielessä}

Analyysiä C:n ominaisuuksista, ja mitä näistä voisi kehittää toisen luvun
kriteerien mukaan. Analyysiä myös ominaisuuksista, joita C:hen voisi lisätä
siten, että toisen luvun kriteerit täyttyvät.

\subsection{Uuden ohjelmointikielen määrittäminen}

Määritetään uusi ohjelmointikieli lukujen 2--4 tulosten ohjaamana.

\subsection{Uuden ohjelmointikielen vertaaminen muihin ohjelmointikieliin}

Verrataan uutta ohjelmointikieltä muihin tutkielmassa käsiteltyihin
ohjelmointikieliin.

\subsection{Yhteenveto}

Yhteenveto tutkimuksen lopputuloksista, pohdintaa mahdollisista tutkimusaiheen
rajoittamisen aiheuttamista ongelmista sekä mahdollisista
jatkotutkimuskohteista.

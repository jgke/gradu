\appendixsection{Hahmotelma kielen syntaksista: ML-perhe}
\appendixlabel{app:grammar-ml}

Tämä on hahmotelma kielen syntaksista ISO/IEC 14977
-standardin~\citep{iso14977} kuvaamassa Backus-Naur -muodossa. Syntaksi on
perustettu Python 3-kielen~\citep{pythonsyntax} ja Haskell
2010-standardin~\citep{haskellsyntax} pohjalta sisennyspohjaiseksi kieleksi.
C-kielen kirjastojen nimisyntaksi on C-standardin~\citep{C11} mukainen.

Syntaksin hyvinä puolina on suuri ilmaisuvoima sekä helppo algebratyyppien
käsittely. Huonoina puolina on monimutkainen syntaksi ja semantiikka.

\inputminted{abnf}{kieli-base.bnf}
\inputminted{abnf}{kieli-imports.bnf}
\newpage
\inputminted{abnf}{kieli-types.bnf}
%\newpage
\inputminted{abnf}{kieli-data.bnf}
\newpage
\inputminted{abnf}{kieli-body.bnf}
\newpage
%\inputminted{abnf}{kieli-macros.bnf}

Esimerkki kielen mukaisesta Hello World -ohjelmasta:

\def\mylexer{kieli_lexer.py:KieliLexer -x}
\inputminted{\mylexer}{example.kieli}

Esimerkki Monad-tyyppiluokan määrittelystä ja käytöstä:

\inputminted{\mylexer}{monad.kieli}

\newpage

Esimerkki Tree-datatyypin määrittelystä ja käytöstä:

\inputminted{\mylexer}{tree.kieli}
\inputminted{text}{tree-output}
\newpage

Tree-esimerkki C-kielellä:

\inputminted{C}{tree.c}

\appendixsection{Hahmotelma kielen syntaksista: ML-perhe}
\appendixlabel{app:grammar-ml}

\bnfdescription.
Syntaksi on perustettu
Python 3-kielen~\citep{pythonsyntax} ja Haskell
2010-standardin~\citep{haskellsyntax} pohjalta sisennyspohjaiseksi kieleksi.
C-kielen kirjastojen nimisyntaksi on C-standardin~\citep{C11} mukainen.

Syntaksin hyvinä puolina on suuri ilmaisuvoima sekä helppo algebratyyppien
käsittely. Huonoina puolina on monimutkainen syntaksi ja semantiikka.

\bnf{1}{17}{kieli.bnf}
\bnf{19}{31}{kieli.bnf} \newpage
\bnf{33}{65}{kieli.bnf} \newpage
\bnf{67}{84}{kieli.bnf}
\bnf{86}{109}{kieli.bnf}

Esimerkki kielen mukaisesta Hello World -ohjelmasta:

\def\mylexer{kieli_lexer.py:KieliLexer -x}
\inputminted{\mylexer}{example.kieli}

Esimerkki Monad-tyyppiluokan määrittelystä ja käytöstä:

\inputminted{\mylexer}{monad.kieli}

\newpage

Esimerkki Tree-datatyypin määrittelystä ja käytöstä:

\inputminted{\mylexer}{tree.kieli}
\inputminted{text}{tree-output}
\newpage

Tree-esimerkki C-kielellä:

\inputminted{C}{tree.c}
